\documentclass{article}
\usepackage{mathtools} 
\usepackage{fontspec}
\usepackage[UTF8]{ctex}
\usepackage{amsthm}
\usepackage{mdframed}
\usepackage{xcolor}
\usepackage{amssymb}
\usepackage{amsmath}

\newmdtheoremenv[
  backgroundcolor=gray!10,
  linewidth=0pt,
  innerleftmargin=10pt,
  innerrightmargin=10pt,
  innertopmargin=10pt,
  innerbottommargin=10pt
]{zgraytheorem}{}
% 定义说明环境样式
\newtheoremstyle{mystyle}% 说明环境样式的名称
  {1em}% 上方间距
  {1em}% 下方间距
  {\normalfont}% 说明内容的字体样式
  {}% 缩进量
  {\bfseries}% 说明标记的字体样式
  {.}% 说明标记和说明内容之间的标点
  {1em}% 说明标记后的水平空间
  {}% 说明标记后的垂直空间
% 使用新定义的样式创建说明环境
\theoremstyle{mystyle}
\newtheorem*{zremark}{说明}


\begin{document}
\title{5.3 习题}
\maketitle

\section*{5.3.1}

\textbf{1.自反性$x=x$。}

证明:

对于任意有理数$\epsilon > 0$,因为$|a_n - a_n| = 0 < \epsilon $,
所以$(a_n)_{n=1}^\infty$与$(a_n)_{n=1}^\infty$是等价的柯西序列,由定义5.3.1可知$x=x$

\textbf{2.对称性$x=y$那么$y=x$。}

证明:

因为$x=y$所以$(a_n)_{n=1}^\infty$与$(b_n)_{n=1}^\infty$是等价的柯西序列,
由柯西序列的等价的定义可知,等价是相互的,所以$y = x$。

\textbf{3.传递性$x=y$和$y=z$那么$x=z$。}

证明:

任意有理数$\epsilon > 0$,$\frac{1}{2}\epsilon > 0$。

由$x=y$可知,两个序列是最终$\frac{1}{/2}\epsilon -$接近的,
所有存在$N \geq 0$使得对于所有的$n \geq N$均有$|a_n - b_n| \leq \frac{1}{2}\epsilon$。

由$y=x$同理可知,存在$N^\prime \geq 0$使得对于所有的$n \geq N^\prime$均有$|b_n - c_n| \leq \frac{1}{2}\epsilon$。

取$M=max(N,N^\prime)$,此时当$n \geq M$,
\begin{align*}
  |a_n - c_n| & = |a_n - b_n + b_n - c_n|                      \\
              & \leq |a_n - b_n| + |b_n - c_n|                 \\
              & \leq \frac{1}{2}\epsilon + \frac{1}{2}\epsilon \\
              & = \epsilon                                     \\
\end{align*}
所以序列$(a_n)_{n=1}^\infty,(c_n)_{n=1}^\infty$是最终$\epsilon -$接近的,于是$x=z$。

\section*{5.3.2}

\textbf{1.$xy$也是实数}

证明:

这里要证明$(a_nb_n)_{n=1}^\infty$也是有理数的一个柯西序列。

我们要证明对每一个$\epsilon > 0$,序列$(a_nb_n)_{n=1}^\infty$是最终$\epsilon-$稳定的。

因为$(a_n)_{n=1}^\infty$是柯西序列,所以是有界的,即存在$M_1 \ge 0$使得$|a_n|\leq M_1$对任意$n \geq 1$都成立。

因为$(b_n)_{n=1}^\infty$是柯西序列,所以是有界的,即存在$M_2 \ge 0$使得$|b_n|\leq M_2$对任意$n \geq 1$都成立。

取$M=max(M_1,M_2)$,则$|a_n|\leq M, |b_n| \leq M$对任意$n \geq 1$都成立。

又因为$(a_n)_{n=1}^\infty$与$(b_n)_{n=1}^\infty$是最终稳定的,
所以任意有理数$\epsilon>0$,存在$N \geq 0$,对$j,k \geq N$有,
\begin{align*}
  |a_j - a_k| \leq \epsilon \\
  |b_j - b_k| \leq \epsilon \\
\end{align*}

由命题4.3.7(h)可知,
\begin{align*}
  |a_jb_j - a_kb_k| & \leq \epsilon|b_j| + \epsilon|a_j| + \epsilon^2 \\
                    & \leq \epsilon (2 + \epsilon)
\end{align*}
由$\epsilon$的任意性可知,$(a_nb_n)_{n=1}^\infty$也是有理数的一个柯西序列。

\textbf{2.$x=x^\prime$,那么$xy = x^\prime y$,即乘法满足替换公理。}

证明:

要证明$xy = x^\prime y$,
即证明序列$(a_nb_n)_{n=1}^\infty$与序列$(a^\prime_nb_n)_{n=1}^\infty$是等价的。

(1)由于$x=x^\prime$可知,序列$(a_n)_{n=1}^\infty$与序列$(a^\prime_n)_{n=1}^\infty$是等价的,
所以对任意$\epsilon > 0$,两个序列是最终$\epsilon -$接近的,
即存在$N \geq 0$使得$n \geq N$有
\begin{align*}
  |a_n - a^\prime_n| & \leq \epsilon \\
\end{align*}

(2)又因为$(b_n)_{n=1}^\infty$是有理数的柯西序列,所以该序列是有界的,
即存在$M \geq 0$,对任意$n \geq 0$有
\begin{align*}
  |b_n| & \leq M \\
\end{align*}

由(1)(2)可知,对任意$n \geq N$有,
\begin{align*}
  |a_nb_n - a^\prime_nb_n| & = |(a_n-a^\prime_n)b_n| \\
                           & = |a_n-a^\prime_n||b_n| \\
                           & \leq M \epsilon
\end{align*}
由$\epsilon$的任意性与$M$是某个确定的有理数可知,序列$(a_nb_n)_{n=1}^\infty$与序列$(a^\prime_nb_n)_{n=1}^\infty$是最终接近的,
所以$xy = x^\prime y$

\section*{5.3.3}
\end{document}