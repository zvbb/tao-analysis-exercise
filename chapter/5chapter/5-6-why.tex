\documentclass{article}
\usepackage{mathtools} 
\usepackage{fontspec}
\usepackage[UTF8]{ctex}
\usepackage{amsthm}
\usepackage{mdframed}
\usepackage{xcolor}
\usepackage{amssymb}
\usepackage{amsmath}

\newmdtheoremenv[
  backgroundcolor=gray!10,
  linewidth=0pt,
  innerleftmargin=10pt,
  innerrightmargin=10pt,
  innertopmargin=10pt,
  innerbottommargin=10pt
]{zgraytheorem}{}
% 定义说明环境样式
\newtheoremstyle{mystyle}% 说明环境样式的名称
  {1em}% 上方间距
  {1em}% 下方间距
  {\normalfont}% 说明内容的字体样式
  {}% 缩进量
  {\bfseries}% 说明标记的字体样式
  {.}% 说明标记和说明内容之间的标点
  {1em}% 说明标记后的水平空间
  {}% 说明标记后的垂直空间
% 使用新定义的样式创建说明环境
\theoremstyle{mystyle}
\newtheorem*{zremark}{说明}


\begin{document}
\title{5.6 为什么}
\maketitle


\textbf{1.设$x \geq 0$是一个非负实数,$n \geq 1$是一个正整数,
  集合$E:=\{y \in R : y \geq 0 \text{且} y^n \leq x\}$,
  此时,集合$E$中包含0}

证明:

按照定义5.6.1 $0^n = 0$,所以$0^n \in E$。

\textbf{2.如果$y > 1$,且$n \geq 1$是一个正整数,那么$y^n > 1$}

证明:

对$n$进行归纳。

$n = 1$时,按照定义5.6.1 $y^n=y^1=y^0 \times y = 1 \times y = y$,因为$y > 1$,
所以$y^n > 1$。

归纳假设$n = k$时,$y^k > 1$。

当$n = k+1$时,$y^{k+1} = y^k \times y$,于是由归纳假设可知,
\begin{align*}
  y^k          & > 1              \\
  y^k \times y & > 1 \times y > 1
\end{align*}

综上,归纳完成。

\textbf{3.如果$y > x$,且$y > 1, n \geq 1$,所以$y^n > x$}

证明:

对$n$进行归纳。

当$n=1$时,$y^1 = y$,所以$y^n > x$。

归纳假设,$n=k$时,$y^k > x$。

当$n = k+1$时,$y^{k+1} = y^k \times y$,由归纳假设可知$y^k > x$,所以,
\begin{align*}
  y^k \times y & > xy
\end{align*}
又因为,
\begin{align*}
  y  & > 1 \\
  yx & > x
\end{align*}
于是$y^k \times y > xy > x$。

综上,归纳完成。

\textbf{4.$x$是一个非负实数,证明$x^{1/1}=x=x^1$}

证明:

由定义5.6.1 可知,$x^1 = x^0 \times x = 1 \times x = x$。

设$E=\{y \in R: y \geq 0 \text{且} y^1 \leq x\}$,由定义5.6.4 可知,$x^{1/1} = sup(E) = x$。

\textbf{5.如果$y$和$z$是正的且$y^n=z^n$,那么$y=z$。}

证明:

假设$y^n = r$,由引理5.6.6(b) 可知$y = r^{1/n}$,

同理$z = r^{1/n}$。

相等的传递性可知$y=z$。

\end{document}