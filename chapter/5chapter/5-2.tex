\documentclass{article}
\usepackage{mathtools} 
\usepackage{fontspec}
\usepackage[UTF8]{ctex}
\usepackage{amsthm}
\usepackage{mdframed}
\usepackage{xcolor}
\usepackage{amssymb}
\usepackage{amsmath}

\newmdtheoremenv[
  backgroundcolor=gray!10,
  linewidth=0pt,
  innerleftmargin=10pt,
  innerrightmargin=10pt,
  innertopmargin=10pt,
  innerbottommargin=10pt
]{zgraytheorem}{}
% 定义说明环境样式
\newtheoremstyle{mystyle}% 说明环境样式的名称
  {1em}% 上方间距
  {1em}% 下方间距
  {\normalfont}% 说明内容的字体样式
  {}% 缩进量
  {\bfseries}% 说明标记的字体样式
  {.}% 说明标记和说明内容之间的标点
  {1em}% 说明标记后的水平空间
  {}% 说明标记后的垂直空间
% 使用新定义的样式创建说明环境
\theoremstyle{mystyle}
\newtheorem*{zremark}{说明}


\begin{document}
\title{5.2 习题}
\maketitle

\section*{5.2.1}

证明:

有理数$\delta >0, \epsilon=\frac{2}{3}\delta > 0$。
由于$(a_n)_{n=1}^\infty$是柯西序列,
所以存在一个$N_a \geq 0$使得$d(a_j,a_k) \leq \frac{1}{2} \epsilon$对所有$j,k \geq N_a$均成立。

由于$(a_n)_{n=1}^\infty$和$(b_n)_{n=1}^\infty$是等价的,
所以存在一个$N_m \geq 0$使得$d(a_n-b_n) \leq \frac{1}{2} \epsilon$对所有$n \geq N_m$均成立。

取$M=max(N_a,N_m)$,所以当取$j,k \geq M$时,

\begin{align*}
  |(a_j-b_j) - (a_k - b_k)|          & \leq |a_j-b_j| + |a_k - b_k| \leq \epsilon       \\
  |(a_j-b_j) - (a_k - b_k)|          & = |a_j-a_k + b_k-b_j| \geq |b_k-b_j| - |a_j-a_k| \\
  \Rightarrow  \epsilon +  |a_j-a_k| & \geq |b_k-b_j|                                   \\
  \Rightarrow \frac{3}{2}\epsilon    & \geq |b_k-b_j|                                   \\
\end{align*}
由此可知$|b_k-b_j| \leq \frac{3}{2}\epsilon = \delta$,所以$(b_n)_{n=1}^\infty$也是柯西序列

\section*{5.2.2}

证明:

(1)充分性

$(a_n)_{n=1}^\infty$是有界的,那么存在$M \geq 0$使得$|a_i| \leq M$对任意的$i \leq 1$均成立。

由于$(a_n)_{n=1}^\infty$和$(b_n)_{n=1}^\infty$是等价的,
所以存在一个$N_m \geq 0$使得$d(a_n-b_n) \leq \epsilon$对所有$n \geq N_m$均成立。
又因为,
\begin{align*}
  |b_n| - |a_n| & \leq |a_n - b_n| \leq \epsilon \\
  |b_n|         & \leq \epsilon + |a_n|          \\
  |b_n|         & \leq \epsilon + M
\end{align*}
且$b_1,b_2,b_3,...,b_{N_m}$是有限序列,所以$b_1,b_2,b_3,...,b_m$也是有界的。
综上,$(b_n)_{n=1}^\infty$是有界的。


\end{document}