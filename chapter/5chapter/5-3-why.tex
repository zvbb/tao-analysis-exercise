\documentclass{article}
\usepackage{mathtools} 
\usepackage{fontspec}
\usepackage[UTF8]{ctex}
\usepackage{amsthm}
\usepackage{mdframed}
\usepackage{xcolor}
\usepackage{amssymb}
\usepackage{amsmath}

\newmdtheoremenv[
  backgroundcolor=gray!10,
  linewidth=0pt,
  innerleftmargin=10pt,
  innerrightmargin=10pt,
  innertopmargin=10pt,
  innerbottommargin=10pt
]{zgraytheorem}{}
% 定义说明环境样式
\newtheoremstyle{mystyle}% 说明环境样式的名称
  {1em}% 上方间距
  {1em}% 下方间距
  {\normalfont}% 说明内容的字体样式
  {}% 缩进量
  {\bfseries}% 说明标记的字体样式
  {.}% 说明标记和说明内容之间的标点
  {1em}% 说明标记后的水平空间
  {}% 说明标记后的垂直空间
% 使用新定义的样式创建说明环境
\theoremstyle{mystyle}
\newtheorem*{zremark}{说明}


\begin{document}
\title{5.3 文中的为什么}
\maketitle

\textbf{$-LIM_{n \rightarrow \infty}a_n = LIM_{n \rightarrow \infty}(-a_n)$}

证明:

由实数负运算的定义可知,
\begin{align*}
  -LIM_{n \rightarrow \infty}a_n & = (-1) \times LIM_{n \rightarrow \infty}a_n                                           \\
                                 & = LIM_{n \rightarrow \infty}-1 \times LIM_{n \rightarrow \infty}a_n                   \\
                                 & = LIM_{n \rightarrow \infty}(-a_n)                                  & \text{【实数乘法定义】} \\
\end{align*}

\textbf{序列$0.1,0.01,0.001,...$等价于零序列$(0)_{n=1}^\infty$}

证明:

其实序列$0.1,0.01,0.001,...$就是$(\frac{1}{n})_{n=1}^\infty$,
对任意有理数$\epsilon > 0$,当$N \geq \frac{1}{\epsilon}$(有命题4.4.1保证$N$是存在的),
使得对所有$n \geq N$有,
\begin{align*}
  |1/n - 0| & = 1/n \leq \epsilon
\end{align*}
所以序列$0.1,0.01,0.001,...$与零序列$(0)_{n=1}^\infty$对任意$\epsilon$是最终$\epsilon -$接近的,
所以两者是等价的。


\textbf{如何推导?}

证明:

这里要先证明,有理数$x,y$具有以下性质$|x|-|y| \leq |x-y|$。

$x=x-y+y$,然后由命题4.3.3(b)(绝对值的三角不等式)可知,
\begin{align*}
  |x|       & \leq |x-y| + |y|                       \\
  |x| - |y| & \leq |x-y|       & \text{【命题4.2.9(d)】} \\
\end{align*}
于是性质$|x|-|y| \leq |x-y|$得证。

利用刚才的性质,可得,
\begin{align*}
  |b_{n_0}| - |b_{n}|     & \leq |b_{n_0} - b_{n}|             \\
  |b_{n_0}| - |b_{n}|     & \leq \frac{1}{2}\epsilon           \\
  \epsilon \leq |b_{n_0}| & \leq \frac{1}{2}\epsilon + |b_{n}| \\
  \epsilon                & \leq \frac{1}{2}\epsilon + |b_{n}| \\
  \frac{1}{2}\epsilon     & \leq  |b_{n}|
\end{align*}

\section*{5.3.4}

证明:

$(b_n)_{n=0}^\infty$等价于$(a_n)_{n=0}^\infty$,即两个序列对有理数$\epsilon > 0$是最终$\epsilon -$接近的,
又因为$(a_n)_{n=0}^\infty$是有界的,由习题5.2.2可知$(b_n)_{n=0}^\infty$也是有界的。


\section*{5.3.5}

证明:

要证明$LIM_{n \rightarrow \infty}\frac{1}{n} = 0$,
也就是要证明$LIM_{n \rightarrow \infty}\frac{1}{n}$与$LIM_{n \rightarrow \infty}0$的等价性。

对任意$\epsilon > 0$,当$n \leq \frac{1}{\epsilon}$有,
\begin{align*}
  |\frac{1}{n} - 0| = \frac{1}{n} \leq \epsilon
\end{align*}
所以两者是等价的,于是命题得证。

\end{document}