\documentclass{article}
\usepackage{mathtools} 
\usepackage{fontspec}
\usepackage[UTF8]{ctex}
\usepackage{amsthm}
\usepackage{mdframed}
\usepackage{xcolor}
\usepackage{amssymb}
\usepackage{amsmath}

\newmdtheoremenv[
  backgroundcolor=gray!10,
  linewidth=0pt,
  innerleftmargin=10pt,
  innerrightmargin=10pt,
  innertopmargin=10pt,
  innerbottommargin=10pt
]{zgraytheorem}{}
% 定义说明环境样式
\newtheoremstyle{mystyle}% 说明环境样式的名称
  {1em}% 上方间距
  {1em}% 下方间距
  {\normalfont}% 说明内容的字体样式
  {}% 缩进量
  {\bfseries}% 说明标记的字体样式
  {.}% 说明标记和说明内容之间的标点
  {1em}% 说明标记后的水平空间
  {}% 说明标记后的垂直空间
% 使用新定义的样式创建说明环境
\theoremstyle{mystyle}
\newtheorem*{zremark}{说明}


\begin{document}
\title{5.6 习题}
\maketitle

\section*{5.6.1}

证明:

\textbf{(a)}

仿照命题5.5.12 的证明过程。

令$E=\{z \in R: z \geq 0 \text{且} z^n \leq x\}$,由定义5.6.4 可知$y = x^{1/n} := sup(E)$。

利用反证法,我们要证明$y^n < x$和$y^n > x$都会导致矛盾。

首先假设$y^n < x$,假设$0<\epsilon<1$是一个较小的正数。由于$\epsilon ^n < \epsilon$。

如果$0< y \leq 1$,那么,
\begin{align}
  (y + \epsilon)^n & = \epsilon ^n + k_0y\epsilon^{n-1} + k_1y^2\epsilon^{n-2} + ... + y^n \\
                   & < \epsilon + y^n + max(k_0,k_1,...,)y\epsilon                         \\
                   & < y^n + \epsilon[1+max(k_0,k_1,...,)y]
\end{align}

设$\delta = x - y^n$,取$\epsilon < \delta / [1+max(k_0,k_1,...,)y]$,就可以保证$(y + \epsilon)^n < x$,
所以$(y + \epsilon) \in E$,从而与$y$是$E$的上确界矛盾。

如果$y>1$,那么,
\begin{align}
  (y + \epsilon)^n & = \epsilon ^n + k_0y\epsilon^{n-1} + k_1y^2\epsilon^{n-2} + ... + y^n \\
                   & < \epsilon + y^n + max(k_0,k_1,...,)y^{n-1}\epsilon                   \\
                   & < y^n + \epsilon[1+max(k_0,k_1,...,)y^{n-1}]
\end{align}
设$\delta = x - y^n$,取$\epsilon < \delta / [1+max(k_0,k_1,...,)y^{n-1}]$,就可以保证$(y + \epsilon)^n < x$,
所以$(y + \epsilon) \in E$,从而与$y$是$E$的上确界矛盾。

现在假设$y^n > x$,假设$0<\epsilon<1$是一个较小的正数。

如果$0< y \leq 1$,那么,
\begin{align}
  (y - \epsilon)^n & > y^n - \epsilon^n - max(|k_0|,|k_1|,...,)y\epsilon \\
                   & > y^n - \epsilon[1-max(|k_0|,|k_1|,...,)y]
\end{align}

设$\delta = y^n - x$,取$\epsilon < \delta / [1-max(|k_0|,|k_1|,...,)y]$,就可以保证$(y - \epsilon)^n > x$,
所以$(y-\epsilon)$也是上界,这与$y$是$E$的最小上界矛盾。

如果$y > 1$,那么,
\begin{align}
  (y - \epsilon)^n & > y^n - \epsilon^n - max(|k_0|,|k_1|,...,)y\epsilon \\
                   & > y^n - \epsilon[1-max(|k_0|,|k_1|,...,)y^{n-1}]
\end{align}

设$\delta = y^n - x$,取$\epsilon < \delta / [1-max(|k_0|,|k_1|,...,)y^{n-1}]$,就可以保证$(y - \epsilon)^n > x$,
所以$(y-\epsilon)$也是上界,这与$y$是$E$的最小上界矛盾。

根据这两个矛盾,我们得到$y^n = x$,命题得证。

证明过程中$k_n$具体的值是什么不重要,这里是定性分析。

\textbf{(b)}

该命题说明了$y$的唯一性,即:只有$y=x^{1/n}$,才能使得$y^n = x$。

假设存在$y^\prime$使得$(y^\prime)^n = x$,那么$(y^\prime)^n = y^n$,对$n$进行归纳,可知$y^\prime = y$,
存在矛盾,所以$y=y^\prime$,即$y=x^{1/n}$是唯一的。

\textbf{(c)}

定义5.6.4 就保证了任何$E=\{y \in R: y \geq 0 \text{且} y^n \leq x\}$的上界$M \geq 0$,
因为上界要大于$E$中的任意元素。所以,$E$的最小上界$sup(E) \geq 0$,所以$x^{1/n}$是非负实数。

\textbf{(d)}

必要性:因为$x^{1/n} > y^{1/n}$,且由命题5.6.3(c)可知,
\begin{align*}
  (x^{1/n})^n   & > (y^{1/n})^n \\
  \Rightarrow x & > y           \\
\end{align*}

充分性:反证法,假设$x > y$时,$x^{1/n} \leq y^{1/n}$。而通过5.6.3(c)可知,
\begin{align*}
  (x^{1/n})^n   & \leq (y^{1/n})^n \\
  \Rightarrow x & \leq y           \\
\end{align*}
这与$x > y$矛盾。所以假设不成立,命题得证。

\textbf{(e)}
(1)$x>1$

首先证明$x>1$时,$x^{1/n}>1$。由(d)可知,$x>1$于是$x^{1/n} > 1^{1/n}$,又因为$1^n=1$,由(b)可知$1=1^{1/n}$,
于是,
\begin{align*}
  x^{1/n} & > 1^{1/n} = 1
\end{align*}

现在证明$x>1$时,$x^n$是严格递增的。只需证明对任意自然数$k, x^k < x^{k+1}$。由于,
\begin{align*}
  x^{k+1} - x^k & = x^k(x-1) > 0
\end{align*}
所以$x^n$是严格递增的。

不妨设$k_0 < k_1$,由(a)可知,
\begin{align}
  (x^{1/k_0})^{k_0} & = x \\
  (x^{1/k_1})^{k_1} & = x
\end{align}
由于$x>1, x^{1/k_1} > 1$,于是$(x^{1/k_1})^n$是严格递增的,且$k_0<k_1$,
所以$(x^{1/k_1})^{k_0} < (x^{1/k_1})^{k_1}=x$,由此可知,
\begin{align}
  (x^{1/k_0})^{k_0} & > (x^{1/k_1})^{k_0}
\end{align}
由5.6.3(c)可知,$x^{1/k_0} > x^{1/k_1}$,所以$x^{1/k}$是关于$k$的减函数得证。

(2)$x<1$ 证明略

(3)$x=1$ 证明略

\textbf{(f)}

按照消去律,只需证明,等式两端的$n$次幂是相等的即可。

由(a)可知
\begin{align*}
  [(xy)^{1/n}]^n = xy
\end{align*}

由命题5.6.3(a)可知,
\begin{align*}
  (x^{1/n}y^{1/n})^n & = (x^{1/n})^n  (y^{1/n})^n \\
                     & = xy
\end{align*}

\textbf{(g)}

按照消去律,只需证明,等式两端的$mn$次幂是相等的即可。

由(a)可知
\begin{align*}
  [(x)^{1/mn}]^mn = x
\end{align*}

有5.6.3 (a)可知,

\begin{align*}
  [(x^{1/n})^{1/m}]^{mn} & = \{[(x^{1/n})^{1/m}]^m\}^n \\
                         & = (x^{1/n})^n               \\
                         & = x
\end{align*}

\section*{5.6.2}

证明:

记$q=a/b$,其中$a$是整数且$b$是正整数。

\textbf{(a)}

$x^q = (x^{1/b})^a$,由定义5.6.4 可知$x^{1/b} \geq 0$,现在只需证明$x^{1/b} \neq 0$,
假设$x^{1/b}=0$,那么,
\begin{align*}
  x^{1/b}     & = 0   \\
  (x^{1/b})^b & = 0^b \\
  x           & = 0
\end{align*}
这与$x>0$矛盾,所以$x^{1/b} > 0$。

$(x^{1/b})^a$的正实数性,通过对$a$进行讨论来完成证明。

(1)$a \leq 0$时,可以对$a$进行归纳,

\end{document}