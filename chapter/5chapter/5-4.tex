\documentclass{article}
\usepackage{mathtools} 
\usepackage{fontspec}
\usepackage[UTF8]{ctex}
\usepackage{amsthm}
\usepackage{mdframed}
\usepackage{xcolor}
\usepackage{amssymb}
\usepackage{amsmath}

\newmdtheoremenv[
  backgroundcolor=gray!10,
  linewidth=0pt,
  innerleftmargin=10pt,
  innerrightmargin=10pt,
  innertopmargin=10pt,
  innerbottommargin=10pt
]{zgraytheorem}{}
% 定义说明环境样式
\newtheoremstyle{mystyle}% 说明环境样式的名称
  {1em}% 上方间距
  {1em}% 下方间距
  {\normalfont}% 说明内容的字体样式
  {}% 缩进量
  {\bfseries}% 说明标记的字体样式
  {.}% 说明标记和说明内容之间的标点
  {1em}% 说明标记后的水平空间
  {}% 说明标记后的垂直空间
% 使用新定义的样式创建说明环境
\theoremstyle{mystyle}
\newtheorem*{zremark}{说明}


\begin{document}
\title{5.4 习题}
\maketitle

\section*{5.4.1}

\textbf{1.实数的三歧性}

证明:

按照以前的思路,先证明(a)(b)(c)至少有一个为真,其次证明(a)(b)(c)最多有一个为真。

按照实数的构造方式,对任意实数$x$,该实数$x$要么是零,要么不是零,不可能同时成立。

这是因为任意实数都是通过柯西序列构造的,两个柯西序列要么等价的,要么不是,
我们固定一个序列是$(0)_{n=1}^\infty$,那么其他的柯西序列要么与其等价,即也等于实数0,要么不等价,即不等于实数0。

如果$x \neq 0$那么由引理5.3.14可知$x$一定存在某个远离0的柯西序列,
由此可知$x$可能是正的或负的,也可能都是;

至此(a)(b)(c)至少有一个为真成立。

现在证明(a)(b)(c)最多有一个为真。

(a)(b)(c)分别对应:
\begin{align}
  x & = LIM_{n \rightarrow \infty}0                 \\
  x & = LIM_{n \rightarrow \infty}a_n & \text{正远离0} \\
  x & = LIM_{n \rightarrow \infty}b_n & \text{负远离0}
\end{align}
如果(a)(b)同时成立,
此时,存在$c>0$使得$a_n \geq c$,
那么对任意$n \geq 1$均有
\begin{align*}
  |a_n - 0| & = |a_n| > c
\end{align*}
所以两个系列不能对任意$c > \epsilon > 0$都是最终$\epsilon -$接近的,所以(a)(b)不能同时成立。

同理(a)(c)不能同时成立。

如果(b)(c)同时成立,此时,
此时,存在$c_0>0$使得$a_n \geq c_0$,存在$c_1 \geq 0$使得$b_n \leq -c_1$,
那么对任意$n \geq 1$均有
\begin{align*}
  |a_n| - |b_n|     & \leq |a_n - b_n|          \\
  |a_n|             & \leq |a_n - b_n|  + |b_n| \\
  c_0               & \leq |a_n - b_n| + |b_n|  \\
  c_0 - |a_n - b_n| & \leq |b_n|                \\
  c_0 - |a_n - b_n| & \leq -c_1                 \\
  c_0 + c_1         & \leq |a_n - b_n|          \\
\end{align*}
所以两个系列不能对任意$c_0 + c_1 > \epsilon > 0$都是最终$\epsilon -$接近的,所以(b)(c)不能同时成立。

至此(a)(b)(c)最多有一个为真。

\textbf{2.实数$x$是负的,当且仅当$-x$是正的。}

证明:


\end{document}