\documentclass{article}
\usepackage{mathtools} 
\usepackage{fontspec}
\usepackage[UTF8]{ctex}
\usepackage{amsthm}
\usepackage{mdframed}
\usepackage{xcolor}
\usepackage{amssymb}
\usepackage{amsmath}

\newmdtheoremenv[
  backgroundcolor=gray!10,
  linewidth=0pt,
  innerleftmargin=10pt,
  innerrightmargin=10pt,
  innertopmargin=10pt,
  innerbottommargin=10pt
]{zgraytheorem}{}
% 定义说明环境样式
\newtheoremstyle{mystyle}% 说明环境样式的名称
  {1em}% 上方间距
  {1em}% 下方间距
  {\normalfont}% 说明内容的字体样式
  {}% 缩进量
  {\bfseries}% 说明标记的字体样式
  {.}% 说明标记和说明内容之间的标点
  {1em}% 说明标记后的水平空间
  {}% 说明标记后的垂直空间
% 使用新定义的样式创建说明环境
\theoremstyle{mystyle}
\newtheorem*{zremark}{说明}


\begin{document}
\title{5.4 习题}
\maketitle

\section*{5.4.1}

\textbf{1.实数的三歧性}

证明:

按照以前的思路,先证明(a)(b)(c)至少有一个为真,其次证明(a)(b)(c)最多有一个为真。

按照实数的构造方式,对任意实数$x$,该实数$x$要么是零,要么不是零,不可能同时成立。

这是因为任意实数都是通过柯西序列构造的,两个柯西序列要么等价的,要么不是,
我们固定一个序列是$(0)_{n=1}^\infty$,那么其他的柯西序列要么与其等价,即也等于实数0,要么不等价,即不等于实数0。

如果$x \neq 0$那么由引理5.3.14可知$x$一定存在某个远离0的柯西序列,
由此可知$x$可能是正的或负的,也可能都是;

至此(a)(b)(c)至少有一个为真成立。

现在证明(a)(b)(c)最多有一个为真。

(a)(b)(c)分别对应:
\begin{align}
  x & = LIM_{n \rightarrow \infty}0                 \\
  x & = LIM_{n \rightarrow \infty}a_n & \text{正远离0} \\
  x & = LIM_{n \rightarrow \infty}b_n & \text{负远离0}
\end{align}
如果(a)(b)同时成立,
此时,存在$c>0$使得$a_n \geq c$,
那么对任意$n \geq 1$均有
\begin{align*}
  |a_n - 0| & = |a_n| > c
\end{align*}
所以两个系列不能对任意$c > \epsilon > 0$都是最终$\epsilon -$接近的,所以(a)(b)不能同时成立。

同理(a)(c)不能同时成立。

如果(b)(c)同时成立,此时,
此时,存在$c_0>0$使得$a_n \geq c_0$,存在$c_1 \geq 0$使得$b_n \leq -c_1$,
那么对任意$n \geq 1$均有
\begin{align*}
  |a_n| - |b_n|     & \leq |a_n - b_n|          \\
  |a_n|             & \leq |a_n - b_n|  + |b_n| \\
  c_0               & \leq |a_n - b_n| + |b_n|  \\
  c_0 - |a_n - b_n| & \leq |b_n|                \\
  c_0 - |a_n - b_n| & \leq -c_1                 \\
  c_0 + c_1         & \leq |a_n - b_n|          \\
\end{align*}
所以两个系列不能对任意$c_0 + c_1 > \epsilon > 0$都是最终$\epsilon -$接近的,所以(b)(c)不能同时成立。

至此(a)(b)(c)最多有一个为真。

\textbf{2.实数$x$是负的,当且仅当$-x$是正的。}

证明:

$x$是负的,所以它可以写成某个负远离0的序列$(a_n)_{n=1}^\infty$的形式极限$x=LIM_{n\rightarrow \infty}a_n$。
由实数的负运算可知$-LIM_{n\rightarrow \infty}a_n = LIM_{n \rightarrow \infty}-a_n$,
由序列$(a_n)_{n=1}^\infty$是负远离0可知,存在有理数$c > 0$使得$a_n \leq -c$对所有的$n \geq 1$均成立,
所以
\begin{align*}
  a_n  & \leq -c                  \\
  -a_n & \geq c  & \text{习题4.2.6}
\end{align*}
于是序列$-(a_n)_{n=1}^\infty$是正远离0的,所以其形式极限$-x$是正的。

\textbf{3.如果$x$和$y$都是正的,那么$x+y$和$xy$都是正的。}

证明:

不妨设$x=LIM_{n \rightarrow \infty}a_n, y = LIM_{n \rightarrow \infty}b_n$。

因为$x,y$是正的,所以它们都是正远离0的,于是存在有理数$c_0,c_1 > 0$使得对任意$n \geq 1$都有
\begin{align}
  |a_n| \geq c_0 \\
  |b_n| \geq c_1
\end{align}
又
\begin{align*}
  x + y & = LIM_{n \rightarrow \infty}a_n + LIM_{n \rightarrow \infty}b_n \\
        & = LIM_{n \rightarrow \infty} a_n + b_n
\end{align*}
因为
\begin{align*}
  |a_n + b_n| = |a_n| + |b_n| \geq c_0 + c_1 > 0
\end{align*}
所以$(a_n+b_n)_{n=1}^\infty$序列正远离0,所以其极限形式$x+y=LIM_{n \rightarrow \infty} a_n + b_n$是正的。

又
\begin{align*}
  xy = LIM_{n \rightarrow \infty} a_nb_n
\end{align*}
因为
\begin{align*}
  |a_nb_n| = |a_n||b_n| \geq c_0c_1 > 0
\end{align*}
所以$(a_nb_n)_{n=1}^\infty$序列正远离0,所以其极限形式$xy=LIM_{n \rightarrow \infty} a_nb_n$是正的。

\section*{5.4.2}

证明:

元证明:命题4.2.4所有的代数定律不仅对实数也是成立的,且实数的三歧性和序的定义都是与有理数一致,
于是有理数通过以上性质得到的命题4.2.9对于实数也应该是成立的。

\begin{zgraytheorem}
  \begin{zremark}
    元证明,就是对证明本身的说明。
    逻辑学中有元对象与目标对象的概念,目标对象是直接讨论的对象,元对象是对目标对象进行讨论或分析的更高层次的对象。
    这里的元证明与元对象类似,目标对象是直接讨论的对象即:实数。
  \end{zremark}
\end{zgraytheorem}

\section*{5.4.3}

证明:

通过实数的三歧性讨论。

(1)$x=0$时,此时取$N=0$就可以满足命题。

(2)$x$是正的。

这里通过反证法证明。假设命题不成立,即不存在自然数N使得$N \leq x < N + 1$,
也就是说,对任意自然数$n$都有$x < n$或$x \geq n + 1$。

接下来我们要通过上面假设的命题,推出一个与命题 5.4.12矛盾的结果。

通过“对任意自然数$n$都有$x < n$或$x \geq n + 1$”,可以推出“不存在自然数大于$x$”。

通过归纳法证明:“不存在自然数大于$x$”

当$N=0$,因为$x$是正实数,所以$x > 0$,由“对任意自然数$n$都有$x < n$或$x \geq n + 1$”可知,
$x < 0$此时不成立,因为与$x > 0$矛盾,所以$x \geq 1$;

归纳假设$N=n$时,$x \geq n + 1$;

$N=n+1$时,$x \geq n + 1$与$x < n + 1$不能同时成立,所以$x \geq (n+1) + 1$;

综上,对任意自然数$N$,都有$x \geq N + 1$,即:不存在自然数大于$x$

与命题5.4.12矛盾,所以命题得证。

(3)$x$是负的。

$-x$是正的,所以存在自然数$N$使得$N \leq -x < N + 1$,所以,
\begin{align*}
  -(N + 1) < x \leq -N
\end{align*}
若$x \neq -N$,则$-(N + 1) \leq x < -N$;若$x = -N$,则$-N \leq x < -N + 1$;

所以负实数也满足命题。

现在证明$N$的唯一性。

假设存在$N_1,N_2,N_1 \neq N_2$都满足命题,即:
\begin{align*}
  N_1 \leq x < N_1 + 1 \\
  N_2 \leq x < N_2 + 1
\end{align*}

如果$N_1 < N_2$,那么$N_1 + 1 \leq N_2$,那么,
\begin{align*}
  N_1 \leq x < N_1 + 1 \leq N_2 \leq x < N_2+1
\end{align*}
可得,$x < x$,存在矛盾。

同理$N_1 > N_2$也存在矛盾。

所以$N$是唯一的。

综上,命题成立。

\section*{5.4.4}

证明:

由命题5.4.8 可知,$x$是正实数,那么$x^{-1}$也是正的。
由习题5.4.3可知,存在自然数$N$使得$N \leq x^{-1} < N + 1$,所以,
\begin{align*}
  x > 1 / (N + 1) > 0
\end{align*}
$N$是自然数,那么$N + 1$是正整数,命题得证。

\section*{5.4.5}

证明:

由习题5.4.3可知,存在整数$N_1, N_2$使得,
\begin{align*}
  N_1 \leq x < N_1 + 1 \\
  N_2 \leq y < N_2 + 1
\end{align*}

因为$x < y$,则$y - x > 0$。

如果$y - x > 1$,此时,$ x + 1 < y $,同时由$N_1 \leq x < N_1 + 1$可得,
\begin{align*}
  N_1 + 1 \leq x + 1 < y \\
  N_1 + 1 < y
\end{align*}
此时,$q = N_1$。

如果$1 \geq y - x > 0$,由推论5.4.13可知,存在$M$使得$M(y-x) > 1$,那么,$My - Mx > 1$。
由之前的论证可知,此时存在整数$N$使得$Mx < N < My$,那么$x < N/M < y$。

综上,命题得证。

\section*{5.4.6}

证明:

充分性

$|x-y| < \epsilon $可以推出$y - \epsilon < x < y + \epsilon$

反证法,假设$x \leq y - \epsilon$或$x > y + \epsilon$,
若
\begin{align*}
  x     & \leq y - \epsilon \\
  x - y & \leq - \epsilon   \\
  |x-y| & \geq \epsilon     \\
\end{align*}
与$|x-y| < \epsilon$矛盾

若
\begin{align*}
  x & > y + \epsilon \\
  x - y > \epsilon   \\
  |x - y| > \epsilon \\
\end{align*}
与$|x-y| < \epsilon$矛盾

综上,充分性得到证明。

必要性

\begin{align*}
  y - \epsilon & < x        & < y + \epsilon                    \\
  -\epsilon    & < x - y    & < y + \epsilon                    \\
  |x - y|      & < \epsilon & \text{分正负讨论即可,利用了命题4.2.9,习题4.2.6}
\end{align*}

必要性得到证明。

$|x-y| \leq \epsilon$当且仅当$y-\epsilon \leq x \leq y + \epsilon$的证明同理。

\section*{5.4.7}

【解题不对,大家不要看了】

证明:

(1)$x \leq y + \epsilon$对所有的实数$\epsilon > 0$均成立 $\Rightarrow$ $x \leq y$。

反证法。假设$x > y$,此时取$\epsilon = (x-y)/2 > 0$,
\begin{align*}
  x   & \leq y + \epsilon    \\
  x   & \leq y + (x - y) / 2 \\
  x   & \leq x/2 + y/2       \\
  x/2 & \leq y/2             \\
  x   & \leq y
\end{align*}
此时$x \leq y$与$x > y$矛盾,所以假设不成立,由实数序的三歧性可知,
$x < y$或$x = y$,即:$x \leq y$。

(2)$x \leq y$ $\Rightarrow$ $x \leq y + \epsilon$对所有的实数$\epsilon > 0$均成立。

由$x \leq y$可知$y - x \geq 0$,那么对任意$\epsilon > 0$,有
\begin{align*}
  y - x            & \geq 0             \\
  y - x + \epsilon & \geq \epsilon  > 0 \\
  y + \epsilon     & > x
\end{align*}
其实按照定义5.4.6,$y + \epsilon > x$表达为$y + \epsilon \geq 0$也是可以的,但$=$的情况好像取不到,
【这也是我让大家不要看这个解答的原因】。

\section*{5.4.8}

假设$LIM_{n \rightarrow \infty}a_n > x$,
由命题5.4.14可知,存在有理数q使得$x < q < LIM_{n \rightarrow \infty}a_n$,
把$q$看做$LIM_{n \rightarrow \infty}q$,由$a_n \leq x, x < q$可知,
\begin{align*}
  q > a_n
\end{align*}
由此可知$q \geq LIM_{n \rightarrow \infty}a_n$,与$q < LIM_{n \rightarrow \infty}a_n$存在矛盾。

同理可证另一个命题。

\end{document}