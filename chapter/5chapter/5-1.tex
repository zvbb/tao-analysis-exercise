\documentclass{article}
\usepackage{mathtools} 
\usepackage{fontspec}
\usepackage[UTF8]{ctex}
\usepackage{amsthm}
\usepackage{mdframed}
\usepackage{xcolor}
\usepackage{amssymb}
\usepackage{amsmath}

\newmdtheoremenv[
  backgroundcolor=gray!10,
  linewidth=0pt,
  innerleftmargin=10pt,
  innerrightmargin=10pt,
  innertopmargin=10pt,
  innerbottommargin=10pt
]{zgraytheorem}{}
% 定义说明环境样式
\newtheoremstyle{mystyle}% 说明环境样式的名称
  {1em}% 上方间距
  {1em}% 下方间距
  {\normalfont}% 说明内容的字体样式
  {}% 缩进量
  {\bfseries}% 说明标记的字体样式
  {.}% 说明标记和说明内容之间的标点
  {1em}% 说明标记后的水平空间
  {}% 说明标记后的垂直空间
% 使用新定义的样式创建说明环境
\theoremstyle{mystyle}
\newtheorem*{zremark}{说明}


\begin{document}
\title{5.1 习题}
\maketitle

\section*{5.1.1}

证明:

由柯西序列的定义可知,
设$\epsilon = 1 > 0$,则存在一个$N \geq 0$使得$d(a_j,a_k) \leq \epsilon$对所有的$j,k \geq N$均成立。
由此可以把柯西序列看成两个部分,
\begin{align*}
  a_1,a_2,a_2,...,a_{N-1} \\
  (a_n)_{n=N}^\infty      \\
\end{align*}

有引理5.1.14(有限序列是有界的)可知对于序列$a_1,a_2,a_2,...,a_{N-1}$是有界的,
即存在有理数$M \geq 0$使得该序列以$M$为界。

对于序列$(a_n)_{n=N}^\infty$,该序列也是有界的,因为由$d(a_j,a_k) \leq \epsilon$对所有的$j,k \geq N$均成立,
取$j=N, i \geq N$可知,
\begin{align*}
  d(a_N,a_i)  & \leq \epsilon                  \\
  |a_N - a_i| & \leq \epsilon                  \\
  |a_i|-|a_N| & \leq |a_N - a_i| \leq \epsilon \\
  |a_i|       & \leq |a_N| + \epsilon
\end{align*}
由$i$的任意性可得$(a_n)_{n=N}^\infty$是有界的,

综上,序列$(a_n)_{n=1}^\infty$是有界的。

\end{document}