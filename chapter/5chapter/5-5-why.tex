\documentclass{article}
\usepackage{mathtools} 
\usepackage{fontspec}
\usepackage[UTF8]{ctex}
\usepackage{amsthm}
\usepackage{mdframed}
\usepackage{xcolor}
\usepackage{amssymb}
\usepackage{amsmath}

\newmdtheoremenv[
  backgroundcolor=gray!10,
  linewidth=0pt,
  innerleftmargin=10pt,
  innerrightmargin=10pt,
  innertopmargin=10pt,
  innerbottommargin=10pt
]{zgraytheorem}{}
% 定义说明环境样式
\newtheoremstyle{mystyle}% 说明环境样式的名称
  {1em}% 上方间距
  {1em}% 下方间距
  {\normalfont}% 说明内容的字体样式
  {}% 缩进量
  {\bfseries}% 说明标记的字体样式
  {.}% 说明标记和说明内容之间的标点
  {1em}% 说明标记后的水平空间
  {}% 说明标记后的垂直空间
% 使用新定义的样式创建说明环境
\theoremstyle{mystyle}
\newtheorem*{zremark}{说明}


\begin{document}
\title{5.5 为什么}
\maketitle


\textbf{1.$R^{+}$没有上界}

证明:

反证法。
假设$M>0$是$R^{+}$的上界,又$M + 1 \in R^+$,但$M + 1 > M$,存在矛盾。

\textbf{2.空集没有最小上界。}

证明:

假设空集存在$M$为其最小上界,由于任何实数都是空集的上界,
所以$M - 1$也是空集的上界,但$M - 1 < M$,与最小上界的定义矛盾。

\textbf{
  3.因为$m_n/n$是$E$的上界而$(m_{n^\prime}-1)/n^\prime$不是,
  所以一定有$m_n/n > (m_{n^\prime}-1)/n^\prime$。
}

证明:

$(m_{n^\prime}-1)/n^\prime$不是$E$的上界,即:存在$x \in E$使得
$x > (m_{n^\prime}-1)/n^\prime$。

又$m_n/n$是$E$的上界,所以,$x \leq m_n/n$,所以,
\begin{align*}
  (m_{n^\prime}-1)/n^\prime < x \leq m_n/n \\
  (m_{n^\prime}-1)/n^\prime < m_n/n
\end{align*}

\textbf{
  4.定理5.5.9的证明过程中,“再次由阿基米德性质可知,存在一个整数$L$使得$L/n < x_0$”的证明。
}

证明:

如果$x_0 > 0$,取$L = 0$即可;

如果$x_0 = 0$,取$L = -1$即可;

如果$x_0 < 0$。

于是$-x_0 > 0$,所以由阿基米德性质可知,存在正整数$L^\prime$使得
\begin{align*}
  L^\prime / n >  - x_0 \\
  \implies              \\
  -L^\prime / n < x_0
\end{align*}
却$L = - L^\prime$即可;


\end{document}