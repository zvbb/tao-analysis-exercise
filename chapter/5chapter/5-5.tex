\documentclass{article}
\usepackage{mathtools} 
\usepackage{fontspec}
\usepackage[UTF8]{ctex}
\usepackage{amsthm}
\usepackage{mdframed}
\usepackage{xcolor}
\usepackage{amssymb}
\usepackage{amsmath}

\newmdtheoremenv[
  backgroundcolor=gray!10,
  linewidth=0pt,
  innerleftmargin=10pt,
  innerrightmargin=10pt,
  innertopmargin=10pt,
  innerbottommargin=10pt
]{zgraytheorem}{}
% 定义说明环境样式
\newtheoremstyle{mystyle}% 说明环境样式的名称
  {1em}% 上方间距
  {1em}% 下方间距
  {\normalfont}% 说明内容的字体样式
  {}% 缩进量
  {\bfseries}% 说明标记的字体样式
  {.}% 说明标记和说明内容之间的标点
  {1em}% 说明标记后的水平空间
  {}% 说明标记后的垂直空间
% 使用新定义的样式创建说明环境
\theoremstyle{mystyle}
\newtheorem*{zremark}{说明}


\begin{document}
\title{5.5 习题}
\maketitle

\section*{5.5.1}

证明:

\section*{5.5.2}

证明:

由于$L,K$都是整数,且$L < K$可知,$K - L$是正自然数,
现在通过对$K - L$进行归纳来完成证明【提示信息中有提到归纳证明】。

归纳基始,$K-L=1$,此时$m=K,m-1=L$,由题设信息可知,该$m$是满足命题的。

归纳假设,$K-L=n$时,存在$m$满足命题。

现在假设$K-L=n+1$时,由于$L < L + 1 < K$,

如果$(L + 1)/n$是集合$E$的上界,
此时可以取$m=L+1$,又由题设可知$(m-1)/n=L/n$不是$E$的上界,此时的$m$满足命题。

如果$(L + 1)/n$不是集合$E$的上界,由归纳假设可知,存在$m, L+1 < m \leq K$满足命题。

至此,完成归纳。

\section*{5.5.3}

证明:

由于$m/n$是$E$的上界,而$(m^\prime -1)/n$不是$E$的上界,所以
\begin{align*}
  m^\prime -1 & < m                                          \\
  m^\prime    & \leq m & \text{【题设说明了$m,m^\prime$是整数,否则无法成立】}
\end{align*}

由于$m^\prime/n$是$E$的上界,而$(m -1)/n$不是$E$的上界,所以
\begin{align*}
  m - 1 & < m^\prime                                          \\
  m     & \leq m^\prime & \text{【题设说明了$m,m^\prime$是整数,否则无法成立】}
\end{align*}

所以$m = m^\prime$

\end{document}