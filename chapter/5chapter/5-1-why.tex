\documentclass{article}
\usepackage{mathtools} 
\usepackage{fontspec}
\usepackage[UTF8]{ctex}
\usepackage{amsthm}
\usepackage{mdframed}
\usepackage{xcolor}
\usepackage{amssymb}
\usepackage{amsmath}

\newmdtheoremenv[
  backgroundcolor=gray!10,
  linewidth=0pt,
  innerleftmargin=10pt,
  innerrightmargin=10pt,
  innertopmargin=10pt,
  innerbottommargin=10pt
]{zgraytheorem}{}
% 定义说明环境样式
\newtheoremstyle{mystyle}% 说明环境样式的名称
  {1em}% 上方间距
  {1em}% 下方间距
  {\normalfont}% 说明内容的字体样式
  {}% 缩进量
  {\bfseries}% 说明标记的字体样式
  {.}% 说明标记和说明内容之间的标点
  {1em}% 说明标记后的水平空间
  {}% 说明标记后的垂直空间
% 使用新定义的样式创建说明环境
\theoremstyle{mystyle}
\newtheorem*{zremark}{说明}


\begin{document}
\title{5.1 习题}
\maketitle


\textbf{1.是0.1稳定的,但不是0.01稳定的} 

证明:

$0.1-0.01=0.09>0.01$

\textbf{2.序列$1,2,4,8,16,...$对于任意的$\epsilon$都不是$\epsilon -$稳定的。}

证明:

序列$1,2,4,8,16,...$中的元素$a_n>n$,设任意有理数$\epsilon > 0$,
由命题4.4.1可知,存在一个自然数$N>\epsilon$,所以,
\begin{align*}
  a_{N+1} - a_0 > N+1 - 1 = N > \epsilon 
\end{align*}
由$\epsilon$的任意性可知,对于任意的$\epsilon$序列都不是$\epsilon -$稳定的。

\textbf{3.序列$10,0,0,0,...$,它是最终$\epsilon -$稳定的。}

证明:

略

\textbf{4.无限序列$1,-2,3,-4,5,-6,...$没有界}

证明:

无限序列中的元素$|a_n| > n$,假设存在$M$使得序列以此为界,即对任意序列中的元素有,
\begin{align*}
  |a_n| \leq M
\end{align*}
但是由命题4.4.1可知,存在自然数$N>M$,又因为,
\begin{align*}
  |a_{N}| > N > M
\end{align*}
由矛盾可知,这样的$M$不存在。

综上,序列是无界的。
\end{document}