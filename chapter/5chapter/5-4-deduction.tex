\documentclass{article}
\usepackage{mathtools} 
\usepackage{fontspec}
\usepackage[UTF8]{ctex}
\usepackage{amsthm}
\usepackage{mdframed}
\usepackage{xcolor}
\usepackage{amssymb}
\usepackage{amsmath}

\newmdtheoremenv[
  backgroundcolor=gray!10,
  linewidth=0pt,
  innerleftmargin=10pt,
  innerrightmargin=10pt,
  innertopmargin=10pt,
  innerbottommargin=10pt
]{zgraytheorem}{}
% 定义说明环境样式
\newtheoremstyle{mystyle}% 说明环境样式的名称
  {1em}% 上方间距
  {1em}% 下方间距
  {\normalfont}% 说明内容的字体样式
  {}% 缩进量
  {\bfseries}% 说明标记的字体样式
  {.}% 说明标记和说明内容之间的标点
  {1em}% 说明标记后的水平空间
  {}% 说明标记后的垂直空间
% 使用新定义的样式创建说明环境
\theoremstyle{mystyle}
\newtheorem*{zremark}{说明}


\begin{document}
\title{5.4 推论}
\maketitle

\textbf{1.书中对命题5.4.12(有理数对实数的界定)的表达感觉有点奇怪。}

不是说命题不正确,而是命题中提到了正整数,虽然正整数是嵌入到有理数中的。
如果把命题中的正整数改为正的有理数话,有理数界定了实数,而整数界定了有理数(命题4.4.1),
这样的表达更加统一。

\textbf{2.书中的命题5.4.12(有理数对实数的界定)只说明了正实数的情况,这里证明对所有实数,命题的正确性。}

\begin{zgraytheorem}
  \begin{zremark}
    这里需要把命题中的$x$改为实数,不限定其是正的,把$q,N$分别改为有理数$q,p$,且不限定为正的。
  \end{zremark}
\end{zgraytheorem}

证明:

通过实数的三歧性分别证明。

当$x$是正的,则书中已经证明过。

当$x=0$,可以直接取$q=0,p=0$此时命题成立。

当$x$是负的,此时$-x$是正的,则存在$q,p$使得$q \leq -x \leq p$,所以$-p \leq x \leq -q$(实数也满足习题4.2.6)。

综上,命题证明完成。

\end{document}