\documentclass{article}
\usepackage{mathtools} 
\usepackage{fontspec}
\usepackage[UTF8]{ctex}
\usepackage{amsthm}
\usepackage{mdframed}
\usepackage{xcolor}
\usepackage{amssymb}
\usepackage{amsmath}


% 定义新的带灰色背景的说明环境 zremark
\newmdtheoremenv[
  backgroundcolor=gray!10,
  % 边框与背景一致,边框线会消失
  linecolor=gray!10
]{zremark}{说明}


\begin{document}
\title{13.3 习题}
\author{张志聪}
\maketitle

\section*{13.3.1}

设$(f(x^{(n)}))_{n = 1}^\infty$是$f(K)$中的任意序列,
序列$(x^{(n)})_{n = 1}^\infty$是$K$中序列,且$f(x^{n})$是$(f(x^{(n)}))_{n = 1}^\infty$中的项。
因为$K$是紧致的,那么存在一个收敛的子序列$(x^{(n_j)})_{j = 1}^\infty$,
不妨设子序列收敛于$x_0 \in K$。

又因为$f$是连续的,所以$f$在$x_0$处连续,由定理13.1.4(b)可知,
序列$(f(x^{(n_j)}))_{j = 1}^\infty$依度量$d_Y$收敛于$f(x_0) \in f(K)$,
又因为$(f(x^{(n_j)}))_{j = 1}^\infty$是$(f(x^{(n)}))_{n = 1}^\infty$的子序列,
由定义12.5.1(紧致性)可知,$f(K)$是紧致的。

\section*{13.3.2}

(1)$f$是有界的。

由定理13.3.1可知,$f(X)$是紧致的,由推论12.5.6可知,$f$是有界的。


(2)$f$在某个点$x_{max} \in X$处取到最大值,并且在某个点$x_{min} \in X$处取到最小值。

我们只证明$f$在某个点$x_{max} \in X$处取到最大值,最小值的证明类似。

因为$f$是有界,那么,$\mathbb{R}$中存在一个包含$f(X)$的球$B(y_0, r), y_0 \in \mathbb{R}, r > 0$。
现在设$E$表示集合
\begin{align*}
  E := \{f(x), x \in X\}
\end{align*}
(即: E := f(X))。根据上述内容可知,这个集合是$B(y_0, r)$的子集,
而且$E$还是非空集合。根据最小上界原理可知($E$是$\mathbb{R}$的子集),
它有一个实数上确界$sup(E)$。

记$m := sup(E)$,根据上确界的定义可知,对所有的$y \in E$均有$y \leq m$。而根据$E$的定义可知,
这意味着$f(x) \leq m$对所有的$x \in X$均成立。因此,为了证明$f$在某个点达到最大值,我们只需要
找到一个$x_{max} \in X$使得$f(x_{max}) = m$即可。

设$n \geq 1$是任意一个整数,那么$m - \frac{1}{n} < m = sup(E)$。因为$sup(E)$是$E$的最小上界,
那么$m - \frac{1}{n}$不可能是$E$的上界,从而存在一个$y \in E$使得$m - \frac{1}{n} < y$。
又由$E$的定义可知,这蕴含着存在一个$x \in X$使得$m - \frac{1}{n} < f(x)$。

现在我们按照下面的方法选取一个序列$(x_n)_{n = 1}^\infty :$对于每一个$n$,选取$x_n$为$x \in X$中
使得$m - \frac{1}{n} < f(x_n)$的元素。(这里需要用到选择公理)这是$X$中的一个序列,因为$X$是紧致的,
我们可以找到一个收敛于某极限$x_{max} \in X$的子序列$(x_{nj})_{j = 1}^\infty$,其中$n_1 < n_2 < ...$。
因为$(x_{nj})_{j = 1}^\infty$收敛于$x_{max}$并且$f$在$x_{max}$处连续,于是由定理13.1.4(b)可知
\begin{align*}
  \lim\limits_{j \to \infty}f(x_{n_j}) = f(x_{max})
\end{align*}
另外,根据该序列的构造过程可知,
\begin{align*}
  f(x_{n_j}) > m - \frac{1}{n_j} \geq m - \frac{1}{j}
\end{align*}
从而对上式两端同时取极限可得,
\begin{align*}
  f(x_{max}) = \lim\limits_{j \to \infty}f(x_{n_j}) \geq \lim\limits_{j \to \infty} m - \frac{1}{j} = m
\end{align*}
另外,$f(x) \leq m$对所有的$x \in X$均成立,从而$f(x_{max}) \leq m$。
联合这两个不等式就得到$f(x_{max}) = m$,结论得证。

\section*{13.3.3}

(1)

设$f : X \to Y$是从度量空间$(X, d_X)$到另一个度量空间$(Y, d_Y)$的映射,$f$是一致连续的,现在我们证明
$f$也是连续的。

对任意$x_0 \in X$,对任意$\epsilon > 0$,因为$f$是一致连续的,那么,存在$\delta > 0$使得只要
$x, x^\prime \in X$满足$d_X(x, x^\prime) < \delta$,就有$d_Y(f(x), f(x^\prime)) < \epsilon$。
不妨设$x^\prime = x_0$,
那么只要满足$d_X(x, x_0) < \delta$,就有$d_Y(f(x), f(x_0)) < \epsilon$。

综上,由定义13.1.1可得$f$是连续的。

(2)举例,连续函数不一定是一致连续的。

$f: \mathbb{R} \to \mathbb{R}, f(x) = \frac{1}{x}$。

对任意$\delta > 0$,存在$n \in \mathbb{N}$,使得$\frac{1}{n} < \delta$,
此时我们有
\begin{align*}
  |\frac{1}{n} - \frac{1}{n+1}| < \delta \\
  |f(\frac{1}{n}) - f(\frac{1}{n + 1})| = |n - (n + 1)| = 1
\end{align*}
由此可知,$f$不是一致连续函数。

\section*{13.3.4}

对于任意的$\epsilon > 0$,因为$g: Y \to Z$是一致连续函数,
那么,存在$\delta_Y > 0$使得只要$y, y^\prime \in Y$满足$d_Y(y, y^\prime) < \delta_Y$,
就有$d_Z(g(y), g(y^\prime)) < \epsilon$。

同理可得,存在$\delta_X > 0$使得只要$x, x^\prime \in X$满足$d_X(x, x^\prime) < \delta_X$,
就有$d_Y(f(x), f(x^\prime)) < \delta_Y$。

综上可得,对任意$\epsilon > 0$,存在$\delta = \delta_X$
使得只要$x, x^\prime \in X$满足$d_X(x, x^\prime) < \delta$,
就有$d_Z(g(f(x)), g(f(x^\prime))) < \epsilon$。

所以,$g \circ f: X \to Z$也是一致连续的。

\section*{13.3.5}

对任意$\epsilon > 0$,因为$f: X \to R$是一致连续函数,
那么,存在$\delta_f > 0$使得只要$x, x^\prime \in X$满足$d_X(x, x^\prime) < \delta_f$,
就有$|f(x) - f(x^\prime)| < \frac{1}{2}\epsilon$($\mathbb{R}$的默认度量是$d^{l^2}$)。

类似的,存在$\delta_g > 0$使得只要$x, x^\prime \in X$满足$d_X(x, x^\prime) < \delta_g$,
就有$|g(x) - g(x^\prime)| < \frac{1}{2}\epsilon$。

综上,对于任意的$\epsilon > 0$,存在$\delta = min(\delta_f, \delta_g)$,
使得只要$x, x^\prime \in X$满足$d_X(x, x^\prime) < \delta$,
就有
\begin{align*}
  d_{l^2}(f \oplus g(x), f \oplus g(x^\prime)) & = d_{l^2}((f(x), g(x)), (f(x^\prime), g(x^\prime))) \\
                                               & = \sqrt{|f(x) - f(x^\prime)|^2 + |g(x) - g(x^\prime)|^2} < \sqrt{\frac{1}{4}\epsilon^2 + \frac{1}{4}\epsilon^2} < \epsilon
\end{align*}

所以,$f \oplus g: X \to R^2$也是一致连续的。

\section*{13.3.6}

(1)证明:加法、减法是一致连续函数。

对任意$\epsilon > 0$,存在$\delta = \frac{1}{2}\epsilon$,
使得只要$d_{l^2}((x, y), (x^\prime, y^\prime)) = \sqrt{|x - x^\prime|^2 + |y - y^\prime|^2} < \delta$,
就有
\begin{align*}
  |x - x^\prime| < \frac{1}{2}\epsilon, |y - y^\prime| < \frac{1}{2}\epsilon
\end{align*}
于是
\begin{align*}
  |(x + y) - (x^\prime + y^\prime)| & =  |(x - x^\prime) + (y - y^\prime)| < \epsilon \\
  |(x - y) - (x^\prime - y^\prime)| & = |(x - x^\prime) - (y - y^\prime)| < \epsilon
\end{align*}
这意味着,$(x, y) \mapsto x + y$和$(x, y) \mapsto x - y$是一致连续函数。


(2)证明:乘法不是一致连续函数。
任意$n \in \mathbb{N}$,点$(n, n), (n + \frac{1}{n}, n)$间的距离
\begin{align*}
  d_{l^2}((n, n), (n + \frac{1}{n}, n)) = \frac{1}{n}
\end{align*}
可以任意小,
此时
\begin{align*}
  |n \times n - (n + \frac{1}{n}) \times n| = 1
\end{align*}
可知,$(x, y) \mapsto xy$不是一致连续函数。

(3)$f + g: X \to \mathbb{R}$和$f - g: X \to \mathbb{R}$也是一致连续函数。

由习题13.3.5可知,$f \oplus g := (f(x), g(x))$的直和$f \oplus g: X \to R^2$是一致连续函数,
另外由(1)可知,函数$(x, y) \mapsto x + y$和$(x, y) \mapsto x - y$是一致连续函数,
那么,由习题13.3.4可知,把这两个函数复合在一起,
此时$f + g = ((x, y) \mapsto x + y)(f \oplus g)$
和$f - g = ((x, y) \mapsto x - y)(f \oplus g)$是一致连续的。

(4)举例$fg: X \to \mathbb{R}$不一定是一致连续的。

%todo

(5)

$max(f,g), min(f,g), cf$是一致连续的,$f/g$不是一致连续的,比如习题13.3.3中的举例。

\end{document}