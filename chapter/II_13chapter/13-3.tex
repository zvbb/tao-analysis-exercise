\documentclass{article}
\usepackage{mathtools} 
\usepackage{fontspec}
\usepackage[UTF8]{ctex}
\usepackage{amsthm}
\usepackage{mdframed}
\usepackage{xcolor}
\usepackage{amssymb}
\usepackage{amsmath}


% 定义新的带灰色背景的说明环境 zremark
\newmdtheoremenv[
  backgroundcolor=gray!10,
  % 边框与背景一致,边框线会消失
  linecolor=gray!10
]{zremark}{说明}


\begin{document}
\title{13.3 习题}
\author{张志聪}
\maketitle

\section*{13.3.1}

设$(f(x^{(n)}))_{n = 1}^\infty$是$f(K)$中的任意序列,
序列$(x^{(n)})_{n = 1}^\infty$是$K$中序列,且$f(x^{n})$是$(f(x^{(n)}))_{n = 1}^\infty$中的项。
因为$K$是紧致的,那么存在一个收敛的子序列$(x^{(n_j)})_{j = 1}^\infty$,
不妨设子序列收敛于$x_0 \in K$。

又因为$f$是连续的,所以$f$在$x_0$处连续,由定理13.1.4(b)可知,
序列$(f(x^{(n_j)}))_{j = 1}^\infty$依度量$d_Y$收敛于$f(x_0) \in f(K)$,
又因为$(f(x^{(n_j)}))_{j = 1}^\infty$是$(f(x^{(n)}))_{n = 1}^\infty$的子序列,
由定义12.5.1(紧致性)可知,$f(K)$是紧致的。

\section*{13.3.2}

(1)$f$是有界的。

由定理13.3.1可知,$f(X)$是紧致的,由推论12.5.6可知,$f$是有界的。


(2)$f$在某个点$x_{max} \in X$处取到最大值,并且在某个点$x_{min} \in X$处取到最小值。

我们只证明$f$在某个点$x_{max} \in X$处取到最大值,最小值的证明类似。

因为$f$是有界,那么,$\mathbb{R}$中存在一个包含$f(X)$的球$B(y_0, r), y_0 \in \mathbb{R}, r > 0$。
现在设$E$表示集合
\begin{align*}
  E := \{f(x), x \in X\}
\end{align*}
(即: E := f(X))。根据上述内容可知,这个集合是$B(y_0, r)$的子集,
而且$E$还是非空集合。根据最小上界原理可知($E$是$\mathbb{R}$的子集),
它有一个实数上确界$sup(E)$。

记$m := sup(E)$,根据上确界的定义可知,对所有的$y \in E$均有$y \leq m$。而根据$E$的定义可知,
这意味着$f(x) \leq m$对所有的$x \in X$均成立。因此,为了证明$f$在某个点达到最大值,我们只需要
找到一个$x_{max} \in X$使得$f(x_{max}) = m$即可。

设$n \geq 1$是任意一个整数,那么$m - \frac{1}{n} < m = sup(E)$。因为$sup(E)$是$E$的最小上界,
那么$m - \frac{1}{n}$不可能是$E$的上界,从而存在一个$y \in E$使得$m - \frac{1}{n} < y$。
又由$E$的定义可知,这蕴含着存在一个$x \in X$使得$m - \frac{1}{n} < f(x)$。

现在我们按照下面的方法选取一个序列$(x_n)_{n = 1}^\infty :$对于每一个$n$,选取$x_n$为$x \in X$中
使得$m - \frac{1}{n} < f(x_n)$的元素。(这里需要用到选择公理)这是$X$中的一个序列,因为$X$是紧致的,
我们可以找到一个收敛于某极限$x_{max} \in X$的子序列$(x_{nj})_{j = 1}^\infty$,其中$n_1 < n_2 < ...$。
因为$(x_{nj})_{j = 1}^\infty$收敛于$x_{max}$并且$f$在$x_{max}$处连续,于是由定理13.1.4(b)可知
\begin{align*}
  \lim\limits_{j \to \infty}f(x_{n_j}) = f(x_{max})
\end{align*}
另外,根据该序列的构造过程可知,
\begin{align*}
  f(x_{n_j}) > m - \frac{1}{n_j} \geq m - \frac{1}{j}
\end{align*}
从而对上式两端同时取极限可得,
\begin{align*}
  f(x_{max}) = \lim\limits_{j \to \infty}f(x_{n_j}) \geq \lim\limits_{j \to \infty} m - \frac{1}{j} = m
\end{align*}
另外,$f(x) \leq m$对所有的$x \in X$均成立,从而$f(x_{max}) \leq m$。
联合这两个不等式就得到$f(x_{max}) = m$,结论得证。

\section*{13.3.3}


\end{document}