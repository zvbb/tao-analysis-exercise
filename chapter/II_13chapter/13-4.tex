\documentclass{article}
\usepackage{mathtools} 
\usepackage{fontspec}
\usepackage[UTF8]{ctex}
\usepackage{amsthm}
\usepackage{mdframed}
\usepackage{xcolor}
\usepackage{amssymb}
\usepackage{amsmath}


% 定义新的带灰色背景的说明环境 zremark
\newmdtheoremenv[
  backgroundcolor=gray!10,
  % 边框与背景一致,边框线会消失
  linecolor=gray!10
]{zremark}{说明}


\begin{document}
\title{13.4 习题}
\author{张志聪}
\maketitle

\section*{13.4.1}

从$E$中任选一个元素组成集合$E_1$,$E_2 := E \setminus E_1$。

当使用离散度量$d_disc$时,所有集合都既是开的又是闭的,
又$E = E_1 \cup E_2$,所以$E$是不连通的。

\section*{13.4.2}

\begin{itemize}
  \item $\Rightarrow$

        因为$(X,d)$是连通的空间,且$f$是连续的,那么由定理13.4.6可知,
        $f(X)$是连通的,又由习题13.4.1可知$f(X)$中只能含有一个元素,
        所以$f$是常数函数。

  \item $\Leftarrow$

        $f$是常数函数,按照连续的定义可知,$f$是连续的。
\end{itemize}

\section*{13.4.3}

\begin{itemize}
  \item (b) $\implies$ (c)

        讨论$X$是非空集合,在广义实数$\mathbb{R}^{*}$中,$\sup(X), \inf(X)$是存在的,
        让$M := \sup(X), m := \inf(X)$。

        任意$z \in (m, M)$,存在$x, y \in X$使得$x < z < y$,因为(b)成立,
        所以$[x, y] \in X$,所以$z \in X$,由$z$的任意性可知$(m, M) \subseteq X$,
        而$X$中的任意元素(除了$m, M$)都属于$(m, M)$,
        于是$X$可以表示成区间($[m, M], [m, M), (m, M], (m, M)$中的任意一种),
        $m, M$是否属于$X$,
        只会影响区间的表示(闭的或开的)。


  \item (c) $\implies$ (b)

        $X$是区间,那么,按照定义9.1.1可知, 任意$x, y \in X$且$x < y$,$[x, y]$包含在$X$中是显然的。
\end{itemize}

\section*{13.4.4}

反证法,假设$f(E)$不是连通的,那么存在两个不相交的非空开集$V$和$W$使得$f(E) = V \cup W$。

由习题13.1.6,习题13.1.7和定理13.1.5(c)可知,集合$f^{-1}(V)$和$f^{-1}(W)$都是非空开集,并且不相交。
(如果存在$x \in f^{-1}(V) \cap f^{-1}(W)$,那么$f(x) \in V \cap W$,这与$V$和$W$不相交矛盾。)
$E = f^{-1}(V) \cup f^{-1}(W)$是易证的。(对任意$x \in E$,$f(x)$要么属于$V$要么属于$W$,
于是可得$x \in f^{-1}(V)$或$x \in f^{-1}(W)$。)

综上,$E$是不连通的,这与题设矛盾。

\section*{13.4.5}

由定理13.4.6可知$f(E)$是连通的。

任意$f(a), f(b) \in f(E)$,
设$f(a) < f(b)$($f(a) \geq f(b)$证明类似)。由定理13.4.5(b)可知,
$[f(a), f(b)] \subseteq f(E)$,因为$f(a) \leq y \leq f(b)$,
于是可得$y \in [f(a), f(b)] \subseteq f(E)$,所以存在$c \in E$使得$f(c) = y$。

\section*{13.4.6}





\end{document}