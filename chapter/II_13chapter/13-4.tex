\documentclass{article}
\usepackage{mathtools} 
\usepackage{fontspec}
\usepackage[UTF8]{ctex}
\usepackage{amsthm}
\usepackage{mdframed}
\usepackage{xcolor}
\usepackage{amssymb}
\usepackage{amsmath}


% 定义新的带灰色背景的说明环境 zremark
\newmdtheoremenv[
  backgroundcolor=gray!10,
  % 边框与背景一致,边框线会消失
  linecolor=gray!10
]{zremark}{说明}


\begin{document}
\title{13.4 习题}
\author{张志聪}
\maketitle

\section*{13.4.1}

从$E$中任选一个元素组成集合$E_1$,$E_2 := E \setminus E_1$。

当使用离散度量$d_disc$时,所有集合都既是开的又是闭的,
又$E = E_1 \cup E_2$,所以$E$是不连通的。

\section*{13.4.2}

\begin{itemize}
  \item $\Rightarrow$

        因为$(X,d)$是连通的空间,且$f$是连续的,那么由定理13.4.6可知,
        $f(X)$是连通的,又由习题13.4.1可知$f(X)$中只能含有一个元素,
        所以$f$是常数函数。

  \item $\Leftarrow$

        $f$是常数函数,按照连续的定义可知,$f$是连续的。
\end{itemize}

\section*{13.4.3}

\begin{itemize}
  \item (b) $\implies$ (c)

        讨论$X$是非空集合,在广义实数$\mathbb{R}^{*}$中,$\sup(X), \inf(X)$是存在的,
        让$M := \sup(X), m := \inf(X)$。

        任意$z \in (m, M)$,存在$x, y \in X$使得$x < z < y$,因为(b)成立,
        所以$[x, y] \in X$,所以$z \in X$,由$z$的任意性可知$(m, M) \subseteq X$,
        而$X$中的任意元素(除了$m, M$)都属于$(m, M)$,
        于是$X$可以表示成区间($[m, M], [m, M), (m, M], (m, M)$中的任意一种),
        $m, M$是否属于$X$,
        只会影响区间的表示(闭的或开的)。


  \item (c) $\implies$ (b)

        $X$是区间,那么,按照定义9.1.1可知, 任意$x, y \in X$且$x < y$,$[x, y]$包含在$X$中是显然的。
\end{itemize}

\section*{13.4.4}

反证法,假设$f(E)$不是连通的,那么存在两个不相交的非空开集$V$和$W$使得$f(E) = V \cup W$。

由习题13.1.6,习题13.1.7和定理13.1.5(c)可知,集合$f^{-1}(V)$和$f^{-1}(W)$都是非空开集,并且不相交。
(如果存在$x \in f^{-1}(V) \cap f^{-1}(W)$,那么$f(x) \in V \cap W$,这与$V$和$W$不相交矛盾。)
$E = f^{-1}(V) \cup f^{-1}(W)$是易证的。(对任意$x \in E$,$f(x)$要么属于$V$要么属于$W$,
于是可得$x \in f^{-1}(V)$或$x \in f^{-1}(W)$。)

综上,$E$是不连通的,这与题设矛盾。

\section*{13.4.5}

由定理13.4.6可知$f(E)$是连通的。

任意$f(a), f(b) \in f(E)$,
设$f(a) < f(b)$($f(a) \geq f(b)$证明类似)。由定理13.4.5(b)可知,
$[f(a), f(b)] \subseteq f(E)$,因为$f(a) \leq y \leq f(b)$,
于是可得$y \in [f(a), f(b)] \subseteq f(E)$,所以存在$c \in E$使得$f(c) = y$。

\section*{13.4.6}

反证法,假设$\bigcup\limits_{\alpha \in I} E_{\alpha}$是不连通的,
那么,$\bigcup\limits_{\alpha \in I} E_{\alpha}$中
存在两个不相交的开集$V$和$W$使得$\bigcup\limits_{\alpha \in I} E_{\alpha} = V \cup W$,
任意$\alpha \in I$,取$V_{\alpha} = V \cap E_{\alpha}, W_{\alpha} = W \cap E_{\alpha}$,
由命题12.3.4(a)可知,$V_{\alpha}$和$W_{\alpha}$都是相对于$E_{\alpha}$的开集,
因为$E_{\alpha}$是连通的,所以$V_\alpha$和$W_{\alpha}$必须有一个是空集,否则$E_{\alpha}$是不连通的,
于是任意$\alpha \in I$,要么$E_{\alpha} \subseteq V$要么$E_{\alpha} \subseteq W$。
因为$V, W$是非空的,那么存在$\alpha, \alpha ^\prime \in I$使得
$E_{\alpha} \subseteq V, E_{\alpha ^\prime} \subseteq W$,
因为$V, W$是不相交的,所以$E_{\alpha} \cap E_{\alpha ^\prime} = \varnothing$,
于是$\bigcap\limits_{\alpha \in I} E_{\alpha} = \varnothing$,与题设矛盾。

\section*{13.4.7}

\begin{zremark}
  这里不能直接用定理13.4.6证明,因为$\gamma ([0, 1]) \subseteq E$,
  而不是$\gamma ([0, 1]) = E$。
\end{zremark}
\begin{zremark}
  逆命题我没有证,主要是没读懂逆命题应该证明什么
\end{zremark}

反证法,假设$E$是不连通的。那么,$E$中存在两个不相交的开集$V$和$W$使得$E = V \cup W$。
取$x \in V, y \in W$,设$Y := \gamma ([0, 1]), Y \subseteq E$,
可得$Y \cap V \not = \varnothing, Y \cap W \not = \varnothing$,
又由命题12.3.4(a)可知$V \cap Y$和$W \cap Y$都是相对于$Y$的开集,
因为$V \cap W = \varnothing$,
所以$(V \cap Y) \cap (W \cap Y) = \varnothing$,
又因为$(V \cap Y) \cup (W \cap Y) = Y$可得,$Y$是不连通的,
但通过定理13.4.6可得,$Y$是连通的,存在矛盾。

\section*{13.4.8}

(1)

反证法,假设$\overline{E}$不连通,
那么,$\overline{E}$中存在两个不相交的开集$V, W$使得$\overline{E} = V \cup W$。
定义$V^\prime := V \cap E, W^\prime := W \cap E$,于是我们有
$V^\prime \cap W^\prime = \varnothing$(因为$V, W$不相交),
并且$V^\prime \cup W^\prime = E$(因为$E \subseteq \overline{E}$)。

接下来,我们需要证明$V^\prime, W^\prime$不是空集。任意$x \in V$,
如果$x \in E$,那么$x \in V^\prime = V \cap E$。
如果$x \in \partial E$,因为$V$是开集,那么存在$r > 0$使得$B(x, r) \subseteq V$,
此时一定存在$y \in B(x, r) \cap E$,否则$x$将是外点,与$x$是边界点矛盾,
于是$y \in V^\prime = V \cap E$。综上,所以$V^\prime$不是空集。
类似的,$W^\prime$也不是空集。

综上可得$E = V^\prime \cup W^\prime$,且$V^\prime, W^\prime$是$E$中的非空开集,
所以$E$是不连通的,与题设矛盾。

(2)逆命题是否成立?

不成立。比如$\overline{E} := [-1, 1]$,$E := (-1, 0) \cup (0, 1)$。

\section*{13.4.9}

(1)证明:这是一种等价关系。

\begin{itemize}
  \item 自反性($x \sim x$)

        任意$x \in X$,集合$\{x\}$是连通的且$x \in \{x\}$,于是可得$x \sim x$。

  \item 对称性($x \sim y \implies y \sim x$)

        因为$x \sim y$,所以存在一个同时包含$x, y$的连通子集$E \subseteq X$,
        这也表明$y \sim x$。

  \item 传递性($x \sim y, y \sim z \implies x \sim z$)

        $x \sim y$,所以存在一个同时包含$x, y$的连通子集$E \subseteq X$。
        $y \sim z$,所以存在一个同时包含$y, z$的连通子集$F \subseteq X$。

        因为$x \in E, x \in F$,所以$E \cap F \not = \varnothing$,
        并且$E, F$都是连通集,
        由习题13.4.6可知$E \cup F$是连通的,且$E \cup F$中包含$x, z$,
        所以$x \sim z$。
\end{itemize}

(2)这种关系的等价类全是连通的闭集。

% todo

\section*{13.4.10}

定理:

$(X,d)$是一个紧致连通度量空间,并设$f: X \to \mathbb{R}$是一个连续函数,
设$M := \sup\limits_{x \in X} f(x)$是$f$的最大值,$m := \inf\limits_{x \in X} f(x)$是$f$的最小值,
并且设$y$是介于$m$和$M$之间的一个实数(即:$m < y < M$),
那么存在一个$c \in X$使得$f(c) = y$。更进一步,$f(X) = [m, M]$。

证明:

由命题13.3.2可知,存在$x_{max} \in X$使得$f(x_{max}) = M$,
存在$x_{min} \in X$使得$f(x_{min}) = m$。由推论13.4.7(介值定理)可知,
存在$c \in X$使得$f(c) = y$。

\end{document}