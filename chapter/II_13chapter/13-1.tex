\documentclass{article}
\usepackage{mathtools} 
\usepackage{fontspec}
\usepackage[UTF8]{ctex}
\usepackage{amsthm}
\usepackage{mdframed}
\usepackage{xcolor}
\usepackage{amssymb}
\usepackage{amsmath}


% 定义新的带灰色背景的说明环境 zremark
\newmdtheoremenv[
  backgroundcolor=gray!10,
  % 边框与背景一致,边框线会消失
  linecolor=gray!10
]{zremark}{说明}


\begin{document}
\title{13.1 习题}
\author{张志聪}
\maketitle

\section*{13.1.1}

\begin{itemize}
  \item $(a) \implies (c)$

        因为$f(x_0) \in V$且$V$是开集,那么存在$r > 0$使得$B(f(x_0), r) \subseteq V$。
        因为$f$在$x_0$处是连续的,那么存在$\delta > 0$,使得只要$d_X(x_0, x) < \delta$,就有
        $d_Y(f(x), f(x_0)) < r$,
        于是令$U = B(x_0, \delta)$即可满足要求,使得$f(U) \subseteq B(f(x_0), r) \subseteq V$。

  \item $(c) \implies (b)$

        对于任意$\epsilon > 0$,令$V := B(f(x_0), \epsilon)$,那么$V \subset Y$。由(c)可知,
        存在一个包含$x_0$的开集$U \subset X$,使得$f(U) \subseteq V$。

        因为$U$是开集,所以存在$B(x_0, r) \subseteq U$,因为序列$(x^{(n)})_{n = 1}^\infty$是$X$中依度量
        $d_X$收敛于$x_0$的序列,于是存在$N \geq 1$使得
        \begin{align*}
          d_X(x_0, x^{(n)}) < r
        \end{align*}
        对所有的$n \geq N$均成立。那么,对所有的$n \geq N$都有
        \begin{align*}
          x^{(n)} \in B(x_0, r) \subseteq U
        \end{align*}
        所以$f(x^{(n)}) \in V$,即
        \begin{align*}
          d_Y(f(x^{(n)}), f(x_0)) < \epsilon
        \end{align*}
        由$\epsilon$的任意性可知,序列$(f(x^{(n)}))_{n = 1}^\infty$是$Y$中依度量$d_Y$收敛于$f(x_0)$的序列。

  \item $(b) \implies (a)$

        反证法,假设$f$在$x_0$处是不连续的,那么存在$\epsilon > 0$使得对任意的$\delta > 0, d_X(x, x_0) < \delta$,
        都有$d_Y(f(x), f(x_0)) > \epsilon$,令$\delta = \frac{1}{n}, n \in \mathbb{N}$,利用选择公理
        可以得到一个在$X$中依度量$d_X$收敛于$x_0$的序列$(x^{(n)})_{n = 1}^\infty$,由(b)可知,
        序列$(f(x^{(n)}))_{n = 1}^\infty$依度量$d_Y$收敛于$f(x_0)$,
        由于对任意的$n$都有$d_Y(f(x^{(n)}), f(x_0)) > \epsilon$,
        所以序列$(f(x^{(n)}))_{n = 1}^\infty$依度量$d_Y$不可能收敛于$f(x_0)$,
        存在矛盾。
\end{itemize}

\section*{13.1.2}

由定理13.1.4可知(a)和(b)是等价的。

\begin{itemize}
  \item $(a) \implies (c)$

        任意$x_0 \in f^{-1}(V), f(x_0) \in V$,因为$V$是开集,所以存在$r > 0$使得$B(f(x_0), r) \subseteq V$。
        $f$是连续的,那么存在$\delta > 0$使得只要$d_X(x_0, x) < \delta$,就有$d_Y(f(x), f(x_0)) < r$。
        所以对于任意$x \in B(x_0, \delta)$,都有$d_Y(f(x), f(x_0)) < r$,
        所以$f(B(x_0, \delta)) \subseteq B(f(x_0), r) \subseteq V$,所以$B(x_0, \delta) \subseteq f^{-1}(V)$。

        由$x_0$的任意性可知,$f^{-1}(V)$就是$X$中的开集。

  \item $(c) \implies (a)$

        设$x_0 \in X$,那么对任意$\epsilon > 0, V := B(f(x_0), \epsilon)$是$Y$中的开集,
        由$(c)$可知$U := f^{-1}(V)$是$X$中的开集,
        所以存在$\delta > 0$使得$B(x_0, \delta) \subseteq U, f(B(x_0, \delta)) \subseteq V = B(f(x_0), \epsilon)$,
        即$d_X(x_0, x) < \delta$,就有
        \begin{align*}
          d_Y(f(x), f(x_0)) < \epsilon
        \end{align*}
        所以$f$在$x_0$处是连续的。

        由$x_0$的任意性可知,$f$是连续的。

  \item $(c) \Leftrightarrow (d)$

        $F$是闭集由命题12.2.15(e)可知当且仅当$Y \setminus F$是开集,由(c)可知$f^{-1}(Y \setminus F)$开集。
        因为$X \setminus f^{-1}(F) = \{x \in X : f(x) \not\in F\} = f^{-1}(Y \setminus F)$,
        所以$f^{-1}(F)$是闭集。

        反向证明类似。
\end{itemize}

\section*{13.1.3}

\begin{itemize}
  \item (a)

        对任意$\epsilon$,
        由于$g$在$f(x_0)$处是连续的,那么存在$\delta > 0$使得只要$d_Y(f(x), f(x_0)) < \delta$,
        就有$d_Z(g(f(x)), g(f(x_0))) < \epsilon$。

        同理,由于$f$,在$x_0 \in X$处是连续的,那么存在$\delta^\prime > 0$使得只要$d_X(x, x_0) < \delta^\prime$,
        就有$d_Y(f(x), f(x_0)) < \delta$。

        综上,只要$d_X(x, x_0) < \delta^\prime$,就有$d_Z(g(f(x)), g(f(x_0))) < \epsilon$。
        所以$g \circ f : X \to Z$在$x_0$处是连续的。

  \item (b)

        与(a)证明类似,略。

\end{itemize}

\section*{13.1.4}

\begin{itemize}
  \item (a)

        \begin{equation*}
          f(x) =
          \begin{cases}
            -1, x < 0 \\
            1, x \geq 0
          \end{cases}
          ,g(x) = x^2
        \end{equation*}
        于是
        \begin{align*}
          g \circ f(x) = 1
        \end{align*}

  \item (b)

        \begin{equation*}
          g(x) =
          \begin{cases}
            -1, x < 0 \\
            1, x \geq 0
          \end{cases}
          ,f(x) = x^2
        \end{equation*}
        于是
        \begin{align*}
          g \circ f(x) = 1
        \end{align*}

  \item (c)

        \begin{equation*}
          f(x) =
          \begin{cases}
            -1, x < 0 \\
            1, x \geq 0
          \end{cases}
          ,g(x) =
          \begin{cases}
            x^2, |x| \geq 1 \\
            0, |x| < 1
          \end{cases}
        \end{equation*}
        于是
        \begin{align*}
          g \circ f(x) = 1
        \end{align*}

\end{itemize}

\section*{13.1.5}
任意$x_0 \in E, \epsilon > 0$,只要$d|_{E \times E}(x, x_0) < \epsilon$, 就有
\begin{align*}
  d_X(\iota _{E \to X}(x), \iota _{E \to X}(x_0)) = d_X(x, x_0) = d|_{E \times E}(x, x_0) < \epsilon
\end{align*}
注意,利用了$E$是$X$的子集且$x, x_0 \in E$。

由$x_0$的任意性可知,$\iota _{E \to X}$是连续的。

\section*{13.1.6}

(1)$f|_E$也在$x_0$处连续。

设$(x_{(n)})_{n = 1}^\infty$是$E$中依度量$d_X$收敛于$x_0$的序列,
因为$E$是$X$的子集,又$f$在$x_0$处是连续的,
由定理13.1.4(b)可知序列$(f(x_{(n)}))_{n = 1}^\infty$在$x_0$依度量$d_Y$收敛于$f(x_0)$,
$f|E(x) = f(x)$所以$f|_E$也在$x_0$处连续。

(2)逆命题。

不成立;定义
\begin{equation*}
  f(x) = \begin{cases}
    1, x \geq 0 \\
    -1, x < 0
  \end{cases}
\end{equation*}
$E := [0, 1]$,于是$x = 0$时$f|_E$是连续的,$f$不是连续的。

(3)$f$是连续的,那么$f|_E$就是连续的。

证明与(1)同理,证明略。


\section*{13.1.7}

(1)对任意的$x_0 \in X$,$f$在$x_0$处是连续的,当且仅当$g$在$x_0$处是连续的。

\begin{itemize}
  \item $\Rightarrow$

        由$f$在$x_0$处连续可知,对任意$\epsilon > 0$,
        存在$\delta > 0$使得只要$d_X(x, x_0) < \delta$,就有$d_Y(f(x), f(x_0)) < \epsilon$。
        又因为对所有的$x \in X$都有$g(x) = f(x)$,所以$d_Y(g(x), g(x_0)) < \epsilon$,
        所以$g$在$x_0$处是连续的。

  \item $\Leftarrow$

        证明类似,略
\end{itemize}

(2)$f$是连续的,当且仅当$g$是连续的。

证明略。
\end{document}