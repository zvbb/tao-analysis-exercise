\documentclass{article}
\usepackage{mathtools} 
\usepackage{fontspec}
\usepackage[UTF8]{ctex}
\usepackage{amsthm}
\usepackage{mdframed}
\usepackage{xcolor}
\usepackage{amssymb}
\usepackage{amsmath}


% 定义新的带灰色背景的说明环境 zremark
\newmdtheoremenv[
  backgroundcolor=gray!10,
  % 边框与背景一致,边框线会消失
  linecolor=gray!10
]{zremark}{说明}


\begin{document}
\title{13.1 习题}
\author{张志聪}
\maketitle

\section*{13.1.1}

\begin{itemize}
  \item $(a) \implies (c)$

        因为$f(x_0) \in V$且$V$是开集,那么存在$r > 0$使得$B(f(x_0), r) \subseteq V$。
        因为$f$在$x_0$处是连续的,那么存在$\delta > 0$,使得只要$d_X(x_0, x) < \delta$,就有
        $d_Y(f(x), f(x_0)) < r$,
        于是令$U = B(x_0, \delta)$即可满足要求,使得$f(U) \subseteq B(f(x_0), r) \subseteq V$。

  \item $(c) \implies (b)$

        对于任意$\epsilon > 0$,令$V := B(f(x_0), \epsilon)$,那么$V \subset Y$。由(c)可知,
        存在一个包含$x_0$的开集$U \subset X$,使得$f(U) \subseteq V$。

        因为$U$是开集,所以存在$B(x_0, r) \subseteq U$,因为序列$(x^{(n)})_{n = 1}^\infty$是$X$中依度量
        $d_X$收敛于$x_0$的序列,于是存在$N \geq 1$使得
        \begin{align*}
          d_X(x_0, x^{(n)}) < r
        \end{align*}
        对所有的$n \geq N$均成立。那么,对所有的$n \geq N$都有
        \begin{align*}
          x^{(n)} \in B(x_0, r) \subseteq U
        \end{align*}
        所以$f(x^{(n)}) \in V$,即
        \begin{align*}
          d_Y(f(x^{(n)}), f(x_0)) < \epsilon
        \end{align*}
        由$\epsilon$的任意性可知,序列$(f(x^{(n)}))_{n = 1}^\infty$是$Y$中依度量$d_Y$收敛于$f(x_0)$的序列。

  \item $(b) \implies (a)$

        
\end{itemize}


\end{document}