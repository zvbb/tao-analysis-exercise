\documentclass{article}
\usepackage{mathtools} 
\usepackage{fontspec}
\usepackage[UTF8]{ctex}
\usepackage{amsthm}
\usepackage{mdframed}
\usepackage{xcolor}
\usepackage{amssymb}
\usepackage{amsmath}


% 定义新的带灰色背景的说明环境 zremark
\newmdtheoremenv[
  backgroundcolor=gray!10,
  % 边框与背景一致,边框线会消失
  linecolor=gray!10
]{zremark}{说明}


\begin{document}
\title{13.2 习题}
\author{张志聪}
\maketitle

\section*{13.2.1}

方法一:使用连续的定义证明

\begin{itemize}
  \item (a)

        \begin{itemize}
          \item $\Rightarrow$

                对任意$\epsilon > 0, \frac{1}{\sqrt{2}}\epsilon$,
                因为$f$在$x_0$处连续,存在$\delta_{f} > 0$使得只要$d_X(x, x_0) < \delta_{f}$,就有
                \begin{align*}
                  d_{l^2}(f(x), f(x_0)) = |f(x) - f(x_0)| < \frac{1}{\sqrt{2}}\epsilon
                \end{align*}

                类似地,存在$\delta_{g} > 0$使得只要$d_X(x, x_0) < \delta_{g}$,就有
                \begin{align*}
                  d_{l^2}(g(x), g(x_0)) = |g(x) - g(x_0)| < \frac{1}{\sqrt{2}}\epsilon
                \end{align*}

                综上,$\delta < min(\delta_{f}, \delta_{g})$,使得只要$d_X(x, x_0) < \delta$,就有
                \begin{align*}
                  d_{l^2}(f \oplus g(x), f \oplus g(x_0)) & = d_{l^2}((f(x), g(x)), (f(x_0), g(x_0)))      \\
                                                          & = \sqrt{|f(x) - f(x_0)|^2 + |g(x) - g(x_0)|^2} \\
                                                          & < \epsilon
                \end{align*}
                所以$f \oplus g$在$x_0$处是连续的。

          \item $\Leftarrow$

                任意$\epsilon > 0$,由于$f \oplus g$在$x_0$处是连续的,所以存在$\delta > 0$使得只要$d_X(x, x_0) < \delta$,就有
                \begin{align*}
                  d_{l^2}(f \oplus g(x), f \oplus g(x_0)) & = d_{l^2}((f(x), g(x)), (f(x_0), g(x_0)))      \\
                                                          & = \sqrt{|f(x) - f(x_0)|^2 + |g(x) - g(x_0)|^2} \\
                                                          & < \epsilon
                \end{align*}
                由此可得
                \begin{align*}
                  |f(x) - f(x_0)| < \epsilon \\
                  |g(x) - g(x_0)| < \epsilon
                \end{align*}
                即
                \begin{align*}
                  d_{l^2}(f(x), f(x_0)) < \epsilon
                \end{align*}
                \begin{align*}
                  d_{l^2}(g(x), g(x_0)) < \epsilon
                \end{align*}
                于是可得$f,g$在$x_0$处是连续的。

        \end{itemize}

  \item (b)

        可以由(a)直接推出。
\end{itemize}

方法二:使用书中的提示

\begin{itemize}
  \item (a)

        \begin{itemize}
          \item $\Rightarrow$

                任意$(x^{(n)})_{n = 1}^\infty$是$X$中依度量$d_X$收敛于$x_0$的序列,
                因为$f,g$在$x_0$处连续,由命题13.1.4(b)可知,
                序列$(f(x^{(n)}))_{n = 1}^\infty$依度量$d_{l^2}$收敛于$f(x_0)$(书中有说在没有特殊说明的时,提到度量空间$R^n(n \geq 1)$指的就是欧几里得度量)。
                序列$(g(x^{(n)}))_{n = 1}^\infty$依度量$d_{l^2}$收敛于$g(x_0)$。

                由命题12.1.18(d)可知,$(f(x^{(n)}), g(x^{(n)}))_{n = 1}^\infty$依度量$d_{l^2}$收敛于$(f(x_0),g(x_0))$,
                由13.1.4(b)可知$f \oplus g$在$x_0$处是连续的。

          \item $\Leftarrow$

                任意$(x^{(n)})_{n = 1}^\infty$是$X$中依度量$d_X$收敛于$x_0$的序列,
                因为$f \oplus g$在$x_0$处是连续的,
                由命题13.1.4(b)可知,序列$(f \oplus g(x^{(n)}))_{n = 1}^\infty = (f(x^{(n)}), g(x^{(n)}))_{n = 1}^\infty$依度量$d_{l^2}$收敛于$(f(x_0),g(x_0))$,
                由命题12.1.18(d)可知序列$(f(x^{(n)}))_{n = 1}^\infty$依度量$d_{l^2}$收敛于$f(x_0)$,
                序列$(g(x^{(n)}))_{n = 1}^\infty$依度量$d_{l^2}$收敛于$g(x_0)$,
                所以由13.1.4(b)可知$f,g$在$x_0$处连续。


        \end{itemize}

  \item (b)

        可以由(a)直接推出。
\end{itemize}

\section*{13.2.2}

\end{document}