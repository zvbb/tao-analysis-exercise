\documentclass{article}
\usepackage{mathtools} 
\usepackage{fontspec}
\usepackage[UTF8]{ctex}
\usepackage{amsthm}
\usepackage{mdframed}
\usepackage{xcolor}
\usepackage{amssymb}
\usepackage{amsmath}


% 定义新的带灰色背景的说明环境 zremark
\newmdtheoremenv[
  backgroundcolor=gray!10,
  % 边框与背景一致,边框线会消失
  linecolor=gray!10
]{zremark}{说明}


\begin{document}
\title{13.2 习题}
\author{张志聪}
\maketitle

\section*{13.2.1}

方法一:使用连续的定义证明

\begin{itemize}
  \item (a)

        \begin{itemize}
          \item $\Rightarrow$

                对任意$\epsilon > 0, \frac{1}{\sqrt{2}}\epsilon$,
                因为$f$在$x_0$处连续,存在$\delta_{f} > 0$使得只要$d_X(x, x_0) < \delta_{f}$,就有
                \begin{align*}
                  d_{l^2}(f(x), f(x_0)) = |f(x) - f(x_0)| < \frac{1}{\sqrt{2}}\epsilon
                \end{align*}

                类似地,存在$\delta_{g} > 0$使得只要$d_X(x, x_0) < \delta_{g}$,就有
                \begin{align*}
                  d_{l^2}(g(x), g(x_0)) = |g(x) - g(x_0)| < \frac{1}{\sqrt{2}}\epsilon
                \end{align*}

                综上,$\delta < min(\delta_{f}, \delta_{g})$,使得只要$d_X(x, x_0) < \delta$,就有
                \begin{align*}
                  d_{l^2}(f \oplus g(x), f \oplus g(x_0)) & = d_{l^2}((f(x), g(x)), (f(x_0), g(x_0)))      \\
                                                          & = \sqrt{|f(x) - f(x_0)|^2 + |g(x) - g(x_0)|^2} \\
                                                          & < \epsilon
                \end{align*}
                所以$f \oplus g$在$x_0$处是连续的。

          \item $\Leftarrow$

                任意$\epsilon > 0$,由于$f \oplus g$在$x_0$处是连续的,所以存在$\delta > 0$使得只要$d_X(x, x_0) < \delta$,就有
                \begin{align*}
                  d_{l^2}(f \oplus g(x), f \oplus g(x_0)) & = d_{l^2}((f(x), g(x)), (f(x_0), g(x_0)))      \\
                                                          & = \sqrt{|f(x) - f(x_0)|^2 + |g(x) - g(x_0)|^2} \\
                                                          & < \epsilon
                \end{align*}
                由此可得
                \begin{align*}
                  |f(x) - f(x_0)| < \epsilon \\
                  |g(x) - g(x_0)| < \epsilon
                \end{align*}
                即
                \begin{align*}
                  d_{l^2}(f(x), f(x_0)) < \epsilon
                \end{align*}
                \begin{align*}
                  d_{l^2}(g(x), g(x_0)) < \epsilon
                \end{align*}
                于是可得$f,g$在$x_0$处是连续的。

        \end{itemize}

  \item (b)

        可以由(a)直接推出。
\end{itemize}

方法二:使用书中的提示

\begin{itemize}
  \item (a)

        \begin{itemize}
          \item $\Rightarrow$

                任意$(x^{(n)})_{n = 1}^\infty$是$X$中依度量$d_X$收敛于$x_0$的序列,
                因为$f,g$在$x_0$处连续,由命题13.1.4(b)可知,
                序列$(f(x^{(n)}))_{n = 1}^\infty$依度量$d_{l^2}$收敛于$f(x_0)$(书中有说在没有特殊说明的时,提到度量空间$R^n(n \geq 1)$指的就是欧几里得度量)。
                序列$(g(x^{(n)}))_{n = 1}^\infty$依度量$d_{l^2}$收敛于$g(x_0)$。

                由命题12.1.18(d)可知,$(f(x^{(n)}), g(x^{(n)}))_{n = 1}^\infty$依度量$d_{l^2}$收敛于$(f(x_0),g(x_0))$,
                由13.1.4(b)可知$f \oplus g$在$x_0$处是连续的。

          \item $\Leftarrow$

                任意$(x^{(n)})_{n = 1}^\infty$是$X$中依度量$d_X$收敛于$x_0$的序列,
                因为$f \oplus g$在$x_0$处是连续的,
                由命题13.1.4(b)可知,序列$(f \oplus g(x^{(n)}))_{n = 1}^\infty = (f(x^{(n)}), g(x^{(n)}))_{n = 1}^\infty$依度量$d_{l^2}$收敛于$(f(x_0),g(x_0))$,
                由命题12.1.18(d)可知序列$(f(x^{(n)}))_{n = 1}^\infty$依度量$d_{l^2}$收敛于$f(x_0)$,
                序列$(g(x^{(n)}))_{n = 1}^\infty$依度量$d_{l^2}$收敛于$g(x_0)$,
                所以由13.1.4(b)可知$f,g$在$x_0$处连续。


        \end{itemize}

  \item (b)

        可以由(a)直接推出。
\end{itemize}

\section*{13.2.2}

任意$(x_0, y_0) \in \mathbb{R}^2$,
设$(x^{(n)})_{n = 1}^\infty$是$\mathbb{R}^2$中依度量$d_{l^2}$收敛于$(x_0, y_0)$的序列,
对任意$n \in \mathbb{N}$,$x^{(n)} = (a_n, b_n)$,由命题12.1.18可知,
序列$(a_n)_{n = 1}^\infty$收敛于$x_0$,
序列$(b_n)_{n = 1}^\infty$收敛于$y_0$。
由定理6.1.19(极限定律)可知
\begin{align*}
  \lim\limits_{n \to \infty} (a_n + b_n) & = \lim\limits_{n \to \infty} a_n + \lim\limits_{n \to \infty} b_n \\
                                         & = x_0 + y_0
\end{align*}
由定理13.1.4(连续性保持收敛性)(b)可知,函数$f(x, y) = x + y$在点$(x_0, y_0)$处是连续的。
由$(x_0, y_0)$的任意性可知$f(x, y) = x + y$是连续的。

同理可证其他函数。

\section*{13.2.3}
定义$g: X \to \mathbb{R}$,任意$x \in X$都有$g(x) = 0$,于是$g: X \to \mathbb{R}$是连续函数。
又由于$|f|(x) = max(f(x), -f(x)) = max(f(x), g(x) - f(x))$,
因为$f: X \to \mathbb{R}, g: X \to \mathbb{R}$是一个连续函数,
由推论13.2.3可知$g - f: X \to \mathbb{R}$是连续函数,
再次利用推论13.2.3可知$|f|: X \to \mathbb{R}$也是连续函数。

\section*{13.2.4}

(1)

任意$(x_0, y_0) \in \mathbb{R}^2$,
设$(x^{(n)})_{n = 1}^\infty$是$\mathbb{R}^2$中依度量$d_{l^2}$收敛于$(x_0, y_0)$的序列,
对任意$n \in \mathbb{N}$,$x^{(n)} = (a_n, b_n)$。

由命题12.1.18可知,
序列$(a_n)_{n = 1}^\infty$收敛于$x_0$,
序列$(b_n)_{n = 1}^\infty$收敛于$y_0$。
于是
\begin{align*}
  \lim\limits_{n \to \infty}  (\pi_1(x^{(n)})) = \lim\limits_{n \to \infty} a_n = x_0 = \pi_1(x_0, y_0)
\end{align*}
所以$\pi_1$是连续的;

同理可证$\pi_2$是连续的。


(2)

$g_1(x, y) = f(\pi_1(x, y)) = f \circ \pi_1(x, y)$,由推论13.1.7可知$g_1$是连续的;

同理可证$g_2$是连续的。

\section*{13.2.5}
(1)

任意$0 \leq i \leq n$和$0 \leq j \leq m$,
\begin{align*}
  c^{ij}x^iy^j = c^{ij}\pi_1^i(x, y) \pi_2^j(x, y)
\end{align*}
由推论13.2.3(b)可知是连续函数。
再次利用推论13.2.3(b)可知,有限个连续函数相加的结果是连续函数。

(2)

证明参考推论13.2.3的证明。

因为$f,g$都是连续的,那么$f \oplus g$是连续的。由(1)可知函数$P$是连续的。
我们把这两个函数复合在一起,那么根据推论13.1.7可知,$P(f,g)(x): X \to \mathbb{R}$是连续的。

\section*{13.2.6}

\begin{itemize}
  \item $\Rightarrow$

        证明方法与习题13.2.1的证明方法(方法二)相同,不再赘述。

  \item $\Leftarrow$

        成立;证明方法与习题13.2.1的证明方法(方法二)相同,不再赘述。
\end{itemize}

\section*{13.2.7}

这道题,没有用书中的提示证明。使用的证明方法与习题13.2.5一致。

任意$(i_1, i_2, \dots, i_k) \in I$,$c(i_1, i_2, \dots, i_k)$是常数,
$x_1^{i_1}, x_2^{i_2}, \dots, x_k^{i_k}$由习题13.2.4和推论13.2.3(b)可知分别都是连续的,
再次利用推论13.2.3(b)可知$c(i_1, i_2, \dots, i_k)x_1^{i_1}x_2^{i_2}\dots x_k^{i_k}$是连续函数。

因为$I$是有限子集,由推论13.2.3(b)可知有限个连续函数相加的结果是连续函数,即$P(x_1, \dots, x_k)$
是连续函数。

\section*{13.2.8}

(1)$(X \times Y, d_{X \times Y})$是度量空间。

证明度量是否满足四个公理
\begin{itemize}
  \item (a)

        对任意的$(x, y) \in X \times Y$,
        \begin{align*}
          d_{X \times Y}((x, y), (x, y)) = d_X(x, x) + d_Y(y, y) = 0
        \end{align*}
        注意,因为$(X, d_X), (Y, d_Y)$都是度量空间,所以$d_X(x, x) = 0, d_Y(y, y) = 0$。

  \item (b) 正性

        对任意两个不同的$(x, y), (x^\prime, y^\prime) \in X \times Y$,
        \begin{align*}
          d_{X \times Y}((x, y), (x^\prime, y^\prime)) = d_X(x, x^\prime) + d_Y(y, y^\prime) > 0
        \end{align*}
        注意,因为$(X, d_X), (Y, d_Y)$都是度量空间,所以$d_X(x, x^\prime) > 0, d_Y(y, y^\prime) > 0$。

  \item (c) 对称性

        对任意两个$(x, y), (x^\prime, y^\prime) \in X \times Y$,
        \begin{align*}
          d_{X \times Y}((x, y), (x^\prime, y^\prime)) & = d_X(x, x^\prime) + d_Y(y, y^\prime)          \\
                                                       & = d_X(x^\prime, x) + d_Y(y^\prime, y)          \\
                                                       & = d_{X \times Y}((x^\prime, y^\prime), (x, y))
        \end{align*}

  \item (d) 三角不等式

        对任意三个$(x, y), (x^\prime, y^\prime), (x^{\prime \prime}, y^{\prime \prime}) \in X \times Y$,
        \begin{align*}
           & d_{X \times Y}((x, y), (x^{\prime \prime}, y^{\prime \prime}))                                                                        \\
           & = d_X(x, x^{\prime \prime}) + d_Y(y,y{\prime \prime})                                                                                 \\
           & \leq d_X(x, x^{\prime}) + d_X(x^{\prime}, x^{\prime \prime}) + d_Y(y, y^{\prime}) + d_Y(y^{\prime}, y^{\prime \prime})                \\
           & = d_{X \times Y}((x, y), (x^{\prime}, y^{\prime})) + d_{X \times Y}((x^{\prime}, y^{\prime}), (x^{\prime \prime}, y^{\prime \prime}))
        \end{align*}
\end{itemize}

综上,$(X \times Y, d_{X \times Y})$是度量空间。

(2)与命题12.1.18类似的结论。

如果$(x^{(k)})_{k = 1}^\infty$是度量空间$(X \times Y, d_{X \times Y})$中的序列,其中
$x^{(k)} = (x_1^{(k)}, x_2^{(k)}), x_1^{(k)} \in X, x_2^{(k)} \in Y$,$x = (x_1, x_2)$是$X \times Y$中的点,
那么下面两个命题是等价的。

(a)$(x^{(k)})_{k = 1}^\infty$收敛于$x$。

(b)序列$(x_1^{(k)})_{k = 1}^\infty$在$X$中收敛于$x_1$,序列$(x_1^{(k)})_{k = 1}^\infty$收敛于$x_2$。

证明:

\begin{itemize}
  \item (a) $\Rightarrow$ (b)



  \item (b) $\Rightarrow$ (a)
\end{itemize}


\end{document}