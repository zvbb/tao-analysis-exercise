\documentclass{article}
\usepackage{mathtools} 
\usepackage{fontspec}
\usepackage[UTF8]{ctex}
\usepackage{amsthm}
\usepackage{mdframed}
\usepackage{xcolor}
\usepackage{amssymb}
\usepackage{amsmath}

\newmdtheoremenv[
  backgroundcolor=gray!10,
  linewidth=0pt,
  innerleftmargin=10pt,
  innerrightmargin=10pt,
  innertopmargin=10pt,
  innerbottommargin=10pt
]{zgraytheorem}{}
% 定义说明环境样式
\newtheoremstyle{mystyle}% 说明环境样式的名称
  {1em}% 上方间距
  {1em}% 下方间距
  {\normalfont}% 说明内容的字体样式
  {}% 缩进量
  {\bfseries}% 说明标记的字体样式
  {.}% 说明标记和说明内容之间的标点
  {1em}% 说明标记后的水平空间
  {}% 说明标记后的垂直空间
% 使用新定义的样式创建说明环境
\theoremstyle{mystyle}
\newtheorem*{zremark}{说明}


\begin{document}
\title{4.3 习题}
\maketitle

\begin{zgraytheorem}
  \begin{zremark}
    本节的证明过程中,用到了一些命题,在书中没有提到,这里提前列出,并证明它。

    \textbf{A. 正有理数 $\geq$ 零 $\geq$ 负有理数}

    证明:

    不妨设$x,y$是任意有理数,并且$x$是正有理数,$y$是负有理数,
    所以存在$a,b,c,d$正整数,使得$x=a/b,y=(-c)/d$,现在只需证明$x \geq 0 \geq y$。
    \begin{align*}
      x - 0 & = a/b - 0   \\
            & = a/b - 0/1 \\
            & = a1-b*0/b  \\
            & = a / b     \\
            & = x         \\
    \end{align*}
    由于$x$是正的,所以$x \geq 0$。

    \begin{align*}
      0 - y & = 0 - (-c)/d \\
            & = 0 - (-c)/d \\
            & = c/d        \\
    \end{align*}
    有$c/d$是正有理数,所以$0 \geq y$。

    综上,命题成立。

    \textbf{A推论1. 正有理数 $>$ 零 $>$ 负有理数}

    证明:由于正有理数不等于零,且由命题A,可知正整数大于零;由于负有理数不等于零,且由命题A,可知负整数小于零。

    \textbf{A推论2. 有理数$x>0$,那么$x$是正有理数;有理数$x < 0$,那么$x$是负有理数}。

    证明:

    由于$x>0$,所以$x-0$是正有理数,不妨设该正有理数是$k$,即:
    \begin{align*}
      x-0 & = k \\
      x   & = k \\
    \end{align*}
    由于$k$是正有理数,所以$x$也是正有理数;

    同理$x < 0$时,$x$是负有理数。


    \textbf{B. 两个正有理数相加,是正有理数}

    证明:不妨设$x,y$是任意正有理数,所以存在$a,b,c,d$正整数,使得$x=a/b,y=c/d$。
    \begin{align*}
      x + y & = a/b + c/d      \\
            & = (ad + bc) / bd \\
    \end{align*}
    由于分子是正整数(命题2.2.8),分母是正整数(命题2.3.3),所以$x+y$是正有理数。


  \end{zremark}
\end{zgraytheorem}

\section*{4.3.1}

\textbf{(a)(绝对值的非退化性)我们有$|x| \geq 0$。另外,$|x|=0$当且仅当$x$为零。}

证明:

$x$是有理数,由引理4.2.7(有理数的三歧性)可知,$x$有三种情况:

(1)$x$是正有理数,此时,$|x|=x$,而正有理数$|x|-0=x-0=x$,由定义4.2.8(有理数的排序)可知$|x|>0$;

(2)$x$是负有理数,此时,$|x|=-x$,$|x|-0=-x-0=-x$,而$-x$是正有理数,由定义4.2.8(有理数的排序)可知$|x|>0$;

(3)$x$等于0,此时$|x|=0$,由定义4.2.8(有理数的排序)可知$|x| \geq 0$;

综上,$|x| \geq 0$。另外,$|x|=0$当且仅当$x$为零。

\textbf{(b)(绝对值的三角不等式)我们有$|x+y| \leq |x| + |y|$。}

证明:

可以通过有理数的三歧性证明,这里情况较多,只证明$x$是正有理数,$y$是负有理数的情况【偷个懒,哈哈哈】。

设$x$是正有理数,$y$是负有理数,不妨设$x=a/b,y=(-c)/d$,其中$a,b,c,d$都是正整数。
\begin{align*}
  |x| + |y| & = a/b + c/d  \\
            & = (ad+bc)/bd \\
\end{align*}

\begin{align*}
  x + y & = a/b + (-c)/d \\
        & = (ad-bc)/bd
\end{align*}
若$x + y$是负有理数,则:
\begin{align*}
  |x+y| & =-(x+y)        \\
        & =[-(ad-bc)]/bd \\
        & =(bc-ad)/bd
\end{align*}

\begin{align*}
  |x|+|y| - (|x+y|) & = (ad+bc)/bd - (bc-ad)/bd    \\
                    & = (ad+bc)/bd + (ad-bc)/bd    \\
                    & = [(ad+bc)bd+(ad-bc)bd]/bdbd \\
                    & = (adbc+adbc) / bdbd         \\
\end{align*}

由于分子是正整数(命题2.2.8),分母是正整数(命题2.3.3),可知$(adbc+adbc) / bdbd$是正的,
所以$ |x| + |y| > |x+y|$。

\textbf{(c)不等式$-y \leq x \leq y$成立,当且仅当$y \geq |x|$。特别地,$-|x| \leq x \leq |x|$。}

证明:

充分性:假设前提$-y \leq x \leq y$成立,该前提隐含$y$不是负有理数(见说明)。由有理数的三歧性,$x$的取值有3种情况:
(1)$x$等于0,此时$|x|=0$,而$y$是正有理数,所以$y \geq 0$。

(2)$x$等于正有理数,此时$|x|=x$,由前提可知$y \geq x$。

(3)$x$等于负有理数,此时$|x|=-x$,不妨设$a,b,c,d$是正有理数,$x=(-a)/b,y=c/d$,
由于$-y \leq x$,所有$-y-x$是负有理数,即:
\begin{align*}
  -y - x & = (-c)/d - (-a)/b \\
         & = (-c)b + a/b     \\
         & = a/b - c/d       \\
         & = (ad - bc) / bd  \\
\end{align*}
由上且$-y-x$是负有理数,可知$(ad-bc)=-(bc-ad)$是负整数,所以$bc-ad$是正整数。

\begin{align*}
  y - (-x) & = c/d - \{-[(-a)/b]\} \\
           & = c/d - a/b           \\
           & = (bc - ad) / bd      \\
\end{align*}
由$bc-ad$是正整数和$bd$是正整数,可知$y-(-x)$是正有理数,所以$y \geq -x$。

综合(1)(2)(3)可知$y \geq |x|$。

必要性:假设$y \geq |x|$,由(a)可知$|x| \geq 0$,又序是可传递的(命题4.2.9),所以$y \geq 0$。由有理数的三歧性,
$x$的取值有3种情况:
(1)$x$等于0,此时$|x|=0$,由前提$y \geq |x|$可知$y \geq 0$,由此可知$y$是零或正有理数,所以$-y$是零或负有理数,
进而$-y \leq 0$。

(2)$x$是正有理数,此时$|x|=x$,由前提$y \geq |x|$可知$y \geq x$,此时$y$是正有理数,$x-(-y)=x+y$,两个正有理数相加是正有理数,所以$-y \leq x$。

(3)$x$是负有理数,此时$|x|=-x$,不妨设$x=(-a)/b,y=c/d$,其中$a,b,c,d$是正整数。
由前提$y \geq |x|$,可知$y \geq -x$,所以:
\begin{align*}
  y - (-x) & = c/d - \{-[(-a)/b]\} \\
           & = c/d - a/b           \\
           & = (bc - ad)/bd        \\
\end{align*}
由于$y \geq -x$,所以$(bc - ad)/bd$是正的。

\begin{align*}
  y - x & = c/d - (-a)/b   \\
        & = c/d + a/b      \\
        & = (ad + bc) / bd \\
\end{align*}
由于$a,b,c,d$都是正整数,由此可知$(ad + bc) / bd$是正的,所以$y > x$。

\begin{align*}
  x - (-y) & = x + y        \\
           & = (-a)/b + c/d \\
           & = (bc - ad)/bd \\
\end{align*}
由于$(bc - ad)/bd$是正的,所以$x \geq -y$。

综上,(1)(2)(3)可知$-y \leq x \leq y$。

特别地,把$y$替换为$|x|$,并且$|x| \geq x$,由必要性可知$-|x| \leq x \leq |x|$。

\begin{zgraytheorem}
  \begin{zremark}
    因为$y$是负有理数,存在正整数$a,b$使得$y = (-a)/b$,现在证明$-y > y$。

    证明:

    由
    \begin{align*}
      (-y) - y & = a/b - [(-a)/b] \\
               & = a/b + a/b      \\
               & = (ab + ab)/bb   \\
    \end{align*}

    由于分子是正整数(命题2.2.8),分母是正整数(命题2.3.3),可知$(ab + ab)/bb$是正的,
    所以$-y > y$。
  \end{zremark}
\end{zgraytheorem}

\textbf{(d)(绝对值的可乘性)$|xy|=|x||y|$。特别地,$|-x|=|x|$}

证明:

由有理数的三歧性,证明过程可以按三种情况说明:

(1)$x,y$有一个是0或都是0,此时,$|xy|=0,|x||y|=0$,所以$|xy|=|x||y|$。

(2)$x,y$同号。如果$x,y$都是正有理数,存在正整数$a,b,c,d$使得$x=a/b,y=c/d$,此时:
\begin{align*}
  |xy| & = |(a/b) * (c/d)| \\
       & = |(ac)/(bd)|     \\
       & = ac/bd           \\
\end{align*}
又
\begin{align*}
  |x||y| & = |a/b||c/d|    \\
         & = (a/b) * (c/d) \\
         & = ac/bd         \\
\end{align*}
所以$|xy|=|x||y|$

如果$x,y$都是负有理数,证明类似。

(3)$x,y$是异号。
如果$x$是正有理数,$y$是负有理数,存在正整数$a,b,c,d$使得$x=a/b,y=(-c)/d$,
\begin{align*}
  |xy| & = |(a/b) * [(-c)/d]| \\
       & = |(-ac)/(bd)|       \\
       & = ad/bd              \\
\end{align*}
又
\begin{align*}
  |x||y| & = |a/b||(-c)/d| \\
         & = (a/b) * (c/d) \\
         & = ac/bd         \\
\end{align*}
所以$|xy|=|x||y|$。
如果$x$是负整数,$y$是正有理数,证明过程类似。

综上,(1)(2)(3)可知$|xy|=|x||y|$。

特别地,$-x=(-1)x$,所以
\begin{align*}
  |-x| & =|(-1)||x|                  \\
       & =1|x|                       \\
       & =|x|       & \text{命题4.2.4} \\
\end{align*}。

\textbf{(e)(距离的非退化性)$d(x,y) \geq 0$。另外,$d(x,y)=0$当且仅当$x=y$。 }

证明:

$d(x,y) = |x-y|$,由于$x-y$结果是有理数,由(a)可知$|x-y| \geq 0$,并且$|x-y|=0$当且仅当$x-y$等于零当且仅当$x=y$

\textbf{(f)(距离的对称性)d(x,y) = d(y,x)。}

证明:

不妨设$z = x - y$,由于$d(x,y)=|z|,d(y,x)=|-z|$,由(d)可知$|-z|=|z|$,所以$d(x,y)=d(y,x)$

\textbf{(g)(距离的三角不等式)$d(x,z) \leq d(x,y) + d(y,z)$。}

证明:

$d(x,z)=|x-z|,d(x,y)=|x-y|,d(y,z)=|y-z|$,由于$x-z = (x-y)+(y-z)$,
由命题(b)可知$|x-z| \leq |x-y| + |y-z|$,所以$d(x,z) \leq d(x,y) + d(y,z)$。

\section*{4.3.2}

\textbf{(a)如果$x=y$,那么对任意的$\varepsilon > 0$,$x$都是$\varepsilon$- 接近于$y$的。
  反过来,如果对于任意的$\varepsilon > 0$,$x$都是$\varepsilon$- 接近于$y$的,那么$x=y$。}

证明:

如果$x=y$,则:
\begin{align*}
  x-y & = y-y \\ &\text{有理数加法是定义明确的}
  x-y & = 0   \\
\end{align*}
由此可知$d(x,y)=0$,所以任意$\varepsilon > 0$总有$\varepsilon > d(x,y)$。

反过来,用反证法证明。不妨设$z=x-y$,由有理数的三歧性可知,$z$的取值有3种情况:

(1)$z$是正有理数,此时$d(x,y)=|x-y|=|z|=z$,
此时取$\varepsilon = (1/2) * z$,那么$d(x,y) > \varepsilon$,与前提矛盾,所以$z$不能是正有理数。

(2)$z$是负有理数,此时$d(x,y)=|x-y|=|z|=-z$,
此时也取$\varepsilon = (1/2) * z$,那么$d(x,y) > \varepsilon$,与前提矛盾,所以$z$不能是负有理数。

由(1)(2)可知$z$只能是零,所以$z=x-y=0 \Rightarrow x=y$。

\textbf{(b)设$\varepsilon > 0$,如果$x$是$\varepsilon$-接近于$y$的,那么$y$也是$\varepsilon$-接近于$x$的。}

证明:

由于$x$是$\varepsilon$-接近于$y$的,所以$d(x,y) \leq \varepsilon$。
由命题4.3.3(f)可知$d(x,y)=d(y,x)$,所以$d(y,x) \leq \varepsilon$,所以$y$也是$\varepsilon$-接近于$x$的。

\textbf{(c)设$\varepsilon,\delta>0$,如果$x$是$\varepsilon$- 接近于$y$的,
  并且$y$是$\delta$- 接近于$z$的,那么$x$和$z$是$(\varepsilon+\delta)$- 接近的。}

证明:

由4.3.3(g)可知$d(x,z) \leq d(x,y) + d(y,z)$,所以$d(x,z) \leq \varepsilon + \delta$,
那么$x$和$z$是$(\varepsilon+\delta)$- 接近的

\textbf{(d)设$\varepsilon,\delta>0$,如果$x$和$y$是$\varepsilon$- 接近的,并且$z$和$w$是$\delta$- 接近的,
  那么$x+z$和$y+w$是$(\varepsilon+\delta)$-接近的,并且$x-z$和$y-w$也是$(\varepsilon+\delta)$- 接近的。}

证明:

记$a:=y-x$,那么$y=x+a$且$|a| \leq \varepsilon$。
类似地,定义$b:=w-z$,那么$w=z+b$且$|b| \leq \delta$。

因为$y=x+a,w=z+b$,所以$d(x+z,y+w)=d(x+z,x+z+a+b)=|a+b|$,
由4.3.3(b)可知$|a+b| \leq |a|+b$,即$d(x+z,y+w) \leq \varepsilon + \delta$,
那么$x+z$和$y+w$是$(\varepsilon+\delta)$-接近的;

因为$y=x+a,w=z+b$,所以$d(x-z,y-w)=d(x-z,x-z+a-b)=|a-b|=|a+(-b)|$,
由4.3.3(b)(d)可知$|a+(-b)| \leq |a| + |b|$,即$d(x-z,y-w) \leq \varepsilon + \delta$,
那么$x-z$和$y-w$也是$(\varepsilon+\delta)$- 接近的

\textbf{(e)设$\varepsilon>0$,如果$x$和$y$是$\varepsilon$- 接近的,
  那么对任意的$\varepsilon^\prime > \varepsilon$,$x$和$y$也是$\varepsilon^\prime$- 接近的。}

证明:

由题设可知$d(x,y) \leq \varepsilon$,又$\varepsilon < \varepsilon^\prime$,
由命题4.2.9(c)(序是可传递的)可知$d(x,y) \leq \varepsilon^\prime$,
那么$x$和$y$也是$\varepsilon^\prime$- 接近的。

\textbf{(f)设$\varepsilon>0$,如果$y$和$z$都是$\varepsilon$- 接近于$x$的,
  并且$w$位于$y$和$z$之间(即$y \leq w \leq z$或$z \leq w \leq y$),
  那么$w$也是$\varepsilon$- 接近于$x$的。}

证明:

情况1:
$w=x$、$w=y$和$w=z$时,显然$w$是$\varepsilon$- 接近于$x$的。

情况2:
$w \neq x, w \neq y, w \neq z$时,
当$y < w < x$时,可知:
\begin{align*}
  d(y,x) & = d(y,w) + d(w,x) \\
\end{align*}
由于命题4.3.3(e)可知$d(y,w) \geq 0$,所以$d(w,x) \leq \varepsilon$,否则与题设矛盾。
当$x<w<z$、$z<w<x$和$x<w<y$证明类似。

综上,命题成立。【感觉证明有点麻烦,没想到好的思路】

\textbf{(g)设$\varepsilon>0$,如果$x$和$y$是$\varepsilon$- 接近的,并且$z$不为零,
  那么$xz$和$yz$是$\varepsilon|z|$- 接近的。}

证明:

记$a:=y-x$,那么$y=x+a$且$|a| \leq \varepsilon$。

因为$y=x+a$,所以,
\begin{align*}
  yz & = (x+a)z = xz + az \\
\end{align*}
于是,
\begin{align*}
  |yz - xz| = |xz + az - xz| = |az| = |a||z|
\end{align*}
又因为$|a| \leq \varepsilon$,所以,
\begin{align*}
  |yz - xz| \leq \varepsilon|z|
\end{align*}
从而$xz$和$yz$是$\varepsilon|z|$- 接近的。

\section*{4.3.3}

\textbf{(a)我们有$x^n x^m = x^{n+m}, (x^n)^m = x^{nm}, (xy)^n = x^n y^n$。}

证明:

(1)$x^n x^m = x^{n+m}$

对$m$进行归纳。当$m=0$时,
\begin{align*}
  x^n x^0 & = x^n * 1 \\
          & = x^n     \\
\end{align*}
又因为,
\begin{align*}
  x^{n+m} & = x^{n+0} \\
          & = x^n
\end{align*}
所以当$m=0$是命题成立。

归纳假设$m=k$时,$x^n x^k = x^{n+k}$。

现在只需证明$m=k++$时,命题成立。由定义4.3.9可知,
\begin{align*}
  x^n x^{k+1} & = x^n (x^k \times x^1) \\
\end{align*}
又由命题4.2.4(有理数的代数定律)可知,
\begin{align*}
  x^n x^{k+1} & = x^n (x^k \times x^1) \\
              & = (x^n x^k) \times x^1 \\
              & = x^{n+k} \times x^1   \\
              & = x^{n+k+1}
\end{align*}
综上,命题成立。

(2)$(x^n)^m = x^{nm}$

对$m$进行归纳。当$m=0时$,由定义4.3.9可知,
\begin{align*}
  (x^n)^m & = (x^n)^0 \\
          & = 1
\end{align*}
又
\begin{align*}
  x^{nm} & = x^{n \times 0} \\
         & = x^0            \\
         & = 1
\end{align*}
所以当$m=0$是命题成立。

归纳假设$m=k$时,$(x^n)^k = x^{nk}$。

现在只需证明$m=k++$时,命题成立。
\begin{align*}
  
\end{align*}

\end{document}