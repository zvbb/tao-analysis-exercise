\documentclass{article}
\usepackage{mathtools} 
\usepackage{fontspec}
\usepackage[UTF8]{ctex}
\usepackage{amsthm}
\usepackage{mdframed}
\usepackage{xcolor}
\usepackage{amssymb}
\usepackage{amsmath}

\newmdtheoremenv[
  backgroundcolor=gray!10,
  linewidth=0pt,
  innerleftmargin=10pt,
  innerrightmargin=10pt,
  innertopmargin=10pt,
  innerbottommargin=10pt
]{zgraytheorem}{}
% 定义说明环境样式
\newtheoremstyle{mystyle}% 说明环境样式的名称
  {1em}% 上方间距
  {1em}% 下方间距
  {\normalfont}% 说明内容的字体样式
  {}% 缩进量
  {\bfseries}% 说明标记的字体样式
  {.}% 说明标记和说明内容之间的标点
  {1em}% 说明标记后的水平空间
  {}% 说明标记后的垂直空间
% 使用新定义的样式创建说明环境
\theoremstyle{mystyle}
\newtheorem*{zremark}{说明}


\begin{document}
\title{4.2 习题}
\maketitle

\begin{zremark}
  \begin{zgraytheorem}
    在本节的证明过程中,用到一些在4.1节未提到的整数代数定律,但都很明显,
    我就没有特别说明证明了
  \end{zgraytheorem}
\end{zremark}

\section*{4.2.1}

证明:

设$x=a//b,y=c//d,z=e//f$为有理数,其中$a,c,e$是整数,$b,d,f$是不为零的整数。

(1)自反性

$ab=ab$,由定义4.2.1(有理数相等的定义)可知$x=x$

(2)对称性

假设$x=y$,由定义4.2.1(有理数相等的定义)可知$ad=bc$,再次利用定义4.2.1(有理数相等的定义)可知$y=x$

(3)传递性

假设$x=y,y=z$,由定义4.2.1(有理数相等的定义)可知$ad=bc,cf=de$,又
\begin{align*}
  ad  & = bc  \\
  adf & = bcf \\
\end{align*}
\begin{align*}
  cf  & = de  \\
  bcf & = bde \\
\end{align*}
所以:$adf = bcf = bde, adf = bde$,由推论4.1.9可知$af=be$,所以$x=z$

\begin{zgraytheorem}
  \begin{zremark}
    其实这里需要引入一个额外的命题,$a=b$,$a,b,c$都是整数,那么$ac=bc$。
    这个命题相对简单,这里说一下证明思路,先证明自然数符合该命题,然后再推广到整数。
  \end{zremark}
\end{zgraytheorem}

\section*{4.2.2}

证明:

(1)乘积的定义是明确的

假设$a // b = a^\prime // b^\prime$,那么$a b^\prime = a^\prime b$,
\begin{align}
  (a//b)*(c//d)                      & =(ac)//(bd)                    \\
  (a^\prime // b^\prime) // (c // d) & = (a^\prime c) // (b^\prime d)
\end{align}
因此我们要证明的是$acb^\prime d = bda^\prime c$,由推论4.1.9(整数的消去律)可知,只需证明
$ab^\prime = ba^\prime$,由假设可知该等式成立;

(2)负数的定义是明确的

假设$a // b = a^\prime // b^\prime$,那么$a b^\prime = a^\prime b$,
\begin{align}
  -(a//b)                 & = (-a)//b               \\
  -(a^\prime // b^\prime) & = (-a^\prime)//b^\prime
\end{align}
因此我们要证明的是$(-a)b^\prime = (-a^\prime)b $,由习题4.1.3可知
\begin{align*}
  (-a)b^\prime & =(-1)\times ab^\prime  \\
  (-a^\prime)b & =(-1)\times a^\prime b \\
\end{align*}
由推论4.1.9(整数的消去律)可知,只需证明
$ab^\prime = a^\prime b$,由假设可知该等式成立;


\section*{4.2.3}

证明:

我们记$x=a//b,y=c//d,z=e//f$,其中$a,c,e$是整数,$b,d,f$是不为零的整数。

(1)$x + y = y + x$

\begin{align*}
  x + y & = (a//b) + (c//d)   \\
        & = (ad + bc) // (bd)
\end{align*}
又
\begin{align*}
  y + x & = (c//d) + (a//b)   \\
        & = (cb + da) // (db)
\end{align*}
由定义4.2.1(有理数的相等定义)可知要证明$x+y=y+x$,只需证明
$(ad + bc)(db) = (cb + da) (bd)$,利用整数的代数定律可知,等式成立。

(2)$x + 0 = 0 + x = x$

\begin{align*}
  x + 0 & = (a//b) + (0//1)  \\
        & = (a1 + b0) // (b) \\
        & = a // b           \\
        & = x
\end{align*}
又
\begin{align*}
  0 + x & = (0//1) + (a//b)  \\
        & = (0b + 1a) // (b) \\
        & = a // b           \\
        & = x
\end{align*}

(3)$x + (-x) = (-x) + (x) = 0$

由(1)可知$x + (-x) = (-x) + (x)$

\begin{align*}
  x + (-x) & = (a//b) + [(-a)//b]         \\
           & = [ab + (-a)b] // bb         \\
           & = [ab + (-1) \times ab]// bb \\
           & = [ab(1+(-1))] // bb         \\
           & = [ab(0)] // bb              \\
           & = 0 // bb                    \\
           & = 0
\end{align*}

(4)$xy = yx$

\begin{align*}
  xy & = (a//b) * (c//d) \\
     & = (ac) // (bd)    \\
\end{align*}
又
\begin{align*}
  yx & = (c//d) * (a//b) \\
     & = (ca) // (db)    \\
\end{align*}
现只需证明$acdb = bdca$,由整数的代数定律可知,等式成立。

(5)$(xy)z = x(yz)$

\begin{align*}
  (xy)z & = [(a//b) * (c//d)] * (e//f) \\
        & = (ac//bd) * (e//f)          \\
        & = ace//bdf                   \\
\end{align*}
又
\begin{align*}
  x(yz) & = a//b * [(c//d)*(e//f)] \\
        & = a//b * (ce//df)        \\
        & = ace//bdf               \\
\end{align*}

(6)$x1 = 1x = x$

由(4)可知$x1 = 1x$

\begin{align*}
  x1 & = (a//b) * (1//1) \\
     & = (a1//b)         \\
     & = a//b            \\
     & = x               \\
\end{align*}

(7)$x(y+z) = xy + xz$

\begin{align*}
  x(y+z) & = (a // b)[c // d + e // f] \\
         & = (a // b)[(cf + de) // df] \\
         & = [a(cf + de)] // bdf       \\
         & = (acf + ade) // bdf        \\
\end{align*}
又
\begin{align*}
  xy + xz & = (a//b) * (c//d) + (a//b) * (e//f) \\
          & = (ac // bd) + (ae // bf)           \\
          & = (acbf + bdae) // bdbf             \\
\end{align*}

由有理数相等的定义可知,我们此时只需证明
$(acf + ade)bdbf = bdf(acbf + bdae)$。
\begin{align*}
  (acf + ade)bdbf  & = acfbdbf + adebdbf \\
                   & = abbcdff + abbddef \\
  bdf(acbf + bdae) & = bdfacbf + bdfbdae \\
                   & = abbcdff + abbddef \\
\end{align*}
所以$(acf + ade)bdbf = bdf(acbf + bdae)$。

(8)$(y+z)x = yx + zx$

由(4)可知
\begin{align*}
  (y+z)x & = x(y+z) \\
  xy     & = yx     \\
  xz     & = zx     \\
\end{align*}
综上,可知$(y+z)x=x(y+z)=xy+xz=yx+zx$,命题得证;

\section*{4.2.4}

\begin{zremark}
  \begin{zgraytheorem}
    由有理数的定义可知,有理数是由整数构造而成,通过整数的三歧性,可以推导出有理数的三歧性。
  \end{zgraytheorem}
\end{zremark}

设$x$为任意有理数,由有理数的定义可知,存在$x=a/b$,其中$a,b$是整数。

首先证明(a)(b)(c)中至少有一个为真。
我们有如下三种情况:$a$、$b$同号,$a$、$b$异号,$a$为0。

如果$a,b$同号,也就是说,要么$a,b$都是正整数,要么$a,b$都是负整数。
如果$a,b$都是正整数,由定义4.2.6可知$x$是正的;
如果$a,b$都是负整数,不妨设$a=-a^\prime, b=-b^\prime$,其中$a^\prime,b^\prime$是正整数,
此时,
\begin{align*}
  x & = a/b                       \\
    & = (-a^\prime) / (-b^\prime) \\
    & = a^\prime / b^\prime       \\
\end{align*}
所以$x$是正的。

如果$a,b$是异号,要么$a$是正整数$b$负整数,或反之。
如果$a$是正整数$b$负整数,不妨设$b=-b^\prime$,其中$b^\prime$是正整数,
此时,
\begin{align*}
  x & = a/b           \\
    & = a/(-b^\prime) \\
    & = -(a/b^\prime) \\
\end{align*}
所以$x$是负的;
同理可证$a$是负整数$b$正整数时,$x$是负的。

如果$a$为0,则$x=a/b=0/1=0$。

所以(a)(b)(c)中至少有一个为真

现在我们证明(a)、(b)、(c)最多有一个为真。

(a)、(b)不能同时为真,如果(a)、(b)同时为真,
当$x=a/b$是正的,则$a$是正整数,
当$x=a/b=0$,则$a=0$,
这与正整数的定义矛盾。

(a)、(c)不能同时为真,如果(a)、(b)同时为真,

\end{document}