\documentclass{article}
\usepackage{mathtools} 
\usepackage{fontspec}
\usepackage[UTF8]{ctex}
\usepackage{amsthm}
\usepackage{mdframed}
\usepackage{xcolor}
\usepackage{amssymb}
\usepackage{amsmath}

\newmdtheoremenv[
  backgroundcolor=gray!10,
  linewidth=0pt,
  innerleftmargin=10pt,
  innerrightmargin=10pt,
  innertopmargin=10pt,
  innerbottommargin=10pt
]{zgraytheorem}{}
% 定义说明环境样式
\newtheoremstyle{mystyle}% 说明环境样式的名称
  {1em}% 上方间距
  {1em}% 下方间距
  {\normalfont}% 说明内容的字体样式
  {}% 缩进量
  {\bfseries}% 说明标记的字体样式
  {.}% 说明标记和说明内容之间的标点
  {1em}% 说明标记后的水平空间
  {}% 说明标记后的垂直空间
% 使用新定义的样式创建说明环境
\theoremstyle{mystyle}
\newtheorem*{zremark}{说明}


\begin{document}
\title{4.2 习题}
\maketitle

\section*{4.2.1}

证明:

设$x=a//b,y=c//d,z=e//f$为有理数,其中$a,c,e$是整数,$b,d,f$是不为零的整数。

(1)自反性

$ab=ab$,由定义4.2.1(有理数相等的定义)可知$x=x$

(2)对称性

假设$x=y$,由定义4.2.1(有理数相等的定义)可知$ad=bc$,再次利用定义4.2.1(有理数相等的定义)可知$y=x$

(3)传递性

假设$x=y,y=z$,由定义4.2.1(有理数相等的定义)可知$ad=bc,cf=de$,又
\begin{align*}
  ad  & = bc  \\
  adf & = bcf \\
\end{align*}
\begin{align*}
  cf  & = de  \\
  bcf & = bde \\
\end{align*}
所以:$adf = bcf = bde, adf = bde$,由推论4.1.9可知$af=be$,所以$x=z$

\begin{zgraytheorem}
  \begin{zremark}
    其实这里需要引入一个额外的命题,$a=b$,$a,b,c$都是整数,那么$ac=bc$。
    这个命题相对简单,这里说一下证明思路,先证明自然数符合该命题,然后再推广到整数。
  \end{zremark}
\end{zgraytheorem}

\section*{4.2.2}


\end{document}