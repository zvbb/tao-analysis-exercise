\documentclass{article}
\usepackage{mathtools} 
\usepackage{fontspec}
\usepackage[UTF8]{ctex}
\usepackage{amsthm}
\usepackage{mdframed}
\usepackage{xcolor}
\usepackage{amssymb}
\usepackage{amsmath}

\newmdtheoremenv[
  backgroundcolor=gray!10,
  linewidth=0pt,
  innerleftmargin=10pt,
  innerrightmargin=10pt,
  innertopmargin=10pt,
  innerbottommargin=10pt
]{zgraytheorem}{}
% 定义说明环境样式
\newtheoremstyle{mystyle}% 说明环境样式的名称
  {1em}% 上方间距
  {1em}% 下方间距
  {\normalfont}% 说明内容的字体样式
  {}% 缩进量
  {\bfseries}% 说明标记的字体样式
  {.}% 说明标记和说明内容之间的标点
  {1em}% 说明标记后的水平空间
  {}% 说明标记后的垂直空间
% 使用新定义的样式创建说明环境
\theoremstyle{mystyle}
\newtheorem*{zremark}{说明}

% 定义证明环境样式
\newtheoremstyle{zproofstyle}
  {0.5em}
  {0.5em}
  {\itshape}
  {}
  {\bfseries}
  {.}
  {\newline}
  {}
\theoremstyle{zproofstyle}
\newtheorem*{zproof}{证明}
\newcommand{\zsub}{\mathbin{\rule{1em}{0.5em}}}
\begin{document}
\title{4.1 习题}
\maketitle

\section*{4.1.1}
\begin{zproof}
  \textcircled{1} 自反性

  设$a \zsub b$是任意整数,现证明$a \zsub b$ = $a \zsub b$。
  由于$a+b=a+b$,所以$a \zsub b$ = $a \zsub b$

  \textcircled{2} 对称性

  设$a \zsub b = c \zsub d$,现证明$c \zsub d=a \zsub b$。
  由于$a \zsub b = c \zsub d$,所以$a+d=c+b$,
  由自然数相等的对称性可知$c+b=a+d$,所以$c \zsub d=a \zsub b$。
\end{zproof}

\section*{4.1.2}
\begin{zproof}
  $-(a \zsub b)=b \zsub a,-(a^\prime \zsub b^\prime)=b^\prime \zsub a^\prime$,
  又$(a \zsub b)=(a^\prime \zsub b^\prime)$则$a+b^\prime = a^\prime + b$,
  由于加法是可以交换的(命题2.2.4)所以$b^\prime + a = b + a^\prime$,
  由此可得$-(a^\prime \zsub b^\prime)=-(a \zsub b)$,
  又由整数相等的对称性可得$-(a \zsub b)=-(a^\prime \zsub b^\prime)$。
\end{zproof}

\section*{4.1.3}
\begin{zproof}
  因为a是整数,不妨设$a = x \zsub y$,其中$x, y$是自然数,则
  \begin{align*}
     & (-1) \times a                                               \\
     & =(0 \zsub 1) \times (x \zsub y)                             \\
     & = (0 \times x + 1 \times y) \zsub (0 \times y + 1 \times x) \\
     & = (0 + y) \zsub (0 + x)                                     \\
     & = y \zsub x                                                 \\
     & = -a
  \end{align*}
\end{zproof}

\section*{4.1.4}
注意: 此时书中已经说明了$\zsub$与$-$的等价性,从此处开始证明中将不使用$\zsub$

记$x = a - b, y = c - d, z = e - f$其中a、b、c、d、e、f是自然数

\textcircled{1} $x + y = y + x$

\begin{zproof}
  \begin{align*}
     & x + y           \\
     & = (a-b)+(c-d)   \\
     & = (a+c) - (b+d) \\
     & y + x           \\
     & =(c-d)+(a-b)    \\
     & =(c+a) - (d+b)  \\
  \end{align*}
  由于加法是可交换(命题2.2.4)可知$a+c=c+a,b+d=d+b$,又由自然数相等的替换公理
  可得$(a+c) + (d+b) = (c+a) + (b+d)$,由此可知$x + y = y + x$
\end{zproof}

\textcircled{2} $(x+y)+z=x+(y+z)$

\begin{zproof}
  \begin{align*}
     & (x+y)+z               \\
     & = [(a-b)+(c-d)]+(e-f) \\
     & = [(a+c)-(b+d)]+(e-f) \\
     & = (a+c+e) - (b+d+f)   \\
     & x+(y+z)               \\
     & = (a-b)+[(c-d)+(e-f)] \\
     & = (a-b)+[(c+e)-(d+f)] \\
     & = (a+c+e) - (b+d+f)   \\
  \end{align*}
  于是$(x+y)+z=x+(y+z)$

\end{zproof}

\textcircled{3} $x+0=0+x=x$

\begin{zproof}
  可以把$0$看做整数$0-0$,由\textcircled{1}可知$x+0=0+x$,
  \begin{align*}
     & 0+x           \\
     & = (0-0)+(a-b) \\
     & = (0+a)-(0+b) \\
     & = a-b         \\
     & = x
  \end{align*}
\end{zproof}

\textcircled{4} $x+(-x)=(-x)+x=0$

\begin{zproof}
  由\textcircled{1}可知$x+(-x)=(-x)+x$,可以把$0$看做整数$0-0$,
  现在证明整数$x+(-x)=0-0$
  \begin{align*}
              & x + (-x)        \\
              & = (a-b) + (b-a) \\
              & = (a+b) - (b+a) \\
    (a+b) + 0 & = (b+a) + 0     \\
    a + b     & = b + a
  \end{align*}
  于是$x+(-x)=(-x)+x=0$
\end{zproof}

\textcircled{5} $xy=yx$

\begin{zproof}
  \begin{align*}
     & xy                  \\
     & = (a-b)(c-d)        \\
     & = (ac+bd) - (ad+bc) \\
     & yx                  \\
     & = (c-d)(a-b)        \\
     & = (ca+db) - (cb+da) \\
  \end{align*}
  由于加法是可以交换的,乘法也是可以交换的,所以
  \begin{align*}
     & = (ca+db) - (cb+da) \\
     & = (ac+bd) - (ad+bc) \\
  \end{align*}
  于是$xy = yx$
\end{zproof}

\textcircled{7} $x1=1x=x$

\begin{zproof}
  由\textcircled{5} 可知$x1=1x$,又
  \begin{align*}
     & x1                                                      \\
     & = (a-b) \times (1-0)                                    \\
     & = (a \times 1 + b \times 0) - (a \times 0 + b \times 1) \\
     & = (a + 0) - (0 + b)                                     \\
     & = a - b                                                 \\
     & = x                                                     \\
  \end{align*}
\end{zproof}

\textcircled{8} $x(y + z) = xy + xz$

\begin{zproof}
  \begin{align*}
     & x(y + z)                                  \\
     & = (a-b)[(c-d)+(e-f)]                      \\
     & = (a-b)[(c+e)-(d+f)]                      \\
     & = [a(c+e)+b(d+f)] - [a(d+f)+b(c+e)]       \\
     & = (ac+ae+bd+bf) - (ad+af+bc+be)           \\
     & xy + xz                                   \\
     & = (a-b)(c-d) + (a-b)(e-f)                 \\
     & = [(ac + bd)-(ad+bc)] + [(ae+bf)-(af+be)] \\
     & = [(ac + bd)+(ae+bf)] - [(ad+bc)+(af+be)] \\
     & = (ac+ae+bd+bf) - (ad+bc+af+be)           \\
  \end{align*}
  于是$x(y + z) = xy + xz$
\end{zproof}

\textcircled{9} $(y+z)x = yx+zx$

\begin{zproof}
  由\textcircled{5}可知$(y+z)x=x(y+z)$,
  又由\textcircled{8}可知$x(y+z)=xy+xz$,
  再次应用\textcircled{5}可得$xy+xz=yx+zx$,
  于是等式成立
\end{zproof}

\section*{4.1.5}

\begin{zproof}
  由引理4.1.5(整数的三歧性)分多种情况讨论。

  (1)如果$a,b$都是正自然数,
  则由2.3.3可知$ab$是正自然数,则$ab \neq = 0$与题设矛盾;

  (2)如果$a,b$都是正自然数的负数,假设分别为$-m,-n$,m、n都是正自然数。
  \begin{align*}
     & ab                                              \\
     & = (-m) \times (-n)                              \\
     & = (0-m) \times (0-n)                            \\
     & = (0 \times 0 + mn) - (0 \times 0 + m \times 0) \\
     & = mn                                            \\
  \end{align*}
  由于m,n都是正自然数,所以$ab=mn \neq 0$,与题设矛盾;

  (3)如果$a=b=0$,$ab=0$,满足题设。

  (4)如果$a=0,b=x-y$,x、y为任意自然数;
  \begin{align*}
     & ab                            \\
     & = (0-0) \times (x-y)          \\
     & = (0 \times x) - (0 \times y) \\
     & = 0 - 0                       \\
     & = 0                           \\
  \end{align*}
  所以$ab=0$,满足题设;

  (5)如果$a=x-y,b=0$,x、y为任意自然数;
  \begin{align*}
     & ab                            \\
     & = (x-y) \times (0-0)          \\
     & = (x \times 0) - (y \times 0) \\
     & = 0 - 0                       \\
     & = 0                           \\
  \end{align*}
  于是$ab=0$,满足题设;

  综上,命题得证。
\end{zproof}

\section*{4.1.6}

\begin{zproof}
  (1)方法一

  由整数加法的替换性与命题4.1.6可知:
  \begin{align*}
    ac - bc & = 0 \\
    (a-b) c & = 0 \\
  \end{align*}
  由命题4.1.8可知$a-b=0$,接下来要证$a=b$,以上等式才能成立。
  \begin{align*}
    a - b           & = 0     \\
    a - b + b       & = 0 + b \\
    a + (- b) + b   & = 0 + b \\
    a + [(- b) + b] & = b     \\
    a + 0           & = b     \\
    a               & = b     \\
  \end{align*}

  \begin{zgraytheorem}
    \begin{zremark}
      上面的证明中用到了一个命题:a,b是整数且$a=b$,则$a+c=b+c$,c是整数。

      该命题对自然数是成立的,但对于整数书中没有该命题,这里需要证明下。

      记$a=x-y,b=p-q,c=w-z$,$x$、$y$、$p$、$q$、$w$、$z$是自然数。
      由$a=b$可得:
      \begin{align*}
        a     & = b     \\
        x - y & = p - q \\
        x + q & = p + y \\
      \end{align*}
      又
      \begin{align*}
         & a + c               \\
         & = (x - y) + (w - z) \\
         & = (x + w) - (y + z) \\
         & b + c               \\
         & = (p - q) + (w - z) \\
         & = (p + w) - (q + z) \\
      \end{align*}
      又由
      \begin{align*}
         & (x + w) + (q + z) \\
         & = x + q + w + z   \\
         & (p + w) + (y + z) \\
         & = p + y + w + z   \\
      \end{align*}
      结合$x + q = p + y$于是$a+c=b+c$
    \end{zremark}
  \end{zgraytheorem}

  (2)方法二

  由引理4.1.5(整数的三歧性)分多种情况讨论。

  (1)a、b、c都是正自然数,则
  \begin{align*}
    ac = bc
  \end{align*}
  由推论2.3.7可知$a=b$

  其余的情况证明类似。(略)

\end{zproof}

\section*{4.1.7}

(a)$a > b$当且仅当$a-b$是一个正的自然数。
\begin{zproof}
  \textcircled{1} 充分性

  假设$a>b$,由引理4.1.5(整数的三歧性)对$a-b$分多种情况讨论。

  (1)如果$a-b=0$,则$a=b$,与题设矛盾;

  (2)$a-b$是正自然数n的负数-n,即
  \begin{align*}
    a - b     & = -n         \\
    a - b + b & = -n + b     \\
    a + 0     & = -n + b     \\
    a         & = -n + b     \\
    a + n     & = -n + b + n \\
    a + n     & = b + -n + n \\
    a + n     & = b          \\
    b         & > a          \\
  \end{align*}
  与题设矛盾;

  综上,$a-b$只能是正自然数

  \textcircled{2} 必要性

  假设$a-b$是一个正的自然数n,那么
  \begin{align*}
    a - b     & = n     \\
    a - b + b & = n + b \\
    a + 0     & = n + b \\
    a         & = n + b \\
  \end{align*}
  于是$a \geq b$,又由于$n$是正自然数,那么$a \neq b$(其实这里需要引入一个额外的命题,下方有说明),所以$a > b$

  \begin{zgraytheorem}
    \begin{zremark}
      整数$a,b$,如果$a \geq b$,c是正自然数,那么$a + c > b$

      \begin{zproof}
        不妨设$a=x-y,b=p-q$
        因为$a \geq b$,那么存在一个自然数n使得$a = b + n$,
        所以$a + c = b + n + c = b + (n + c)$,于是$a + c \geq b$,

        若$a+c=b$,则
        \begin{align*}
          a + c            & = b        \\
          b + n + c        & = b        \\
          b + n + c + (-b) & = b + (-b) \\
          n + c            & = 0        \\
        \end{align*}
        由推论2.2.9可知$c=0$,这与$c$是正自然数矛盾,所以$a+c \neq b$,所以$a + c > b$,命题得证。
      \end{zproof}
    \end{zremark}
  \end{zgraytheorem}

  综上,命题得证。
\end{zproof}

(b)(加法保持序不变)如果$a > b$,那么$a + c > b + c$

\begin{zproof}
  因为$a>b$,由(a)可知$a-b=n$,n是正自然数;
  \begin{align*}
     & (a+c) - (b+c)        \\
     & =(a+c) + [-(b+c)]    \\
     & =(a+c) + [(-c)+(-b)] \\
     & =a + c + (-c) + (-b) \\
     & =a + 0 + (-b)        \\
     & =a - b
  \end{align*}
  由此可知$(a+c) - (b+c)=a - b$是正自然数,所以$a + c > b + c$。
\end{zproof}

\begin{zgraytheorem}
  \begin{zremark}
    以上的证明中$-(b+c)=(-b)+(-c)$,不是显然的,需要证明以下命题。

    $a,b$是整数,则$-(a+b)=(-a)+(-b)$。
    \begin{zproof}
      由于a,b是整数,所以存在$a=x-y,b=p-q$,$x,y,p,q$是自然数。
      \begin{align*}
         & -(a+b)                \\
         & =-[(x-y)+(p-q)]       \\
         & =-[(x+p)-(y+q)]       \\
         & =(y+p)-(x+p)          \\
         & (-a)+(-b)             \\
         & = [-(x-y)] + [-(p-q)] \\
         & = (y-x) + (q-p)       \\
         & = (y+q) - (x+p)       \\
      \end{align*}
      于是$-(a+b)=(-a)+(-b)$,命题得证。
    \end{zproof}
  \end{zremark}
\end{zgraytheorem}

(c)(正的乘法保持序不变)如果$a>b$并且c是正的,那么$ac > bc$。

证明:

因为$a>b$所以存在正自然数$x$使得$a=b+x$,此时
\begin{align*}
  ac & = (b+x) \times c          \\
     & = b \times c + x \times c \\
\end{align*}
又
\begin{align*}
  ac - bc & = b \times c + x \times c - b \times c \\
          & = x \times c                           \\
\end{align*}
由于$x,c$都是正的自然数,所以$ac-bc = x\times c > 0$,通过(a)可知$ac > bc$

(d)(负运算反序)如果$a > b$,那么$-a < -b$。

证明:

不妨设$b = p - q$,$p$、$q$是自然数。由题设$a > b$可知存在正自然数c,使得$a = b + c$,
\begin{align*}
  -b - (-a) & = -b - (b + c)        \\
            & = q-p - [(p - q) + c] \\
            & = q-p - (p + c - q)   \\
            & = q-p - [q - (p + c)] \\
            & = q-p - q + p + c     \\
            & = c                   \\
\end{align*}
(note: 上面把$p$,$q$,$c$既看做自然数也看做整数,这样自然数与整数的代数定律都可以使用)
由于$c > 0$,所以$-a < -b$

(e)(序是可以传递的) 如果$a > b$且$b > c$ ,那么 $a > c$。

证明:



\end{document}