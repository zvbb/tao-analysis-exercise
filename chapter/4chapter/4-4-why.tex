\documentclass{article}
\usepackage{mathtools} 
\usepackage{fontspec}
\usepackage[UTF8]{ctex}
\usepackage{amsthm}
\usepackage{mdframed}
\usepackage{xcolor}
\usepackage{amssymb}
\usepackage{amsmath}

\newmdtheoremenv[
  backgroundcolor=gray!10,
  linewidth=0pt,
  innerleftmargin=10pt,
  innerrightmargin=10pt,
  innertopmargin=10pt,
  innerbottommargin=10pt
]{zgraytheorem}{}
% 定义说明环境样式
\newtheoremstyle{mystyle}% 说明环境样式的名称
  {1em}% 上方间距
  {1em}% 下方间距
  {\normalfont}% 说明内容的字体样式
  {}% 缩进量
  {\bfseries}% 说明标记的字体样式
  {.}% 说明标记和说明内容之间的标点
  {1em}% 说明标记后的水平空间
  {}% 说明标记后的垂直空间
% 使用新定义的样式创建说明环境
\theoremstyle{mystyle}
\newtheorem*{zremark}{说明}


\begin{document}
\title{4.4 文中的为什么}
\maketitle

\textbf{1.有理数$\epsilon > 0$,对于每个自然数$n$,$n\epsilon$都是非负的。}

证明:

对$n$进行归纳。

(1)$n=0$,此时$n\epsilon = 0$,是非负的。

(2)归纳假设$n=k$时,$k\epsilon$是非负的。

(3)$n=k++$时,$(k++)\epsilon=k\epsilon + \epsilon$,所以,
\begin{align*}
  (k++)\epsilon - k\epsilon & = \epsilon > 0     \\
  \Rightarrow (k++)\epsilon & > k\epsilon \geq 0 \\
  \Rightarrow (k++)\epsilon & > 0
\end{align*}
于是$(k++)\epsilon$是正的。

综上,归纳完成。

\end{document}