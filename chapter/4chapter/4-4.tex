\documentclass{article}
\usepackage{mathtools} 
\usepackage{fontspec}
\usepackage[UTF8]{ctex}
\usepackage{amsthm}
\usepackage{mdframed}
\usepackage{xcolor}
\usepackage{amssymb}
\usepackage{amsmath}

\newmdtheoremenv[
  backgroundcolor=gray!10,
  linewidth=0pt,
  innerleftmargin=10pt,
  innerrightmargin=10pt,
  innertopmargin=10pt,
  innerbottommargin=10pt
]{zgraytheorem}{}
% 定义说明环境样式
\newtheoremstyle{mystyle}% 说明环境样式的名称
  {1em}% 上方间距
  {1em}% 下方间距
  {\normalfont}% 说明内容的字体样式
  {}% 缩进量
  {\bfseries}% 说明标记的字体样式
  {.}% 说明标记和说明内容之间的标点
  {1em}% 说明标记后的水平空间
  {}% 说明标记后的垂直空间
% 使用新定义的样式创建说明环境
\theoremstyle{mystyle}
\newtheorem*{zremark}{说明}


\begin{document}
\title{4.4 习题}
\maketitle

\section*{4.4.1}
证明:

\textbf{1.证明$n$的存在性}

由有理数的三歧性分情况讨论。

(1)$x = 0$时,$n=0$满足命题$n \leq x < n+1$。

(2)$x$是正有理数时,存在正整数$a,b$使得$x = a / b$。

当$a<b$时,
因为$x$是正有理数,所以$x \geq 0$,又因为,
\begin{align*}
  1 - x & = 1 - a/b \\
        & = (b-a)/b \\
\end{align*}
由于$b>a$可知,$b-a>0$,由此可知$1-x$是正有理数,所以$1 > x$。
从而可取$n=0$。

当$a>b$时,由命题$2.3.9$可知,存在自然数$m,r$使得$a=mb+r$且$0 \leq r < b$。
因为$a=mb+r$,所以,
\begin{align*}
  a/b & = (mb + r) / b \\
      & = m + r/b      \\
\end{align*}
由于$0 \leq r/b < 1$,所以可取$n=m$,满足命题。

(3)$x$是负有理数时,存在正整数$a,b$使得$x = (-a) / b$。

当$a<b$时,取$n=-1$,证明过程与上面类似,不在赘述

当$a>b$时,取$n=-(m+1)$,证明过程与上面类似,不在赘述

\textbf{2.证明$n$的唯一性}

假设存在整数$n_1 \neq n_2$并且满足
\begin{align}
  n_1  \leq x & < n_1 + 1 \\
  n_2  \leq x & < n_2 + 1
\end{align}

由于$n_1 \neq n_2$,不妨假设$n_1 < n_2$,
所以存在正自然数$a \geq 1$使得$n_2 = n_1 + a$,
又由假设可知$n_2 \leq x < n_1 + 1$,因为$n_2 = n_1 + a$,所以
\begin{align*}
  n_1 + a  \leq x & < n_1 + 1 \\
\end{align*}
由$a \geq 1$可知,以上公式矛盾,所以$n_1 < n_2$不成立。

同理可知$n_1 > n_2$不成立。

综上$n_1 \neq n_2$时无法同时满足命题,至此$n$的唯一性得证。

\section*{4.4.2}

证明:

\textbf{a.不存在无穷递降的自然数列}

利用反证法。假设存在无穷递降的自然数列$a_0,a_1,a2,...$。证明无穷递降的自然数列具有性质$p$:
对任意的$k \in N$和任意的$n \in N$都有$a_n \geq k$,然后利用性质$p$得到矛盾,以此达到“不存在无穷递降的自然数列”的目的。

利用归纳法证明性质p:


\end{document}