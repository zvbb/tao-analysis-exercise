\documentclass{article}
\usepackage{mathtools} 
\usepackage{fontspec}
\usepackage[UTF8]{ctex}
\usepackage{amsthm}
\usepackage{mdframed}
\usepackage{xcolor}
\usepackage{amssymb}
\usepackage{amsmath}

\newmdtheoremenv[
  backgroundcolor=gray!10,
  linewidth=0pt,
  innerleftmargin=10pt,
  innerrightmargin=10pt,
  innertopmargin=10pt,
  innerbottommargin=10pt
]{zgraytheorem}{}
% 定义说明环境样式
\newtheoremstyle{mystyle}% 说明环境样式的名称
  {1em}% 上方间距
  {1em}% 下方间距
  {\normalfont}% 说明内容的字体样式
  {}% 缩进量
  {\bfseries}% 说明标记的字体样式
  {.}% 说明标记和说明内容之间的标点
  {1em}% 说明标记后的水平空间
  {}% 说明标记后的垂直空间
% 使用新定义的样式创建说明环境
\theoremstyle{mystyle}
\newtheorem*{zremark}{说明}


\begin{document}
\title{4.4 习题}
\maketitle

证明:

由有理数的三歧性分情况讨论。

(1)$x = 0$时,$n=0$满足命题$n \leq x < n+1$。

(2)$x$是正有理数时,存在正整数$a,b$使得$x = a / b$。

当$a<b$时,
因为$x$是正有理数,所以$x \geq 0$,又因为,
\begin{align*}
  1 - x & = 1 - a/b \\
        & = (b-a)/b \\
\end{align*}
由于$b>a$可知,$b-a>0$,由此可知$1-x$是正有理数,所以$1 > x$。
从而可取$n=0$。

当$a>b$时,由命题$2.3.9$可知,存在自然数$m,r$使得$a=mb+r$且$0 \leq r < b$。
因为$a=mb+r$,所以,
\begin{align*}
  a/b & = (mb + r) / b \\
      & = m + r/b      \\
\end{align*}
由于$0 \leq r/b < 1$,所以可取$n=m$,满足命题。

(3)$x$是负有理数时,存在正整数$a,b$使得$x = (-a) / b$。

当$a<b$时,取$n=-1$,证明过程与上面类似,不在赘述

当$a>b$时,取$n=m-1$,证明过程与上面类似,不在赘述


\end{document}