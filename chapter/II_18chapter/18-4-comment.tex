\documentclass{article}
\usepackage{mathtools} 
\usepackage{fontspec}
\usepackage[UTF8]{ctex}
\usepackage{amsthm}
\usepackage{mdframed}
\usepackage{xcolor}
\usepackage{amssymb}
\usepackage{amsmath}


% 定义新的带灰色背景的说明环境 zremark
\newmdtheoremenv[
  backgroundcolor=gray!10,
  % 边框与背景一致,边框线会消失
  linecolor=gray!10
]{zremark}{说明}

% 通用矩阵命令: \flexmatrix{矩阵名}{元素符号}{行数}{列数}
\newcommand{\flexmatrix}[4]{
  \[
  #1 = \begin{pmatrix}
    #2_{11}     & #2_{12}     & \cdots & #2_{1#4}   \\
    #2_{21}     & #2_{22}     & \cdots & #2_{2#4}   \\
    \vdots      & \vdots      & \ddots & \vdots     \\
    #2_{#31}    & #2_{#32}    & \cdots & #2_{#3#4}
  \end{pmatrix}
  \]
}

% 简化版命令(默认矩阵名为A,元素符号为a): \quickmatrix{行数}{列数}
\newcommand{\quickmatrix}[2]{\flexmatrix{A}{a}{#1}{#2}}


\begin{document}
\title{18.4 注释}
\author{张志聪}
\maketitle

\begin{zremark}
  陶哲轩定义勒贝格可测性时,需要$A \subseteq \mathbb{R}^n$有:
  $m^{\ast}(A) = m^{\ast}(A \cap E) + m^{\ast}(A \setminus E)$,但没有说$m^{\ast}(A)$一定存在啊?
  也就是说他默认$A$一定有开盒覆盖么,而且一定有相应的外测度。
\end{zremark}

\textbf{证明:}

是的,陶哲轩默认了外测量$m^{\ast}(A)$是对所有的$A \subseteq \mathbb{R}^n$定义良好的。

在书中18.2节的开头,他就有说明,原文:“外测度适用于每一个集合,
并且满足除了可加性(ix)和(xi)之外的所有性质(v)$\sim$(xiii)”。


\end{document}