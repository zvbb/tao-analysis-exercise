\documentclass{article}
\usepackage{mathtools} 
\usepackage{fontspec}
\usepackage[UTF8]{ctex}
\usepackage{amsthm}
\usepackage{mdframed}
\usepackage{xcolor}
\usepackage{amssymb}
\usepackage{amsmath}


% 定义新的带灰色背景的说明环境 zremark
\newmdtheoremenv[
  backgroundcolor=gray!10,
  % 边框与背景一致,边框线会消失
  linecolor=gray!10
]{zremark}{说明}

% 通用矩阵命令: \flexmatrix{矩阵名}{元素符号}{行数}{列数}
\newcommand{\flexmatrix}[4]{
  \[
  #1 = \begin{pmatrix}
    #2_{11}     & #2_{12}     & \cdots & #2_{1#4}   \\
    #2_{21}     & #2_{22}     & \cdots & #2_{2#4}   \\
    \vdots      & \vdots      & \ddots & \vdots     \\
    #2_{#31}    & #2_{#32}    & \cdots & #2_{#3#4}
  \end{pmatrix}
  \]
}

% 简化版命令(默认矩阵名为A,元素符号为a): \quickmatrix{行数}{列数}
\newcommand{\quickmatrix}[2]{\flexmatrix{A}{a}{#1}{#2}}


\begin{document}
\title{18.5 习题}
\author{张志聪}
\maketitle

\section*{18.5.1}

\begin{itemize}
  \item $\Rightarrow$

        如果$f$是可测的,那么对任意开集$V \subseteq \mathbb{R}^m$,$f^{-1}(V)$都是可测的,
        而开盒子$B$本身就是开集,所以$f^{-1}(B)$是可测的。

  \item $\Leftarrow$

        对任意开集$V \subseteq \mathbb{R}^m$,由引理18.5.10可知,$V$可写成可数个或有限个开盒子的并集,
        即
        \begin{align*}
          V = \bigcup_{B \in \Sigma} B
        \end{align*}
        $\Sigma$是一个可数集或者有限集。
        我们有
        \begin{align*}
          f^{-1}(V) = \bigcup_{B \in \Sigma} f^{-1}(B)
        \end{align*}
        因为$f^{-1}(B)$是可测的,利用(iv)($\sigma-$代数性质可知),
        $\bigcup_{B \in \Sigma} f^{-1}(B)$是可测的,即$f^{-1}(V)$是可测的。

\end{itemize}

\section*{18.5.2}

对任意开集$V \subseteq \mathbb{R}^m$,由引理18.5.10可知,$V$可写成可数个或有限个开盒子的并集,
即
\begin{align*}
  V = \bigcup_{B \in \Sigma} B
\end{align*}
$\Sigma$是一个可数集或者有限集。
对任意开盒子$B \in \Sigma$可以表示成
\begin{align*}
  B = \prod\limits_{i=1}^m (a_i, b_i)
\end{align*}
对任意$1 \leq j \leq m$,
令$b_j = (a_j, b_j)$,
由题设可知,$f_j^{-1}(b_j)$是可测的。

接下来证明:
\begin{align*}
  f^{-1}(B) = \bigcap\limits_{j=1}^m f_j^{-1}(b_j)
\end{align*}
因为任意$x_0 \in f^{-1}(B)$,那么$f(x_0) \in B$,
所以对任意$1 \leq j \leq m$都有$f_j(x_0) \in b_j$,否则与$f(x_0) \in B$矛盾,
于是可得$x_0 \in \bigcap\limits_{j=1}^m f_j^{-1}(b_j)$,进而$f^{-1}(B) \subseteq \bigcap\limits_{j=1}^m f_j^{-1}(b_j)$。

任意$x_0 \in \bigcap\limits_{j=1}^m f_j^{-1}(b_j)$,
那么对任意$1 \leq j \leq m$都有$f_j(x_0) \in b_j$,
于是可得$(f_1(x_0), \cdots, f_m(x_0)) \in B$,
所以$x_0 \in f^{-1}(B)$,进而$\bigcap\limits_{j=1}^m f_j^{-1}(b_j) \subseteq f^{-1}(B)$。

综上可得$f^{-1}(B) = \bigcap\limits_{j=1}^m f_j^{-1}(b_j)$。

因为任意$1 \leq j \leq m$,$f_j^{-1}(b_j)$都是可测的,
利用引理18.4.4(d)(布尔代数性质)可知$\bigcap\limits_{j=1}^m f_j^{-1}(b_j)$
是可测的,即$f^{-1}(B)$是可测的。

我们有
\begin{align*}
  f^{-1}(V) = \bigcup\limits_{B \in \Sigma} f^{-1}(B)
\end{align*}

由开盒子$B$的任意性和$\sigma$代数性质可知$\bigcup\limits_{B \in \Sigma} f^{-1}(B)$是可测的,
即$f^{-1}(V)$是可测的。

综上可得,$f$是可测函数。

\section*{18.5.3}

对任意开集$V \subseteq \mathbb{R}^p$,
由引理18.5.2(连续函数是可测的)可知,
$g^{-1}(V)$是可测的。

因为$g^{-1}(V) \subseteq W$,且由题设可知$f$是可测的,
所以$f^{-1}(g^{-1}(V))$是可测的。

接下来我们需要证明:
\begin{align*}
  f^{-1}(g^{-1}(V)) = (g \circ f)^{-1}(V)
\end{align*}

任意$x_0 \in f^{-1}(g^{-1}(V))$,
那么$f(x_0) \in g^{-1}(V)$,
进而$g(f(x_0)) \in V$,即$g \circ f(x_0) \in V$,
所以$x_0 \in (g \circ f)^{-1}(V)$,
从而可得$f^{-1}(g^{-1}(V)) \subseteq (g \circ f)^{-1}(V)$。

任意$x_0 \in (g \circ f)^{-1}(V)$,
那么$g \circ f(x_0) \in V$,
即$g(f(x_0)) \in V$,
于是可得$f(x_0) \in g^{-1}(V)$,
进而$x_0 \in f^{-1}(g^{-1}(V))$,
从而可得$(g \circ f)^{-1}(V) \subseteq f^{-1}(g^{-1}(V))$。

综上可得$f^{-1}(g^{-1}(V)) = (g \circ f)^{-1}(V)$。

所以$(g \circ f)^{-1}(V)$也是可测的。

\section*{18.5.4}

\begin{itemize}
  \item $\Rightarrow$

        因为$(a, \infty) \subseteq \mathbb{R}$,
        于是由$f$是可测的可得,$f^{-1}((a, \infty))$是可测的。

  \item $\Leftarrow$

        先证明如果对于所有的$a$,$f^{-1}((a, \infty))$都是可测的,
        那么对于所有的$a$,$f^{-1}([a, \infty))$也是可测的。

        我们有
        \begin{align*}
          [a, \infty) = \bigcap\limits_{n = 1}^\infty (a - \frac{1}{n}, \infty)
        \end{align*}
        $f^{-1}$保持集合运算(函数的逆像与集合的基本运算(交并补)有着良好的兼容性),
        即我们有:
        \begin{align*}
          f^{-1}([a, \infty)) & = f^{-1}\left(\bigcap\limits_{n = 1}^\infty (a - \frac{1}{n}, \infty)\right) \\
                              & = \bigcap\limits_{n = 1}^\infty f^{-1}((a - \frac{1}{n}, \infty))
        \end{align*}
        由于$f^{-1}((a - \frac{1}{n}, \infty))$都是可测的,由$\sigma-$代数性质可得
        \begin{align*}
          f^{-1}([a, \infty)) = \bigcap\limits_{n = 1}^\infty f^{-1}((a - \frac{1}{n}, \infty))
        \end{align*}
        是可测的。

        任意开集$V \subseteq \mathbb{R}$,
        由引理18.4.10可知,$V$可写成可数个或有限个开盒子的并集,
        即
        \begin{align*}
          V = \bigcup_{B \in \Sigma} B
        \end{align*}
        $\Sigma$是一个可数集或者有限集。

        对任意开盒子$B \in \Sigma$,
        $B$是一维空间中的开盒子,于是可表示成$(a, b)$其中$a, b$都是实数。
        我们有
        \begin{align*}
          B = (a, b) = (a, \infty) \setminus [b, \infty)
        \end{align*}
        由题设可知$f^{-1}((a, \infty))$是可测的,且$f^{-1}([b, \infty))$都是可测的,
        于是我们有
        \begin{align*}
          f^{-1}(B) & = f^{-1}((a, \infty) \setminus [b, \infty))         \\
                    & = f^{-1}((a, \infty)) \setminus f^{-1}([b, \infty))
        \end{align*}
        利用推论18.4.7可知,$f^{-1}(B)$是可测的。 
\end{itemize}

\section*{18.5.5}



\end{document}