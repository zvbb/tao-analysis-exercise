\documentclass{article}
\usepackage{mathtools} 
\usepackage{fontspec}
\usepackage[UTF8]{ctex}
\usepackage{amsthm}
\usepackage{mdframed}
\usepackage{xcolor}
\usepackage{amssymb}
\usepackage{amsmath}


% 定义新的带灰色背景的说明环境 zremark
\newmdtheoremenv[
  backgroundcolor=gray!10,
  % 边框与背景一致,边框线会消失
  linecolor=gray!10
]{zremark}{说明}

% 通用矩阵命令: \flexmatrix{矩阵名}{元素符号}{行数}{列数}
\newcommand{\flexmatrix}[4]{
  \[
  #1 = \begin{pmatrix}
    #2_{11}     & #2_{12}     & \cdots & #2_{1#4}   \\
    #2_{21}     & #2_{22}     & \cdots & #2_{2#4}   \\
    \vdots      & \vdots      & \ddots & \vdots     \\
    #2_{#31}    & #2_{#32}    & \cdots & #2_{#3#4}
  \end{pmatrix}
  \]
}

% 简化版命令(默认矩阵名为A,元素符号为a): \quickmatrix{行数}{列数}
\newcommand{\quickmatrix}[2]{\flexmatrix{A}{a}{#1}{#2}}


\begin{document}
\title{18.4 习题}
\author{张志聪}
\maketitle

\section*{18.4.1}

$A$是$\mathbb{R}$中的开集,不妨设$A = (a, b)$其中$a < b, a, b \in \mathbb{R}$。
于是
\begin{align*}
  m^{\ast}(A) = b - a
\end{align*}

(1)当$b \leq 0$时,$A \cap (0, \infty) = \varnothing, A \setminus (0, \infty) = A$,
于是
\begin{align*}
  m^{\ast}(A \cap (0, \infty)) + m^{\ast}(A \setminus (0, \infty))
   & = m^{\ast}(\varnothing) + m^{\ast}(A) \\
   & = m^{\ast}(A)
\end{align*}

(2)当$b > 0 > a$时,$A \cap (0, \infty) = (0, b), A \setminus (0, \infty) = (a, 0]$。
\begin{align*}
  m^{\ast}(A \cap (0, \infty)) + m^{\ast}(A \setminus (0, \infty))
   & = m^{\ast}((0, b)) + m^{\ast}((a, 0]) \\
   & = b + m^{\ast}((a, 0])
\end{align*}
接下来,我们计算$m^{\ast}((a, 0])$。

因为,我们有
\begin{align*}
  (a, 0) \subseteq (a, 0]
\end{align*}
对任意$\epsilon > 0$,我们又有
\begin{align*}
  (a, 0] \subseteq (a, \epsilon)
\end{align*}
由引理18.2.5(vii)可知,
\begin{align*}
  m^{\ast}((a, 0)) \leq m^{\ast}((a, 0]) \leq m^{\ast}((a, \epsilon)) \\
  \implies                                                            \\
  -a \leq m^{\ast}((a, 0]) \leq \epsilon - a
\end{align*}
令$\epsilon \to 0$,然后使用夹逼定理,可以得到
\begin{align*}
  m^{\ast}((a, 0]) = -a
\end{align*}
综上可得
\begin{align*}
  m^{\ast}(A \cap (0, \infty)) + m^{\ast}(A \setminus (0, \infty)) = b - a = m^{\ast}(A)
\end{align*}

(3)当$a \geq 0$时,$A \cap (0, \infty) = A, A \setminus (0, \infty) = \varnothing$。
于是
\begin{align*}
  m^{\ast}(A \cap (0, \infty)) + m^{\ast}(A \setminus (0, \infty))
   & = m^{\ast}(A) + m^{\ast}(\varnothing) \\
   & = m^{\ast}(A)
\end{align*}

综上可得,命题$m^{\ast}(A) = m^{\ast}(A \cap (0, \infty)) + m^{\ast}(A \setminus (0, \infty))$得证。

\section*{18.4.2}
有一点需要注意$E := \{(x_1, x_2, \cdots, x_n) \in \mathbb{R}^n: x_n > 0\}$,其中
$x_1, x_2, \cdots, x_{n - 1}$不是固定值,在集合定义中,未被限制的变量都是自由的,
所以在这个集合中,前面的分量$x_1, x_2, \cdots, x_{n - 1}$完全自由,只有最后
一个分量$x_n > 0$。

$A$是$\mathbb{R}^n$中的开盒子,那么,对任意$A$可以表示成
\begin{align*}
  A = \prod\limits_{i = 1}^n (a_i, b_i)
\end{align*}

于是
\begin{align*}
  A \cap E = \prod\limits_{i = 1}^{n - 1} (a_i, b_i) \times ((a_n, b_n) \cap (0, \infty)) \\
  A \setminus E = \prod\limits_{i = 1}^{n - 1} (a_i, b_i) \times ((a_n, b_n) \setminus (0, \infty))
\end{align*}
由习题18.2.2可知,
\begin{align*}
  m^{\ast}(A \cap E) \leq m^{\ast}(\prod\limits_{i = 1}^{n - 1} (a_i, b_i))m^{\ast}( (a_n, b_n) \cap (0, \infty)) \\
  m^{\ast}(A \setminus E) \leq m^{\ast}(\prod\limits_{i = 1}^{n - 1} (a_i, b_i))m^{\ast}( (a_n, b_n) \setminus (0, \infty))
\end{align*}
于是可得
\begin{align*}
  m^{\ast}(A \cap E) + m^{\ast}(A \setminus E) \leq m^{\ast}(\prod\limits_{i = 1}^{n - 1} (a_i, b_i))(m^{\ast}( (a_n, b_n) \cap (0, \infty)) + m^{\ast}( (a_n, b_n) \setminus (0, \infty)))
\end{align*}
利用习题18.4.1(viii)可得
\begin{align*}
  m^{\ast}(A \cap E) + m^{\ast}(A \setminus E) & \leq m^{\ast}(\prod\limits_{i = 1}^{n - 1} (a_i, b_i))m^{\ast}((a_n, b_n)) \\
                                               & = m^{\ast}(\prod\limits_{i = 1}^{n} (a_i, b_i))                            \\
                                               & = m^{\ast}(A)
\end{align*}
又由引理18.2.5可知
\begin{align*}
  m^{\ast}(A) \leq m^{\ast}(A \cap E) + m^{\ast}(A \setminus E)
\end{align*}
综上可得
\begin{align*}
  m^{\ast}(A) = m^{\ast}(A \cap E) + m^{\ast}(A \setminus E)
\end{align*}

\section*{18.4.3}

令$E$等于半空间。对任意$A \subseteq \mathbb{R}^n$,引理18.2.5(viii)可得
\begin{align*}
  m^{\ast}(A) \leq m^{\ast}(A \cap E) + m^{\ast}(A \setminus E)
\end{align*}

对任意$\epsilon > 0$,由外测度的定义可知,存在
$(B_j)_{j \in J}$覆盖$A$,使得
\begin{align*}
  \sum\limits_{j \in J} vol(B_j) \leq m^{\ast}(A) + \epsilon \\
  \implies                                                   \\
  \sum\limits_{j \in J} vol(B_j) - \epsilon \leq m^{\ast}(A)
\end{align*}
由推论18.2.7,我们有
\begin{align*}
  m^{\ast}(B_j) = vol(B_j)
\end{align*}
于是由习题18.4.2可知,对任意$j \in J$,都有
\begin{align*}
  m^{\ast}(B_j) = m^{\ast}(B_j \cap E) + m^{\ast}(B_j \setminus E)
\end{align*}
于是,我们有
\begin{align*}
  \sum\limits_{j \in J} vol(B_j) = \sum\limits_{j \in J} m^{\ast}(B_j \cap E)  + \sum\limits_{j \in J} m^{\ast}(B_j \setminus E)
\end{align*}
因为
\begin{align*}
  \bigcup \limits_{j \in J} (B_j \cap E) \subseteq A \cap E \\
  \bigcup \limits_{j \in J} (B_j \setminus E) \subseteq A \setminus E
\end{align*}
于是由引理18.2.5(viii)或(x)可得
\begin{align*}
  m^{\ast}(A \cap E) \leq \sum\limits_{j \in J} m^{\ast}(B_j \cap E) \\
  m^{\ast}(A \setminus E) \leq \sum\limits_{j \in J} m^{\ast}(B_j \setminus E)
\end{align*}
综上可得
\begin{align*}
  m^{\ast}(A \cap E) + m^{\ast}(A \setminus E) -\epsilon \leq m^{\ast}(A)
\end{align*}
由$\epsilon$的任意性可知,并结合$m^{\ast}(A) \leq m^{\ast}(A \cap E) + m^{\ast}(A \setminus E)$,
我们就有
\begin{align*}
  m^{\ast}(A) = m^{\ast}(A \cap E) + m^{\ast}(A \setminus E)
\end{align*}

于是由定义18.4.1可知,$E$是可测的。

\section*{18.4.4}

\begin{itemize}
  \item (a)

        对$\mathbb{R}^n$的每一个子集$A$,我们有
        \begin{align*}
          A \cap E = A \setminus (\mathbb{R}^n \setminus E) \\
          A \setminus E = A \cap (\mathbb{R}^n \setminus E)
        \end{align*}
        因为$E$是可测的,所以
        \begin{align*}
          m^{\ast}(A) & = m^{\ast}(A \cap E) + m^{\ast}(A \setminus E)                                                   \\
                      & = m^{\ast}(A \setminus (\mathbb{R}^n \setminus E)) + m^{\ast}(A \cap (\mathbb{R}^n \setminus E))
        \end{align*}
        于是可得,$\mathbb{R}^n \setminus E$是可测的。

  \item (b)

        因为$E$是可测的,所以对$\mathbb{R}^n$的每一个子集$A$,我们有
        \begin{align*}
          m^{\ast}(A) & = m^{\ast}(A \cap E) + m^{\ast}(A \setminus E)
        \end{align*}
        又因为
        \begin{align*}
          A \cap (E + x) = (A - x) \cap E + x \\
          A \setminus (E + x) = (A - x) \setminus E + x
        \end{align*}
        因为$A - x$也是$\mathbb{R}^n$的子集,于是以下等式依然成立:
        \begin{align*}
          m^{\ast}(A - x) & = m^{\ast}((A - x) \cap E) + m^{\ast}((A - x) \setminus E)
        \end{align*}
        利用引理18.2.5(xiii)(平移不变性),我们有
        \begin{align*}
          m^{\ast}(A - x + x) & = m^{\ast}((A - x) \cap E + x) + m^{\ast}((A - x) \setminus E + x) \\
          \implies                                                                                 \\
          m^{\ast}(A)         & = m^{\ast}(A \cap (E + x)) + m^{\ast}(A \setminus (E + x))
        \end{align*}
        于是可得$x + E$是可测的。

        由定义18.4.1可知$m(E) = m^{\ast}(E)$,
        再次利用引理18.2.5(xiii)(平移不变性)
        \begin{align*}
          m(E) = m^{\ast}(E) = m^{\ast}(x + E) = m(x + E)
        \end{align*}

  \item (c) $\circledast$

        $E_1$是可测的,所以对任意$A \subseteq \mathbb{R}^n$,我们有
        \begin{align}
          m^{\ast}(A) = m^{\ast}(A \cap E_1) + m^{\ast}(A \setminus E_1)
        \end{align}
        $E_2$也是可测的,且因为$(A \cap E_1) \subseteq \mathbb{R}^n$,于是我们有
        \begin{align*}
          m^{\ast}(A \cap E_1) & = m^{\ast}((A \cap E_1) \cap E_2) + m^{\ast}((A \cap E_1) \setminus E_2) \\
                               & = m^{\ast}(A \cap E_1 \cap E_2) + m^{\ast}(A \cap E_1 \setminus E_2)
        \end{align*}
        同样的,$(A \setminus E_1) \subseteq \mathbb{R}^n$,于是
        \begin{align*}
          m^{\ast}(A \setminus E_1) & = m^{\ast}((A \setminus E_1) \cap E_2) + m^{\ast}((A \setminus E_1) \setminus E_2) \\
                                    & = m^{\ast}(A \cap E_2 \setminus E_1) + m^{\ast}(A \setminus (E_1 \cup E_2))
        \end{align*}
        代入(1)式可得:
        \begin{align*}
          m^{\ast}(A) = m^{\ast}(A \cap E_1 \cap E_2) + m^{\ast}(A \cap E_1 \setminus E_2) + m^{\ast}(A \cap E_2 \setminus E_1) + m^{\ast}(A \setminus (E_1 \cup E_2))
        \end{align*}
        \textcolor{red}{令这个为式子(0)}

        观察$A \setminus (E_1 \cap E_2)$,利用命题3.1.28(h),我们有
        \begin{align*}
          A \setminus (E_1 \cap E_2) & = (A \setminus E_1) \cup (A \setminus E_2)                                                    \\
                                     & =(A \cap E_1 \setminus E_2) \cup (A \cap E_2 \setminus E_1) \cup (A \setminus (E_1 \cup E_2))
        \end{align*}
        于是由命题18.2.5(viii)可知,
        \begin{align*}
          m^{\ast}(A \setminus (E_1 \cap E_2)) \leq m^{\ast}(A \cap E_1 \setminus E_2) + m^{\ast}(A \cap E_2 \setminus E_1) + m^{\ast}(A \setminus (E_1 \cup E_2))
        \end{align*}
        这里有:
        \begin{align*}
          m^{\ast}(A \setminus (E_1 \cap E_2)) = m^{\ast}(A \cap E_1 \setminus E_2) + m^{\ast}(A \cap E_2 \setminus E_1) + m^{\ast}(A \setminus (E_1 \cup E_2))
        \end{align*}
        反证法,假设上式不成立,那么就有
        \begin{align*}
          m^{\ast}(A \setminus (E_1 \cap E_2)) < m^{\ast}(A \cap E_1 \setminus E_2) + m^{\ast}(A \cap E_2 \setminus E_1) + m^{\ast}(A \setminus (E_1 \cup E_2))
        \end{align*}
        又因为
        \begin{align*}
          A = (A \cap (E_1 \cap E_2)) \cup (A \setminus (E_1 \cap E_2))
        \end{align*}
        再次利用命题18.2.5(viii)可知,
        \begin{align*}
           & m^{\ast}(A) \leq m^{\ast}(A \cap E_1 \cap E_2) + m^{\ast}(A \setminus (E_1 \cap E_2))                                                                        \\
           & \implies                                                                                                                                                     \\
           & m^{\ast}(A) < m^{\ast}(A \cap E_1 \cap E_2) + m^{\ast}(A \cap E_1 \setminus E_2) + m^{\ast}(A \cap E_2 \setminus E_1) + m^{\ast}(A \setminus (E_1 \cup E_2))
        \end{align*}
        这与式(0)存在矛盾,假设不成立。

        结合式(0)可得
        \begin{align}
          m^{\ast}(A) = m^{\ast}(A \cap (E_1 \cap E_2)) + m^{\ast}(A \setminus (E_1 \cap E_2))
        \end{align}
        所以$E_1 \cap E_2$是可测的。

        利用集合运算关系(命题3.1.28),我们有
        \begin{align*}
          E_1 \cup E_2 = \mathbb{R}^n \setminus ((\mathbb{R}^n \setminus E_1) \cap (\mathbb{R}^n \setminus E_2))
        \end{align*}
        由(a)可知,$\mathbb{R}^n \setminus E_1$和$\mathbb{R}^n \setminus E_2$都是可测的。
        由刚才的证明可知$(\mathbb{R}^n \setminus E_1) \cap (\mathbb{R}^n \setminus E_2)$也是可测的,
        再次利用(a)可知,$\mathbb{R}^n \setminus ((\mathbb{R}^n \setminus E_1) \cap (\mathbb{R}^n \setminus E_2))$是可测的。

  \item (d)

        对$N$进行归纳即可证明。
  \item (e)

        \begin{itemize}
          \item 开盒子

                任意开盒子为$B = \prod\limits_{i = 1}^n (a_i, b_i)$,
                于是
                \begin{align*}
                  B = \left(\bigcap \limits_{i = 1}^n \{(x_1, x_2, \dots, x_n) \in \mathbb{R}^n : x_i > a_i\}\right)
                  \bigcap
                  \left(\bigcap \limits_{i = 1}^n \{(x_1, x_2, \dots, x_n) \in \mathbb{R}^n : x_i < b_i\}\right)
                \end{align*}
                (注意,右侧括号内都是半空间平移后的形式)

                由引理18.4.2(半空间都是可测的),并且利用引理(b)(d)(c)可知,$B$是可测的。

          \item 闭盒子

                任意闭盒子为$B = \prod\limits_{i = 1}^n [a_i, b_i]$,
                因为
                \begin{align*}
                  B = (a_1, a_2, \dots, a_n) + \prod\limits_{i = 1}^n [0, b_i - a_i]
                \end{align*}
                令$E = \prod\limits_{i = 1}^n [0, b_i - a_i]$,
                由(b)(平移不变性)可知,我们只需证明$E$是可测的即可。

                又因为
                \begin{align*}
                  \mathbb{R}^n \setminus E & = \left(\bigcup \limits_{i = 1}^n \{(x_1, \dots, x_n) \in \mathbb{R}^n : x_i < 0\}\right)
                  \bigcup \left(\bigcup \limits_{i = 1}^n \{(x_1, \dots, x_n) \in \mathbb{R}^n : x_i > (b_i - a_i)\}\right)
                \end{align*}
                由引理18.4.2(半空间都是可测的),并且利用引理(b)(d)(c)可知,
                $E$是可测的。
        \end{itemize}
  \item (f)

        对于$\mathbb{R}^n$的任意子集,我们有
        \begin{align*}
          A \cap E \subseteq E
        \end{align*}
        利用引理18.2.5(vii)(单调性)可知,
        \begin{align*}
          m^{\ast}(A \cap E) \leq m^{\ast}(E) = 0
        \end{align*}
        结合引理18.2.5(vi)(正性)可知,
        \begin{align*}
          m^{\ast}(A \cap E) = 0
        \end{align*}
        因为
        \begin{align*}
          A \setminus E \subseteq A
        \end{align*}
        再次利用引理18.2.5(vii)(单调性)
        \begin{align*}
          m^{\ast}(A \setminus E) \leq m^{\ast}(A)
        \end{align*}
        又因为
        \begin{align*}
          (A \setminus E) \cup E = A
        \end{align*}
        利用引理18.2.5(viii)(有限次可加性)
        \begin{align*}
          m^{\ast}(A) & \leq m^{\ast}(A \setminus E) + m^{\ast}(E) \\
                      & = m^{\ast}(A \setminus E) + 0              \\
                      & = m^{\ast}(A \setminus E)
        \end{align*}
        综上可得
        \begin{align*}
          m^{\ast}(A) = m^{\ast}(A \setminus E)
        \end{align*}
        于是我们有
        \begin{align*}
          m^{\ast}(A) = m^{\ast}(A \cap E) + m^{\ast}(A \setminus E)
        \end{align*}
        所以,$E$是可测的。

\end{itemize}

\section*{18.4.5}

反证法,假设$E$是可测的。
如同在命题18.3.3中描述的,
由$m^{\ast}(E) \neq 0$可知,存在一个有限的整数$n > 0$使得
$m^{\ast}(E) = m(E) > 1/n$。现在令$J$表示$\mathbb{Q} \cap [-1, 1]$的基数
为$3n$的有限子集。$E$是可测的,那么由引理18.4.4(b)和引理18.4.5可知,
\begin{align*}
  m\left(\bigcup\limits_{q \in J} q + E\right)
   & = \sum\limits_{q \in J} m(q + E) \\
   & = \sum\limits_{q \in J} m(E)     \\
   & = 3n m^{\ast}(E)                 \\
   & > 3n \frac{1}{n}                 \\
   & = 3
\end{align*}
即(按照定义18.4.1中的定义)
\begin{align*}
  m^{\ast}\left(\bigcup\limits_{q \in J} q + E\right) > 3
\end{align*}

而由命题18.3.3中有$\bigcup\limits_{q \in J} q + E \subseteq X$,
那么
\begin{align*}
  m^{\ast}\left(\bigcup\limits_{q \in J} q + E\right) \leq m^{\ast}(X)
\end{align*}
这与$1 \leq m^{\ast}(X) \leq 3$矛盾。

\section*{18.4.6}
(1)

对$J$的基数$n$进行归纳。$\bigcup \limits_{j \in J} E_j$表示成$\bigcup \limits_{j = 1}^n E_j$。
(可以建立对应双射,细节不做说明了。)

归纳基始$n = 1$时,这是显然的:
\begin{align*}
  m^{\ast}(A \cap E_1) = m^{\ast}(A \cap E_1)
\end{align*}

归纳假设$n = k$时,我们有
\begin{align*}
  m^{\ast}(A \cap \bigcup\limits_{j = 1}^k E_j) = \sum \limits_{j = 1}^k m^{\ast}(A \cap E_j)
\end{align*}

$n = k + 1$时,
令$F = \bigcup\limits_{j = 1}^k E_j$,
因为$E_{k + 1}$是可测的,由引理18.4.4(d)可知,$F \cup E_{k + 1}$也是可测的,
于是我们有
\begin{align*}
  m^{\ast}(A \cap (F \cup E_{k + 1}))
  = m^{\ast}(A \cap (F \cup E_{k + 1}) \cap E_{k + 1}) + m^{\ast}(A \cap (F \cup E_{k + 1}) \setminus E_{k + 1})
\end{align*}

由命题命题3.1.18可得(有用到$E_j$之间不相交这个条件)
\begin{align*}
  A \cap (F \cup E_{k + 1}) \cap E_{k + 1}
   & = ((A \cap F) \cup (A \cap E_{k + 1})) \cap E_{k + 1}                  \\
   & = (E_{k + 1} \cap (A \cap F)) \cup (E_{k + 1} \cap (A \cap E_{k + 1})) \\
   & = E_{k + 1}
\end{align*}
又有
\begin{align*}
  A \cap (F \cup E_{k + 1}) \setminus E_{k + 1}
   & = ((A \cap F) \cup (A \cap E_{k + 1})) \setminus E_{k + 1} \\
   & = A \cap F
\end{align*}
(第二个等式可能只能使用集合相等(定义3.1.4)的定义证明了)

综上可得,并结合归纳假设
\begin{align*}
  m^{\ast}(A \cap (F \cup E_{k + 1}))
   & = m^{\ast}(A \cap (F \cup E_{k + 1}) \cap E_{k + 1}) + m^{\ast}(A \cap (F \cup E_{k + 1}) \setminus E_{k + 1}) \\
   & = m^{\ast}(E_{k + 1}) + m^{\ast}(A \cap F)                                                                     \\
   & = \sum\limits_{j = 1}^{k + 1} m^{\ast}(A \cap E_{j})
\end{align*}
归纳完成。

(2)

令$A = \bigcup\limits_{j \in J} E_j$,代入
\begin{align*}
  m^{\ast}(A \cap \bigcup\limits_{j \in J} E_j) & = \sum\limits_{j \in J} m^{\ast}(A \cap E_{j}) \\
  \implies                                                                                       \\
  m^{\ast}(\bigcup\limits_{j \in J} E_j)        & = \sum\limits_{j \in J} m^{\ast}(E_{j})
\end{align*}

\section*{18.4.7}

我们有
\begin{align*}
  B \setminus A = B \cap (\mathbb{R}^n \setminus A)
\end{align*}
因为$A$是可测的,所以由引理18.4.4(a)可知$\mathbb{R}^n \setminus A$是可测的,
再次利用引理18.4.4(c)可知,$B \cap (\mathbb{R}^n \setminus A)$也是可测的。
(通过证明过程可以知道,没有用到$A \subseteq B$这个题设)

因为$A \subseteq B$,所以$A, B \setminus A$是不相交的,
又因为
\begin{align*}
  A \cup (B \setminus A) = B
\end{align*}
于是由引理18.4.5可知
\begin{align*}
  m(A \cup (B \setminus A)) = m(B) = m(A) + m(B \setminus A) \\
  \implies                                                   \\
  m(B \setminus A) = m(B) - m(A)
\end{align*}

\section*{18.4.8}

(1)

按书中的提示,记:
\[
  J = \{j_1, j_2, \dots\},
  F_N = \bigcup\limits_{k = 1}^N \Omega_{j_k},
  E_N = F_{N} \setminus F_{N - 1}
\]

于是可得
\begin{align*}
  F_N = \bigcup \limits_{n = 1}^N E_n \\
  \bigcup\limits_{j \in J} \Omega_{j} = \lim\limits_{N \to \infty} F_N = \bigcup \limits_{n = 1}^\infty E_n
\end{align*}
因为$E_N$的可测性由推论18.4.8保证,
又因为$E_N$之间都是不相交的可测集,
于是由引理18.4.8(可数可加性)可知,
$\bigcup \limits_{n = 1}^\infty E_n$是可测集,
进而$\bigcup\limits_{j \in J} \Omega_{j}$是可测集。

(2)

利用命题3.1.28(h)(德摩根定律),我们有
\begin{align*}
  \bigcap\limits_{k = 1}^N \Omega_{k} = \mathbb{R}^n \setminus \bigcup\limits_{k = 1}^N (\mathbb{R}^n \setminus \Omega_{k})
\end{align*}
于是,我们有
\begin{align*}
  \bigcap\limits_{k = 1}^\infty \Omega_{k} = \mathbb{R}^n \setminus \bigcup\limits_{k = 1}^\infty (\mathbb{R}^n \setminus \Omega_{k})
\end{align*}
(这里不是从有限版本直接推导出可数无限版本的,需要单独证明下,3-1-comment.tex中有证明)

利用引理16.4.4(a)可知,$\mathbb{R}^n \setminus \Omega_{k}$是可测的,
并利用刚才的结论,可以完成证明。



\end{document}