\documentclass{article}
\usepackage{mathtools} 
\usepackage{fontspec}
\usepackage[UTF8]{ctex}
\usepackage{amsthm}
\usepackage{mdframed}
\usepackage{xcolor}
\usepackage{amssymb}
\usepackage{amsmath}


% 定义新的带灰色背景的说明环境 zremark
\newmdtheoremenv[
  backgroundcolor=gray!10,
  % 边框与背景一致,边框线会消失
  linecolor=gray!10
]{zremark}{说明}

% 通用矩阵命令: \flexmatrix{矩阵名}{元素符号}{行数}{列数}
\newcommand{\flexmatrix}[4]{
  \[
  #1 = \begin{pmatrix}
    #2_{11}     & #2_{12}     & \cdots & #2_{1#4}   \\
    #2_{21}     & #2_{22}     & \cdots & #2_{2#4}   \\
    \vdots      & \vdots      & \ddots & \vdots     \\
    #2_{#31}    & #2_{#32}    & \cdots & #2_{#3#4}
  \end{pmatrix}
  \]
}

% 简化版命令(默认矩阵名为A,元素符号为a): \quickmatrix{行数}{列数}
\newcommand{\quickmatrix}[2]{\flexmatrix{A}{a}{#1}{#2}}


\begin{document}
\title{18.2 注释}
\author{张志聪}
\maketitle

\begin{zremark}
  开盒子$A, B$,且$A$是$\mathbb{R}^n$的子集,$B$是$\mathbb{R}^m$的子集,
  我们有
  \begin{align*}
    vol(A \times B) = vol(A) vol(B)
  \end{align*}
\end{zremark}

\textbf{证明:}

$A \times B$是$\mathbb{R}^{n + m}$的子集,
且可表示为
\begin{align*}
  A \times B = \{(a, b): a \in A, b \in B\}
\end{align*}
设
\begin{align*}
  A = \prod\limits_{i = 1}^n (a_i, b_i) \\
  B = \prod\limits_{j = 1}^m (c_j, d_j)
\end{align*}
于是按照笛卡尔积定义,我们有
\begin{align*}
  A \times B = \{(x_1, \cdots, x_n, y_1, \cdots, y_m): a_i < x_i < b_i, c_j < y_j < d_j\}
\end{align*}
也就是说$A \times B$可以表示成开盒子的形式:
\begin{align*}
  A \times B = \prod\limits_{k = 1}^{n + m} (e_k, f_k)
\end{align*}
其中
\begin{align*}
  (e_k, f_k) & = (a_i, b_i), \ 1 \leq k \leq n                     \\
  (e_k, f_k) & = (c_{k - n}, d_{k - n}), \ n + 1 \leq k \leq n + m
\end{align*}
所以
\begin{align*}
  vol(A \times B) & = \prod\limits_{k = 1}^{n + m} (f_k - e_k)                                \\
                  & = \prod\limits_{i = 1}^{n} (a_i, b_i) \prod\limits_{j = 1}^{m} (c_j, d_j) \\
                  & = vol(A) vol(B)
\end{align*}

\begin{zremark}
  $\mathbb{R}^n$自身就被可数个单位立方体$(0,1)^n$覆盖,如何覆盖?
\end{zremark}

\textbf{证明:}

我们用以下方式覆盖$\mathbb{R}^n$:
\begin{align*}
  \mathbb{R}^n = \bigcup\limits_{q \in \mathbb{Q}^n}((0, 1)^n + q)
\end{align*}
其中,有理数$\mathbb{Q}$是可数的(推论8.1.15),又由推论8.1.14可知$\mathbb{Q}^n$也是可数的。
$(0, 1)^n + q$表示单位立方体平移到$q$这个位置。

接下来,需要证明这个集合确实可以覆盖$\mathbb{R}^n$。

对任意$x = (x_1, \cdots, x_n) \in \mathbb{R}^n$,
由实数的构造方式可得,对任意分量$1 \leq j \leq n$,
存在有理数$q_j$,
使得
\begin{align*}
  x_j - q_j \in (0, 1)
\end{align*}
令$q = (q_1, \cdots, q_n)$,则
$x \in (0, 1)^n + q$。

$\{A^{(j)}: j \in J; x_n \in (a, b)\}$

\begin{zremark}
  虽然$\mathbb{R}$的一维测度是$+\infty$,但是$\mathbb{R}^2$的整个$x$轴的二维外测度却是$0$。
\end{zremark}

\textbf{证明:}

设$\mathbb{R}^2$的整个$x$轴是区间$X = \{(x, 0): x \in \mathbb{R}\}$。

对于每一个整数$z$,$B_z : = \prod \limits_{i = 1}^2 [a_i, b_i]$,
其中$[a_1, b_1]= [z - 1, z + 1]$,$[a_2, b_2] = [0, 0]$,
于是
\begin{align*}
  m^{\ast}(B_z) = 2 \times 0 = 0
\end{align*}
全体的$z \in \mathbb{Q}, B_z$的并集就是整个目标集合$X$,
所以
\begin{align*}
  m^{\ast}(X) \leq \sum \limits_{z \in \mathbb{Q}} m^{\ast}(B_z) = 0
\end{align*}

\begin{zremark}
  默认情况下,两个无限级数的乘积是否收敛,与其“乘积”的求和顺序有关,于是需要定义其顺序。
  但在绝对收敛的情况,情况特殊:

  这里引入一个命题(参考了数学分析第五版(下册)华东师范大学版):
  设有绝对收敛序列
  \begin{align*}
    \sum \limits_{n = 1}^{\infty} |u_n| = A \\
    \sum \limits_{n = 1}^{\infty} |v_n| = B
  \end{align*}
  则把级数$\sum \limits_{n = 1}^{\infty} u_n$和$\sum \limits_{n = 1}^{\infty} v_n$
  中每一项所有可能的乘积列出(任何方式都可以)得到级数$\sum w_n$也是绝对收敛的,且其和等于$AB$。
\end{zremark}

\textbf{证明:}

以$s_n$表示级数$\sum |w_n|$的部分和,即:
\begin{align*}
  s_n = |w_1| + |w_2| + \cdots + |w_n|
\end{align*}
其中$w_k = u_{i_k}v_{j_k}(k = 1, 2, \cdots, n)$,记
\begin{align*}
  m = max(i_1, j_1, \cdots, i_n, j_n),  \\
  A_m = |u_1| + |u_2| + \cdots + |u_m|, \\
  B_m = |v_1| + |v_2| + \cdots + |v_m|
\end{align*}
于是有(有限项求和)
\begin{align*}
  s_n \leq A_m B_m
\end{align*}
因为$\sum \limits_{n = 1}^{\infty} |u_n|$和$\sum \limits_{n = 1}^{\infty} |v_n|$
的部分和${A_n}$和${B_n}$序列都是有界的。
于是可得${s_n}$是有界的,从而级数$\sum w_n$绝对收敛。

绝对收敛的级数,具有可重排序的性质,于是把$\sum w_n$以正方形顺序并加相关括号,即
\begin{align*}
  u_1v_1 + (u_1v_2 + u_2v_2 + u_2v_1) + (u_1v_3 + u_2v_3 + u_3v_3 + u_3v_2 + u_3v_1) + \cdots,
\end{align*}
\[
  \begin{array}{c|cccc}
           & v_1     & v_2     & v_3     & \cdots \\
    \hline
    u_1    & u_1 v_1 & u_1 v_2 & u_1 v_3 & \cdots \\
    u_2    & u_2 v_1 & u_2 v_2 & u_2 v_3 & \cdots \\
    u_3    & u_3 v_1 & u_3 v_2 & u_3 v_3 & \cdots \\
    \vdots & \vdots  & \vdots  & \vdots  & \ddots
  \end{array}
\]

把每个括号作为一项的新级数:
\begin{align*}
  p_1 + p_2 + p_3 + \cdots
\end{align*}
于是部分和$P_n$有
\begin{align*}
  P_n = A_n B_n
\end{align*}
从而由之前的结论有
\begin{align*}
  \lim\limits_{n \to \infty} P_n
  = \lim\limits_{n \to \infty} A_n B_n
  = \lim\limits_{n \to \infty} A_n \lim\limits_{n \to \infty} B_n
  = AB
\end{align*}



\end{document}


