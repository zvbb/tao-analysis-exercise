\documentclass{article}
\usepackage{mathtools} 
\usepackage{fontspec}
\usepackage[UTF8]{ctex}
\usepackage{amsthm}
\usepackage{mdframed}
\usepackage{xcolor}
\usepackage{amssymb}
\usepackage{amsmath}


% 定义新的带灰色背景的说明环境 zremark
\newmdtheoremenv[
  backgroundcolor=gray!10,
  % 边框与背景一致,边框线会消失
  linecolor=gray!10
]{zremark}{说明}

% 通用矩阵命令: \flexmatrix{矩阵名}{元素符号}{行数}{列数}
\newcommand{\flexmatrix}[4]{
  \[
  #1 = \begin{pmatrix}
    #2_{11}     & #2_{12}     & \cdots & #2_{1#4}   \\
    #2_{21}     & #2_{22}     & \cdots & #2_{2#4}   \\
    \vdots      & \vdots      & \ddots & \vdots     \\
    #2_{#31}    & #2_{#32}    & \cdots & #2_{#3#4}
  \end{pmatrix}
  \]
}

% 简化版命令(默认矩阵名为A,元素符号为a): \quickmatrix{行数}{列数}
\newcommand{\quickmatrix}[2]{\flexmatrix{A}{a}{#1}{#2}}


\begin{document}
\title{18.2 注释}
\author{张志聪}
\maketitle

\begin{zremark}
  $\mathbb{R}^n$自身就被可数个单位立方体$(0,1)^n$覆盖,如何覆盖?
\end{zremark}

\textbf{证明:}

我们用以下方式覆盖$\mathbb{R}^n$:
\begin{align*}
  \mathbb{R}^n = \bigcup\limits_{q \in \mathbb{Q}^n}((0, 1)^n + q)
\end{align*}
其中,有理数$\mathbb{Q}$是可数的(推论8.1.15),又由推论8.1.14可知$\mathbb{Q}^n$也是可数的。 
$(0, 1)^n + q$表示单位立方体平移到$q$这个位置。

接下来,需要证明这个集合确实可以覆盖$\mathbb{R}^n$。

对任意$x = (x_1, \cdots, x_n) \in \mathbb{R}^n$,
由实数的构造方式可得,对任意分量$1 \leq j \leq n$,
存在有理数$q_j$,
使得
\begin{align*}
  x_j - q_j \in (0, 1) 
\end{align*}
令$q = (q_1, \cdots, q_n)$,则
$x \in (0, 1)^n + q$。

$\{A^{(j)}: j \in J; x_n \in (a, b)\}$

\begin{zremark}
  虽然$\mathbb{R}$的一维测度是$+\infty$,但是$\mathbb{R}^2$的整个$x$轴的二维外测度却是$0$。  
\end{zremark}

\textbf{证明:}

设$\mathbb{R}^2$的整个$x$轴是区间$X = \{(x, 0): x \in \mathbb{R}\}$。

对于每一个整数$z$,$B_z : = \prod \limits_{i = 1}^2 [a_i, b_i]$,
其中$[a_1, b_1]= [z - 1, z + 1]$,$[a_2, b_2] = [0, 0]$,
于是
\begin{align*}
  m^{\ast}(B_z) = 2 \times 0 = 0
\end{align*}
全体的$z \in \mathbb{Q}, B_z$的并集就是整个目标集合$X$,
所以
\begin{align*}
  m^{\ast}(X) \leq \sum \limits_{z \in \mathbb{Q}} m^{\ast}(B_z) = 0
\end{align*}


\end{document}


