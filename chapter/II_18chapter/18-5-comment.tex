\documentclass{article}
\usepackage{mathtools} 
\usepackage{fontspec}
\usepackage[UTF8]{ctex}
\usepackage{amsthm}
\usepackage{mdframed}
\usepackage{xcolor}
\usepackage{amssymb}
\usepackage{amsmath}


% 定义新的带灰色背景的说明环境 zremark
\newmdtheoremenv[
  backgroundcolor=gray!10,
  % 边框与背景一致,边框线会消失
  linecolor=gray!10
]{zremark}{说明}

% 通用矩阵命令: \flexmatrix{矩阵名}{元素符号}{行数}{列数}
\newcommand{\flexmatrix}[4]{
  \[
  #1 = \begin{pmatrix}
    #2_{11}     & #2_{12}     & \cdots & #2_{1#4}   \\
    #2_{21}     & #2_{22}     & \cdots & #2_{2#4}   \\
    \vdots      & \vdots      & \ddots & \vdots     \\
    #2_{#31}    & #2_{#32}    & \cdots & #2_{#3#4}
  \end{pmatrix}
  \]
}

% 简化版命令(默认矩阵名为A,元素符号为a): \quickmatrix{行数}{列数}
\newcommand{\quickmatrix}[2]{\flexmatrix{A}{a}{#1}{#2}}


\begin{document}
\title{18.5 注释}
\author{张志聪}
\maketitle

\begin{zremark}
  证明:
  \begin{align*}
    g^{-1}((a, +\infty]) = \bigcup\limits_{n \geq 1} f_n^{-1}((a, +\infty])
  \end{align*}
\end{zremark}

\textbf{证明:}

这是上确界函数,书中没找到明确定义的地方,这里先说明一下:

$\sup\limits_{n \geq 1} f_n$ 表示一系列函数${f_n}$(其中 $n \geq 1$)的上确界函数。
具体来说,对于每一个自变量$x$,
这个函数的值是所有函数$f_n$在$x$处的上确界(即最小的上界)。
数学表达式为:
\begin{align*}
  \left( \sup\limits_{n \geq 1} f_n \right)(x) = \sup(f_1(x), f_2(x), \cdots)
\end{align*}

\begin{itemize}
  \item 从右到左

        设任意$x_0 \in g^{-1}((a, +\infty])$,
        那么$g(x_0) \in (a, +\infty]$,
        由上确界函数的定义可知,存在$f_n(x_0) = g(x_0)$,
        从而$g^{-1}((a, +\infty]) \subseteq \bigcup\limits_{n \geq 1} f_n^{-1}((a, +\infty])$。

  \item 从左到右

        设任意$x_0 \in \bigcup\limits_{n \geq 1} f_n^{-1}((a, +\infty])$,
        那么存在某个$n$,使得$f_n(x_0) \in (a, +\infty]$,
        于是我们有
        \begin{align*}
          g(x_0) \geq f_n(x_0) > a
        \end{align*}
        所以$x_0 \in g^{-1}((a, +\infty])$,
        从而$\bigcup\limits_{n \geq 1} f_n^{-1}((a, +\infty]) \subseteq g^{-1}((a, +\infty])$。
\end{itemize}



\end{document}