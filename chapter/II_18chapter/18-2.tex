\documentclass{article}
\usepackage{mathtools} 
\usepackage{fontspec}
\usepackage[UTF8]{ctex}
\usepackage{amsthm}
\usepackage{mdframed}
\usepackage{xcolor}
\usepackage{amssymb}
\usepackage{amsmath}


% 定义新的带灰色背景的说明环境 zremark
\newmdtheoremenv[
  backgroundcolor=gray!10,
  % 边框与背景一致,边框线会消失
  linecolor=gray!10
]{zremark}{说明}

% 通用矩阵命令: \flexmatrix{矩阵名}{元素符号}{行数}{列数}
\newcommand{\flexmatrix}[4]{
  \[
  #1 = \begin{pmatrix}
    #2_{11}     & #2_{12}     & \cdots & #2_{1#4}   \\
    #2_{21}     & #2_{22}     & \cdots & #2_{2#4}   \\
    \vdots      & \vdots      & \ddots & \vdots     \\
    #2_{#31}    & #2_{#32}    & \cdots & #2_{#3#4}
  \end{pmatrix}
  \]
}

% 简化版命令(默认矩阵名为A,元素符号为a): \quickmatrix{行数}{列数}
\newcommand{\quickmatrix}[2]{\flexmatrix{A}{a}{#1}{#2}}


\begin{document}
\title{18.2 习题}
\author{张志聪}
\maketitle

\section*{18.2.1}

\begin{itemize}
  \item (v) (空集)

        因为$\varnothing \subseteq \varnothing$,
        于是我们可以这样定义开盒子
        \begin{align*}
          B = \prod\limits_{i=1}^{1}
        \end{align*}
        其中$(a_1, b_1) = (0, 0)$,
        所以$B$是空集,
        \begin{align*}
          m^{\ast}(\varnothing) \leq vol(B) = 0
        \end{align*}
        又因为按照定义,外测量是非负的,于是
        \begin{align*}
          m^{\ast}(\varnothing) = 0
        \end{align*}

  \item (vi)(正性)

        到目前为止,可测集是否能够被开盒覆盖是不确定的,
        但这里应该指的是可以被开盒覆盖的可测集。

        设$\Omega$被任意有限个或者可数个盒子$(B_j)_{j \in J}$覆盖。
        由盒子体积的定义可知,对任意$j \in J$,都有
        \begin{align*}
          vol(B_j) \geq 0
        \end{align*}
        所以
        \begin{align*}
          \sum\limits_{j \in J} vol(B_j) \geq 0 \\
          m^{\ast}(\Omega) \geq 0
        \end{align*}
        而$m^{\ast}(\Omega) \leq +\infty$是显然的。

  \item (vii) (单调性)

        如果$m^{\ast}(B) = +\infty$,命题显然是正确的。

        如果$m^{\ast}(B) < +\infty$,即$m^{\ast}(B)$是某个实数。
        由定义18.2.4可知,$m^{\ast}(B)$是下确界,于是对任意$\epsilon > 0$,
        存在$(B_j)_{j \in J}$覆盖$B$,使得
        \begin{align*}
          \sum\limits_{j \in J} vol(B_j) \leq m^{\ast}(B) + \epsilon
        \end{align*}
        (因为如果不存在,那么$m^{\ast}(B) + \epsilon$就成为了下确界,存在矛盾)

        因为$A \subseteq B$,所以$(B_j)_{j \in J}$也覆盖$A$,
        所以
        \begin{align*}
          m^{\ast}(A) \leq \sum\limits_{j \in J} vol(B_j) \leq m^{\ast}(B) + \epsilon
        \end{align*}
        由$\epsilon$的任意可知,$m^{\ast}(A) \leq m^{\ast}(B)$。

  \item (viii)(有限次可加性)

        可以直接通过(x)(v)推导,设$J$的基数为$n$,
        我们可以定义一个双射$f: \{i \in \mathbb{N}: 1 \leq i \leq n \} \to (A_j)_{j \in J}$。
        并令
        \begin{equation*}
          A_k =
          \begin{cases*}
            f(k), \  k \leq n \\
            \varnothing, \  k > n
          \end{cases*}
        \end{equation*}
        于是$\bigcup\limits_{k \in \mathbb{N}} A_k$是可数无限集合,由(x)可得,
        \begin{align*}
          m^{\ast}(\bigcup\limits_{k \in \mathbb{N}} A_k)  \leq  \sum \limits_{k \in \mathbb{N}} m^{\ast}(A_k)
        \end{align*}
        又因为(可以利用反证法证明)
        \begin{align*}
          \bigcup\limits_{k \in \mathbb{N}} A_k = \bigcup\limits_{j \in J} A_j
        \end{align*}
        于是
        \begin{align*}
          m^{\ast}(\bigcup\limits_{k \in \mathbb{N}} A_k) = m^{\ast}(\bigcup\limits_{j \in J} A_j)
        \end{align*}
        而且
        \begin{align*}
          \sum \limits_{k \in \mathbb{N}} m^{\ast}(A_k) & = \sum \limits_{k = 1}^n m^{\ast}(A_k) + \sum \limits_{k = n + 1}^\infty m^{\ast}(A_k) \\
                                                        & = \sum \limits_{j \in J} m^{\ast}(A_j)
        \end{align*}
        综上,
        \begin{align*}
          m^{\ast}(\bigcup\limits_{j \in J} A_j) \leq \sum \limits_{j \in J} m^{\ast}(A_j)
        \end{align*}


  \item (x) (可数次可加性)

        $\sum \limits_{j \in J} m^{\ast}(A_j) = +\infty$,命题显然是正确的。

        接下来,证明$\sum \limits_{j \in J} m^{\ast}(A_j) < +\infty$的情况。

        对任意$\epsilon$,
        对任意$A_j$,存在开盒覆盖,即存在一簇盒子$(B^{(j)}_k)_{k \in K}$,
        使得
        \begin{align*}
          A_j \subseteq \bigcup \limits_{k \in K} B^{(j)}_k
        \end{align*}
        且
        \begin{align*}
          m^{\ast}(A_j) \leq \sum \limits_{k \in K} vol(B^{(j)}_k) \leq m^{\ast}(A_j) + \frac{\epsilon}{2^j}
        \end{align*}
        于是,整个并集$\bigcup \limits_{j \in J} A_j$
        可以被$\bigcup \limits_{j \in J}\left(\bigcup \limits_{k \in K} B^{(j)}_k\right)$表示。

        我们有
        \begin{align*}
          m^{\ast}(\bigcup \limits_{j \in J} A_j)
          \leq \sum \limits_{j \in J} \sum \limits_{k \in K} vol(B^{(j)}_k)
          \leq \sum \limits_{j \in J} (m^{\ast}(A_j) + \frac{\epsilon}{2^j})
          = \sum \limits_{j \in J} m^{\ast}(A_j)  + \epsilon
        \end{align*}
        由$\epsilon$的任意性可知,
        \begin{align*}
          m^{\ast}(\bigcup \limits_{j \in J} A_j)  \leq  \sum \limits_{j \in J} m^{\ast}(A_j)
        \end{align*}

        \begin{zremark}
          其实以上的证明使用了“可数无限次加”,而是本书中,陶哲轩通过极限来严格定义无限级数的和。
          这样做的目的应该是避免逻辑问题:\\
          “无限次加法”在数学基础中缺乏严格定义(如:交换律和结合律在无限情况下是否需要额外条件?)。\\
          而极限理论已经是一套严谨框架了。
        \end{zremark}

  \item (xiii)

\end{itemize}



\end{document}


