\documentclass{article}
\usepackage{mathtools} 
\usepackage{fontspec}
\usepackage[UTF8]{ctex}
\usepackage{amsthm}
\usepackage{mdframed}
\usepackage{xcolor}
\usepackage{amssymb}
\usepackage{amsmath}


% 定义新的带灰色背景的说明环境 zremark
\newmdtheoremenv[
  backgroundcolor=gray!10,
  % 边框与背景一致,边框线会消失
  linecolor=gray!10
]{zremark}{说明}

% 通用矩阵命令: \flexmatrix{矩阵名}{元素符号}{行数}{列数}
\newcommand{\flexmatrix}[4]{
  \[
  #1 = \begin{pmatrix}
    #2_{11}     & #2_{12}     & \cdots & #2_{1#4}   \\
    #2_{21}     & #2_{22}     & \cdots & #2_{2#4}   \\
    \vdots      & \vdots      & \ddots & \vdots     \\
    #2_{#31}    & #2_{#32}    & \cdots & #2_{#3#4}
  \end{pmatrix}
  \]
}

% 简化版命令(默认矩阵名为A,元素符号为a): \quickmatrix{行数}{列数}
\newcommand{\quickmatrix}[2]{\flexmatrix{A}{a}{#1}{#2}}


\begin{document}
\title{18.2 习题}
\author{张志聪}
\maketitle

\section*{18.2.1}

\begin{itemize}
  \item (v) (空集)

        因为$\varnothing \subseteq \varnothing$,
        于是我们可以这样定义开盒子
        \begin{align*}
          B = \prod\limits_{i=1}^{1}
        \end{align*}
        其中$(a_1, b_1) = (0, 0)$,
        所以$B$是空集,
        \begin{align*}
          m^{\ast}(\varnothing) \leq vol(B) = 0
        \end{align*}
        又因为按照定义,外测量是非负的,于是
        \begin{align*}
          m^{\ast}(\varnothing) = 0
        \end{align*}

  \item (vi)(正性)

        到目前为止,可测集是否能够被开盒覆盖是不确定的,
        但这里应该指的是可以被开盒覆盖的可测集。

        设$\Omega$被任意有限个或者可数个盒子$(B_j)_{j \in J}$覆盖。
        由盒子体积的定义可知,对任意$j \in J$,都有
        \begin{align*}
          vol(B_j) \geq 0
        \end{align*}
        所以
        \begin{align*}
          \sum\limits_{j \in J} vol(B_j) \geq 0 \\
          m^{\ast}(\Omega) \geq 0
        \end{align*}
        而$m^{\ast}(\Omega) \leq +\infty$是显然的。

  \item (vii) (单调性)

        如果$m^{\ast}(B) = +\infty$,命题显然是正确的。

        如果$m^{\ast}(B) < +\infty$,即$m^{\ast}(B)$是某个实数。
        由定义18.2.4可知,$m^{\ast}(B)$是下确界,于是对任意$\epsilon > 0$,
        存在$(B_j)_{j \in J}$覆盖$B$,使得
        \begin{align*}
          \sum\limits_{j \in J} vol(B_j) \leq m^{\ast}(B) + \epsilon
        \end{align*}
        (因为如果不存在,那么$m^{\ast}(B) + \epsilon$就成为了下确界,存在矛盾)

        因为$A \subseteq B$,所以$(B_j)_{j \in J}$也覆盖$A$,
        所以
        \begin{align*}
          m^{\ast}(A) \leq \sum\limits_{j \in J} vol(B_j) \leq m^{\ast}(B) + \epsilon
        \end{align*}
        由$\epsilon$的任意可知,$m^{\ast}(A) \leq m^{\ast}(B)$。

  \item (viii)(有限次可加性)

        可以直接通过(x)(v)推导,设$J$的基数为$n$,
        我们可以定义一个双射$f: \{i \in \mathbb{N}: 1 \leq i \leq n \} \to (A_j)_{j \in J}$。
        并令
        \begin{equation*}
          A_k =
          \begin{cases*}
            f(k), \  k \leq n \\
            \varnothing, \  k > n
          \end{cases*}
        \end{equation*}
        于是$\bigcup\limits_{k \in \mathbb{N}} A_k$是可数无限集合,由(x)可得,
        \begin{align*}
          m^{\ast}(\bigcup\limits_{k \in \mathbb{N}} A_k)  \leq  \sum \limits_{k \in \mathbb{N}} m^{\ast}(A_k)
        \end{align*}
        又因为(可以利用反证法证明)
        \begin{align*}
          \bigcup\limits_{k \in \mathbb{N}} A_k = \bigcup\limits_{j \in J} A_j
        \end{align*}
        于是
        \begin{align*}
          m^{\ast}(\bigcup\limits_{k \in \mathbb{N}} A_k) = m^{\ast}(\bigcup\limits_{j \in J} A_j)
        \end{align*}
        而且
        \begin{align*}
          \sum \limits_{k \in \mathbb{N}} m^{\ast}(A_k) & = \sum \limits_{k = 1}^n m^{\ast}(A_k) + \sum \limits_{k = n + 1}^\infty m^{\ast}(A_k) \\
                                                        & = \sum \limits_{j \in J} m^{\ast}(A_j)
        \end{align*}
        综上,
        \begin{align*}
          m^{\ast}(\bigcup\limits_{j \in J} A_j) \leq \sum \limits_{j \in J} m^{\ast}(A_j)
        \end{align*}


  \item (x) (可数次可加性)

        $\sum \limits_{j \in J} m^{\ast}(A_j) = +\infty$,命题显然是正确的。

        接下来,证明$\sum \limits_{j \in J} m^{\ast}(A_j) < +\infty$的情况。

        对任意$\epsilon$,
        对任意$A_j$,存在开盒覆盖,即存在一簇盒子$(B^{(j)}_k)_{k \in K}$,
        使得
        \begin{align*}
          A_j \subseteq \bigcup \limits_{k \in K} B^{(j)}_k
        \end{align*}
        且
        \begin{align*}
          m^{\ast}(A_j) \leq \sum \limits_{k \in K} vol(B^{(j)}_k) \leq m^{\ast}(A_j) + \frac{\epsilon}{2^j}
        \end{align*}
        于是,整个并集$\bigcup \limits_{j \in J} A_j$
        可以被$\bigcup \limits_{j \in J}\left(\bigcup \limits_{k \in K} B^{(j)}_k\right)$表示。

        我们有
        \begin{align*}
          m^{\ast}(\bigcup \limits_{j \in J} A_j)
          \leq \sum \limits_{j \in J} \sum \limits_{k \in K} vol(B^{(j)}_k)
          \leq \sum \limits_{j \in J} (m^{\ast}(A_j) + \frac{\epsilon}{2^j})
          = \sum \limits_{j \in J} m^{\ast}(A_j)  + \epsilon
        \end{align*}
        由$\epsilon$的任意性可知,
        \begin{align*}
          m^{\ast}(\bigcup \limits_{j \in J} A_j)  \leq  \sum \limits_{j \in J} m^{\ast}(A_j)
        \end{align*}

        \begin{zremark}
          其实以上的证明使用了“可数无限次加”,而是本书中,陶哲轩通过极限来严格定义无限级数的和。
          这样做的目的应该是避免逻辑问题:\\
          “无限次加法”在数学基础中缺乏严格定义(如:交换律和结合律在无限情况下是否需要额外条件?)。\\
          而极限理论已经是一套严谨框架了。
        \end{zremark}

  \item (xiii)

        对任意$\epsilon > 0$,存在一簇开盒子$(B_j)_{j \in J}$覆盖了$\Omega$,使得
        \begin{align*}
          \sum\limits_{j \in J} vol(B_j) \leq m^{\ast}(\Omega) + \epsilon
        \end{align*}
        任意$x_0 \in \Omega$,存在$j \in J$使得
        \begin{align*}
          x_0 \in B_j
        \end{align*}
        于是
        \begin{align*}
          (x + x_0) \in (x + B_j)
        \end{align*}
        于是可得$(x + \Omega) \subseteq \bigcup \limits_{j \in J} (x + B_j)$。
        又由$vol$的平移不变性(公理(xiii)),
        \begin{align*}
          vol(x + B_j)  = vol(B_j)
        \end{align*}
        综上可得,
        \begin{align*}
          m^{\ast}(x + \Omega) \leq \sum\limits_{j \in J} vol(x + B_j) = \sum\limits_{j \in J} vol(B_j) \leq m^{\ast}(\Omega) + \epsilon
        \end{align*}
        由$\epsilon$的任意性可知,$m^{\ast}(x + \Omega) \leq m^{\ast}(\Omega)$。

        类似地,可得
        \begin{align*}
          m^{\ast}(\Omega) \leq m^{\ast}(x + \Omega) + \epsilon
        \end{align*}
        由$\epsilon$的任意性可知,$m^{\ast}(\Omega) \leq m^{\ast}(x + \Omega)$。

        综上,
        \begin{align*}
          m^{\ast}(\Omega) = m^{\ast}(x + \Omega)
        \end{align*}
\end{itemize}

\section*{18.2.2}

% $A \subseteq \mathbb{R}^n$,$B \subseteq \mathbb{R}^m$,有
% \begin{align*}
%   m^{\ast}_{n + m}(A \times B) \leq m^{\ast}_{n}(A)m^{\ast}_{m}(B)
% \end{align*}

对任意$\epsilon > 0$,存在一簇盒子$(A_j)_{j \in J}$覆盖了$A$且有
\begin{align*}
  \sum\limits_{j \in J} vol(A_j) \leq m^{\ast}_{n}(A) + \epsilon
\end{align*}
同理可得,存在一簇盒子$(B_k)_{k \in K}$覆盖了$B$且有
\begin{align*}
  \sum\limits_{k \in K} vol(B_k) \leq m^{\ast}_{m}(B) + \epsilon
\end{align*}

接下来,我们证明$A \times B$被$(A_j)_{j \in J} \times (B_k)_{k \in K}$覆盖。

$A \times B = \{(a, b): a \in A, b \in B\}$,对任意$x_0 = (a_0, b_0), x_0 \in A \times B$,
其中$a_0 \in A, b_0 \in B$,
所以存在$j_0 \in J, k_0 \in K$使得
\begin{align*}
  a_0 \in A_{j_0} \\
  b_0 \in B_{k_0}
\end{align*}
所以
\begin{align*}
  x_0 \in A_{j_0} \times B_{k_0}
\end{align*}
即$x_0 \in (A_j)_{j \in J} \times (B_k)_{k \in K}$,
由$x_0$的任意性可知,$A \times B \subseteq (A_j)_{j \in J} \times (B_k)_{k \in K}$。

于是我们有
\begin{align*}
  m^{\ast}_{n + m}(A \times B)
   & \leq \sum\limits_{j \in J} \sum\limits_{k \in K} vol(A_j \times B_k)                      \\
   & = \sum\limits_{j \in J} \sum\limits_{k \in K} vol(A_j)vol(B_k)                            \\
   & = \sum\limits_{j \in J} vol(A_j) \sum\limits_{k \in K} vol(B_k)                           \\
   & \leq (m^{\ast}_{n}(A) + \epsilon) + (m^{\ast}_{m}(B) + \epsilon)                          \\
   & = m^{\ast}_{n}(A)m^{\ast}_{m}(B) + \epsilon(m^{\ast}_{n}(A) + m^{\ast}_{m}(B) + \epsilon)
\end{align*}
(注意$\sum\limits_{j \in J} \sum\limits_{k \in K} vol(A_j)vol(B_k) = \sum\limits_{j \in J} vol(A_j) \sum\limits_{k \in K} vol(B_k)$
需要额外证明下,本节注释部分有证明。)
因为$m^{\ast}_{n}(A), m^{\ast}_{m}(B)$是定值且$\epsilon$是任意的,所以
\begin{align*}
  m^{\ast}_{n + m}(A \times B) \leq m^{\ast}_{n}(A)m^{\ast}_{m}(B)
\end{align*}

\section*{18.2.3}

\begin{itemize}
  \item (a)
        $A_1 \subseteq A_2 \subseteq A_3 \subseteq \cdots$,于是有$ m(\bigcup\limits_{j = 1}^\infty A_j) = \lim\limits_{j \to \infty} m(A_j)$。

        证明:
        $m(A_j)$是单调递增的,所以要么$\lim\limits_{j \to \infty} m(A_j) = +\infty$,要么
        $\lim\limits_{j \to \infty} m(A_j) = L$(其中$L \in \mathbb{R}$)。接下里分情况讨论:

        (1)如果$\lim\limits_{j \to \infty} m(A_j) = +\infty$,即序列$(m(A_j))_{j = 1}^\infty$
        没有上界,即对任意$M > 0$,都存在$N$,使得只要$j \geq N$,就有
        \begin{align*}
          m(A_j) > M
        \end{align*}
        因为对任意$n$都有
        \begin{align*}
          \bigcup\limits_{j = 1}^n A_j = A_n
        \end{align*}
        于是当$n \geq N$时,也有
        \begin{align*}
          m(\bigcup\limits_{j = 1}^n A_j) > M
        \end{align*}
        由$M$的任意性和$m(\bigcup\limits_{j = 1}^\infty A_j)$的单调递增性可知,
        $m(\bigcup\limits_{j = 1}^\infty A_j)$发散,即$m(\bigcup\limits_{j = 1}^\infty A_j) = +\infty$。

        (2)如果$\lim\limits_{j \to \infty} m(A_j) = L$,
        即序列$(m(A_j))_{j = 1}^\infty$
        有上界且$L$是最小上界(单调递增序列的性质),
        于是对任意$n \in \mathbb{N}^+$都有
        \begin{align*}
          m(A_n) \leq L
        \end{align*}
        于是可得
        \begin{align*}
          m(\bigcup\limits_{j = 1}^n A_j) = m(A_n) \leq L
        \end{align*}
        $m(\bigcup\limits_{j = 1}^\infty A_j)$的单调递增性且有上界可知,
        $m(\bigcup\limits_{j = 1}^\infty A_j)$收敛。

        反证法,假设$m(\bigcup\limits_{j = 1}^\infty A_j) = L^\prime < L$,
        那么,$L^\prime$是$m(\bigcup\limits_{j = 1}^\infty A_j)$的上界,
        对任意$n \in \mathbb{N}^+$都有
        \begin{align*}
          m(\bigcup\limits_{j = 1}^n A_j) \leq L^\prime
        \end{align*}
        因为$m(\bigcup\limits_{j = 1}^n A_j) = m(A_n)$,
        于是可得,对任意$n \in \mathbb{N}^+$都有
        \begin{align*}
          m(A_n) \leq L^\prime < L
        \end{align*}
        于是$L$不是$(m(A_j))_{j = 1}^\infty$的最小上界,
        存在矛盾。

        综合(1)(2)可得
        \begin{align*}
          m(\bigcup\limits_{j = 1}^\infty A_j) = \lim\limits_{j \to \infty} m(A_j)
        \end{align*}

  \item (b)

        因为对于每一个正整数$j$都有$A_j \supseteq A_{j + 1}$,
        于是可得$(m(A_j))_{j = 1}^\infty$是单调递减的,
        又因为对任意可测集$\Omega$都有$m(\Omega) \geq 0$,且有$m(A_1) < +\infty$。
        综上可得,$(m(A_j))_{j = 1}^\infty$是单调递减且有下界,
        所以$(m(A_j))_{j = 1}^\infty$极限存在,且等于下确界$l \in \mathbb{R}$:
        \begin{align*}
          \lim\limits_{j \to \infty} m(A_j) = l
        \end{align*}
        因为$l$是下确界,那么对任意$n \in \mathbb{N}^+$都有
        \begin{align*}
          m(A_n) \geq l
        \end{align*}
        由题设可知
        \begin{align*}
          \bigcap\limits_{j = 1}^n A_j = A_n
        \end{align*}
        于是,我们有
        \begin{align*}
          m(\bigcap\limits_{j = 1}^n A_j) = m(A_n) \geq l
        \end{align*}
        所以,$(m(\bigcap\limits_{j = 1}^n A_j))_{n = 1}^\infty$也是
        单调递减且有下界$l$,
        进而极限$\lim\limits_{n \to \infty} m(\bigcap\limits_{j = 1}^n A_j) \geq l$。

        接下来证明$\lim\limits_{n \to \infty} m(\bigcap\limits_{j = 1}^n A_j) = l$。

        反证法,假设$\lim\limits_{n \to \infty} m(\bigcap\limits_{j = 1}^n A_j) = l^\prime > l $,
        即$l^\prime$是$(m(\bigcap\limits_{j = 1}^n A_j))_{j = 1}^\infty$下确界,
        那么,由下确界的性质,对$\epsilon = \frac{l - l^\prime}{2} > 0$,
        存在$n \in \mathbb{N}^+$使得
        \begin{align*}
          m(\bigcap\limits_{j = 1}^n A_j) = m(A_n) & < l^\prime + \frac{l - l^\prime}{2} \\
                                                   & < l
        \end{align*}
        这与$l$是$(m(A_j))_{j = 1}^\infty$的下确界矛盾。

        综上,
        \begin{align*}
          m(\bigcap\limits_{j = 1}^\infty A_j) = \lim\limits_{j \to \infty} m(A_j)
        \end{align*}
\end{itemize}

\section*{18.2.4}

令$A = \{(0, 1/q)^n + (1/q)(k_1,\cdots, k_n): k_1,\cdots, k_n \in \{0, 1, \cdots, q - 1\}\}$,
于是$A$中有$q^n$个不相交的集合,且集合都是通过平移$(0, 1/q)^n$得到的。
又因为$A \subseteq [0, 1]^n$,
由勒贝格测度的性质(ix)(xii)(xiii)可知,
\begin{align*}
  m(A) = q^n m((0, 1/q)^n) \leq m([0, 1]^n) = 1 \\
  \implies                                      \\
  m((0, 1/q)^n) \leq q^{-n}
\end{align*}

令$B = \{[0, 1/q]^n + (1/q)(k_1,\cdots, k_n): k_1,\cdots, k_n \in \{0, 1, \cdots, q - 1\}\}$,
于是$B$中$q^n$个相交的集合(在边界的地方重合),且集合都是通过平移$[0, 1/q]^n$得到的。
又因为$\bigcup\limits_{b \in B}  b = [0, 1]^n$,
于是$[0, 1]^n \subseteq \bigcup\limits_{b \in B}  b$。
由勒贝格测度的性质(viii)(xii)(xiii)可知,
\begin{align*}
  m(\bigcup\limits_{b \in B}  b) = m([0, 1]^n) = 1 \leq q^n m([0, 1/q]^n) \\
  \implies                                                                \\
  m([0, 1/q]^n) \geq q^{-n}
\end{align*}

因为
\begin{align*}
  [0, 1/q]^n \setminus (0, 1/q)^n = \{0, 1/q\}
\end{align*}

令$p$是任意正整数,
于是$\{0\} \subseteq (-1/2p) + (0, 1/p)^n, \{1/q\} \subseteq 1/q + (-1/2p) + (0, 1/p)$。
由之前的讨论可知
\begin{align*}
  m(\{0\}) + m(\{1/q\}) \leq 2 m((0, 1/p)^n) \leq 2 p^{-n}
\end{align*}
因为
\begin{align*}
  \lim\limits_{p \to \infty} 2 p^{-n} = 0
\end{align*}
于是,对任意$\epsilon > 0$,存在$N$使得只要$p \geq N$,就有
\begin{align*}
  2 p^{-n} \leq \epsilon
\end{align*}
综上可得,
\begin{align*}
  m([0, 1/q]^n \setminus (0, 1/q)^n) \leq \epsilon
\end{align*}
于是
\begin{align*}
  m([0, 1/q]^n) \leq m((0, 1/q)^n) + \epsilon \\
  \implies                                    \\
  m([0, 1/q]^n) \leq m((0, 1/q)^n) \leq q^{-n}
\end{align*}
结合$m([0, 1/q]^n) \geq q^{-n}$可知,
\begin{align*}
  m([0, 1/q]^n) = q^{-n}
\end{align*}
进而
\begin{align*}
  m((0, 1/q)^n) = q^{-n}
\end{align*}

\section*{18.2.5$\circledast$}

设任意闭盒子$B : = \prod\limits_{j = 1}^n (a_j, b_j)$。

如果$a_j = b_j$,则$B$是空集,于是$m(B) = 0$,
又因为$vol(B) = \prod\limits_{j = 1}^n (b_j - a_j) = \prod\limits_{j = 1}^n 0 = 0$,
于是$m(B) = vol(B)$。

如果$a_j \neq b_j$,先假设$a_j, b_j$都是有理数,于是$b_j - a_j$也是有理数,
于是可表示$b_j - a_j = \frac{p_j}{q_j}$,其中$p_j, q_j$都是正整数。
我们有
\begin{align*}
   & vol(B) = \prod\limits_{j = 1}^n (b_j - a_j) = \prod\limits_{j = 1}^n \frac{p_j}{q_j}                                              \\
   & B = (a_1, \cdots, a_n) + \prod\limits_{j = 1}^n (0, b_j - a_j) = (a_1, \cdots, a_n) + \prod\limits_{j = 1}^n (0, \frac{p_j}{q_j})
\end{align*}
由(xiii)(平移不变性)可知,我们只需证明$vol(B) = m(\prod\limits_{j = 1}^n (0, \frac{p_j}{q_j}))$。

为了方便描述,令$B^\prime : = \prod\limits_{j = 1}^n (0, \frac{p_j}{q_j})$。

取$q = q_1 \times q_2 \cdots q_n$,
于是每个$\frac{p_j}{q_j}$可以写作$\frac{p_j^\prime}{q}$(只需保证$\frac{p_j}{q_j} = \frac{p_j^\prime}{q}$)。
所以
\begin{align*}
  B^\prime & = \prod\limits_{j = 1}^n (0, \frac{p_j}{q_j})                                                            \\
           & = \prod\limits_{j = 1}^n (0, \frac{p_j^\prime}{q})                                                       \\
           & \supseteq \{(0, 1/q)^n + (1/q)(k_1,\cdots, k_n): k_j \in \{0, 1, \cdots, p_j^\prime\}; 1 \leq j \leq n\}
\end{align*}
可得$\{(0, 1/q)^n + (1/q)(k_1,\cdots, k_n): k_j \in \{0, 1, \cdots, p_j^\prime\}; 1 \leq j \leq n\}$中
元素个数为
\begin{align*}
  p_1^\prime \times p_2^\prime \times \cdots \times p_n^\prime
   & = \frac{p_1 q}{q_1} \times \frac{p_2 q}{q_2} \times \cdots \times \frac{p_n q}{q_n} \\
   & = (p_1 \times p_2 \times \cdots \times p_n)q^{n - 1}
\end{align*}
因为元素都是$(0, 1/q)^n$通过平移得到,所以有
\begin{align*}
  m(B^\prime) & \leq (p_1 \times p_2 \times \cdots \times p_n)q^{n - 1} m((0, 1/q)^n) \\
              & \leq (p_1 \times p_2 \times \cdots \times p_n)q^{n - 1} q^{-n}        \\
              & = \frac{p_1 \times p_2 \times \cdots \times p_n}{q}                   \\
              & = \prod\limits_{j = 1}^n \frac{p_j}{q_j}                              \\
              & = vol(B)
\end{align*}
又因为对任意$\epsilon > 0$,
都有$m(B^\prime \setminus \{(0, 1/q)^n + (1/q)(k_1,\cdots, k_n): k_j \in \{0, 1, \cdots, p_j^\prime\}; 1 \leq j \leq n\}) \leq \epsilon$,
(习题18.2.4中有类似证明,这里省略)可得
\begin{align*}
  m(B^\prime) = vol(B)
\end{align*}
即
\begin{align*}
  m(B) = vol(B)
\end{align*}

如果$a_j, b_j$是实数,
这里把下标$a_j, b_j$改为上标$a^j, b^j$,以方便接下来序列的表示。
对每一个分量$1 \leq j \leq n$,由实数的构造方式,我们可以构造一个
单调递减收敛于$a^j$的序列$(a_m^j)_{m = 1}^\infty$,
一个单调递增收敛于$b^j$序列$(b_m^j)_{m = 1}^\infty$,
并保证对任意$m$都有$a_m < b_m$。

令$A_k := \prod\limits_{j = 1}^n (a^j_k, b^j_k)$,
于是$A_k \subseteq A_{k + 1}$,
且我们有
\begin{align*}
  \bigcup\limits_{k = 1}^\infty A_k = \prod\limits_{j = 1}^n (a^j, b^j) \\
  \lim\limits_{k \to \infty} A_k = \prod\limits_{j = 1}^n (a^j, b^j)
\end{align*}
(可以通过任意$x \in \textbf{左侧}$都有$x \in \textbf{右侧}$来证明)

所以
\begin{align*}
  m(\bigcup\limits_{k = 1}^\infty A_k) & = m(\prod\limits_{j = 1}^n (a^j, b^j))                                  & (1) \\
  \lim\limits_{k \to \infty} m(A_k)    & = \lim\limits_{k \to \infty} m(\prod\limits_{j = 1}^n (a^j_k, b^j_k))         \\
                                       & = \lim\limits_{k \to \infty} vol(\prod\limits_{j = 1}^n (a^j_k, b^j_k))       \\
                                       & = \lim\limits_{k \to \infty} vol(A_k)                                         \\
                                       & = vol(\prod\limits_{j = 1}^n (a^j, b^j))                                & (2)
\end{align*}
利用18.2.3(a)可知
\begin{align*}
  m(\bigcup\limits_{k = 1}^\infty A_k) = \lim\limits_{k \to \infty} m(A_k)
\end{align*}
于是
\begin{align*}
  m(\prod\limits_{j = 1}^n (a^j, b^j)) = vol(\prod\limits_{j = 1}^n (a^j, b^j))  
\end{align*}
即
\begin{align*}
  m(B) = vol(B)
\end{align*}

\section*{18.2.6}

反证法,假设$\mathbb{R}$是可数的,
于是由例18.2.9可知
\begin{align*}
  m^{\ast}(\mathbb{R}) \leq \sum\limits_{r \in \mathbb{R}} m^{\ast}(\{r\}) = \sum\limits_{r \in \mathbb{R}} = 0 
\end{align*}
因为$[0, 1] \subseteq \mathbb{R}$,
由命题18.2.6可知,
\begin{align*}
  m^{\ast}([0, 1]) = 1 \leq m^{\ast}(\mathbb{R})
\end{align*}
存在矛盾。

\end{document}


