\documentclass{article}
\usepackage{mathtools} 
\usepackage{fontspec}
\usepackage[UTF8]{ctex}
\usepackage{amsthm}
\usepackage{mdframed}
\usepackage{xcolor}
\usepackage{amssymb}
\usepackage{amsmath}


% 定义新的带灰色背景的说明环境 zremark
\newmdtheoremenv[
  backgroundcolor=gray!10,
  % 边框与背景一致,边框线会消失
  linecolor=gray!10
]{zremark}{说明}


\begin{document}
\title{8.4 注释}
\author{张志聪}
\maketitle

\section*{公理 8.1(选择公理)}

一看到这个公理我就感到困惑,“为什么笛卡尔积的有限选取(引理3.5.12)不能自动推广到无限选取”。
而3.5.12中是对$n$进行了归纳的,是可以推广到无限自然数的。

无法推广的原因可能如下:

\begin{itemize}
  \item $I$不一定是可数集,此时数学归纳法失效。
  \item 循环论证的问题。书中8.4.3 说明了选择公理可以推导出一些$\textbf{非构造性}$的证明,而
        引理3.1.6就是选择公理的极端情况,而引理3.5.12的证明用到了引理3.1.6
\end{itemize}


\end{document}