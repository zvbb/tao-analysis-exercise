\documentclass{article}
\usepackage{mathtools} 
\usepackage{fontspec}
\usepackage[UTF8]{ctex}
\usepackage{amsthm}
\usepackage{mdframed}
\usepackage{xcolor}
\usepackage{amssymb}
\usepackage{amsmath}


% 定义新的带灰色背景的说明环境 zremark
\newmdtheoremenv[
  backgroundcolor=gray!10,
  % 边框与背景一致,边框线会消失
  linecolor=gray!10
]{zremark}{说明}


\begin{document}
\title{8.4 习题}
\author{张志聪}
\maketitle

\section*{8.4.1}

\begin{itemize}
  \item $\Rightarrow$ 书中的提示已经很明显了,这里不做证明了。
  \item $\Leftarrow$ 把$X$看做选择公理的集合$I$,由题设可知对每一个$\alpha \in I$,
        都至少存在一个$y \in Y$使得$P(\alpha,y)$为真,
        定义$X_{\alpha} := \{y \in Y : P(\alpha,y) \text{为真}\}$,显然,$X_{\alpha}$是非空的。

        接下来,要验证$\prod \limits_{\alpha \in I} X_\alpha$是非空的。

        由命题8.4.7可知,存在一个函数$f: I \rightarrow Y$使得$P(\alpha,f(\alpha))$对所有的$\alpha \in I$均成立。
        又由$X_{\alpha}$的定义可知$f(\alpha) \in X_{\alpha}$,于是,$f \in \prod \limits_{\alpha \in I} X_\alpha$,
        所以$\prod \limits_{\alpha \in I} X_\alpha$是非空的。
\end{itemize}

\section*{8.4.2}

\begin{itemize}
  \item $\Rightarrow$ 由题设与选择公理可知,
        $\prod \limits_{\alpha \in I} X_{\alpha}$是非空的,
        即:存在一个函数$f$对每一个$\alpha \in I$都指定了一个元素$f(\alpha) \in X_\alpha$。
        定义$Y := \{f(\alpha) : \alpha \in I\}$,此时$\#(Y \cap X_\alpha) = 1$。
        反证法,假设存在某个$\alpha$使得$\#(Y \cap X_\alpha) \neq 1$,即:集合$Y$中有多个元素
        属于$X_\alpha$,即存在$\alpha , \beta \in I$使得$f(\alpha), f(\beta) \in X_\alpha$,
        这与题设$X_\alpha \cap X_\beta = \varnothing$矛盾。
  \item $\Leftarrow$ 假设$I, X_\alpha$满足选择公理的前置条件,
        通过$\{\alpha\} \times X_\alpha = \{(\alpha, x) : x \in X_\alpha \}$替换$X_\alpha$,可以构造出
        一个不相交的集合簇$X_\alpha$,此时$\Rightarrow$的前置条件已满足,
        于是,可以找到一个集合$Y$使得$\#(Y \cap X_\alpha) = 1$,
        此时,我们可以定义一个函数$f$对每一个$\alpha \in I$都指定一个元素$(\alpha, x_\alpha) = Y \cap X_\alpha$,
        这个$x_\alpha \in X_\alpha$(这里是一开始的$X_\alpha$)
\end{itemize}

\section*{8.4.3}

\begin{itemize}
  \item $\Rightarrow$对每一个$\alpha \in A$,定义$X_\alpha := \{ x : x \in B, g(x) = \alpha\}$,
        由于$g$是满射,所以$X_\alpha$是非空的,由选择公理可知,存在一个函数$f$对每一个$\alpha \in A$
        都指定一个元素$x_\alpha \in X_\alpha$,因为$x_\alpha \in B$,所以$f$是$A \rightarrow B$的单射。

  \item $\Leftarrow$ 提示让我们利用习题8.4.2。这里显然是利用其逆命题,
        所以我们要找到满足8.4.2的前置条件的集合$Y$。

        公理8.1的前置条件中,显然是不满足对任意$\alpha, \beta \in I$都有$X_\alpha \cap X_\beta = \varnothing$,
        但可以像之前8.4.2中一样处理,通过$\{\alpha\} \times X_\alpha = \{(\alpha, x) : x \in X_\alpha \}$替换$X_\alpha$,
        此时的$X_\alpha$就是满足8.4.2的前置条件了。

        现在我们需要找到对任意$\alpha \in I$的满射函数$g_\alpha : X_\alpha \rightarrow \{\alpha\}$为
        \begin{align*}
          g_\alpha(x) = \alpha
        \end{align*}
        所以,由习题8.4.3找到一个单射$f_\alpha : \{\alpha\} \rightarrow X_\alpha$。

        现在定义集合$Y$如下:
        \begin{align*}
          Y := \{f_\alpha(\alpha) : \alpha \in I\}
        \end{align*}
        显然$\#(Y \cap X_\alpha) = 1$对所有的$\alpha \in I$均成立。

        既然$Y$已经构造出来,那么由8.4.2的逆命题,可以说明此时选择公理也成立。

\end{itemize}

\begin{zremark}
  8.4.2的逆命题,需要保证$Y$的存在性,
  为了避免循环论证,我们不能使用选择公理来说明$Y$的存在性,
  我们需要自己构造出$Y$,且不能使用选择公理构造(这里是利用习题8.4.3构造的)。
\end{zremark}

\end{document}
