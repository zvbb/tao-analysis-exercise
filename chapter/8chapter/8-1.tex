\documentclass{article}
\usepackage{mathtools} 
\usepackage{fontspec}
\usepackage[UTF8]{ctex}
\usepackage{amsthm}
\usepackage{mdframed}
\usepackage{xcolor}
\usepackage{amssymb}
\usepackage{amsmath}

\newmdtheoremenv[
  backgroundcolor=gray!10,
  linewidth=0pt,
  innerleftmargin=10pt,
  innerrightmargin=10pt,
  innertopmargin=10pt,
  innerbottommargin=10pt
]{zgraytheorem}{}
% 定义说明环境样式
\newtheoremstyle{mystyle}% 说明环境样式的名称
  {1em}% 上方间距
  {1em}% 下方间距
  {\normalfont}% 说明内容的字体样式
  {}% 缩进量
  {\bfseries}% 说明标记的字体样式
  {.}% 说明标记和说明内容之间的标点
  {1em}% 说明标记后的水平空间
  {}% 说明标记后的垂直空间
% 使用新定义的样式创建说明环境
\theoremstyle{mystyle}
\newtheorem*{zremark}{说明}


\begin{document}
\title{8.1 习题}
\maketitle

\section*{8.1.1}

$\bigstar \Leftarrow$

由命题3.6.14(c)知$X$不能为有限集,所以$X$是无限集(因为集合要么是无限的,要么是有限的)。


$\bigstar \Rightarrow$

\textbf{方法1}

在无限集$X$中,一定能取出一列互不相同的元素$a_1,a_2,…$。
事实上,在$X$中任取一个元素,记为$a_1$,
因为$X$是无限集,集合$X-\{a_1\}$显然不空,
这时再从集合$X-\{a_1\}$取一个元素$a_2$,
同样,$X-\{a_1, a_2\}$不会是空集,可以不停地做下去,
将从$X$中取出一列互不相同的元素$a_1,a_2,…$,记余集为$\hat{X} := X-\{a_1,a_2,…\}$。
在$X$中取出一个真子集
\begin{align*}
  Y =: \hat{X} \cup \{a_2,a_3,…\}
\end{align*}
定义函数$f: X \rightarrow Y$如下:
\begin{align*}
  f(a_i) & = a_{i+1}, a_i \in \{a_1,a_2,…\} \\
  f(x)   & = x, x \in \hat{X}
\end{align*}
显然$f$是双射,所以$X,Y$有相同的基数。

\textcolor{red}{注意}
方法1是非严格的证明,文中的“不停地”不够准确,引理8.5.14中有说明

\textbf{方法2}

todo

\section*{8.1.2}
(1)如果$X$是有限集合,则由自然数的三歧性,经过有限次比较,就可以得到最小元素存在。

(2)$X$是无限集

$\bigstar$ \textbf{最大下界方式}

因为$X$是自然数集的非空子集,那么对任意$x \in X, x \geq 0$,即:集合$X$有下界,
由定理5.5.9可知集合$X$有最大下界,不妨设为$m$。

现在需要证明$m \in X$。

反证法,假设$m \not \in X$。

由假设可知任意$x \in X, x > m$。
因为$m \geq \lfloor m \rfloor$(注4.4.2中$\lfloor m \rfloor$表示$m$的整数部分),
于是$x > \lfloor m \rfloor$,
由命题2.2.12(e)可知$x \geq \lfloor m \rfloor + 1$,
那么,$\lfloor m \rfloor + 1$也是$X$的下界,
而$\lfloor m \rfloor + 1 > m$,这与$m$是$X$的最大下界矛盾。

$\bigstar$ \textbf{无穷递降原理方式}

假设$X$没有最小元素,即:任意$x \in X$,存在$x^\prime \in X, x^\prime < x$。

现在构造出序列$(a_n)_{n=0}^\infty$。因为$X$是非空的,所以存在$x_0 \in X$,定义$a(0) := x_0$,
递归定义$a(n+1) := x_{n+1} (x_{n+1} < a(n))$,由之前的说明可知$x_{n+1}$是存在的。

显然这个序列与无穷递降原理矛盾。

$\bigstar \textbf{自然数替换成整数}$

“最大下界方式”的证明方式,显然对整数也是合适,所以替换成整数,良序定理对整数是成立。

$\bigstar \textbf{自然数替换成有理数}$

\section*{8.1.3}

$\bigstar \textbf{因为$X$是无限集,所以集合$\{x \in X: \text{对所有的$m < n$均有$x \neq a_m$}\}$也是无限集}$

设$A =: \{x \in X: \text{对所有的$m < n$均有$x \neq a_m$}\}$,
设$B =: \{a_m: i < n\}$。

反证法,假设$A$不是无限集。又$B$是有限集,
那么,由命题3.6.14(b)可知,$X$是有限集,这与$X$是无限集矛盾。

$\bigstar (a_n)_{n=0}^\infty \textbf{是一个递增序列} $

反证法,假设$(a_n)_{n=0}^\infty$不是递增序列。

由假设可知,存在$k, a_k \geq a_{k+1}$。

由序列的定义可知$a_{k+1} := min\{x \in X: \text{对所有的$m < k+1$均有$x \neq a_m$}\}$,
因为$k < k+1$,所以$a_k = a_{k+1}$是不存在的。

另外,$a_k > a_{k+1}$也是不可能的,因为$a_k$比$a_{k+1}$先定义,
由$min$函数的定义可知,先取出的$a_k$肯定小于$a_{k+1}$。

于是,与假设相悖。

$\bigstar \textbf{对所有的}n \neq m \textbf{均有} a_n \neq a_m$

由$(a_n)_{n=0}^\infty$严格递增保证。

$\bigstar a_n \in X$。

由$a_n$的定义方式保证。

$\bigstar \textbf{表明}$

由$n$的任意性保证的,因为如果存在$m$使得$a_m = x$,则前提不成立了。


$\bigstar a_n \geq n$。

对$n$进行归纳。

归纳基始,$n=0$,由于$a_n$是自然数,所以$a_n \geq 0$。

归纳假设,$n=k$,$a_k \geq k$。

当$n=k+1$时,因为$a_n$是一个递增序列,所以$a_{k+1} > a_k$,即$a_{k+1} > k$,
由命题2.2.12(e)可知$a_{k+1} \geq k+1$。

归纳完成。

$\bigstar \textbf{必然有?}$

良序定理保证最小值的唯一性。

\section*{8.1.4}

有一点需要注意,值域$f(N)$可能不是$Y$,但在以下的证明中,可以把$Y$视为值域$f(N)$。

$\bigstar$ $Y$是有限集时,命题显然成立。

$\bigstar$ $Y$是无限集时。

通过提示可知,$f$是从$A$到$f(A)$的双射是显然的。

下面我们需要证明的是$f(A)$与$Y$是相等的集合,那个$Y$就是可数的了。

由$f(A)$的定义方式可知,$f(A)$中的元素,一定属于$Y$。

对任意$y \in Y$,存在自然数集合$B$,任意$b \in B$使得$f(b) = y$,
取$B$中的最小值$n$,现在需要证明$n \in A$。

对$n$进行强归纳。

归纳基始,$n = 0$,因为不存在自然数$m < 0$,所以$0 \in A$。

归纳假设,$n \leq k$时,$n \in A$。

$n = k+1$时,反证法,假设$k + 1 \neq A$,那么,存在$m < k+1$使得$f(m) = f(k+1)$,
此时,$m \in B$,且$m < k+1$,与$n$是$B$中的最小值矛盾,于是$k + 1 \in A$。

归纳完成。

所以,$n \in A$,即:$y \in f(A)$。

\section*{8.1.5}

因为$X$是可数集,那么存在一个双射$g: \mathbb{N} \rightarrow X$。

所以,由定义3.3.10(复合)可以构造出函数$f \circ g: \mathbb{N} \rightarrow Y$,
由命题8.1.8可知$f \circ g(\mathbb{N})$是至多可数的。

又因为$f(X)$与$f \circ g(\mathbb{N})$是相等的集合(证明略),于是$f(X)$是至多可数的。

\section*{8.1.6}

$\bigstar \Rightarrow$

$A$是至多可数的,如果$A$是有限集,则结论是显然的。

如果$A$是可数集,那么存在一个双射$g: A \rightarrow \mathbb{N}$,
令$f = g$。

$\bigstar \Leftarrow$

(1)如果$f(A)$是有限集,则结论是显然的。

(2)如果$f(A)$是无限集,则需要进一步证明。

因为$f$是单射,那么存在$f(A) \subseteq \mathbb{N}$。

因为$A$不能是空集,存在元素$a_0 \in A$。

现在定义函数$h: \mathbb{N} \rightarrow A$为
\begin{align*}
  h(n) & := f^{-1}(n) & \{n \in \mathbb{N}: n \in f(A) \}      \\
  h(n) & := a_0       & \{n \in \mathbb{N}: n \not \in f(A) \} \\
\end{align*}

由命题8.1.8可知,$A$是至多可数的的。
\section*{8.1.7}

$h(\mathbb{N}) = X \cup Y$是显然的(证明略),由命题8.1.8可知$X \cup Y$是至多可数的。

假设$X \cup Y$是有限集,那么由命题3.6.14(c)可知$X$是有限的,这与题设矛盾。

\section*{8.1.8}

由推论8.1.9可知,只要找到一个函数$f: \mathbb{N} \times \mathbb{N} \rightarrow X \times Y$,
那么,$f(\mathbb{N} \times \mathbb{N})$是至多可数的。
然后只需说明$X \times Y = f(\mathbb{N} \times \mathbb{N})$。那么,$X \times Y$也是至多可数的。

因为$X$是可数集,所以存在一个双射函数$g: \mathbb{N} \rightarrow X$;

同理,存在一个双射函数$h: \mathbb{N} \rightarrow Y$;

现在定义函数$f: \mathbb{N} \times \mathbb{N} \rightarrow X \times Y$为
\begin{align*}
  f(n,m) := (g(n), h(m))
\end{align*}

接下来,要证明$X \times Y = f(\mathbb{N} \times \mathbb{N})$。

反证法,假设$X \times Y \neq f(\mathbb{N} \times \mathbb{N})$,
那么,至少存在一个元素$(x,y) \in X \times Y$且$(x,y) \not \in f(\mathbb{N} \times \mathbb{N})$。
因为$x \in X, y\in Y$,所以存在$(n,m)$使得$x = g(n), y=h(m)$,
又因为$(n,m) \in \mathbb{N} \times \mathbb{N}$,
而$(g(n),h(m)) \in f(\mathbb{N} \times \mathbb{N})$,即:
$(x,y) \in f(\mathbb{N} \times \mathbb{N})$,与假设矛盾。


\section*{8.1.9}

todo 


\section*{8.1.10}
没找到!

\end{document}