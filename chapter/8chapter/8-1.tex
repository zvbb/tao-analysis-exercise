\documentclass{article}
\usepackage{mathtools} 
\usepackage{fontspec}
\usepackage[UTF8]{ctex}
\usepackage{amsthm}
\usepackage{mdframed}
\usepackage{xcolor}
\usepackage{amssymb}
\usepackage{amsmath}

\newmdtheoremenv[
  backgroundcolor=gray!10,
  linewidth=0pt,
  innerleftmargin=10pt,
  innerrightmargin=10pt,
  innertopmargin=10pt,
  innerbottommargin=10pt
]{zgraytheorem}{}
% 定义说明环境样式
\newtheoremstyle{mystyle}% 说明环境样式的名称
  {1em}% 上方间距
  {1em}% 下方间距
  {\normalfont}% 说明内容的字体样式
  {}% 缩进量
  {\bfseries}% 说明标记的字体样式
  {.}% 说明标记和说明内容之间的标点
  {1em}% 说明标记后的水平空间
  {}% 说明标记后的垂直空间
% 使用新定义的样式创建说明环境
\theoremstyle{mystyle}
\newtheorem*{zremark}{说明}


\begin{document}
\title{8.1 习题}
\maketitle

\section*{8.1.1}

$\bigstar \Leftarrow$

由命题3.6.14(c)知$X$不能为有限集,所以$X$是无限集(因为集合要么是无限的,要么是有限的)。


$\bigstar \Rightarrow$

\textbf{方法1}

在无限集$X$中,一定能取出一列互不相同的元素$a_1,a_2,…$。
事实上,在$X$中任取一个元素,记为$a_1$,
因为$X$是无限集,集合$X-\{a_1\}$显然不空,
这时再从集合$X-\{a_1\}$取一个元素$a_2$,
同样,$X-\{a_1, a_2\}$不会是空集,可以不停地做下去,
将从$X$中取出一列互不相同的元素$a_1,a_2,…$,记余集为$\hat{X} := X-\{a_1,a_2,…\}$。
在$X$中取出一个真子集
\begin{align*}
  Y =: \hat{X} \cup \{a_2,a_3,…\}
\end{align*}
定义函数$f: X \rightarrow Y$如下:
\begin{align*}
  f(a_i) & = a_{i+1}, a_i \in \{a_1,a_2,…\} \\
  f(x)   & = x, x \in \hat{X}
\end{align*}
显然$f$是双射,所以$X,Y$有相同的基数。

\textcolor{red}{注意}
方法1是非严格的证明,文中的“不停地”不够准确,引理8.5.14中有说明

\textbf{方法2}

todo 还未找到

\section*{8.1.2}
(1)如果$X$是有限集合,则由自然数的三歧性,经过有限次比较,就可以得到最小元素存在。

(2)$X$是无限集

$\bigstar$ \textbf{最大下界方式}

因为$X$是自然数集的非空子集,则$x \in X, x \leq 0$,即:集合$X$有下界,
由定理5.5.9可知集合$X$有最大下界,不妨设为$m$。

现在需要证明$m \in X$。

反证法,假设$m \not \in X$。

由假设可知任意$x \in X, x > m$。
因为$m \geq \lfloor m \rfloor$(注4.4.2中$\lfloor m \rfloor$表示$m$的整数部分),
于是$x > \lfloor m \rfloor$,
由命题2.2.12(e)可知$x \geq \lfloor m \rfloor + 1$,
那么,$\lfloor m \rfloor + 1$也是$X$的下界,
而$\lfloor m \rfloor + 1 > m$,这与$m$是$X$的最大下界矛盾。

$\bigstar$ \textbf{无穷递降原理方式}

假设$X$没有最小元素,即:任意$x \in X$,存在$x^\prime \in X, x^\prime < x$。

现在构造出序列$(a_n)_{n=0}^\infty$。因为$X$是非空的,所以存在$x_0 \in X$,定义$a(0) := x_0$,
递归定义$a(n+1) := x_{n+1} < a(n)$,由之前的说明可知$x_{n+1}$是存在的。

显然这个序列与无穷递降原理矛盾。

\end{document}