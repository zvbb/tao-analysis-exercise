\documentclass{article}
\usepackage{mathtools} 
\usepackage{fontspec}
\usepackage[UTF8]{ctex}
\usepackage{amsthm}
\usepackage{mdframed}
\usepackage{xcolor}
\usepackage{amssymb}
\usepackage{amsmath}


% 定义新的带灰色背景的说明环境 zremark
\newmdtheoremenv[
  backgroundcolor=gray!10,
  % 边框与背景一致,边框线会消失
  linecolor=gray!10
]{zremark}{说明}


\begin{document}
\title{8.3 习题}
\author{张志聪}
\maketitle

\section*{8.3.1}

对$n$进行归纳。

归纳基始,$n=0$,此时,$X$是空集,$\#(X) = 0$,$2^0=1$,这与空集的子集只有它本身是一致的。

归纳假设,$n=k$时,$\#(2^{X}) = 2^k$。

当$n=k+1$时, 在$X$中任取一个元素$x_0$,此时,设$X^\prime = X \setminus \{x_0\}$。
对$2^{X}$的任意子集$A$:
\begin{itemize}
  \item 如果$x_0 \not \in A$,此时$A \subseteq 2^{X^\prime}$,由归纳假设可知,这样的子集有$2^k$个。
  \item 如果$x_0 \in A$,定义$A^\prime := A \setminus \{x_0\}$,显然$A^\prime \subseteq 2^{X^\prime}$,
        因为$A^\prime$有$2^k$个,所以$A^\prime \cup \{x_0\}$有$2^k$。
\end{itemize}
综上,$2^k + 2^k = 2^{k+1}$。

\section*{8.3.2}

\begin{zremark}
  一开始,觉得题目不对!
  理由如下:由题设,$A \subseteq C$且单射$f: C \rightarrow A$可知,
  $f(C)$与$C$是双射,而$f(C) \subseteq A$,所以只有$C=A$才能满足题设,进而$A=B=C$。
  那么,$D_0 = B \setminus A = \varnothing$,就没有证明的必要了。

  问题出在对习题3.6.7的理解上了,这里只能证明$\#(A) = \#(B) = \#(C)$,而无法证明$A = B = C$,
  举一个反例,自然数$N$与偶数集合的基数相等,也可以构建一个单射,
  但不妨碍偶数集合是自然数子集这一事实。
\end{zremark}

(1)命题与$D_{n} \cap D_{n+1} = \varnothing$等价。对$n$进行归纳。

归纳基始,$n = 0$时,$D_0 := B \setminus A, D_1 := f(D_0)$。
反证法,假设$D_0 \cap D_1 \neq \varnothing $,由题设可知$D_1 \subseteq A$,
因为$D_0 \cap D_1 \neq \varnothing $,所以存在元素$x \in D_0, D_1, A$,
这与$D_0 := B \setminus A$矛盾。

归纳假设,$n = k$时,命题$D_{k} \cap D_{k+1} = \varnothing$成立。

当$n = k+1$时,$D_{k+2} := f(D_{k+1})$。
反证法,假设$D_{k+2} \cap D_{k+1} \neq \varnothing$,
即存在$d_0 \in D_{k+2}, D_{k+1}$,又因为$D_{k+1} = f(D_k)$,
于是,存在$x_0, x_1$使得
\begin{equation*}
  \begin{cases*}
    d_0 = f(x_0) \text{ 其中} x_0 \in D_k, f(x_0) \in D_{k+1} \\
    d_0 = f(x_{1}) \text{ 其中}  x_{1} \in D_{k+1}, f(x_1) \in D_{k+2}
  \end{cases*}
\end{equation*}
由归纳假设可知$x_0 \neq x_1$,这与$f$是单射的矛盾。

(2)
\begin{itemize}
  \item 单射;函数$g$的定义域被定义成两个部分,各自显然是单射的,现在要证明两个部分的值域没有交集。
        反证法,假设存在$x_0 \in \bigcup \limits_{n=0}^\infty D_n, x_1 \not \in \bigcup \limits_{n=0}^\infty D_n, x_0 \neq x_1$,
        使得$g(x_0) = g(x_1)$,即:$f^{-1}(x_0) = x_1$,$f(x_1) = x_0$。

        因为$x_0 \in \bigcup \limits_{n=0}^\infty D_n$,
        所以存在$x^\prime \in \bigcup \limits_{n=0}^\infty D_n$使得$f(x^\prime) = x_0$。
        因为$x_1 \not \in \bigcup \limits_{n=0}^\infty D_n$,所以$x^\prime \neq x_1$,
        于是$f(x^\prime) = f(x_1)$,这与$f$是单射矛盾。


  \item 满射;对任意$y \in B$,
        如果$y \in \bigcup \limits_{n=0}^\infty D_n$,由$f$的定义可知,
        $f(y) \in A, f(y) \in \bigcup \limits_{n=0}^\infty D_n$,
        满足$g$的定义,于是$f^{-1}(f(y)) = y$;
        如果$y \not \in \bigcup \limits_{n=0}^\infty D_n$,有$g(y) = y$;
\end{itemize}

\section*{8.3.3}

\begin{zremark}
  这个习题,容易觉得无需证明,
  \begin{equation*}
    \begin{cases*}
      \#(A) \leq \#(B) \\
      \#(A) \geq \#(B)
    \end{cases*}
  \end{equation*}
  得到$\#(A) = \#(B)$是显然。
  但是,这里是把基数看做自然数了,这个假设不成立的。
  自然数集合的基数明显不是自然数。

  于是,我们只能老老实实利用基数相等的定义证明了。
\end{zremark}

目标是构造一个双射函数$g: A \rightarrow B$。利用习题8.3.2,只需找到满足条件的$A,B,C$与$f$即可。
不妨设$f_1: A \rightarrow B$的单射,$f_2: B \rightarrow A$的单射。
显然,$A,f_1(A)$的基数相同,$f_1(A) \subseteq B$,于是,如下定义,即可满足条件:
\begin{equation*}
  \begin{cases*}
    A := f_1(A) \\
    B := B      \\
    C := B      \\
    f := f_2
  \end{cases*}
\end{equation*}

\end{document}