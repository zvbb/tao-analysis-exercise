\documentclass{article}
\usepackage{mathtools} 
\usepackage{fontspec}
\usepackage[UTF8]{ctex}
\usepackage{amsthm}
\usepackage{mdframed}
\usepackage{xcolor}
\usepackage{amssymb}
\usepackage{amsmath}


% 定义新的带灰色背景的说明环境 zremark
\newmdtheoremenv[
  backgroundcolor=gray!10,
  % 边框与背景一致,边框线会消失
  linecolor=gray!10
]{zremark}{说明}


\begin{document}
\title{8.2 习题}
\author{张志聪}
\maketitle

\section*{8.2.1}

令
\begin{align*}
  S = \left \{ \sum \limits_{x \in A} |f(x)|: A \subseteq X , A \text{是有限集} \right \}
\end{align*}

\begin{itemize}
  \item $\Rightarrow$
        如果$X$是有限集,这是命题是显然的;

        如果$X$是可数集。
        因为$X$是可数集,那么存在双射函数$g: \mathbb{N} \rightarrow X $,
        又因为级数$\sum \limits_{x \in X}f(x)$绝对收敛,那么
        \begin{align*}
          \sum \limits_{x \in X} |f(x)| = \sum \limits_{n=0}^\infty |f(g(n))|
        \end{align*}

        任意元素$e \in S, e = \sum \limits_{x \in A} |f(x)|: A \subseteq X$,所以
        有一个有限集$N^\prime \subseteq \mathbb{N}, A = g(N^\prime)$,
        因为$N^\prime$是有限集,所有存在自然数$k$,使得$max(N^\prime) \leq k$,于是,
        \begin{align*}
          e = \sum \limits_{x \in A} |f(x)| = \sum \limits_{n \in N^\prime} |f(g(n))| \leq \sum \limits_{n=0}^k |f(g(n))| \leq \sum \limits_{n=0}^\infty |f(g(n))|
        \end{align*}
        所以,$e$是有限的,由$e$的任意性可知,集合$S$有上界,
        所以,
        \begin{align*}
          \sup \left \{ \sum \limits_{x \in A} |f(x)|: A \subseteq X , A \text{是有限集} \right \} < \infty
        \end{align*}
  \item $\Leftarrow$
        反证法,假设级数$\sum \limits_{x \in X}f(x)$绝对发散。

        因为
        \begin{align*}
          \sup \left \{ \sum \limits_{x \in A} |f(x)|: A \subseteq X , A \text{是有限集} \right \} < \infty
        \end{align*}
        设$\sup S \leq M$。因为$\sum \limits_{x \in X}f(x)$绝对发散,
        所以存在自然数$N$使得
        \begin{align*}
          \sum \limits_{n=0}^N |f(g(n))| > M
        \end{align*}
        令$A = g({n \in \mathbb{N}: 0 \leq n \leq N})$,因为$A$是有限集,且$A \in S$,
        所以
        \begin{align*}
          \sum \limits_{x \in A} |f(x)| = \sum \limits_{n=0}^N |f(g(n))| \leq M
        \end{align*}
        存在矛盾。
\end{itemize}

\section*{8.2.2}

这道题没有证明的必要了,提示就是一个简要的证明了。

\section*{8.2.3}

\begin{zremark}
  当X是不可数集时,
  由于$\sum \limits_{x \in X}f(x) := \sum \limits_{x \in X: f(x)\not = 0}f(x)$,
  所以,由引理8.2.5 可知,$\{x \in X: f(x)\not = 0\}$是至多可数的,
  于是证明时只需说明至多可数的情况即可。

  只讨论$X$可数集,$X$是有限集,命题7.2.14已经覆盖。
\end{zremark}

因为$\sum \limits_{x \in X} f(x)$,$\sum \limits_{x \in X} g(x)$绝对收敛,
所以存在某个双射$h: \mathbb{N} \rightarrow X$,使得
$\sum \limits_{n=0}^\infty f(h(n))$与
$\sum \limits_{n=0}^\infty g(h(n))$是绝对收敛点的,且
\begin{align*}
  \sum \limits_{x \in X} f(x) = \sum \limits_{n=0}^\infty f(h(n)) \\
  \sum \limits_{x \in X} g(x) = \sum \limits_{n=0}^\infty g(h(n))
\end{align*}

\begin{itemize}
  \item (a)
        由定义8.2.1可知,$\sum \limits_{n=0}^\infty \big(f(h(n))+g(h(n))\big)$绝对收敛,
        可以说明$\sum \limits_{x \in X} \big(f(x)+g(x)\big)$是绝对收敛的。

        因为$\sum \limits_{n=0}^\infty f(h(n))$与$\sum \limits_{n=0}^\infty g(h(n))$绝对收敛,
        由命题7.2.14(a)可知,$\sum \limits_{n=0}^\infty |f(h(n))|+|g(h(n))|$收敛,
        设其收敛与$M$,又因为
        \begin{align*}
          |f(n) + g(n)| \leq |f(n)| + |g(n)|
        \end{align*}
        于是由命题7.3.1可知,$\sum \limits_{n=0}^\infty |f(h(n))+g(h(n))|$收敛,
        所以$\sum \limits_{x \in X} \big(f(x)+g(x)\big)$是绝对收敛的。

        \begin{align*}
          \sum \limits_{x \in X} \big(f(x)+g(x)\big) = \sum \limits_{x \in X} f(x) + \sum \limits_{x \in X} g(x) \\
        \end{align*}
        该公式的所有项都是绝对收敛的,也就是说其也是条件条件收敛的,由命题7.2.14(a)保证了该公式的正确性。
  \item (b)$\sum \limits_{x \in X} f(x)$绝对收敛,
        可知$\sum \limits_{n=0}^\infty f(h(n))$绝对收敛于某个实数$M$。
        $\sum \limits_{n=0}^\infty |cf(h(n))|$的部分和$S_{Nc} \leq |c|M$,命题7.3.1保证了该级数收敛。

        \begin{align*}
          \sum \limits_{x \in X} cf(x) = c \sum \limits_{x \in X} f(x)
        \end{align*}
        由命题7.2.14(b)可知保证。

  \item (c)$\Rightarrow$:$X_1,X_2$一定有一个是可数集,不妨设$X_1$是可数集。
        反证法,假设$\sum \limits_{x \in X_{1}} f(x)$绝对发散,
        不妨设$\sum \limits_{x \in X} f(x)$绝对收敛于$M$。
        因为$X_1$是可数集,所以存在某个双射$h_1: \mathbb{N} \rightarrow X_1$,
        由于$\sum \limits_{x \in X_{1}} f(x)$绝对发散,所以存在一个整数$N$使得
        \begin{align*}
          \sum \limits_{n=0}^N |f(h_1(n))| > M
        \end{align*}
        因为$X_1$是$X$的子集,所以有限集$Y := \{x \in X_1, f(x) \leq N\}$也是$X$的子集,
        取$m = max(h^{-1}(Y))$,此时,
        \begin{align*}
          \sum \limits_{n=0}^m |f(h(n))| > M
        \end{align*}
        存在矛盾。

        至于等式,证明起来,不是那么简单,因为这里的集合是不一致的,接下来的证明注意对集合的处理。

        $\circ$ 如果$X_2$也是可数集,那么存在双射
        $h_1: \mathbb{N} \rightarrow X_1$,
        $h_2: \mathbb{N} \rightarrow X_2$,
        定义$h: \mathbb{N} \rightarrow X$如下:
        \begin{equation}
          \begin{cases*}
            h(2n) = h_1(n) \\
            h(2n+1) = h_2(n)
          \end{cases*}
        \end{equation}
        以上定义的$h$是双射。因为
        \begin{align*}
           & \lim \limits_{N \rightarrow \infty} \sum \limits_{n=0}^N \big(f(h_1(n)) + f(h_2(n))\big) \\
           & = \lim \limits_{N \rightarrow \infty} \sum \limits_{n=0}^N  f(h(n))                      \\
          %  & = \lim \limits_{N \rightarrow \infty} \sum \limits_{n=0}^N f(h_1(n)) + \lim \limits_{N \rightarrow \infty} \sum \limits_{n=0}^N f(h_2(n)) \\
        \end{align*}
        于是,
        \begin{align*}
          \sum \limits_{x \in X_1} f(x) + \sum \limits_{x \in X_2} f(x) = \sum \limits_{x \in X} f(x)
        \end{align*}

        $\circ$如果$X_2$是有限集,设基数是$m$,定义$Y := \{i \in \mathbb{N} : i < m\}$,存在双射
        $h_1: \mathbb{N} \setminus Y \rightarrow X_1$
        \begin{zremark}
          因为$X_1$是可数集,则存在双射$h_1: \mathbb{N} \rightarrow X_1$,
          又因为$\sum \limits_{x \in X_1} f(x) $是绝对收敛的,所以
          \begin{align*}
             & \sum \limits_{x \in X_1} f(x)           \\
             & = \sum \limits_{n=0}^\infty f(h_1(n))   \\
             & = \sum \limits_{n=m}^\infty f(h_1(n-m)) \\
          \end{align*}
          第二个等式的成立可以通过部分和序列的相等证明,思路如下:
          设$\sum \limits_{n=0}^\infty f(h_1(n))$、$\sum \limits_{n=m}^\infty f(h_1(n-m))$
          的部分和分别为$S_N, S_{N}^\prime$。由$(S_N)_{N=0}^\infty$序列的收敛性,
          并由命题7.2.5 可知,对任意$\epsilon > 0$,存在一个整数$K, n \geq K+m$,使得
          \begin{align*}
             & |S_n - S_{n}^\prime|                                                 \\
             & = |\sum \limits_{n=p}^{p+m} f(h_1(n))| \leq \epsilon & \text{p大于等于K} \\
          \end{align*}
          这里的$n \geq K + m$的原因是,$S_0 = S_m^\prime$,所以$S_n$比$S_n^\prime$
          多$m$个项,所以这$m$个项必须都是大于$K$的项,相加才会小于等于$\epsilon$,
          所有两者的部分和是最终$-\epsilon$相等的,所以收敛于同一个值。
        \end{zremark}

        $h_2: Y \rightarrow X_2$,
        定义$h: \mathbb{N} \rightarrow X$如下:
        \begin{equation}
          \begin{cases*}
            h(n) = h_1(n-m) \text{  if  } n \geq m \\
            h(n) = h_2(n) \text{  if  } n < m
          \end{cases*}
        \end{equation}

        以上定义的$h$都是双射。因为
        \begin{align*}
           & = \sum \limits_{n=0}^{m-1} f(h_2(n)) + \lim \limits_{N \rightarrow \infty} \sum \limits_{n=m}^N f(h_1(n)) \\
           & = \lim \limits_{N \rightarrow \infty} \sum \limits_{n=0}^N  f(h(n))
        \end{align*}
        于是,
        \begin{align*}
          \sum \limits_{x \in X_1} f(x) + \sum \limits_{x \in X_2} f(x) = \sum \limits_{x \in X} f(x)
        \end{align*}
        $\Leftarrow$ 绝对收敛的证明与上一个等式的等式一致(上个等式的证明不需要其是绝对收敛的),这里说一下思路:
        因为,
        \begin{align*}
           & \sum \limits_{x \in X_1} |h(x)| + \sum \limits_{x \in X_2} |h(x)| = \sum \limits_{x \in X} |h(x)| \\
        \end{align*}
        证明方法与条件收敛的等式一致,因为,
        $\sum \limits_{x \in X_1} |h(x)|,\sum \limits_{x \in X_2} |h(x)|$都是收敛的,所以
        $\sum \limits_{x \in X} |h(x)|$也是收敛的。
  \item (d)是命题7.1.11(c)的扩展,证明方式也是一致的,都是利用级数的值与双射的选取无关,具体证明略。
        % 因为$X$是可数集,$\phi : Y \rightarrow X$是一个双射,所以$Y$与$X$的基数是相同的,
        % 于是$Y$也是可数集。
\end{itemize}

\section*{8.2.4}
\begin{zremark}
  $\sum \limits_{n \in A_{+}} a_n$的含义是什么,因为$A_{+}$是一个集合,
  也就意味着其中的元素是没有顺序的,那么,$\sum \limits_{n \in A_{+}} a_n$收敛,
  是否任意顺序的求和都是收敛的?

  回顾书中定义 8.2.1是没有明确说明的,我个人推测是任意顺序的收敛(如果能找到佐证,我会回来纠正)
  。因为存在异议,所以接下来的证明,没有对顺序进行要求(或假设),
  直观感受就是:你爱什么顺序就什么顺序,我们只利用收敛与发散的性质。

  当然这里的$A_{+}$具有特殊性,因为$a_n: n \in A_{+}$都是非负数,
  只要存在某个顺序的求和是收敛的,
  也就意味着是绝对收敛的,命题7.4.3(级数的重排列)进而保证了,任意顺序求和也是收敛的。

  $A_{-}$也具有特殊性,有命题7.2.14(b)也能保证,其是绝对收敛的,从而再次利用命题7.4.3(级数的重排列),
  可以说明其任意顺序求和也是收敛的。
\end{zremark}

通过说明可知,这里的条件收敛与绝对收敛是一致的。

\begin{itemize}
  \item  如果两个级数都是收敛的,则由命题8.2.6(绝对收敛级数的定律)可知,
        $\sum \limits_{n=0}^\infty a_n$也是绝对收敛的,与题设矛盾。
  \item  如果只有一个是收敛的,也就意味着另一个是发散的。
        这里不妨设$\sum \limits_{n \in A_{+}}$是收敛于$M$。
        那么,对任意的$\epsilon > 0$,存在整数$N_{+}$使得$n \geq N_{+}$部分和
        \begin{align*}
          |S_{n}^{+} - M| \leq \epsilon
        \end{align*}
        也就是说$S_{n}^{+}$是一个有限值,
        而$\sum \limits_{n \in A_{-}} a_n$是发散的,其部分和$S_{n}^{-}$可以小于任何实数。
        即对任意实数$r$,存在$n$使得
        \begin{align*}
          |S_{n}^{-} + S_{n}^{+}| \geq r
        \end{align*}
        与题设$\sum \limits_{n=0}^\infty$收敛矛盾。
\end{itemize}

\section*{8.2.5}

\begin{itemize}
  \item 因为如果是有限集,则必定是条件收敛的,与命题8.2.7矛盾。
  \item 与命题8.2.7矛盾。
  \item 单射由$n_j$的定义保证的。【感觉没啥好说的啊,或者是我没读懂题】
  \item 反证法,假设情形$I$出现了有限次,不妨设$K$后情形$I$不再满足,
        即:$i = K$时,$\sum \limits_{0 \leq i < K} a_{n_i} = M \geq L$,
        此时,$M = M_{+} + M_{-}$,其中$M_{+}$是$A_{+}$中元素$j$对应的$a_j$求和,
        其中$M_{-}$是$A_{-}$中元素$j$对应的$a_j$求和。
        因为$\sum \limits_{n \in A_{-}} a_{n}$是发散的,存在$N_{-}$,当$n \geq N_{-}$使得
        其部分和$S_{n}^{-} < L - M^{+}$,此时,$M^{+} + S_{n}^{-} < L$,情形$II$不再满足。
  \item 满射由$n_j$的定义保证的。【感觉没啥好说的啊,或者是我没读懂题】
  \item 由推论7.2.6 可知 $\lim \limits_{n \rightarrow \infty} a_n = 0$,
        即对任意$\epsilon > 0$,都存在整数$N, n \geq N$使得$|a_n - 0| \leq \epsilon$。

        因为情形$I$与情形$II$的都是无限多次的,且在$A_{+},A_{-}$都是以递增的方式取出的,
        所以$a_{j+}, a_{j-}$最终都会大于$N$,于是$\lim \limits_{j \rightarrow \infty} a_{n_j} = 0$。
  \item 说明其是柯西序列来证明其收敛性(定理6.4.18)??? 翻转
\end{itemize}

\section*{8.2.6}

$\sum \limits_{n \in A_{+}} a_n$就是发散的,所以其是发散到$+\infty$

\end{document}