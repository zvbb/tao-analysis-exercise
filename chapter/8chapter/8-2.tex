\documentclass{article}
\usepackage{mathtools} 
\usepackage{fontspec}
\usepackage[UTF8]{ctex}
\usepackage{amsthm}
\usepackage{mdframed}
\usepackage{xcolor}
\usepackage{amssymb}
\usepackage{amsmath}

\newmdtheoremenv[
  backgroundcolor=gray!10,
  linewidth=0pt,
  innerleftmargin=10pt,
  innerrightmargin=10pt,
  innertopmargin=10pt,
  innerbottommargin=10pt
]{zgraytheorem}{}
% 定义说明环境样式
\newtheoremstyle{mystyle}% 说明环境样式的名称
  {1em}% 上方间距
  {1em}% 下方间距
  {\normalfont}% 说明内容的字体样式
  {}% 缩进量
  {\bfseries}% 说明标记的字体样式
  {.}% 说明标记和说明内容之间的标点
  {1em}% 说明标记后的水平空间
  {}% 说明标记后的垂直空间
% 使用新定义的样式创建说明环境
\theoremstyle{mystyle}
\newtheorem*{zremark}{说明}


\begin{document}
\title{8.2 习题}
\maketitle

\section*{8.2.1}

令
\begin{align*}
  S = \left \{ \sum \limits_{x \in A} |f(x)|: A \subseteqq X , A \text{是有限集} \right \}
\end{align*}

$\bigstar \Rightarrow$

如果$X$是有限集,这是命题是显然的。

如果$X$是可数集。
因为$X$是可数集,那么存在双射函数$g: \mathbb{N} \rightarrow X $,
又因为级数$\sum \limits_{x \in X}f(x)$绝对收敛,那么
\begin{align*}
  \sum \limits_{x \in X} |f(x)| = \sum \limits_{n=0}^\infty |f(g(n))|
\end{align*}

任意元素$e \in S, e = \sum \limits_{x \in A} |f(x)|: A \subseteqq X$,所以
有一个有限集$N^\prime, A = g(N^\prime)$,因为$N^\prime$是有限集,所有存在自然数$k$,
使得$max(N^\prime) \leq k$,于是,
\begin{align*}
  e = \sum \limits_{x \in A} |f(x)| = \sum \limits_{n \in N^\prime} |f(g(n))| \leq \sum \limits_{n=0}^k |f(g(n))| \leq \sum \limits_{n=0}^\infty |f(g(n))| 
\end{align*}
所以,$e$是有限的,由$e$的任意性可知,集合$S$有上界,
所以,
\begin{align*}
  \sup \left \{ \sum \limits_{x \in A} |f(x)|: A \subseteqq X , A \text{是有限集} \right \} < \infty
\end{align*}

$\bigstar \Leftarrow$

反证法,假设级数$\sum \limits_{x \in X}f(x)$绝对发散。

因为
\begin{align*}
  \sup \left \{ \sum \limits_{x \in A} |f(x)|: A \subseteqq X , A \text{是有限集} \right \} < \infty
\end{align*}
设$\sup S \leq M$。因为$\sum \limits_{x \in X}f(x)$绝对发散,
所以存在自然数$N$使得
\begin{align*}
  \sum \limits_{n=0}^N |f(g(n))| > M
\end{align*}
令$A = g({n \in \mathbb{N}: 0 \leq n \leq N})$,因为$A$是有限集,且$A \in S$,
所以
\begin{align*}
  \sum \limits_{x \in A} |f(x)| = \sum \limits_{n=0}^N |f(g(n))| \leq M
\end{align*}
存在矛盾。

\end{document}