\documentclass{article}
\usepackage{mathtools} 
\usepackage{fontspec}
\usepackage[UTF8]{ctex}
\usepackage{amsthm}
\usepackage{mdframed}
\usepackage{xcolor}
\usepackage{amssymb}
\usepackage{amsmath}


% 定义新的带灰色背景的说明环境 zremark
\newmdtheoremenv[
  backgroundcolor=gray!10,
  % 边框与背景一致,边框线会消失
  linecolor=gray!10
]{zremark}{说明}


\begin{document}
\title{8.5 习题}
\author{张志聪}
\maketitle

这一节题太多了,我只写正文中提到的习题了。

\section*{8.5.3}

证明是偏序集。
\begin{itemize}
  \item (自反性)因为对任意的正整数$x$都有$x = x \times 1$,所以$x | x$
  \item (反对称性)如果正整数$x, y$满足$x | y$且$y | x$,那么存在正整数$a,b$使得
        \begin{equation*}
          \begin{cases*}
            y = x \times a \\
            x = y \times b
          \end{cases*}
        \end{equation*}
        $\Rightarrow$
        \begin{equation*}
          \begin{cases*}
            a = 1 \\
            b = 1
          \end{cases*}
        \end{equation*}
        于是$x = y$
  \item (传递性)如果正整数$x, y, z$满足$x | y$且$y | z$,那么存在正整数$a,b$使得
        \begin{equation*}
          \begin{cases*}
            y = x \times a \\
            z = y \times b
          \end{cases*}
        \end{equation*}
        $\Rightarrow$
        \begin{equation*}
          \begin{cases*}
            z = x \times a \times b
          \end{cases*}
        \end{equation*}
        于是$x | z$
\end{itemize}

证明不是全序集,举一个反例即可,正整数$2,3$是不满足$2 | 3$或$3 | 2$的,因为不存在正整数$a$使得
$2 = 3a$或$3 = 2a$。

\section*{8.5.7}

设$\leq_{X}$是$X$上的序关系。

反证法,假设$Y$有多个最小元素。
假设$y_1, y_2 \in Y$且$y_1 \neq y_2$都是$Y$的最小元素。
由于$Y$是$X$的一个全序子集,则由定义8.5.3可知,$y_1 \leq_{X} y_2$或$y_2 \leq_{X} y_1$。

如果$y_1 \leq_{X} y_2$则与$y_2$是最小值相悖;
如果$y_2 \leq_{X} y_1$则与$y_1$是最小值相悖。

最大值的证明同上。

\section*{8.5.8}
为了描述方便,不妨设$X$是全序集,$Y$是$X$的一个非空子集,$\leq_X$是$X$上的序关系。

按照定义8.5.3可知,全序集的非空子集也是全序集。

由题设可知$Y$是有限集合,所以不妨设$\#(Y) = n$,$n$是任意自然数。

对$n$进行归纳。

归纳基始,$n = 1$,即$Y$中只有一个元素,由定义8.5.5可知,该元素既是最大值也是最小值。

归纳假设,$n = k$时命题成立。

$n = k + 1$,设$Y^\prime = Y \setminus \{x\}$,$x$可以是$Y$中的任意元素。
由引理3.6.9可知,$\#(Y^\prime) = k$,于是利用归纳假设可得$min(Y^\prime) = y_1$,
因为$Y$是全序集,所以$x, y_1$是可以比较大小的,
即:要么$x \leq_X y_1$(此时$x$是最小值),要么$y_1 \leq_X x$(此时$y_1$是最小值)。

最大值证明类似。

\section*{8.5.10}

\begin{zremark}
  “强归纳原理和弱归纳原理是等价的”。个人感觉这个命题还是挺重要的,接下来我会证明这个命题。

  这里的证明,参考了《符号逻辑讲义$_\text{徐明}$》命题681.
  \begin{itemize}
    \item 强归纳原理 $\Rightarrow$ 弱归纳原理;
          即强归纳原理成立的前提下,可以推出弱归纳原理成立。

          令$P(n)$是关于元素$n \in X$的任意性质。假设弱归纳原理的前提成立,即假设$P(0)$成立,
          并归纳假设对每一个$n \in X, P(n)$成立则$P(n+1)$成立。
          现在需要用强归纳原理证明弱归纳原理的结论(对所有的$n$都有$P(n)$)。而强归纳原理的前提
          对所有$m \leq n$的$P(m)$成立,那么$P(n+1)$成立,这显然已由弱归纳原理的前提保证了。
          强归纳原理的前提满足后,结论也就有了。


    \item 弱归纳原理 $\Rightarrow$ 强归纳原理

          即弱归纳原理成立的前提下,可以推出强归纳原理成立。

          令$P(n)$是关于元素$n \in X$的任意性质。假设强归纳原理的前提成立,
          即对所有$m \leq n$的$P(m)$成立,那么$P(n+1)$成立。
          现在需要用弱归纳原理证明强归纳原理的结论。而弱归纳原理的前提$P(0)$成立,
          与对每一个$n \in X, P(n)$成立则$P(n+1)$成立,这显然也已被强归纳原理的前提保证了,
          弱归纳原理的前提满足后,结论也就有了。
  \end{itemize}
  上面的证明是在自然数集上证明的,但该命题在良序集也是成立的【当前还未想好怎么表述...】。
  % 这里,你可能会说书中是$m < n$的$m \in X$都为真,那么$P(n)$也为真。
  % 而不是以上证明中的$m \leq n$,
  % 两种方式表达的效果是一样的:比良序集$Y$,良序集$\{W : x \in X \setminus Y, x \not \in Y\}$的最小值是唯一的,
  % 具体来说$m \leq n$是$n$,$m \leq n$是$n+1$,$n+1$只是在自然数集中的表达方式。
\end{zremark}



\end{document}
