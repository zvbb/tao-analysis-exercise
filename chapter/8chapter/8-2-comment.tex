\documentclass{article}
\usepackage{mathtools} 
\usepackage{fontspec}
\usepackage[UTF8]{ctex}
\usepackage{amsthm}
\usepackage{mdframed}
\usepackage{xcolor}
\usepackage{amssymb}
\usepackage{amsmath}
\begin{document}

\title{8.2 注释}
\author{张志聪}
\maketitle

$\bigstar \textbf{定义}8.2.1 \textbf{定义明确性}$

假设双射函数$h: \mathbb{N} \rightarrow X$。
需要证明:
\begin{align*}
    \sum \limits_{n=0}^\infty f(g(n)) = \sum \limits_{m=0}^\infty f(h(m))
\end{align*}

可以定义$a_n$为
\begin{align*}
    a_n := f(g(n))
\end{align*}
现在,根据命题7.4.3只需要找到一个双射。

定义$w: \mathbb{N} \rightarrow \mathbb{N}$为
\begin{align*}
    w(m) := h^{-1} \circ g(m)
\end{align*}

因为$g$是$\mathbb{N} \rightarrow X$的双射,且
$h$是$\mathbb{N} \rightarrow X$的双射,那么$h^{-1}$是$\mathbb{N} \rightarrow X$的双射,
所以,这两个函数的复合函数$w$是双射函数。又因为
\begin{align*}
    \sum \limits_{m=0}^\infty f(h(m)) = \sum \limits_{m=0}^\infty a_{w(m)}
\end{align*}
由命题7.4.3可知
\begin{align*}
    \sum \limits_{n=0}^\infty a_n                 & = \sum \limits_{m=0}^\infty a_{w(m)} \\
    \Rightarrow \sum \limits_{n=0}^\infty f(g(n)) & = \sum \limits_{m=0}^\infty f(h(m))
\end{align*}


$\bigstar \textbf{定理}8.2.2$

(1)书中
\begin{align*}
    \sum \limits_{n=0}^N \sum \limits_{m=0}^M \leq \sum \limits_{(n,m) \in X}f(n,m)
\end{align*}
应该是错的,这里应该是相等的,即:
\begin{align*}
    \sum \limits_{n=0}^N \sum \limits_{m=0}^M = \sum \limits_{(n,m) \in X}f(n,m)
\end{align*}

(2)书中“当$M \rightarrow \infty$时,对上面的式子取上确界可得(利用极限定律并对$N$使用归纳法)”

应该是错的,这里是对$M$进行归纳,$N$是看做固定值的。

(3)证明:$\sum \limits_{(n,m) \in \mathbb{N} \times \mathbb{N}}f(n,m)$是绝对收敛的,那么
$\sum \limits_{(n,m) \in \mathbb{N} \times \mathbb{N}}f_{+}(n,m)$
和$\sum \limits_{(n,m) \in \mathbb{N} \times \mathbb{N}}f_{-}(n,m)$也都是绝对收敛的。

反证法,假设$f_{+}$是发散的,
由$\sum \limits_{(n,m) \in \mathbb{N} \times \mathbb{N}}f(n,m)$的部分和$S_N$等于
$\sum \limits_{(n,m) \in \mathbb{N} \times \mathbb{N}}f_{+}(n,m)$的部分和$S_{N+}$,
$\sum \limits_{(n,m) \in \mathbb{N} \times \mathbb{N}}f_{-}(n,m)$的部分和$S_{N-}$相加,
即:
\begin{align*}
    S_N = S_{N+} + S_{N-}
\end{align*}
(注意:这里的部分和都是序列每项取绝对值的部分和)。

由于$(S_N)_{N=0}^\infty$收敛,所以,对任意$\epsilon > 0$,都存在一个整数$M^\prime$,
当$N \geq M^\prime$使得
\begin{align}
    |S_N - L| \leq \epsilon
\end{align}

当如果$f_{+}$是发散的,那个,存在一个整数$M^{\prime\prime}$,
当$N \geq M^{\prime\prime}$使得
\begin{align*}
     & S_{N+} > L + \epsilon
\end{align*}
又因为$S_{N-} \geq 0$,所以,取$M = max(M^\prime, M^{\prime\prime})$,
当$N \geq M$使得
\begin{align*}
    |S_N - L| & = |S_{N+} + S_{N-} - L | \\
              & > \epsilon
\end{align*}
显然,与(1)式存在矛盾。所以$f_{+}$是收敛的。同理可证$f_{-}$收敛。

$\textbf{(4)最后一句话,有个需要证明的部分。}$
\begin{align*}
    \sum \limits_{n=0}^\infty \sum \limits_{m=0}^\infty f(n,m)
    = \sum \limits_{n=0}^\infty \sum \limits_{m=0}^\infty f_{+}(n,m)
    +
    \sum \limits_{n=0}^\infty \sum \limits_{m=0}^\infty f_{-}(n,m)
\end{align*}
其中,$\sum \limits_{n=0}^\infty \sum \limits_{m=0}^\infty f_{+}(n,m)$与
$ \sum \limits_{n=0}^\infty \sum \limits_{m=0}^\infty f_{-}(n,m)$绝对收敛的。

感觉是结论是明显的,但这里有以下问题:
\begin{itemize}
    \item 书中没有明确定义双重级数。
    \item 回到定义7.2.2(级数的收敛),部分和$S_N = \sum \limits_{n=0}^N \sum \limits_{m=0}^\infty f(n,m)$
          因为其包含一个无限级数,处理起来没有想象中那么简单。
\end{itemize}
不妨设$\sum \limits_{n=0}^\infty \sum \limits_{m=0}^\infty f_{+}(n,m)$与
$ \sum \limits_{n=0}^\infty \sum \limits_{m=0}^\infty f_{-}(n,m)$的部分和分别是$S_{N+}$,$S_{N-}$。
那么,如果能证明:
\begin{align*}
    S_N = S_{N+} + S_{N-}
\end{align*}
则利用极限定律,可以完成证明。

对$m$进行归纳,$m = 0$时,由命题7.1.11(b)可知
\begin{align*}
    S_N    & = \sum \limits_{n=0}^N \sum \limits_{m=0}^0 f(n,m)          \\
           & = \sum \limits_{n=0}^N f(n,0)                               \\
    S_{N+} & = \sum \limits_{n=0}^\infty \sum \limits_{m=0}^0 f_{+}(n,m) \\
           & = \sum \limits_{n=0}^N f_{+}(n,0)                           \\
    S_{N-} & = \sum \limits_{n=0}^\infty \sum \limits_{m=0}^0 f_{-}(n,m) \\
           & = \sum \limits_{n=0}^N f_{-}(n,0)                           \\
\end{align*}
因为$f(n,0) = f_{+}(n,0) + f_{-}(n,0)$,由引理7.1.4(c)可知
\begin{align*}
    S_N = S_{N+} + S_{N-}
\end{align*}

归纳假设$m=k$时,命题成立。

$m = k+1$时,由引理7.1.4(a)可知
\begin{align*}
    S_N & = \sum \limits_{n=0}^N \sum \limits_{m=0}^{k+1} f(n,m)                       \\
        & = \sum \limits_{n=0}^N \big (\sum \limits_{m=0}^{k} f(n,m) + f(n,k+1) \big )
\end{align*}
同理可知
\begin{align*}
    S_{N+} & = \sum \limits_{n=0}^N \big (\sum \limits_{m=0}^{k} f_{+}(n,m) + f_{+}(n,k+1) \big ) \\
    S_{N-} & = \sum \limits_{n=0}^N \big (\sum \limits_{m=0}^{k} f_{-}(n,m) + f_{-}(n,k+1) \big )
\end{align*}
因为$k \in \mathbb{N}$(即:是有限值),所以
\begin{align*}
    S_N & = \sum \limits_{n=0}^N \big (\sum \limits_{m=0}^{k} f(n,m) + f(n,k+1) \big )         \\
        & = \sum \limits_{n=0}^N \sum \limits_{m=0}^{k} f(n,m) + \sum \limits_{n=0}^N f(n,k+1) \\
\end{align*}
同理$S_{N+},S_{N-}$同样成立。

利用归纳假设,可以证得$S_N = S_{N+} + S_{N-}$。

至此,归纳完成。

然后利用极限定律,即可完成证明。

$\textbf{(5)前提一定要是绝对收敛么?} $

是的。首先第二个等式用到了命题7.4.3,而该命题的前提就是要求级数是绝对收敛的。

对于第一个等式,在刚刚(3)中,$f$是绝对收敛,则$f_{+},f_{-}$也是绝对收敛的。
而在$f$是条件收敛的情况下,是没有这个性质的,这里举一个反例
\begin{align*}
    \sum \limits_{n=1}^\infty (-1)^n / n
\end{align*}

由命题7.2.12(交错级数判别法)可知,该级数收敛。
而级数$f_{+} := \sum \limits_{n=1}^\infty 1 / n$是调和函数,是发散的(推论7.3.7)。

\end{document}