\documentclass{article}
\usepackage{mathtools} 
\usepackage{fontspec}
\usepackage[UTF8]{ctex}
\usepackage{amsthm}
\usepackage{mdframed}
\usepackage{xcolor}
\usepackage{amssymb}
\usepackage{amsmath}


% 定义新的带灰色背景的说明环境 zremark
\newmdtheoremenv[
  backgroundcolor=gray!10,
  % 边框与背景一致,边框线会消失
  linecolor=gray!10
]{zremark}{说明}


\begin{document}
\title{12.5 习题}
\author{张志聪}
\maketitle

\section*{12.5.1}

等价的,指的是相互之间可以推导。

\begin{itemize}
  \item $\Rightarrow$

        定义9.1.22概念下,$Y$是有界的,那么存在实数$M > 0$使得$Y \subset [-M, M]$,
        那么令$r > M$,此时$Y \subset [-M, M] \subset B(0, r)$,满足定义12.5.3。

  \item $\Leftarrow$

        定义12.5.3概念下,$Y$是有界的,那么存在一个包含$Y$的球$B(0, r)$,令$M > r$,
        此时$Y \subset B(0, r) \subset [-M, M]$,满足定义9.1.22。

\end{itemize}

\section*{12.5.2}

\begin{itemize}
  \item 完备性

        反证法,假设不是完备的,那么存在柯西序列$(x_n)_{n = 1}^\infty$不在$(X,d)$中收敛。

        度量空间$(X,d)$是紧致的,由定义12.5.1可知,$(x_n)_{n = 1}^\infty$存在一个子序列
        收敛于某个$x_0, x_0 \in X$,由引理12.4.9可知,原序列$(x_n)_{n = 1}^\infty$也收敛于$x_0$。
        与假设矛盾。

  \item 有界性

        反证法,不是有界的。那么,对$x \in X$,任意正整数$n$都有
        \begin{align*}
          X_n := X \setminus B(x, n)
        \end{align*}
        都是非空的。

        由选择公理,能够找到一个序列$(x_n)_{n = 1}^\infty$使得$x_n \in X_n$,对所有的$n \geq 1$均成立。
        由于$(X,d)$是紧致的,所以$(x_n)_{n = 1}^\infty$存在一个收敛于$L \in X$的子序列$(x_{n_j})_{j = 1}^\infty$。

        于是对任意$\epsilon > 0$,存在$N$使得
        \begin{align*}
          d(x_{n_j}, L) < \epsilon
        \end{align*}
        对任意$j \geq N$均成立。

        于是我们有
        \begin{align*}
          d(x_{n_j}, x) \leq d(x_{n_j}, L) + d(L, x) < \epsilon + d(L, x)
        \end{align*}
        对任意$j \geq N$均成立。

        当取$N^\prime = max(N, [\epsilon + d(L, x)] + 1) $
        \begin{align*}
          d(x_{n_j}, x) > n_j \geq N^\prime > \epsilon + d(L, x)
        \end{align*}
        对任意的$j \geq N^\prime$均成立。

        存在矛盾,假设不成立。

\end{itemize}

\section*{12.5.3}

\begin{itemize}
  \item $\Rightarrow$

        利用推论12.5.6可证。

  \item $\Leftarrow$

        $E$是$\mathbb{R}^n$的子集,所以按照欧几里得空间的定义:
        \begin{align*}
          \mathbb{R}^n = \{(x_1,x_2,...,x_n): x_1,x_2,...,x_n \in \mathbb{R}\}
        \end{align*}
        于是$E$是有界闭集,那么任意坐标分量构成的集合都是有界闭集(这里可以通过反证法证明,
        如果坐标分量$x^{(j)}$构成的集合不是有界闭集,那么$E$将不会是有界闭集。)

        对$E$中的任意序列$(x_n)_{n = 1}^\infty$,由定理9.1.24可知,
        对于坐标分量$j = 1$可以得到一个收敛的子序列$(x_n^{(1)})_{n = 1}^\infty$,
        再次利用定理9.1.24,可在子序列$(x_n^{(1)})_{n = 1}^\infty$基础上得到
        坐标分量$j=2$上收敛的子序列$(x_n^{(2)})_{n = 1}^\infty$,
        依次进行下去,最后得到一个序列$(x_n^{(n)})_{n = 1}^\infty$对于
        每个坐标分量$1 \leq j \leq n$,都是收敛的,利用命题12.1.18(d)可知,
        序列在欧几里得度量、出租车度量、上确界范数上收敛。
\end{itemize}

\section*{12.5.4}

定义$f: \mathbb{R}^n \rightarrow \mathbb{R}, f(x) = 0$,令$V = (-1, 1)$。
此时$V$是开集,$f(V) = {1}$是闭集。

\section*{12.5.5}

定义$f: \mathbb{R}^n \rightarrow \mathbb{R}, f(x) = 1/x$,令$F = [1, +\infty)$。
此时$F$是闭集,$f(F) = (0, 1]$是开集。

\section*{12.5.6}

在紧致度量空间$(K_1, d|_{K_1 \times K_1})$中,考察集合$V_n := K_1 \setminus K_n$,
$K_i$是闭的,则$V_n$是开的,否则与命题12.2.15(e)矛盾。

我们有
\begin{align*}
  V_1 \subset V_2 \subset V_3 \subset ...
\end{align*}

反证法,假设$\bigcap\limits_{n = 1}^{\infty} K_n = \varnothing$。
由命题3.1.26(h)(德$\circ$摩根定律)可知
\begin{align*}
  K_1 \setminus \bigcap\limits_{n = 1}^{\infty} K_n & = (K_1 \setminus K_1) \cup (K_1 \setminus K_2) \cup (K_1 \setminus K_3) \cup ... \\
                                                    & = \bigcup \limits_{n = 1}^{\infty} V_n
\end{align*}

由假设$\bigcap\limits_{n = 1}^{\infty} K_n = \varnothing$可知
\begin{align*}
  \bigcup \limits_{n = 1}^{\infty} V_n = K_1
\end{align*}
于是由定理12.5.8可知,存在$N \geq 1$使得$K_1 \subseteq \bigcup \limits_{n = 1}^{N} V_n$。

由$V_1 \subset V_2 \subset V_3 \subset ...$可知,存在$j > N$,
$x_0 \in V_j, x_0 \notin \bigcup \limits_{n = 1}^{N} V_n$,于是$x_0 \notin K_1$,
这与$V_j \subset K_1$矛盾,故假设不成立。

\section*{12.5.7}

\begin{itemize}
  \item (a)

        \begin{itemize}
          \item $\Rightarrow$

                $Z$是紧致的,由推论12.5.6可知$Z$是闭的。
          \item $\Leftarrow$

                $Z$是闭的,设$(z_n)_{n = 1}^\infty$是$Z$中的序列,因为$Z \subseteq Y$,
                所以$(z_n)_{n = 1}^\infty$也是$Y$中的序列,因为$Y$是紧致的,
                于是存在一个收敛的子序列$(z_{n_k})_{k = 1}^\infty$。又因为$Z$是闭集,
                由命题12.2.15(b)可知,子序列$(z_{n_k})_{k = 1}^\infty$收敛于$Z$中的值,
                所以$Z$也是紧致的。
        \end{itemize}
  \item (b)

        设$(x_n)_{n = 1}^\infty$是$Y_1 \cup Y_2 ... \cup Y_n$中的序列,
        那么,在某个$Y_j(1 \leq j \leq n)$中有无限多个项(反证法可以证明)。
        于是在$Y_j$中可以得到$(x_n)_{n = 1}^\infty$的子序列$(x_{j_k})_{k = 1}^\infty$,
        又因为$Y_j$是紧致子集,所以子序列$(x_{j_k})_{k = 1}^\infty$存在收敛的子序列。
        综上可得$Y_1 \cup Y_2 ... \cup Y_n$是紧致的。

  \item (c)

        子集$Y : \{x_0\}$是单点集,那么$Y$中的序列只能是常数序列$(x_0)_{n = 1}^\infty$,
        显然,此时$Y$是紧致的。

        子集$Y$不是单点集且是有限子集,那么可以通过有限个单点集$Y_1, Y_2, ..., Y_n$的并集得到$Y$,
        由(b)可得$Y$是紧致的。

\end{itemize}

\section*{12.5.8}

任意$k \in \mathbb{N}$。

$k \not = 1$有
\begin{align*}
  d_{l^1}(e^{(1)}, e^{(k)}) & = |e_1^{(1)} - e_1^{(k)}| + |e_k^{(1)} - e_k^{(k)}| \\
                            & = |1 - 0| + |0 - 1|                                 \\
                            & = 2
\end{align*}

$k = 1$有
\begin{align*}
  d_{l^1}(e^{(1)}, e^{(1)}) & = 0
\end{align*}
于是$e^{(n)} \subseteq B(e^{(1)}, 3)$,所以$e^{(n)}$是有界的。

设$(x_n)_{n = 1}^\infty$是$\{e^{(n)}, n \in \mathbb{N}\}$中的收敛序列,由之前的讨论可知,存在$N \geq 1$使得
\begin{align*}
  x_j = x_k
\end{align*}
对所有的$j,k \geq N$均成立。
所以$\lim\limits_{n \to \infty} x_n$的极限值属于$\{e^{(n)}, n \in \mathbb{N}\}$。

\section*{12.5.9}

\begin{itemize}
  \item $\Rightarrow$

        $(X,d)$是紧致的,那么,对于$X$中的任意序列$(x_n)_{n = 1}^\infty$,
        存在一个收敛的子序列$(x_{nj})_{j = 1}^\infty$某个值$L$,由命题6.6.6可知,
        $L$是$(x_n)_{n = 1}^\infty$的极限点。


  \item $\Leftarrow$

        $X$中的任意序列都至少有一个极限点,由命题6.6.6可知,序列存在一个收敛的子序列,
        所以$(X,d)$是收敛的。
\end{itemize}

\section*{12.5.10}

\begin{itemize}
  \item (a)

        因为$n$正整数,可取$r = max(d(x^{(1)}, x^{(2)}), d(x^{(1)}, x^{(3)}), ..., d(x^{(1)}, x^{(n)}))$。

        由题设可知,对任意$x \in X$,都存在$1 \leq i \leq n$使得$x \in B(x^{(i)}, \epsilon)$,
        由三角不等式可知
        \begin{align*}
          d(x, x^{(1)}) \leq d(x, x^{(i)}) + d(x^{(i)}, x^{(1)}) < \epsilon + r
        \end{align*}
        可得$X \subseteq B(x^{(1)}, \epsilon + r)$,所以$X$是有界的。

  \item (b)

        对任意$\epsilon > 0$, 定义$B_x := B(x, \epsilon), x \in X$,于是我们有$X \subseteq \bigcup\limits_{x \in X} B_x$,
        由命题12.5.8可知,存在$X$的有限子集$F$使得
        \begin{align*}
          X \subseteq \bigcup\limits_{x \in F} B_x
        \end{align*}

  \item (c)

        假设$(x_n)_{n = 1}^\infty$是$X$中的序列,按照定义12.5.2(紧致性),我们需要证明$(x_n)_{n = 1}^\infty$存在一个收敛的子序列。

        对任意$\epsilon > 0$,由于$X$是完全有界的,
        那么,存在一个正整数$n$使得$X = \bigcup\limits_{i = 1}^n B(x^{(i)}, \frac{1}{2}\epsilon)$,
        因为$n$是正整数,于是存在某个$B(x^{(i)}, \frac{1}{2}\epsilon), 1 \leq i \leq n$中有无限个项(序列$(x_n)_{n = 1}^\infty$的项)。
        位于$B(x^{(i)}, \epsilon)$中的项组成一个子序列$(x^{ij})_{j=1}^\infty$,$x_0, x_1 \in B(x^{(i)}, \epsilon)$都有
        \begin{align*}
          d(x_0, x_1) \leq d(x_0, x^{(i)}) + d(x^{(i)}, x_1) < \epsilon
        \end{align*}
        $N$为$x^{i1}$在原序列中的下标,那么
        \begin{align*}
          d(x^{(j)}, x^{(k)}) < \epsilon
        \end{align*}
        对所有的$j,k \geq N$均成立。

        类似地,在子序列$(x^{ij})_{j=1}^\infty$可以递归处理得到一个柯西序列(这里也可以使用选择公理,操作方式类似于引理8.4.5的证明),
        由$X$的完备性可知,该柯西序列收敛。

\end{itemize}

\section*{12.5.11}

按照书中的提示进行证明。

反证法,假设$X$不是紧致的,那么由习题12.5.9可知存在一个序列$(x^{(n)})_{n = 1}^\infty$,它没有极限点。
“于是,对于每一个$x \in X$,都存在一个包含$x$的球$B(x, \epsilon)$,它最多包含序列中有限多个元素”。
又因为$X$可以被有限子覆盖(即有限个$B(x, \epsilon)$可以包含$X$),
这意味着序列$(x^{(n)})_{n = 1}^\infty$在$X$中的项是有限的,存在矛盾。

\begin{zremark}
  于是,对于每一个$x \in X$,都存在一个包含$x$的球$B(x, \epsilon)$,它最多包含序列中有限多个元素,其实不是太明显,
  需要额外说明下。

  没找到证明方法!!!
  % todo
\end{zremark}

\section*{12.5.12}

\begin{itemize}
  \item (a)

        对$X$中的任意柯西序列$(x_n)_{n = 1}^\infty$,
        因为是柯西序列,对任意$1 > \epsilon > 0$,存在$N \geq 1$使得
        \begin{align*}
          d_{disc}(x_j, x_k) < \epsilon
        \end{align*}
        对所有的$j,k \geq N$均成立,由离散度量$d_{disc}$的定义,
        对所有的$j,k \geq N$都有$x_j = x_k$,
        于是可以得到一个收敛的子序列$(x_{n})_{n = N}^\infty$收敛于$x_N, x_N \in X$。

  \item (b)

        利用习题12.5.10(c),当$X$是完全有界的,那么$X$是紧致的。

        当$X$不是完全有界的,那么$X$不是紧致的。证明框架:不是完全有界的,可以取一个等距的序列,比如: $(2 + n)_{n = 1}^\infty$。


\end{itemize}

\section*{12.5.13}

由定理12.5.7(海涅-博雷尔定理)可知,我们只需证明$E \times F$是一个有界闭集。

设$(a_n)_{n = 1}^\infty$是$E \times F$中依$d_{l^2}$的收敛序列,其中$a_n = (x_n, y_n), x_n \in E, y_n \in F$,
假设$\lim \limits_{n \to \infty} x_n = L, L = (x, y)$,由命题12.1.18可知,$x_n \to x, y_n \to y$当$n \to \infty$。
$E,F$是$\mathbb{R}$的两个紧致子集,于是$E,F$都是完备的,所以$x \in E, y \in F$。由命题12.2.15(b)可知,$E \times F$是闭集。

因为$E,F$是$\mathbb{R}$的两个紧致子集,那么由命题12.5.5可知$E,F$是完备和有界的,
所以存在$M_1, M_2$使得$E \subseteq B(0, M_1) = [-M_1, M_2], F \subseteq B(0, M_2) = [-M_2, M_2]$。
令
\begin{align*}
  M = \sqrt{M_1^2 + M_2^2}
\end{align*}
任意$(x, y) \in E \times F$,有
\begin{align*}
  d_{l^2}((x, y), (0, 0)) = \sqrt{x^2 + y^2} \leq M
\end{align*}
由此可得$E \times F \subseteq B_{d_{l^2}}(0, M+1)$,所以$E \times F$是有界。

\section*{12.5.14}

设$R := \inf\{d(x_0, y) : y \in E\}$($d(x_0, y) \geq 0$利用定理5.5.9可知$R$是存在的。),
由下确界的定义,对任意$n \in \mathbb{N}$,存在$y \in E$使得$d(x_0, y) \leq R + \frac{1}{n}$,利用选择公理,
可以构造$E$中的序列
\begin{align*}
  (x^{(n)})_{n=1}^\infty
\end{align*}
满足$d(x_0, x^{(n)}) \leq R + \frac{1}{n}$。

因为$E$是紧致的,那么$(x^{(n)})_{n=1}^\infty$存在收敛的子序列$(x^{(nk)})_{k=1}^\infty$,设$x := \lim \limits_{k \to \infty} x^{(nk)}$,
因为$E$是紧致的,那么也是完备的,所以$x \in E$。

对任意$\epsilon > 0$,存在$K \geq 1$使得
\begin{align*}
  d(x^{(nk)}, x) < \frac{1}{2}\epsilon
\end{align*}
对所有的$k \geq K$均成立。

可以让$k \geq K$并且$\frac{1}{nk} < \frac{1}{2}\epsilon$,我们有
\begin{align*}
  R \leq d(x_0, x) \leq d(x_0, x^{(nk)}) + d(x^{(nk)}, x) < R + \frac{1}{nk} + \frac{1}{2}\epsilon < R + \frac{1}{2}\epsilon + \frac{1}{2}\epsilon
\end{align*}
由$\epsilon$的任意性可知,$d(x_0, x) = R$。

\section*{12.5.15}

反证法,假设$\bigcap\limits_{\alpha \in I} K_{\alpha} = \varnothing$。

任意$\alpha \in I$,由命题12.2.15(e)可知,$X \setminus K_{\alpha}$是开集,
于是
\begin{align*}
  X = X \setminus \bigcap\limits_{\alpha \in I} K_{\alpha} = \bigcup\limits_{\alpha \in I} (X \setminus K_{\alpha}) 
\end{align*}
因为$X$是紧致的,由定理12.5.8可知,存在$I$的一个有限子集$F$使得
\begin{align*}
  X \subseteq \bigcup\limits_{\alpha \in F} (X \setminus K_{\alpha}) = X \setminus \bigcap\limits_{\alpha \in F} K_{\alpha}
\end{align*}
可得
\begin{align*}
  \bigcap\limits_{\alpha \in F} K_{\alpha} = \varnothing
\end{align*}
这与题设中的有限交性质矛盾。

\end{document}