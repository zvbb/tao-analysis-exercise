\documentclass{article}
\usepackage{mathtools} 
\usepackage{fontspec}
\usepackage[UTF8]{ctex}
\usepackage{amsthm}
\usepackage{mdframed}
\usepackage{xcolor}
\usepackage{amssymb}
\usepackage{amsmath}


% 定义新的带灰色背景的说明环境 zremark
\newmdtheoremenv[
  backgroundcolor=gray!10,
  % 边框与背景一致,边框线会消失
  linecolor=gray!10
]{zremark}{说明}


\begin{document}
\title{12.4 习题}
\author{张志聪}
\maketitle

\section*{12.4.1}

$(x^{(n)})_{n = m}^\infty$是$(X,d)$中收敛于极限$x_0$的序列,
由收敛的定义(定义12.1.14)可知,对任意$\epsilon > 0$,存在一个$N \geq m$使得
\begin{align*}
  d(x^{(n)},x_0) \leq \epsilon
\end{align*}
对所有的$n \geq N$均成立。

如果$(x^{n_j})_{j=1}^\infty$是$(x^{(n)})_{n = m}^\infty$的子序列,
令$N^\prime = N - m$,由定义12.4.1可知
\begin{align*}
  n_{N^\prime} \geq N
\end{align*}
于是
\begin{align*}
  d(x^{(n_j)}, x_0) \leq \epsilon
\end{align*}
对所有的$j \geq N^\prime$均成立,所以子序列收敛于$x_0$。

\section*{12.4.2}

\begin{itemize}
  \item $\Rightarrow$

        先构造出子序列,注意要满足子序列定义,然后证明该子序列收敛于$L$。

        (1)以递归的方式定义:

        $j=1$时,定义$x^{n_1}=x^{(m)}$。

        归纳假设,$n_j$时,项$x^{(n_j)}$是存在的。

        $j+1$时,
        由$(x^{(n)})_{n = m}^\infty$收敛于$L$,所以取$\epsilon = 1/(j+1) > 0$,
        存在$n \geq n_j$使得$d(x^{(n)}, L) \leq \epsilon$,满足该条件的$x^{(n)}$是一个非空集合,任取一个作为$x^{(n_{j+1})}$。

        (2)序列的收敛性

        对任意$\epsilon > 0$,存在$1/j \leq \epsilon$(存在的原因是$1/j$收敛于$0$)。
        通过序列$(x^{(n_j)})_{j=1}^\infty$的构造方式可知,
        取$N = j$,$n \geq N$使得$d(x^{(n_n)}, L) \leq \epsilon$,序列收敛得证。

  \item $\Leftarrow$

        任意$N \geq m$和$\epsilon > 0$,由子序列$(x^{(n_j)})_{j=1}^\infty$收敛于$L$,
        那么存在$N^\prime \geq 1, j \geq max(N^\prime, N) \geq N$使得
        \begin{align*}
          d(x^{(n_j)}, L) \leq \epsilon
        \end{align*}

        由子序列定义(定义12.4.1)可知,$n_j \geq j \geq N$,于是由之前的说明可知,
        存在一个$n = n_j \geq N$,使得
        \begin{align*}
          d(x^{(n)}, L) \leq \epsilon
        \end{align*}
        命题得证。

\end{itemize}

\section*{12.4.3}

任意$\epsilon > 0, \frac{1}{2}\epsilon > 0$,$(x^{(n)})_{n = m}^\infty$收敛于$x_0$,
那么存在$N \geq m$使得
\begin{align*}
  d(x^{(n)}, x_0) < \frac{1}{2}\epsilon
\end{align*}
对所有的$n \geq N$均成立。

所以,对$j,k \geq N$有
\begin{align*}
  d(x^{(j)}, x_0) < \frac{1}{2}\epsilon \\
  d(x^{(k)}, x_0) < \frac{1}{2}\epsilon
\end{align*}
由三角不等式可知,
\begin{align*}
  d(x^{(j)}, x^{(k)}) < \epsilon
\end{align*}
对所有的$j,k \geq N$均成立。

综上可知,$(x^{(n)})_{n = m}^\infty$是柯西序列。

\section*{12.4.4}

对任意的$\epsilon > 0$,由于原序列$(x^{(n)})_{n = m}^\infty$是柯西序列,
所以存在一个$N^\prime \geq m$使得
\begin{align*}
  d(x^{(j)}, x^{(k)}) < \frac{1}{2}\epsilon
\end{align*}
对所有的$j,k \geq N^\prime$均成立。

由命题12.4.5可知, $x_0$也是原序列$(x^{(n)})_{n = m}^\infty$的一个极限点,
令$N = N^\prime$,都存在一个$n^\prime \geq N$使得
\begin{align*}
  d(x^{(n^\prime)}, x_0) < \frac{1}{2}\epsilon
\end{align*}

对任意$n \geq N$我们有,
\begin{align*}
  d(x^{(n)}, x_0) & \leq d(x^{(n)}, x^{(n^\prime)}) + d(x^{(n^\prime)}, x_0) \\
                  & < \frac{1}{2}\epsilon + \frac{1}{2}\epsilon              \\
                  & = \epsilon
\end{align*}
(注意,这里的$n, n^\prime$满足$n, n^\prime \geq N^\prime$要求)
于是,原序列$(x^{(n)})_{n = m}^\infty$收敛于$x_0$。

\section*{12.4.5}
(1)

令$E := \{x^{(n)}: n \geq m\}$。

任意半径$r > 0$,$B(L, r) := \{x \in X: d(L, x) < r\}$,

因为$L$是序列$(x^{(n)})_{n = m}^\infty$的一个极限点,
那么对任意的$N \geq m$,存在一个$n \geq N$使得
\begin{align*}
  d(L, x^{(n)}) < r
\end{align*}
所以,$x^{(n)} \in B(L, r)$,于是
$x^{(n)} \in B(L, r) \cap E$,所以$B(L, r) \cap E \neq \varnothing$。

综上,由$r$的任意性可得,$L$是集合$\{x^{(n)}: n \geq m\}$的附着点。

(2)逆命题成立么?

不成立,比如$x^{(m)}$就是集合$\{x^{(n)}: n \geq m\}$的附着点,
但是$x^{(m)}$却不一定是集合$\{x^{(n)}: n \geq m\}$的极限点。

\section*{12.4.6}

假设柯西序列$(x^{(n)})_{n = m}^\infty$收敛于$L,L^\prime$且$L \neq L^\prime$。

序列$(x^{(n)})_{n = m}^\infty$收敛于$L$,那么对$\epsilon = \frac{1}{3}d(L-L^\prime) > 0$,
存在$N \geq m$使得
\begin{align*}
  d(x^{(n)}, L) < \epsilon
\end{align*}
对所有的$n \geq N$均成立。

类似地,存在$N^\prime \geq m$使得
\begin{align*}
  d(x^{(n)}, L^\prime) < \epsilon
\end{align*}
对所有的$n \geq N^\prime$均成立。

取$M = max(N, N^\prime)$,于是对所有的$n \geq M$都有
\begin{equation*}
  \begin{cases*}
    d(x^{(n)}, L) < \epsilon \\
    d(x^{(n)}, L^\prime) < \epsilon
  \end{cases*}
\end{equation*}
但上式不可能同时成立,如果成立会导致以下矛盾:
\begin{align*}
  d(L, L^\prime) \leq d(x^{(n)}, L) + d(x^{(n)}, L^\prime) < 2\epsilon = \frac{2}{3}d(L, L^\prime)
\end{align*}

\section*{12.4.7}

\begin{itemize}
  \item (a)

        \begin{align*}
           & (Y, d|_{Y \times Y}) \text{是完备的}        \\
           & \implies                                \\
           & \text{柯西序列} \Leftrightarrow \text{收敛序列}
        \end{align*}

        反证法,假设$Y$不是闭集,那么,由命题12.2.15(d)可知,
        存在一个$Y$中的收敛序列的极限值不属于$Y$,
        由引理12.4.7可知收敛序列是柯西序列,
        即存在一个$Y$中的柯西序列不在$(Y, d|_{Y \times Y})$中收敛,
        这与定义12.4.10矛盾。

  \item (b)

        设$(x^{(n)})_{n=m}^\infty$是$Y$中任意柯西序列,由于$Y$是$X$的子集,而$(X,d)$又是完备度量空间,
        所以序列收敛,不妨设收敛于$x_0, x_0 \in X$。
        又因为$Y$是$X$的一个闭子集,由命题12.2.15(d)可得$x_0 \in Y$。

        综上,由序列$(x^{(n)})_{n=m}^\infty$的任意性可得,$Y$中的柯西序列在$(Y, d|_{Y \times Y})$中都是收敛的。

        所以,子空间$(Y, d|_{Y \times Y})$也是完备的。
\end{itemize}

\section*{12.4.8}

\begin{itemize}
  \item (a)

        \begin{itemize}
          \item 自反性

                $d(x,x) = 0$保证了自反性的正确性,证明略

          \item 对称性

                $d(x,y) = d(y,x)$保证了对称性的正确性,证明略

          \item 传递性

                $d(x,z) \leq d(x,y) + d(y,z)$保证了传递性的正确性,证明略
        \end{itemize}
  \item (b)

        (1)

        对任意$\epsilon > 0, \frac{1}{2}\epsilon > 0$,
        因为序列$(x_n)_{n=1}^\infty$是柯西序列,所以存在$N \geq 1$使得
        \begin{align*}
          d(x_j, x_k) < \frac{1}{2}\epsilon
        \end{align*}
        对所有的$j,k \geq N$均成立。

        类似地,存在$N^\prime \geq 1$使得
        \begin{align*}
          d(y_j, y_k) < \frac{1}{2}\epsilon
        \end{align*}
        对所有的$j,k \geq N^\prime$均成立。

        取$M = max(N, N^\prime)$,于是对所有的$j,k \geq M$都有
        \begin{align*}
          |d(x_i, y_i) - d(x_j, y_j)| & \leq |d(x_i, y_i) - d(y_i, x_j)| + |d(y_i, x_j) - d(x_j, y_j)| \\
                                      & \leq d(x_i, x_j) + d(y_i, y_j)                                 \\
                                      & < \epsilon
        \end{align*}
        对所有的$i,j \geq M$均成立。

        由此可得$(d(x_n, y_n)_{n = 1}^\infty)$是柯西序列,由于对任意$n, d(x_n, y_n) \in \mathbb{R}$,
        而$(\mathbb{R}, d)$是完备度量空间,所以$(d(x_n, y_n)_{n = 1}^\infty)$在$(\mathbb{R},d)$中收敛,
        即$\lim\limits_{n \to \infty} d(x_n, y_n)$存在。

        以上证明使用了命题:
        \begin{zremark}
          \textbf{$|a - b| \leq |a - c| + |c - b|$}

          证明:

          因为绝对值的三角不等式(命题4.3.3(b)),以下命题成立
          \begin{align*}
            |x + y| \leq |x| + |y|
          \end{align*}
          令$x = a - c, y = c - b$,代入上式可证。
        \end{zremark}

        \begin{zremark}
          \textbf{$d(x, y) - d(y, z) \leq d(x, z)$}

          证明:

          因为三角不等式(定义12.1.2),以下命题成立
          \begin{align*}
             & d(x, y) \leq d(x, z) + d(z, y) \\
             & \implies                       \\
             & d(x, y) - d(z, y) \leq d(x, z) \\
             & \implies                       \\
             & d(x, y) - d(y, z) \leq d(x, z)
          \end{align*}

        \end{zremark}

        (2)

        因为$LIM_{n \to \infty} x_n$和$LIM_{n \to \infty} x_n^\prime$是两个相等的形式极限,即满足
        \begin{align*}
          \lim\limits_{n \to \infty} d(x_n, x_n^\prime) = 0
        \end{align*}
        那么,对任意的$\epsilon > 0$,存在$N \geq 1$使得
        \begin{align*}
          d(x_n, x_n^\prime) \leq \epsilon
        \end{align*}
        我们需要证明序列$(d(x_n, y_n))_{n = 1}^\infty$和$(d(x_n^\prime, y_n))_{n = 1}^\infty$
        是等价序列(满足定义5.2.6)。

        因为
        \begin{align*}
          |d(x_n, y_n) - d(x_n^\prime, y_n)| \leq d(x_n, x_n^\prime) \leq \epsilon
        \end{align*}
        对所有的$n \geq N$均成立。

        综上,两个序列是等价序列。
  \item (c)

        我们需要证明$\overline{X}$中的任意柯西序列是收敛序列。
        由(b)可知$\overline{X}$中的元素(或点)都是柯西序列的形式极限,于是
        不妨设任意$\overline{X}$中的任意柯西序列为$(a_n)_{n=1}^\infty$,其中
        $a_k = LIM_{n \to \infty} x_n^{(k)}$是$X$中的柯西序列$(x_n^{(k)})_{n=1}^\infty$的形式极限。

        对于任意柯西序列$(x_n^{(k)})_{n=1}^\infty$,对$\frac{1}{k} > 0$,
        存在$N_k \geq 1$使得
        \begin{align}
          d(x_i^{(k)}, x_j^{(k)}) < \frac{1}{k}
        \end{align}
        对所有的$i,j \geq N_k$均成立,元素集合$E_k := \{x_n^{(k)} : n \geq N_k\}$是一个非空集合。
        利用选择公理,能够找到一个序列$(b_n)_{n=1}^\infty$使得$b_n \in E_n$对所有的$n \geq 1$均成立。

        首先证明$(b_n)_{n=1}^\infty$是柯西序列。因为$(a_n)_{n=1}^\infty$是柯西序列,
        那么,对任意$\epsilon > 0$,存在$N_a \geq 1$使得
        \begin{align*}
          d_{\overline{X}}(a_i, a_j) < \frac{\epsilon}{3}
        \end{align*}
        对所有的$i, j \geq N_a$均成立。

        即$\lim\limits_{n \to \infty} d(x_n^{(i)}, x_n^{(j)}) < \frac{\epsilon}{3}$
        对任意$i,j \geq N_a$均成立。
        由极限的定义可知,存在$N^\prime$使得
        \begin{align*}
          d(x_n^{(i)}, x_n^{(j)}) < \frac{\epsilon}{3}
        \end{align*}
        对任意$n \geq N^\prime, i,j \geq N_a$均成立。

        于是可取$N = max(N_a,N_i,N_j,N^\prime)$,我们有
        \begin{align*}
          d(b_i, b_j) & \leq d(b_i, x_{N}^{(i)}) + d(x_{N}^{(i)}, x_{N}^{(j)}) + d(x_{N}^{(j)}, b_j) \\
                      & < 1/i + \frac{\epsilon}{3} + 1/j                                             \\
                      & < \epsilon
        \end{align*}
        (注意这里的$i,j$只要足够大,就能满足条件),
        对任意$i,j \geq N_a$均成立,所以$(b_n)_{n=1}^\infty$是柯西序列。

        接下来证明$(a_n)_{n=1}^\infty$收敛,对任意$\epsilon > 0$,存在$N$使得
        \begin{align*}
          d(b_i, b_j) < \frac{\epsilon}{2}
        \end{align*}
        对所有的$i,j \geq N$均成立。

        由$b_k$的构造方式可知,存在$k \geq N$使得
        \begin{align*}
          d(x_n^{(k)}, b_k) < \frac{1}{k} < \frac{\epsilon}{2}
        \end{align*}
        对所有的$n \geq N_k$均成立。

        我们有
        \begin{align*}
          d(x_n^{(k)}, b_n) \leq d(x_n^{(k)}, b_k) + d(b_k, b_n) < \frac{\epsilon}{2} + \frac{\epsilon}{2} = \epsilon
        \end{align*}
        对所有的$n > N_k, k \geq N$均成立。

        设$L = LIM_{n \to \infty} b_n$,
        即$d_{\overline{X}}(a_k, L) = \lim\limits_{n \to \infty} d(x_n^{(k)}, b_n) < \epsilon$
        对所有的$k \geq N$均成立,
        可得$(a_n)_{n=1}^\infty$收敛于$L$。

  \item (d)

        \begin{align*}
          x, y \in X; x = y                      \\
          \Leftrightarrow                        \\
          d(x, y) = 0                            \\
          \Leftrightarrow                        \\
          \lim\limits_{n \to \infty} d(x, y) = 0 \\
          \Leftrightarrow                        \\
          LIM_{n \to \infty} x = LIM_{n \to \infty} y
        \end{align*}
        所以元素$x \in X$与$\overline{X}$中$x$所对应的形式极限$LIM_{n \to \infty} x$等同起来是合理的。

        对任意$x, y \in X$,我们有
        \begin{align*}
          d_{\overline{X}} (x, y) & = d_{\overline{X}} (LIM_{n \to \infty} x, LIM_{n \to \infty} y) \\
                                  & = \lim\limits_{n \to \infty} d(x, y)                            \\
                                  & = d(x, y)
        \end{align*}
        从而$(X, d)$可以看作$\overline{X}, d_{\overline{X}}$的子空间。
  \item (e)

        \begin{itemize}
          \item 任意$x \in \overline{X}$都是$X$的附着点。

                有题设可知$x = LIM_{n \to \infty} x_n$,所以$x$可以表示成$X$中的柯西序列$(x_n)_{n=1}^\infty$。
                对任意$\epsilon > 0$,存在$N \geq 1$使得
                \begin{align*}
                  d(x_n, x_N) < \frac{\epsilon}{2}
                \end{align*}
                对所有的$n \geq N$存成立。

                因为$x_N \in X$,所以
                \begin{align*}
                  d_{\overline{X}}(x, x_N) & = d_{\overline{X}}(LIM_{n \to \infty} x_n, LIM_{n \to \infty} x_N) \\
                                           & = \lim \limits_{n \to \infty} d(x_n, x_N)                          \\
                                           & \leq \frac{\epsilon}{2}                                            \\
                                           & < \epsilon
                \end{align*}
                即对任意$\epsilon$半径,都存在$x_N$使得$d_{\overline{X}}(x, x_N) < \epsilon$,
                所以$x$是附着点。

          \item $X$的任意附着点$x, x \in \overline{X}$

                证明框架:由附着点定义和选择公理可以得到一个$X$中的任意柯西序列$(x_n)_{n=1}^\infty$,
                该序列收敛于$x = \lim\limits_{n \to \infty} x_n$,利用了(f)可知,$x \in \overline{X}$。
        \end{itemize}

  \item (f)

        由(c)可知柯西序列$(x_n)_{n=1}^\infty$是收敛的,不妨设$L = \lim\limits_{n \to \infty} x_n$,
        因为$L \in \overline{X}$,所以存在柯西序列$(y_n)_{n=1}^\infty$使得$L = LIM_{n \to \infty} y_n$。
        接下来我们要证明$LIM_{n \to \infty} y_n = LIM_{n \to \infty} x_n = L$。

        因为$L = \lim\limits_{n \to \infty} x_n$,那么,对任意$\epsilon > 0$,存在$K$使得
        \begin{align*}
          d_{\overline{X}}(LIM_{n \to \infty} y_n, x_k) = \lim\limits_{n \to \infty} d(y_n, x_k) < \frac{\epsilon}{2}
        \end{align*}
        对所有的$k \geq K$均成立。

        由于$(x_n)_{n=1}^\infty$是柯西序列,所以存在$N \geq K$使得
        \begin{align*}
          d(x_n, x_N) < \frac{\epsilon}{2}
        \end{align*}
        对所有的$n \geq N$均成立。

        因为$\lim\limits_{n \to \infty} d(y_n, x_k) < \frac{\epsilon}{2}$所以存在$N^\prime \geq N$使得
        \begin{align*}
          d(y_n, x_{N^\prime}) < \frac{\epsilon}{2}
        \end{align*}
        对所有$n \geq N^\prime$均成立。

        于是我们有
        \begin{align*}
          d(x_n, y_n) \leq d(x_n, x_{N^\prime}) + d(x_{N^\prime}, y_n) < \frac{\epsilon}{2} + \frac{\epsilon}{2} = \epsilon
        \end{align*}
        对所有的$n \geq N^\prime$均成立。

        所以
        \begin{align*}
          d_{\overline{X}}(LIM_{n \to \infty} x_n, LIM_{n \to \infty} y_n) = \lim\limits_{n \to \infty} d(x_n, y_n) = 0
        \end{align*}
        于是
        \begin{align*}
          \lim\limits_{n \to \infty} x_n = L = LIM_{n \to \infty} y_n = LIM_{n \to \infty} x_n
        \end{align*}
\end{itemize}


\end{document}