\documentclass{article}
\usepackage{mathtools} 
\usepackage{fontspec}
\usepackage[UTF8]{ctex}
\usepackage{amsthm}
\usepackage{mdframed}
\usepackage{xcolor}
\usepackage{amssymb}
\usepackage{amsmath}

\newmdtheoremenv[
  backgroundcolor=gray!10,
  linewidth=0pt,
  innerleftmargin=10pt,
  innerrightmargin=10pt,
  innertopmargin=10pt,
  innerbottommargin=10pt
]{zgraytheorem}{}
% 定义说明环境样式
\newtheoremstyle{mystyle}% 说明环境样式的名称
  {1em}% 上方间距
  {1em}% 下方间距
  {\normalfont}% 说明内容的字体样式
  {}% 缩进量
  {\bfseries}% 说明标记的字体样式
  {.}% 说明标记和说明内容之间的标点
  {1em}% 说明标记后的水平空间
  {}% 说明标记后的垂直空间
% 使用新定义的样式创建说明环境
\theoremstyle{mystyle}
\newtheorem*{zremark}{说明}


\begin{document}
\title{6.3 习题}
\maketitle

说在开头的话:文中的\textbf{上确界与最小上界不是一回事},
最小上界是一个集合$E$有上界为前提的,此时的最小上界与上确界一致。
而如果集合没有最小上界,那么集合的上确界被指定为$+\infty$【空集时被指定为$-\infty$】。
由此可知,最小上界定义是包含在上确界中的定义中,反之则不然。

\section*{6.3.1}

证明$sup(a_n)_{n=1}^\infty=1$,
首先$1$是上界,因为$a_n$是递减的,且$n=1$时,$a_1 = 1$。
假设存在上界$M < 1$,由$a_1=1$可知,$M$不存在。

证明$inf(a_n)_{n=1}^\infty=0$,
首先,因为$n$是正整数,所以$1/n>0$,于是$0$是下界。假设存在下界$m > 0$,由推论5.4.13(阿基米德性质)可知,
存在正整数$M$使得$Mm > 1$,所以$m > 1/M$,取$n = M$,此时$a_n = 1/M < m$,
与$m$是下界矛盾。

\section*{6.3.2}

设$E := \{a_n : n \geq m\}$,$E$是非空的实数集合,$x := sup(E)$。

(1)
由定义6.2.6可知,$x$要么是实数,要么是$+\infty$。

如果$x$是实数,由最小上界定义可知,$a_n \leq x$对所有的$n \geq m$均成立。

如果$x$是$+\infty$,定义6.2.3可知$a_n \leq x$。


(2)$M$是$E$的上界。反证法$x > M$。
如果$x$是实数,那么此时与$x$是$E$的最小上界定义矛盾。
如果$x=+\infty$,那么由定义6.2.3可知,这样的$x \geq M$,按照定义5.5.10此时$E$是没有上界的,
于是$M = +\infty$,所以不存在$x>M$。

综上,$ x \leq M$。

(3)反证法。假设不存在$n \geq m$使得$y < a_n \leq x$。

由假设可知$a_n \leq y$或$a_n > x$。

如果$x$是实数,那么,$x$是$E$的最小上界,
如果存在$y<x, a_n \leq y$,那么$y$才是$E$的最小上界,所以该情况不可能发生。
如果存在$a_n > x$,那么与$x$是最小上界矛盾,所以该情况不可能发生。

如果$x=+\infty$,表明$E$没有上界。
所以$a_n > x$是不可能的。
如果存在$y < x, a_n \leq y$,那么,$y$是实数,即$E$是有上界的,这与$E$没有上界矛盾。

综上,假设不成立。

\section*{6.3.3}

由于序列$(a_n)_{n=m}^\infty$是有界的实数序列,所以集合$E := \{a_n : n \geq m\}$按定理5.5.9
可知集合$E$有一个最小上界,即存在$sup(E)$。

$M$是$E$的上界,由命题6.3.6可知,那么$sup(E) \leq M$。

现在要证明序列$(a_n)_{n=m}^\infty$是收敛的,并且收敛于$sup(E)$,为了描述方便,设$x:=sup(E)$。

对于任意实数$\epsilon > 0$,$x - \epsilon < x$,
由命题6.3.6可知,存在一个$n \geq m$使得$x-\epsilon < a_n \leq x$,不妨设这里的$n$为$N$,
由于序列是递增的,且$x$是最小上界,所以$x-\epsilon < a_n \leq x$对$n \geq N$均成立,
所以$|x - a_n| \leq \epsilon$,即序列是最终$\epsilon -$接近与$x$,由于$\epsilon$是任意的,
所以序列收敛于$x$,即:$\lim\limits_{n \to \infty} a_n = x = sup(a_n)_{n=m}^\infty$

\end{document}