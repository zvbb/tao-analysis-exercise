\documentclass{article}
\usepackage{mathtools} 
\usepackage{fontspec}
\usepackage[UTF8]{ctex}
\usepackage{amsthm}
\usepackage{mdframed}
\usepackage{xcolor}
\usepackage{amssymb}
\usepackage{amsmath}

\newmdtheoremenv[
  backgroundcolor=gray!10,
  linewidth=0pt,
  innerleftmargin=10pt,
  innerrightmargin=10pt,
  innertopmargin=10pt,
  innerbottommargin=10pt
]{zgraytheorem}{}
% 定义说明环境样式
\newtheoremstyle{mystyle}% 说明环境样式的名称
  {1em}% 上方间距
  {1em}% 下方间距
  {\normalfont}% 说明内容的字体样式
  {}% 缩进量
  {\bfseries}% 说明标记的字体样式
  {.}% 说明标记和说明内容之间的标点
  {1em}% 说明标记后的水平空间
  {}% 说明标记后的垂直空间
% 使用新定义的样式创建说明环境
\theoremstyle{mystyle}
\newtheorem*{zremark}{说明}


\begin{document}
\title{6.2 习题}
\maketitle

\section*{6.2.1}

\textbf{自反性}

当$x \in R$时,有命题5.3.3可知$x = x$,由实数排序定义5.4.6可知$x \leq x$,
又由广义实数的排序定义6.2.3可知$x \leq x$。

当$x = +\infty$时,由广义实数的排序定义6.2.3可知$x \leq x$

当$x = -\infty$时,由广义实数的排序定义6.2.3可知$x \leq x$

\textbf{三歧性}

(1)如果$x,y \in R$,由实数的三歧性可得$x,y$满足三歧性。

(2)分情况讨论$x,y$属于广义实数系附加的两个额外元素$+\infty$和$-\infty$。

这里只讨论一种情况,其他的情况类似不做赘述。

如果$x = +\infty$,$y \in R$,由定义6.2.3可知$x \geq y$,即该种情况属于三种情况之一,
按照定义,不满足其他两种情况,所以满足三歧性。

\textbf{传递性}

(1)如果$x,y,z \in R$,由实数的传递性可得$x \leq z$。

(2)分情况讨论$x,y,z$属于广义实数系附加的两个额外元素$+\infty$和$-\infty$。

证明略

\textbf{负运算使序改变}

如果$x, y \in R$,显然成立。

分情况讨论$x,y$属于广义实数系附加的两个额外元素$+\infty$和$-\infty$。

证明略

\section*{6.2.2}

\textbf{(a)}

根据定义6.2.6 和 定义6.2.3,$sup(E) \geq x \text{且} inf(E) \leq x, x \in E$。

\textbf{(b)}

反证法。假设$sup(E) > M$。

如果$sup(E)$是实数,按照定义6.2.6可知$E$中只能包括$-\infty$和实数,
而6.2.6(c)的情形最终会归入6.2.6(a),由定义5.5.10可知,$sup(E)$是$E$的最小上界,
如果$sup(E) > M$,那么与最小上界的定义5.5.5矛盾。

如果$sup(E)=-\infty$,由定义6.2.3可知,$-\infty \leq M$。

如果$sup(E)=+\infty$,由定义6.2.3可知,$M$必须等于$+\infty$,所以$sup(E) \leq M$。

\textbf{(c)}

证明方法与(b)类似,略
\end{document}