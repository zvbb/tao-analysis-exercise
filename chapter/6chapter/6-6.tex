\documentclass{article}
\usepackage{mathtools} 
\usepackage{fontspec}
\usepackage[UTF8]{ctex}
\usepackage{amsthm}
\usepackage{mdframed}
\usepackage{xcolor}
\usepackage{amssymb}
\usepackage{amsmath}

\newmdtheoremenv[
  backgroundcolor=gray!10,
  linewidth=0pt,
  innerleftmargin=10pt,
  innerrightmargin=10pt,
  innertopmargin=10pt,
  innerbottommargin=10pt
]{zgraytheorem}{}
% 定义说明环境样式
\newtheoremstyle{mystyle}% 说明环境样式的名称
  {1em}% 上方间距
  {1em}% 下方间距
  {\normalfont}% 说明内容的字体样式
  {}% 缩进量
  {\bfseries}% 说明标记的字体样式
  {.}% 说明标记和说明内容之间的标点
  {1em}% 说明标记后的水平空间
  {}% 说明标记后的垂直空间
% 使用新定义的样式创建说明环境
\theoremstyle{mystyle}
\newtheorem*{zremark}{说明}


\begin{document}
\title{6.6 习题}
\maketitle

\section*{6.6.1}

(1)自反性

定义$f(n)=n$的函数$f : \mathbb{N} \rightarrow \mathbb{N}$
是从$\mathbb{N}$到$\mathbb{N}$的严格递增函数,使得
\begin{align*}
  a_n = a_{f(n)} = a_n \text{对所有的$n \in \mathbb{N}$均成立}
\end{align*}
由定义6.6.1 可知,此时$(a_n)_{n=0}^\infty$是$(a_n)_{n=0}^\infty$的一个子序列。

(2)传递性

因为$(b_n)_{n=0}^\infty$是$(a_n)_{n=0}^\infty$的子序列,
那么存在一个函数$f : \mathbb{N} \rightarrow \mathbb{N}$
是从$\mathbb{N}$到$\mathbb{N}$的严格递增函数,使得
\begin{align*}
  b_n = a_{f(n)} \text{对所有的$n \in \mathbb{N}$均成立}
\end{align*}

因为$(c_n)_{n=0}^\infty$是$(b_n)_{n=0}^\infty$的子序列,
那么存在一个函数$g : \mathbb{N} \rightarrow \mathbb{N}$
是从$\mathbb{N}$到$\mathbb{N}$的严格递增函数,使得
\begin{align*}
  c_n = b_{g(n)} \text{对所有的$n \in \mathbb{N}$均成立}
\end{align*}

因为$f$的值域与$g$的定义域是同一个集合,我们可以把$g,f$复合,
得到函数$g \circ f : \mathbb{N} \rightarrow \mathbb{N}$,
该函数是从$\mathbb{N}$到$\mathbb{N}$的严格递增函数,使得
\begin{align*}
  c_n = a_{(g \circ f)(n)} \text{对所有的$n \in \mathbb{N}$均成立}
\end{align*}

由定义6.6.1 可知,此时$(c_n)_{n=0}^\infty$是$(a_n)_{n=0}^\infty$的子序列
\end{document}