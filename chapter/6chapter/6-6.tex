\documentclass{article}
\usepackage{mathtools} 
\usepackage{fontspec}
\usepackage[UTF8]{ctex}
\usepackage{amsthm}
\usepackage{mdframed}
\usepackage{xcolor}
\usepackage{amssymb}
\usepackage{amsmath}

\newmdtheoremenv[
  backgroundcolor=gray!10,
  linewidth=0pt,
  innerleftmargin=10pt,
  innerrightmargin=10pt,
  innertopmargin=10pt,
  innerbottommargin=10pt
]{zgraytheorem}{}
% 定义说明环境样式
\newtheoremstyle{mystyle}% 说明环境样式的名称
  {1em}% 上方间距
  {1em}% 下方间距
  {\normalfont}% 说明内容的字体样式
  {}% 缩进量
  {\bfseries}% 说明标记的字体样式
  {.}% 说明标记和说明内容之间的标点
  {1em}% 说明标记后的水平空间
  {}% 说明标记后的垂直空间
% 使用新定义的样式创建说明环境
\theoremstyle{mystyle}
\newtheorem*{zremark}{说明}


\begin{document}
\title{6.6 习题}
\maketitle

\section*{6.6.1}

(1)自反性

定义$f(n)=n$的函数$f : \mathbb{N} \rightarrow \mathbb{N}$
是从$\mathbb{N}$到$\mathbb{N}$的严格递增函数,使得
\begin{align*}
  a_n = a_{f(n)} = a_n \text{对所有的$n \in \mathbb{N}$均成立}
\end{align*}
由定义6.6.1 可知,此时$(a_n)_{n=0}^\infty$是$(a_n)_{n=0}^\infty$的一个子序列。

(2)传递性

因为$(b_n)_{n=0}^\infty$是$(a_n)_{n=0}^\infty$的子序列,
那么存在一个函数$f : \mathbb{N} \rightarrow \mathbb{N}$
是从$\mathbb{N}$到$\mathbb{N}$的严格递增函数,使得
\begin{align*}
  b_n = a_{f(n)} \text{对所有的$n \in \mathbb{N}$均成立}
\end{align*}

因为$(c_n)_{n=0}^\infty$是$(b_n)_{n=0}^\infty$的子序列,
那么存在一个函数$g : \mathbb{N} \rightarrow \mathbb{N}$
是从$\mathbb{N}$到$\mathbb{N}$的严格递增函数,使得
\begin{align*}
  c_n = b_{g(n)} \text{对所有的$n \in \mathbb{N}$均成立}
\end{align*}

因为$f$的值域与$g$的定义域是同一个集合,我们可以把$g,f$复合,
得到函数$g \circ f : \mathbb{N} \rightarrow \mathbb{N}$,
该函数是从$\mathbb{N}$到$\mathbb{N}$的严格递增函数,使得
\begin{align*}
  c_n = a_{(g \circ f)(n)} \text{对所有的$n \in \mathbb{N}$均成立}
\end{align*}

由定义6.6.1 可知,此时$(c_n)_{n=0}^\infty$是$(a_n)_{n=0}^\infty$的子序列

\section*{6.6.2}

略

\section*{6.6.3}

证明存在性。这里采用的方法,是先构造出目标对象。
这里需要考察的是,构造的目标是否满足要求。
具体来说,对于本习题,需要确定构造的序列是存在的,并确定构造的序列的倒数是收敛于$0$的(习题6.6.5与本题类似)。

(1)证明序列的每一项都是存在的

归纳法证明。

$j=0$,因为序列$(a_n)_{n=0}^\infty$是无界的,所以肯定存在$|a_n| \geq 0$,取第一个满足要求的$n$即可。

\begin{zgraytheorem}
  \begin{zremark}
    $j=0$时,如果$a_{n_0}=0$,会导致错误,所以$b_0$应该是要限制为非零的。
  \end{zremark}
\end{zgraytheorem}

归纳假设,$j-1$时,项是存在的。

$j$时,由于序列$(a_n)_{n=0}^\infty$是无界的,所以肯定存在$|a_n| \geq j$,此时满足条件的$n$至少有一个,
可以看做是一个集合,使用公理3.5(分类公理),可以得到所有元素都大于$n_{j-1}$的集合$A$,
取该集合的下确界作为$n_j$(这个下确界肯定是存在的,因为集合是有下界的。定理 5.5.9的推论)。

(2)$\lim\limits_{n \rightarrow \infty}1/b_n$存在且等于$0$。

对任意$\epsilon > 0$,都存在$\epsilon \geq 1/j$(因为$1/j$递增且极限为$0$),
由$(b_n)_{n=0}^\infty$的构造方式,可知,取$n_j$时,$|b_j| = |a_{n_j}| \geq j$,
且由序列$(b_n)_{n=0}^\infty$是递增的,可知,当$n \geq n_j$时,$|b_n| \geq j$均成立,
于是$|1/b_n| \leq 1/j \leq \epsilon$对$n \geq n_j$均成立。

由$\epsilon$的任意性,可知,$\lim\limits_{n \rightarrow \infty}1/b_n$存在且等于$0$。

\section*{6.6.4}

\textbf{(a)$\Rightarrow$ (b)}

序列$(a_n)_{n=0}^\infty$收敛于$L$,
那么,对任意$\epsilon > 0$,存在$N \geq 0$,
使得$|a_n - L| \leq \epsilon$对$n \geq N$均成立。

由子序列的定义(定义 6.6.1)可知,
序列$(a_n)_{n=0}^\infty$的任意子序列$(b_n)_{n=0}^\infty$,
都会存在一个严格递增的函数$f : \mathbb{N} \rightarrow \mathbb{N}$使得
\begin{align*}
  b_n = a_{f(n)} \text{对所有的$n \in \mathbb{N}$均成立}
\end{align*}
由$f$的定义可知$f(n) \geq n$,所以$n \geq N$时,$f(n) \geq N$,
所以$|b_n - L| = |a_{f(n)} - L | \leq \epsilon$对$n \geq N$均成立。
所以,序列$(b_n)_{n=0}^\infty$收敛于$L$。

由于$(b_n)_{n=0}^\infty$是任意的子序列,所以命题得证。

\textbf{(b)$\Rightarrow$ (a)}

由自反性 可知$(a_n)_{n=0}^\infty$也是$(a_n)_{n=0}^\infty$的子序列,
题设已经说明$(a_n)_{n=0}^\infty$收敛于$L$。

\section*{6.6.5}

\textbf{(a)$\Rightarrow$ (b)}

(1)证明序列的每一项都是存在的

归纳法证明。

$j=0$时,定义$a_{n_0}=a_0$。

归纳假设,$j-1$时,项是存在的。

$j$时,现在要证明$b_j:=a_{n_j}$是存在的。
由$L$是极限点,所以取$\epsilon = 1/j > 0$,对$N=n_{j-1}$(归纳假设,保证了$n_{j-1}$存在),
存在$n \geq N$使得$|a_n - L| \leq \epsilon$,满足该条件的$n$是一个集合,我们取其中最小值,
此时的最小值就是$n_j$且可知$a_{n_j}$是存在的。

(2)序列的收敛性

对任意实数$\epsilon > 0$,存在$1/j \leq \epsilon$(存在的原因是$1/j$收敛于$0$)。
通过序列$(a_{n_j})_{j=0}^\infty$的构造方式,
可知,只要证明存在$N, n = N$有$|a_n - L| \leq 1/j$,那么,就有$n > N$有$|a_n - L| < 1/j$,
即:$n \geq N$有$|a_n - L| \leq 1/j$。接下来只要证明这个$N$是存在的即可。
由构造方式可知$|a_{n_j} - L | \leq 1/j \leq \epsilon$,
所以,可取$N = n_j$,即$N$是存在的。


\textbf{(b)$\Rightarrow$ (a)}

设收敛于$L$的子序列是$(b_n)_{n=0}^\infty$,
因为是子序列,存在一个严格递增的函数$f : \mathbb{N} \rightarrow \mathbb{N}$使得
\begin{align*}
  b_m = a_{f(n)} \text{对所有的$n \in \mathbb{N}$均成立}
\end{align*}
(注意:这里为了讨论的方便,把子序列的下标改为$m$)
因为收敛于$L$,那么,对任意$\epsilon > 0$,存在$M \geq 0$,
$|b_m - L| \leq \epsilon$对$m \geq M$均成立,
因为$f$是严格递增的函数,且没有上界,
对每一个$N$,都存在$n$使得$f(n) \geq max(M, N)$,
因为$f(n) \geq M$,所以,
\begin{align*}
  |b_m - L|      & \leq \epsilon \\
  |a_{f(n)} - L| & \leq \epsilon
\end{align*}
由$\epsilon$的任意性,可知,$L$是$(a_n)_{n=0}^\infty$的极限点。

\end{document}