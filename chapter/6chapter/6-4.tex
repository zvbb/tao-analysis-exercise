\documentclass{article}
\usepackage{mathtools} 
\usepackage{fontspec}
\usepackage[UTF8]{ctex}
\usepackage{amsthm}
\usepackage{mdframed}
\usepackage{xcolor}
\usepackage{amssymb}
\usepackage{amsmath}

\newmdtheoremenv[
  backgroundcolor=gray!10,
  linewidth=0pt,
  innerleftmargin=10pt,
  innerrightmargin=10pt,
  innertopmargin=10pt,
  innerbottommargin=10pt
]{zgraytheorem}{}
% 定义说明环境样式
\newtheoremstyle{mystyle}% 说明环境样式的名称
  {1em}% 上方间距
  {1em}% 下方间距
  {\normalfont}% 说明内容的字体样式
  {}% 缩进量
  {\bfseries}% 说明标记的字体样式
  {.}% 说明标记和说明内容之间的标点
  {1em}% 说明标记后的水平空间
  {}% 说明标记后的垂直空间
% 使用新定义的样式创建说明环境
\theoremstyle{mystyle}
\newtheorem*{zremark}{说明}


\begin{document}
\title{6.4 习题}
\maketitle

\section*{6.4.1}

(1)
序列$(a_n)_{n=m}^\infty$收敛于$c$,那么对任意实数$\epsilon > 0$,
都是最终$\epsilon -$接近于$c$的,
即: 能够找到某个$N \geq m$使得$(a_n)_{n=N}^\infty$是$\epsilon-$接近于$c$的。
并且对于任意$N^\prime \geq m$,取$N_0 := max(N, N^\prime)$,
此时$(a_n)_{n=N_0}^\infty$是$\epsilon-$接近于$c$的,
即:$a_n$是$\epsilon-$接近于$c$,对$n \geq N_0$均成立,所以$c$是$\epsilon-$附着于$(a_n)_{n=N^\prime}^\infty$的。
由$\epsilon$的任意性,可知$c$是$(a_n)_{n=m}^\infty$的极限点。

(2)反证法,存在另一个极限点$d$,且$d \neq c$。
$(a_n)_{n=m}^\infty$收敛于$c$,那么对实数$\epsilon > 0$,是最终$\epsilon -$接近于$c$的。
即: 能够找到$N \geq m$使得$(a_n)_{n=N}^\infty$是$\epsilon-$接近于$c$的。

同时$d$是$(a_n)_{n=m}^\infty$的极限点,那么,$d$是$\epsilon-$附着于$(a_n)_{n=N}^\infty$的,
那么存在一个$n \geq N$使得$a_n$是$\epsilon-$接近于$d$的,
如果$d > c$,取$0 <\epsilon < (d-c)/2$,
此时,$|a_n - d| \leq \epsilon$与$|a_n - c| \leq \epsilon$无法同时满足,即$a_n$
无法同时$\epsilon-$接近于$c,d$。

$d \leq c$同理。

\section*{6.4.2}

这里只说明极限点和上极限,因为下极限的证明可以用上极限类推。

设$(a_n)_{n=m}^\infty$是一个实数序列,$c$是一个实数,且$m^\prime \geq m$是一个整数,$k \geq 0$是一个非负整数。

\textbf{(1)与习题6.1.3类似的结论}

(1.1)$c$是$(a_n)_{n=m}^\infty$极限点,当且仅当$c$是$(a_n)_{n=m^\prime}^\infty$极限点。

\textbf{$\Rightarrow$} $c$是$(a_n)_{n=m}^\infty$极限点,
当且仅当“对任意$\epsilon > 0$,对每一个$N \geq m$,$c$都是$\epsilon-$附着于$(a_n)_{n=N}^\infty$的”,
我们把引号中的性质定义声明为$P$,即对任意$N$,只要$N \geq m$都具有性质$P$。
因为$m^\prime \geq m$,于是对任意$N$,$N \geq m^\prime \geq m$都具有性质$P$,
所以$c$是$(a_n)_{n=m^\prime}^\infty$的极限点。

\textbf{$\Leftarrow$} $c$是$(a_n)_{n=m^\prime}^\infty$的极限点。
对任意$\epsilon > 0$,对每一个$N$,

如果$N \geq m^\prime$,由于$c$是$(a_n)_{n=m^\prime}^\infty$的极限点,
那么,$c$都是$\epsilon-$附着于$(a_n)_{n=N}^\infty$的;

如果$m \leq N < m^\prime$,我们要证明此时$c$也是$\epsilon-$附着于$(a_n)_{n=N}^\infty$,
即:要证明存在一个$n \geq N$使得$a_n$是$\epsilon-$接近于$c$。
我们可以取$n \geq m^\prime$,那么$n$也是大于$N$,还是由$c$是$(a_n)_{n=m^\prime}^\infty$的极限点,
保证了$n$的存在性。

综上$c$是$(a_n)_{n=m}^\infty$的极限点。

(1.2)$c$是$(a_n)_{n=m}^\infty$的上极限,当且仅当$c$是$(a_n)_{n=m^\prime}^\infty$的上极限。

\textbf{$\Rightarrow$} $c$是$(a_n)_{n=m}^\infty$的上极限,即:序列$(a_N^+)_{N=m}^\infty$的下确界是$c$。
序列$(a_N^+)_{N=m^\prime}^\infty$是序列$(a_N^+)_{N=m}^\infty$的子集。

反证法,假设$c$不是$(a_n)_{n=m^\prime}^\infty$的上极限,设$(a_n)_{n=m^\prime}^\infty$的上极限是$c^\prime$
【这里其实要证明$c^\prime$的存在性。可以通过以下命题得到$c^\prime$是存在的:
\textbf{有上界序列存在实数上极限,否则上极限不是实数,而是$+\infty$}】。

如果$c^\prime > c$,那么,存在$m \leq N_0 < m^\prime$
使得$c \leq a_{N_0}^+ < c^\prime$,因为$(a_n)_{n=m^\prime}^\infty$是$(a_n)_{n=N_0}^\infty$的子集,
所以$sup((a_n)_{n=m^\prime}^\infty) \leq sup((a_n)_{n=N_0}^\infty)$,
又因为$c^\prime \leq sup((a_n)_{n=m^\prime}^\infty)$,
于是$c^\prime \leq sup((a_n)_{n=N_0}^\infty)$,即:$c^\prime \leq a_{N_0}^+$。
这与$c \leq a_{N_0}^+ < c^\prime$矛盾。

如果$c > c^\prime$,因为序列$(a_N^+)_{N=m^\prime}^\infty$是序列$(a_N^+)_{N=m}^\infty$的子集,
所以$inf((a_N^+)_{N=m^\prime}^\infty) \geq inf((a_N^+)_{N=m}^\infty)$,即: $c^\prime \geq c$,
这与$c > c^\prime$矛盾。

综上,$c = c^\prime$。

\textbf{$\Leftarrow$} $c$是$(a_n)_{n=m^\prime}^\infty$的上极限,即:序列$(a_N^+)_{N=m^\prime}^\infty$的下确界是$c$。
序列$(a_N^+)_{N=m^\prime}^\infty$是序列$(a_N^+)_{N=m}^\infty$的子集。

反证法,假设$c$不是$(a_n)_{n=m}^\infty$的上极限,设$(a_n)_{n=m}^\infty$的上极限是$c^\prime$。

如果$c > c^\prime$,那么,存在$m \leq N_0 < m^\prime$
使得$c^\prime \leq a_{N_0}^+ < c$,因为$(a_n)_{n=m^\prime}^\infty$是$(a_n)_{n=N_0}^\infty$的子集,
所以$sup((a_n)_{n=m^\prime}^\infty) \leq sup((a_n)_{n=N_0}^\infty)$,
又因为$c \leq sup((a_n)_{n=m^\prime}^\infty)$,
于是$c \leq sup((a_n)_{n=N_0}^\infty)$,即:$c < a_{N_0}^+$。这与$c^\prime \leq a_{N_0}^+ < c$矛盾。

如果$c < c^\prime$,因为序列$(a_N^+)_{N=m^\prime}^\infty$是序列$(a_N^+)_{N=m}^\infty$的子集,
所以$inf((a_N^+)_{N=m^\prime}^\infty) \geq inf((a_N^+)_{N=m}^\infty)$,即:$c^\prime \geq c$,
这与$c < c^\prime$矛盾。

综上,$c = c^\prime$。

\textbf{与习题6.1.4类似的结论}

该问题是6.1.3的拓展,这里我只证明一种情况。

(2.1) $c$是$(a_n)_{n=m}^\infty$的极限点,当且仅当$c$是$(a_{n+k})_{n=m}^\infty$的极限点。

如果我们能证明$(a_{n})_{n=m^\prime}^\infty$与$(a_{n+k})_{n=m}^\infty$相等的,然后通过(1.1)就可以证明该命题,
接下来我们证明这两个序列的相等的。

通过定义5.5.1 可知,序列就是函数,是一个从集合$Z$到$R$的函数。
于是我们要证明两个序列相等,只需要证明其对应函数相等。
通过定义3.3.7(函数的相等)来进行接下来的证明。

设$f: N \rightarrow R$为函数$f(n)=a_{n+k}$,设$g: N \rightarrow N$为函数$g(m)=m$。
那么$f \circ g = f(g(m)) = a_{g(m)+k} = a_{m+k}$。

设$f^\prime: N \rightarrow R$为函数$f^\prime(n)=a_{n}$,设$g^\prime: N \rightarrow N$为函数$g^\prime(m)=m+k$。
那么$f^\prime \circ g^\prime = f^\prime(g^\prime(m)) = a_{m+k}$。

由$f \circ g,f^\prime \circ g^\prime$的构造过程可知两个具有相同的定义域,
又对于任意的$x \in N$,$f \circ g(x) = a_{x+k}, f^\prime \circ g^\prime(x)=a_{x+k}$,
所以$f \circ g(x) = f^\prime \circ g^\prime(x)$,由此可知 两个函数相等,即两个序列相等。

\section*{6.4.3}

不妨设$E := \{a_n: n \geq m\}$,$M=sup(E),M^\prime=inf(E)$。

\textbf{(c)}

由例6.2.10 可知$M \geq M^\prime$,接下来我只证明$L^+ \leq M$(可以类推$M^\prime \leq L^-$)和$L^- \leq L+$。

反证法,假设$L^+ > M$。由命题6.3.6 可知对任意$n \geq m$,都有$a_n \leq M$。
因为$L^+ := inf(a_N^+)_{N=m}^\infty$则也由命题6.3.6 可知存在$N \geq m$使得$a_N^+ > L^+$,
由$a_N^+ := sup(a_n)_{n=N}^\infty$,可知存在$n \geq N$使得$a_n > L^+$,这与任意$a_n \leq M$矛盾。

反证法,假设$L^- > L^+$,由$L^- := sup(a_N^-)_{N=m}^\infty$可知存在$N_0 \geq m$使得$a_{N_0}^- > L^+$,
由因为$L^+ := inf(a_N^+)_{N=m}^\infty$,所以存在$N_1 \geq m$使得$a_{N_0}^- > a_{N_1}^+$【否则上极限就不是$L^+$了,
而是一个大于等于$a_{N_0}^-$的数了】。
由$a_{N_0}^- := inf(a_n)_{n=N_0}^\infty$定义, 可知 对$n \geq N_0$都有$a_n \geq a_{N_0}^-$,
由$a_{N_1}^+ := sup(a_n)_{n=N_1}^\infty$定义,可知对$n \geq N_1$都有$a_n \leq a_{N_1}^+$,
取$n \geq max(N_0,N_1)$此时$a_{N_0}^- \leq a_n \leq a_{N_1}^+$,这与$a_{N_0}^- > a_{N_1}^+$矛盾。

\textbf{(d)}

这里我只证明$c \leq L^+$,因为$L^- \leq c$可以类推。

反证法,假设$c > L^+$,由$L^+ := inf(a_N^+)_{N=m}^\infty$可知,
由命题6.3.6 可知,存在$N_0 \geq m$使得$a_{N_0}^+ < c$,
又因为$a_{N_0}^+ := sup(a_n)_{n=N_0}^\infty$,所以任意$n \geq N_0$都有$a_n \leq a_{N_0}^+$,
由此可知,
\begin{align*}
  |c - a_n| & = |c - a_{N_0}^+ + a_{N_0}^+ - a_n|   \\
            & = |c - a_{N_0}^+| + |a_{N_0}^+ - a_n| \\
            & > |c - a_{N_0}^+|
\end{align*}
此时$c,a_n$的距离总是大于$|c - a_{N_0}^+|$,这与$c$是极限点的定义矛盾。

\textbf{(e)}

这里我只证明$L^+$是极限点,因为$L^-$可以类推。

反证法,假设$L^+$不是极限点,那么通过极限点的定义6.4.1 可知,存在$\epsilon > 0, N_0 \geq m$,
此时$L^+$不是$\epsilon -$附着于$(a_n)_{n=N_0}^\infty$的,即对任意$n \geq N_0$,都有,
\begin{align*}
   & |L^+ - a_n| > \epsilon                             \\
   & \Rightarrow                                        \\
   & a_n > L^+ + \epsilon \text{或} a_n < L^+ - \epsilon
\end{align*}

因为$L^+ := inf(a_N^+)_{N=m}^\infty$,
那么对任意$N \geq m$都有$a_N^+ \geq L^+$。
又$a_N^+ := sup(a_n)_{n=N}^\infty$。综上,我们可以得到,对任意$N \geq m, n \geq N$都有:
\begin{equation}
  \begin{cases*}
    a_n \leq a_N^+ \\
    L^+ \leq a_N^+
  \end{cases*}
\end{equation}

(1)如果$n \geq N_0, a_n > L^+ + \epsilon$,那么,
\begin{align*}
  a_N^+ \geq a_n > L^+ + \epsilon
\end{align*}
而对于哪些$N < N_0$,由$a_N^+$的定义可知,$a_N+ \geq a_{N_0}^+$,
于是此时$L^+ + \epsilon$是上极限,这与下确界的唯一性矛盾(上极限其实就是集合的下确界)。

(2)如果$n \geq N_0, a_n < L^+ - \epsilon$,由此可知,$N \geq N_0$时,
\begin{align*}
  a_N^+ \leq L^+ - \epsilon
\end{align*}
这与$L^+ \leq a_N^+$矛盾。

\textbf{(f)}

\textbf{$\Rightarrow$}

由命题6.4.5 可知$c$是极限点,如果$L^+ \neq c$,那么由(e)可知$L^+$也是极限点,这与命题6.4.5
的后半部分相悖。

\textbf{$\Leftarrow$}

由于$L^+ = L^-$,由(e)可知,$(a_n)_{n=m}^\infty$有且只有一个极限点,也就是说$c$是极限点。
接下来要证明序列收敛与$c$。

反证法,假设$c$序列不收敛于$c$,那么,存在$\epsilon > 0$,找不到$N \geq m$,使得$n \geq N$时,
都有$|a_n -c| \leq \epsilon$,即:总是存在$|a_n -c| > \epsilon$。

(1)如果$a_n > c+\epsilon$,由$L^+,a_N^+$的定义 可知对任意$N \geq m, n \geq N$都有,
\begin{equation}
  \begin{cases*}
    a_n \leq a_N^+ \\
    L^+ \leq a_N^+
  \end{cases*}
\end{equation}
由此可得$a_N^+ \geq c+\epsilon=L^+ + \epsilon$对任意$N$均成立,由此可知上极限不是$L^+$,这与题设相悖。

(2)如果$a_n < c - \epsilon$,同理可证其与下极限是$L^-$相悖。

\section*{6.4.4}

这里我只证明(1)(3),其他的可以类推。

\textbf{(1)}

不妨设
\begin{align*}
  M        & = sup(b_n)_{n=m}^\infty \\
  M^\prime & = sup(a_n)_{n=m}^\infty
\end{align*}
反证法,假设$M^\prime > M$,取$m, M < m < M^\prime$,
由命题6.3.6 可知至少存在一个$n \geq m$使得$m < a_n \leq M^\prime$,此时$a_n > m > M$,
由于$M$是上确界,所以$b_n \leq M$,于是$a_n > b_n$,与题设相悖。

\textbf{(3)}

不妨设
\begin{align*}
  L^+          & = inf(b_N^+)_{n=m}^\infty \\
  L^{+^\prime} & = inf(a_N^+)_{n=m}^\infty
\end{align*}
又因为对任意$N \geq m$都有
\begin{align*}
  a_N^+ & := sup(a_n)_{n=N}^\infty \\
  b_N^+ & := sup(b_n)_{n=N}^\infty
\end{align*}
由(1)可知$b_N^+ \geq a_N^+$,于是由(2)可知 $L^{+^\prime} \leq L^+$

\section*{6.4.5}

由命题6.4.12(f)可知,$(a_n)_{n=m}^\infty,(c_n)_{n=m}^\infty$收敛于$L$,
那么,两者的上极限$L^+$和下极限$L^-$都等于$L$,即:$L^+ = L^- = L$。

设$(b_n)_{n=m}^\infty$的上极限和下极限分别为$L^{+^\prime},L^{-^\prime}$。
由引理6.4.15 可知,
\begin{equation}
  \begin{cases*}
    L^- \leq L^{-^\prime} \leq L^- \\
    L^+ \leq L^{+^\prime} \leq L^+
  \end{cases*}
\end{equation}
由此可知$L^{+^\prime}=L^{-^\prime}=L$,
由命题6.4.12(f)可知$(b_n)_{n=m}^\infty$收敛于$L$

\end{document}