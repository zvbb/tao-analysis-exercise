\documentclass{article}
\usepackage{mathtools} 
\usepackage{fontspec}
\usepackage[UTF8]{ctex}
\usepackage{amsthm}
\usepackage{mdframed}
\usepackage{xcolor}
\usepackage{amssymb}
\usepackage{amsmath}

\newmdtheoremenv[
  backgroundcolor=gray!10,
  linewidth=0pt,
  innerleftmargin=10pt,
  innerrightmargin=10pt,
  innertopmargin=10pt,
  innerbottommargin=10pt
]{zgraytheorem}{}
% 定义说明环境样式
\newtheoremstyle{mystyle}% 说明环境样式的名称
  {1em}% 上方间距
  {1em}% 下方间距
  {\normalfont}% 说明内容的字体样式
  {}% 缩进量
  {\bfseries}% 说明标记的字体样式
  {.}% 说明标记和说明内容之间的标点
  {1em}% 说明标记后的水平空间
  {}% 说明标记后的垂直空间
% 使用新定义的样式创建说明环境
\theoremstyle{mystyle}
\newtheorem*{zremark}{说明}


\begin{document}
\title{6.4 习题}
\maketitle

\section*{6.4.1}

(1)
序列$(a_n)_{n=m}^\infty$收敛于$c$,那么对任意实数$\epsilon > 0$,
都是最终$\epsilon -$接近于$c$的,
即: 能够找到某个$N \geq m$使得$(a_n)_{n=N}^\infty$是$\epsilon-$接近于$c$的。
并且对于任意$N^\prime \geq m$,取$N_0 := max(N, N^\prime)$,
此时$(a_n)_{n=N_0}^\infty$是$\epsilon-$接近于$c$的,
即:$a_n$是$\epsilon-$接近于$c$,对$n \geq N_0$均成立,所以$c$是$\epsilon-$附着于$(a_n)_{n=N^\prime}^\infty$的。
由$\epsilon$的任意性,可知$c$是$(a_n)_{n=m}^\infty$的极限点。

(2)反证法,存在另一个极限点$d$,且$d \neq c$。
$(a_n)_{n=m}^\infty$收敛于$c$,那么对实数$\epsilon > 0$,是最终$\epsilon -$接近于$c$的。
即: 能够找到$N \geq m$使得$(a_n)_{n=N}^\infty$是$\epsilon-$接近于$c$的。

同时$d$是$(a_n)_{n=m}^\infty$的极限点,那么,$d$是$\epsilon-$附着于$(a_n)_{n=N}^\infty$的,
那么存在一个$n \geq N$使得$a_n$是$\epsilon-$接近于$d$的,
如果$d > c$,取$0 <\epsilon < (d-c)/2$,
此时,$|a_n - d| \leq \epsilon$与$|a_n - c| \leq \epsilon$无法同时满足,即$a_n$
无法同时$\epsilon-$接近于$c,d$。

$d \leq c$同理。

\section*{6.4.2}

这里只说明极限点和上极限,因为下极限的证明可以用上极限类推。

设$(a_n)_{n=m}^\infty$是一个实数序列,$c$是一个实数,且$m^\prime \geq m$是一个整数。

(1)与习题6.1.3类似的结论

(1.1)$c$是$(a_n)_{n=m}^\infty$极限点,当且仅当$c$是$(a_n)_{n=m^\prime}^\infty$极限点。

$\Rightarrow$ $c$是$(a_n)_{n=m}^\infty$极限点,
那么,“对任意$\epsilon > 0$,对每一个$N \geq m$,$c$都是$\epsilon-$附着于$(a_n)_{n=N}^\infty$的”,
我们把引号中的性质定义声明为$P(N)$,即对任意$N$,只要$N \geq m$都具有性质$P$。
因为$m^\prime \geq m$,于是对任意$N$,$N \geq m^\prime \geq m$都具有性质$P$,
所以$c$是$(a_n)_{n=m^\prime}^\infty$的极限点。

$\Leftarrow$ $c$是$(a_n)_{n=m^\prime}^\infty$的极限点。
对任意$\epsilon > 0$,对每一个$N$,
如果$N \geq m^\prime$,由于$c$是$(a_n)_{n=m^\prime}^\infty$的极限点,
那么,$c$都是$\epsilon-$附着于$(a_n)_{n=N}^\infty$的;
如果$m \leq N < m^\prime$,我们要证明此时$c$也是$\epsilon-$附着于$(a_n)_{n=N}^\infty$,
即:要证明存在一个$n \geq N$使得$a_n$是$\epsilon-$接近于$c$。
我们可以取$n \geq m^\prime$,那么$n$也是大于$N$,还是由$c$是$(a_n)_{n=m^\prime}^\infty$的极限点,
保证了$n$的存在性。

综上$c$是$(a_n)_{n=m}^\infty$的极限点。

(1.2)$c$是$(a_n)_{n=m}^\infty$的上极限,当且仅当$c$是$(a_n)_{n=m^\prime}^\infty$的上极限。

$\Rightarrow$ $c$是$(a_n)_{n=m}^\infty$的上极限,即:序列$(a_N^+)_{N=m}^\infty$的下确界是$c$。
序列$(a_N^+)_{N=m^\prime}^\infty$是序列$(a_N^+)_{N=m}^\infty$的子集。

反证法,假设$c$不是$(a_n)_{n=m^\prime}^\infty$的上极限,设$(a_n)_{n=m^\prime}^\infty$的上极限是$c^\prime$
【这里其实要证明$c^\prime$的存在性。可以通过以下命题得到$c^\prime$是存在的:
\textbf{有上界序列存在实数上极限,否则上极限不是实数,而是$+\infty$}】。

如果$c^\prime > c$,那么,存在$m \leq N_0 < m^\prime$
使得$c \leq a_{N_0}^+ < c^\prime$,因为$(a_n)_{n=m^\prime}^\infty$是$(a_n)_{n=N_0}^\infty$的子集,
所以$sup((a_n)_{n=m^\prime}^\infty) \leq sup((a_n)_{n=N_0}^\infty)$,
又因为$c^\prime \leq sup((a_n)_{n=m^\prime}^\infty)$,
于是$c^\prime \leq sup((a_n)_{n=N_0}^\infty)$,即:$c^\prime < a_{N_0}^+$。这与$c < a_{N_0}^+$矛盾。

如果$c > c^\prime$,因为序列$(a_N^+)_{N=m^\prime}^\infty$是序列$(a_N^+)_{N=m}^\infty$的子集,
所以$inf((a_N^+)_{N=m^\prime}^\infty) \geq inf((a_N^+)_{N=m}^\infty)$,这与$c > c^\prime$矛盾。

综上,$c = c^\prime$。

$\Leftarrow$ $c$是$(a_n)_{n=m^\prime}^\infty$的上极限,即:序列$(a_N^+)_{N=m^\prime}^\infty$的下确界是$c$。
序列$(a_N^+)_{N=m^\prime}^\infty$是序列$(a_N^+)_{N=m}^\infty$的子集。

反证法,假设$c$不是$(a_n)_{n=m}^\infty$的上极限,设$(a_n)_{n=m}^\infty$的上极限是$c^\prime$。

如果$c > c^\prime$,那么,存在$m \leq N_0 < m^\prime$
使得$c^\prime \leq a_{N_0}^+ < c$,因为$(a_n)_{n=m^\prime}^\infty$是$(a_n)_{n=N_0}^\infty$的子集,
所以$sup((a_n)_{n=m^\prime}^\infty) \leq sup((a_n)_{n=N_0}^\infty)$,
又因为$c \leq sup((a_n)_{n=m^\prime}^\infty)$,
于是$c \leq sup((a_n)_{n=N_0}^\infty)$,即:$c < a_{N_0}^+$。这与$c^\prime \leq a_{N_0}^+ < c$矛盾。

如果$c < c^\prime$,因为序列$(a_N^+)_{N=m^\prime}^\infty$是序列$(a_N^+)_{N=m}^\infty$的子集,
所以$inf((a_N^+)_{N=m^\prime}^\infty) \geq inf((a_N^+)_{N=m}^\infty)$,这与$c < c^\prime$矛盾。

综上,$c = c^\prime$。

\end{document}