\documentclass{article}
\usepackage{mathtools} 
\usepackage{fontspec}
\usepackage[UTF8]{ctex}
\usepackage{amsthm}
\usepackage{mdframed}
\usepackage{xcolor}
\usepackage{amssymb}
\usepackage{amsmath}

\newmdtheoremenv[
  backgroundcolor=gray!10,
  linewidth=0pt,
  innerleftmargin=10pt,
  innerrightmargin=10pt,
  innertopmargin=10pt,
  innerbottommargin=10pt
]{zgraytheorem}{}
% 定义说明环境样式
\newtheoremstyle{mystyle}% 说明环境样式的名称
  {1em}% 上方间距
  {1em}% 下方间距
  {\normalfont}% 说明内容的字体样式
  {}% 缩进量
  {\bfseries}% 说明标记的字体样式
  {.}% 说明标记和说明内容之间的标点
  {1em}% 说明标记后的水平空间
  {}% 说明标记后的垂直空间
% 使用新定义的样式创建说明环境
\theoremstyle{mystyle}
\newtheorem*{zremark}{说明}


\begin{document}
\title{6.4 习题}
\maketitle

\section*{6.4.1}

(1)
序列$(a_n)_{n=m}^\infty$收敛于$c$,那么对任意实数$\epsilon > 0$,
都是最终$\epsilon -$接近于$c$的,
即: 能够找到某个$N \geq m$使得$(a_n)_{n=N}^\infty$是$\epsilon-$接近于$c$的。
并且对于任意$N^\prime \geq m$,取$N_0 := max(N, N^\prime)$,
此时$(a_n)_{n=N_0}^\infty$是$\epsilon-$接近于$c$的,
即:$a_n$是$\epsilon-$接近于$c$,对$n \geq N_0$均成立,所以$c$是$\epsilon-$附着于$(a_n)_{n=N^\prime}^\infty$的。
由$\epsilon$的任意性,可知$c$是$(a_n)_{n=m}^\infty$的极限点。

(2)反证法,存在另一个极限点$d$,且$d \neq c$。
$(a_n)_{n=m}^\infty$收敛于$c$,那么对实数$\epsilon > 0$,是最终$\epsilon -$接近于$c$的。
即: 能够找到$N \geq m$使得$(a_n)_{n=N}^\infty$是$\epsilon-$接近于$c$的。

同时$d$是$(a_n)_{n=m}^\infty$的极限点,那么,$d$是$\epsilon-$附着于$(a_n)_{n=N}^\infty$的,
那么存在一个$n \geq N$使得$a_n$是$\epsilon-$接近于$d$的,
如果$d > c$,取$0 <\epsilon < (d-c)/2$,
此时,$|a_n - d| \leq \epsilon$与$|a_n - c| \leq \epsilon$无法同时满足,即$a_n$
无法同时$\epsilon-$接近于$c,d$。

$d \leq c$同理。

\end{document}