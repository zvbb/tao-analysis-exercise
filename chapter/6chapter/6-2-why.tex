\documentclass{article}
\usepackage{mathtools} 
\usepackage{fontspec}
\usepackage[UTF8]{ctex}
\usepackage{amsthm}
\usepackage{mdframed}
\usepackage{xcolor}
\usepackage{amssymb}
\usepackage{amsmath}

\newmdtheoremenv[
  backgroundcolor=gray!10,
  linewidth=0pt,
  innerleftmargin=10pt,
  innerrightmargin=10pt,
  innertopmargin=10pt,
  innerbottommargin=10pt
]{zgraytheorem}{}
% 定义说明环境样式
\newtheoremstyle{mystyle}% 说明环境样式的名称
  {1em}% 上方间距
  {1em}% 下方间距
  {\normalfont}% 说明内容的字体样式
  {}% 缩进量
  {\bfseries}% 说明标记的字体样式
  {.}% 说明标记和说明内容之间的标点
  {1em}% 说明标记后的水平空间
  {}% 说明标记后的垂直空间
% 使用新定义的样式创建说明环境
\theoremstyle{mystyle}
\newtheorem*{zremark}{说明}


\begin{document}
\title{6.2 为什么}
\maketitle

\textbf{1.设$E$是空集,那么$sup(E)=-\infty$且$inf(E)=+\infty$。}

因为$+\infty, -\infty$都不是空集$E$中的元素,由定义6.2.6 可知,空集被归入情形(a),
即按照定义5.5.10 确定$sup(E)$,定义5.5.10 直接就说明了空集$E$的$sup(E)=-\infty$。


因为$-E$还是空集,所以,$inf(E) := -sup(-E) = -sup(E) = -(-\infty) = +\infty$。

\textbf{2.设$E$是广义实数集合,当$E$为空集,是上确界能够小于下确界的唯一形式。}

如果$E$不是空集。如果$E$是没有上界或存在包含$+\infty$,那么由定义6.2.6 和 定义5.5.10 可知,
$sup(E) = +\infty$,那么$inf(E) = -\infty$,由定义6.2.3(广义实数的排序)可知,$sup(E) > inf(E)$。

如果$E$有上界,设$sup(E) = M$,反证法,假设存在下确界$inf(E) = m$,并且$m > M$。
任意$x \in E$,由上确界与下确界的定义可知,
\begin{align}
  m & \leq x \\
  x & \leq M
\end{align}
可得$M \geq m$,与$m > M$矛盾,所以$sup(E) \geq inf(E)$。

\end{document}