\documentclass{article}
\usepackage{mathtools} 
\usepackage{fontspec}
\usepackage[UTF8]{ctex}
\usepackage{amsthm}
\usepackage{mdframed}
\usepackage{xcolor}
\usepackage{amssymb}
\usepackage{amsmath}

\newmdtheoremenv[
  backgroundcolor=gray!10,
  linewidth=0pt,
  innerleftmargin=10pt,
  innerrightmargin=10pt,
  innertopmargin=10pt,
  innerbottommargin=10pt
]{zgraytheorem}{}
% 定义说明环境样式
\newtheoremstyle{mystyle}% 说明环境样式的名称
  {1em}% 上方间距
  {1em}% 下方间距
  {\normalfont}% 说明内容的字体样式
  {}% 缩进量
  {\bfseries}% 说明标记的字体样式
  {.}% 说明标记和说明内容之间的标点
  {1em}% 说明标记后的水平空间
  {}% 说明标记后的垂直空间
% 使用新定义的样式创建说明环境
\theoremstyle{mystyle}
\newtheorem*{zremark}{说明}


\begin{document}
\title{6.1 习题}
\maketitle

\section*{6.1.1}

证明框架如下:由于$n,m$都是自然数,且$m>n$,所以存在正自然数$k$使得$k = n + k$,对$k$进行归纳。

\section*{6.1.2}

证明:

与定义6.1.5 说的是一个意思,证明略

\section*{6.1.3}

证明:

充分性:

如果$(a_n)_{n=m}^\infty$收敛与$c$,那么对任意的$\epsilon > 0$,该序列都是最终$\epsilon -$接近$c$的,
所以存在$N \geq m$使得$|a_n - c| \leq \epsilon$对所有的$n \geq N$均成立。
由题设可知$m^\prime > m$,于是存在$N^\prime := max(m^\prime, N),N^\prime \geq m^\prime$,使得$|a_n - c| \leq \epsilon$对所有的$n \geq N^\prime$均成立,
由于$\epsilon$是任意的,由习题6.1.2可知,$(a_n)_{n=m^\prime}^\infty$收敛与$c$。

必要性:

$(a_n)_{n=m^\prime}^\infty$收敛与$c$,那么对任意的$\epsilon > 0$,该序列都是最终$\epsilon -$接近$c$的。
所以存在$N \geq m^\prime$使得$|a_n - c| \leq \epsilon$对所有的$n \geq N$均成立。由于$m^\prime > m$,
所以$N \geq m$,该性质对序列$(a_n)_{n=m}^\infty$也成立,由于$\epsilon$是任意的,
由习题6.1.2可知,$(a_n)_{n=m^\prime}^\infty$收敛与$c$。

\section*{6.1.4}

证明:

$(a_n)_{n=m}^\infty$收敛于$c$,于是对任意$\epsilon > 0$,
存在一个$N \geq m$使得$|a_n - c| \leq \epsilon$对所有的$n \geq N$
均成立,由于$k \geq 0$是一个非负整数,所以$n + k \geq N$,
于是$|a_{n+k} - c| \leq \epsilon$对所有的$n \geq N$
均成立,由习题6.1.2可知,$(a_{n+k})_{n=m}^\infty$收敛与$c$。

\section*{6.1.5}

证明:

要证明序列$(a_n)_{n=m}^\infty$是柯西序列,
则对于任意$\epsilon>0$,我们需要证明序列$(a_n)_{n=m}^\infty$是最终$\epsilon -$接近的。
于是设$\epsilon > 0$是一个任意的实数,那么$\epsilon/2 > 0$。因为$(a_n)_{n=m}^\infty$是收敛的
实数序列,不妨设收敛于实数$L$,可知$(a_n)_{n=m}^\infty$是最终$\epsilon -$接近与$L$的,
于是存在一个$N \geq m$使得$d(a_n,L)\leq \epsilon/2$对所有的$n \geq N$均成立。
任意$j,k \geq N$,有$d(a_j,L)\leq \epsilon/2$,$d(a_k,L)\leq \epsilon/2$,
于是根据三角不等式可得,$d(a_j,a_k) \leq \epsilon$,因此$(a_n)_{n=m}^\infty$是最终$\epsilon -$接近的。
由于$\epsilon$是任意选取的,因此$(a_n)_{n=m}^\infty$是柯西序列。

\section*{6.1.6}

证明:

证明为什么$a_n > L + \epsilon / 2$或$a_n < L - \epsilon / 2$,其余的按书中的提示证明就可以了。

序列$(a_n)_{n=m}^\infty$不是最终$\epsilon -$接近与$L$的,
即对任意的$N \geq m$都存在$|a_n - L| > \epsilon$对所有的$n \geq N$均成立。

序列$(a_n)_{n=m}^\infty$是柯西序列,所以存在$N_0$使得$|a_j - a_k| \leq \epsilon/2$对所有的$j,k \geq N_0$均成立。

固定$a_n=j_k$,所以,
\begin{align*}
  |a_j - a_n|                  & \leq \epsilon/2                      \\
  \Rightarrow a_n - \epsilon/2 & \leq a_j       \leq a_n + \epsilon/2
\end{align*}
又因为,$|a_n - L| > \epsilon$所以$a_n > \epsilon + L$或$a_n < L - \epsilon$。

如果$a_n > \epsilon + L$,那么,
\begin{align*}
  a_n - \epsilon/2          & \leq a_j \\
  L + \epsilon - \epsilon/2 & < a_j    \\
  L + \epsilon/2            & < a_j
\end{align*}

如果$a_n < L - \epsilon$,那么,

\begin{align*}
  a_j & \leq a_n + \epsilon/2       \\
  a_j & < L - \epsilon + \epsilon/2 \\
  a_j & < L - \epsilon/2
\end{align*}

\section*{6.1.7}

证明:

证明方法与命题6.1.4的类似。

首先假设$(a_n)_{n=m}^\infty$是定义5.1.12 意义下的有界序列,那么存在有理数$M$,该序列以$M$为界,
由于有理数$M$也是实数,所以$(a_n)_{n=m}^\infty$是定义6.1.16 意义下的有界序列。

现在假设是定义6.1.16 下的有界序列,那么存在实数$M$,该序列以$M$为界,根据命题5.4.12 可知,
存在一个比$M$大的有理数$M^\prime$,由于$M^\prime$是有理数,且$M < M^\prime$,所有该序列也以$M^\prime$为界,
所以$(a_n)_{n=m}^\infty$是定义5.1.12 意义下的有界序列。

\section*{6.1.8}

\textbf{(a)}

我们必须证明$(a_n+b_n)_{n=m}^\infty$收敛于$x+y$。换言之,对于任意的$\epsilon > 0$,
我们需要证明序列$(a_n+b_n)_{n=m}^\infty$是最终$\epsilon -$接近$x+y$的。

因为$(a_n)_{n=m}^\infty$收敛于$x$且$\epsilon /2 > 0$,则序列是最终$\epsilon /2 -$接近$x$,
即存在$N_a \geq m$使得$|a_n - x| \leq \epsilon /2$对所有的$n \geq N_a$均成立。

同理对序列$(b_n)_{n=m}^\infty$存在$N_b \geq m$使得$|b_n - y| \leq \epsilon /2$对所有的$n \geq N_b$均成立。

取$N := max(N_a, N_b)$,于是对所有的$n \geq N$都有,
\begin{align*}
   & |a_n + b_n - (x+y)|                   \\
   & =|(a_n - x) + (b_n - y)|              \\
   & \leq |a_n - x| + |b_n - y| = /epsilon
\end{align*}
因此$(a_n+b_n)_{n=m}^\infty$是最终$\epsilon -$接近$x+y$的。
由于$\epsilon$是任意的,所以$(a_n+b_n)_{n=m}^\infty$收敛于$x+y$

\textbf{(b)}

设$\epsilon_0> 0$,
因为$(a_n)_{n=m}^\infty$收敛于$x$,即存在$N_a \geq m$使得$|a_n - x| \leq \epsilon _0$对所有的$n \geq N_a$均成立。

因为$(b_n)_{n=m}^\infty$收敛于$y$,即存在$N_b \geq m$使得$|b_n - y| \leq \epsilon _0$对所有的$n \geq N_b$均成立。

取$N := max(N_a, N_b)$,

由命题4.3.7(h) 对实数也成立,那么对任意$n \geq N$都有,
\begin{align*}
  d(a_nb_n,xy) \leq \epsilon_0|y| + \epsilon_0|x| + \epsilon_0\epsilon_0
\end{align*}
对任意的$\epsilon > 0$,只要$\epsilon_0$足够小,那么$d(a_nb_n,xy)\leq \epsilon$,
因此$(a_nb_n)_{n=m}^\infty$是最终$\epsilon -$接近$xy$的。
由于$\epsilon$是任意的,所以$(a_nb_n)_{n=m}^\infty$收敛于$xy$


\textbf{(c)}

(c)是(b)的特例。由实数的定义可知,存在一个有理数序列使得$LIM_{n\rightarrow\infty}c_n = c$,
由命题6.1.15可知 $c = \lim\limits_{n \rightarrow \infty}c_n$,即$(c_n)_{n=m}^\infty$收敛于$c$,
由命题(b)可知,
\begin{align*}
  \lim\limits_{n\rightarrow \infty}(ca_n) & = (\lim\limits_{n\rightarrow \infty}c) (\lim\limits_{n\rightarrow \infty}a_n) \\
                                          & = cx
\end{align*}

\textbf{(d)}

由(c)可知,$(-b_n)_{n=m}^\infty$是收敛于$-y$的。再利用(a)可证明该命题。


\textbf{(e)}

因为$(b_n)_{n=m}^\infty$收敛于$y$,
那么对任意$\delta > 0$,存在$N_b \geq m$使得$|b_n - y| \leq \delta$,对任意的$n \geq N_b$均成立。

我们必须要证明序列$(b_n^{-1})_{n=m}^\infty$收敛于$y^{-1}$。换言之,对于任意的$\epsilon > 0$,我们需要证明
序列$(b_n^{-1})_{n=m}^\infty$是最终$\epsilon -$接近于$y^{-1}$的。
于是设$\epsilon > 0$是任意的实数,
\begin{align*}
  |b_n^{-1} - y^{-1}| & = \left| \frac{1}{b_n} - \frac{1}{y} \right| \\
                      & = \left| \frac{y - b_n}{yb_n} \right|        \\
\end{align*}
此时,分子分母都是可变的,无法定量分析,需要固定分母的范围,这也是书中提示要证明辅助命题
\textbf{"如果一个序列的所有元素都不为零,并且该序列收敛于一个非零极限,那么这个序列是远离 0 的。"}原因。

\begin{zgraytheorem}
  辅助命题证明:

  设序列$(a_n)_{n=m}^\infty$收敛于$L \neq 0$,那么设$\epsilon = \frac{1}{2}|L| > 0$,存在$N \geq m, n \geq N$使得
  \begin{align*}
    |a_n - L|                      & \leq \epsilon                    \\
    |a_n - L|                      & \leq \frac{1}{2}|L|              \\
    \Rightarrow L - \frac{1}{2}|L| & \leq a_n \leq L + \frac{1}{2}|L|
  \end{align*}

  如果$L > 0$,那么,$0 < \frac{1}{2}|L| \leq a_n$,由于序列的所有元素都不为零,即:$a_n > 0$对$m \leq n < N$均成立,
  取$c := min(\frac{1}{2}|L|, (a_n)_{n=m}^{N-1})$【$min$生效的前提是$(a_n)_{n=m}^{N-1}$是有限集合】,
  综上可知$c > 0$,且任意$a_n \geq c$对任意$n \geq m$均成立,
  所以序列是远离$0$的。如果$L < 0$,同理可证。

  综上,命题得证。
\end{zgraytheorem}

由辅助命题可知,序列$(b_n)_{n=m}^\infty$是远离$0$的,那么存在一个实数$c > 0$使得$|b_n| \geq c$对所有的$n \geq m$均成立。
所以,
\begin{align*}
   & \left| \frac{y - b_n}{yb_n} \right| \\
   & \leq |y - b_n| / c|y|               \\
   & \leq \delta/c|y|                    \\
\end{align*}

因为$\delta>0$是任意实数,可以通过调整$\delta$的取值,使得,
\begin{align*}
   & \left| \frac{y - b_n}{yb_n} \right| \\
   & \leq \epsilon
\end{align*}
即$|b_n^{-1} - y^{-1}| \leq \epsilon$,所以序列是最终$\epsilon -$接近于$y^{-1}$的,
因为$\epsilon$是任意的,所以序列收敛于$y^{-1}$。

\textbf{(f)}

序列$(a_n/b_n)_{n=m}^\infty$可以看做$(a_n \times b_n^{-1})_{n=m}^\infty$,
由(e)可知$(b_n^{-1})_{n=m}^\infty$收敛于$y^{-1}$。
由(a)可知,
\begin{align*}
   & \lim\limits_{n \rightarrow \infty}(a_n \times b_n^{-1})                               \\
   & = (\lim\limits_{n \rightarrow \infty}a_n)(\lim\limits_{n \rightarrow \infty}b_n^{-1}) \\
   & = x \times y^{-1}                                                                     \\
   & = \frac{x}{y}                                                                         \\
   & = \frac{\lim\limits_{n \rightarrow \infty}a_n}{\lim\limits_{n \rightarrow \infty}b_n}
\end{align*}

\textbf{(g)}

我们必须证明$(max(a_n,b_n))_{n=m}^\infty$收敛于$max(x,y)$。换言之,对于任意的$\epsilon > 0$,
我们需要证明序列$(max(a_n,b_n))_{n=m}^\infty$是最终$\epsilon -$接近$max(x,y)$的。

任意实数$\delta > 0$,
因为$(a_n)_{n=m}^\infty$收敛于$x$,即存在$N_a \geq m$使得$|a_n - x| \leq \delta$对所有的$n \geq N_a$均成立。

因为$(b_n)_{n=m}^\infty$收敛于$y$,即存在$N_b \geq m$使得$|b_n - y| \leq \delta$对所有的$n \geq N_b$均成立。

取$N := max(N_a, N_b)$,于是对所有的$n \geq N$都有,
\begin{align*}
  |a_n - x|              & \leq \delta              \\
  \Rightarrow x - \delta & \leq a_n \leq x + \delta \\
  |b_n - y|              & \leq \delta              \\
  \Rightarrow y - \delta & \leq b_n \leq y + \delta
\end{align*}

如果$y > x$,我们可以取$0 < \delta < (y-x)/2$,此时$b_n > a_n$对任意$n \geq N$均成立。也就是说,
当$n \geq N$后$max(a_n,b_n) = b_n$,由习题6.1.3可知 序列$(max(a_n,b_n))_{n=m}^\infty$与$(b_n)_{n=m}^\infty$
收敛于同一个值$y$。

同理可证,$y \leq x$时,序列$(max(a_n,b_n))_{n=m}^\infty$收敛于$x$。

综上,命题得证。

\textbf{(h)}

证明与(g)类似。


\section*{6.1.9}

暂时还证明不了,要看完10.5节后才可以解答

\section*{6.1.10}


\end{document}