\documentclass{article}
\usepackage{mathtools} 
\usepackage{fontspec}
\usepackage[UTF8]{ctex}
\usepackage{amsthm}
\usepackage{mdframed}
\usepackage{xcolor}
\usepackage{amssymb}
\usepackage{amsmath}

\newmdtheoremenv[
  backgroundcolor=gray!10,
  linewidth=0pt,
  innerleftmargin=10pt,
  innerrightmargin=10pt,
  innertopmargin=10pt,
  innerbottommargin=10pt
]{zgraytheorem}{}
% 定义说明环境样式
\newtheoremstyle{mystyle}% 说明环境样式的名称
  {1em}% 上方间距
  {1em}% 下方间距
  {\normalfont}% 说明内容的字体样式
  {}% 缩进量
  {\bfseries}% 说明标记的字体样式
  {.}% 说明标记和说明内容之间的标点
  {1em}% 说明标记后的水平空间
  {}% 说明标记后的垂直空间
% 使用新定义的样式创建说明环境
\theoremstyle{mystyle}
\newtheorem*{zremark}{说明}


\begin{document}
\title{6.1 习题}
\maketitle

\section*{6.1.1}

证明框架如下:由于$n,m$都是自然数,且$m>n$,所以存在正自然数$k$使得$k = n + k$,对$k$进行归纳。

\section*{6.1.2}

证明:

与定义6.1.5 说的是一个意思,证明略

\section*{6.1.3}

证明:

充分性:

如果$(a_n)_{n=m}^\infty$收敛与$c$,那么对任意的$\epsilon > 0$,该序列都是最终$\epsilon -$接近$c$的,
所以存在$N \geq m$使得$|a_n - c| \leq \epsilon$对所有的$n \geq N$均成立。
由题设可知$m^\prime > m$,于是存在$N^\prime := max(m^\prime, N),N^\prime \geq m^\prime$,使得$|a_n - c| \leq \epsilon$对所有的$n \geq N^\prime$均成立,
由于$\epsilon$是任意的,由习题6.1.2可知,$(a_n)_{n=m^\prime}^\infty$收敛与$c$。

必要性:

$(a_n)_{n=m^\prime}^\infty$收敛与$c$,那么对任意的$\epsilon > 0$,该序列都是最终$\epsilon -$接近$c$的。
所以存在$N \geq m^\prime$使得$|a_n - c| \leq \epsilon$对所有的$n \geq N$均成立。由于$m^\prime > m$,
所以$N \geq m$,该性质对序列$(a_n)_{n=m}^\infty$也成立,由于$\epsilon$是任意的,
由习题6.1.2可知,$(a_n)_{n=m^\prime}^\infty$收敛与$c$。

\section*{6.1.4}

证明:

$(a_n)_{n=m}^\infty$收敛于$c$,于是对任意$\epsilon > 0$,
存在一个$N \geq m$使得$|a_n - c| \leq \epsilon$对所有的$n \geq N$
均成立,由于$k \geq 0$是一个非负整数,所以$n + k \geq N$,
于是$|a_{n+k} - c| \leq \epsilon$对所有的$n \geq N$
均成立,由习题6.1.2可知,$(a_{n+k})_{n=m}^\infty$收敛与$c$。

\section*{6.1.5}

证明:

要证明序列$(a_n)_{n=m}^\infty$是柯西序列,
则对于任意$\epsilon>0$,我们需要证明序列$(a_n)_{n=m}^\infty$是最终$\epsilon -$接近的。
于是设$\epsilon > 0$是一个任意的实数,那么$\epsilon/2 > 0$。因为$(a_n)_{n=m}^\infty$是收敛的
实数序列,不妨设收敛于实数$L$,可知$(a_n)_{n=m}^\infty$是最终$\epsilon -$接近与$L$的,
于是存在一个$N \geq m$使得$d(a_n,L)\leq \epsilon/2$对所有的$n \geq N$均成立。
任意$j,k \geq N$,有$d(a_j,L)\leq \epsilon/2$,$d(a_k,L)\leq \epsilon/2$,
于是根据三角不等式可得,$d(a_j,a_k) \leq \epsilon$,因此$(a_n)_{n=m}^\infty$是最终$\epsilon -$接近的。
由于$\epsilon$是任意选取的,因此$(a_n)_{n=m}^\infty$是柯西序列。

\section*{6.1.6}

证明:

证明为什么$a_n > L + \epsilon / 2$或$a_n < L - \epsilon / 2$,其余的按书中的提示证明就可以了。

序列$(a_n)_{n=m}^\infty$不是最终$\epsilon -$接近与$L$的,
即对任意的$N \geq m$都存在$|a_n - L| > \epsilon$对所有的$n \geq N$均成立。

序列$(a_n)_{n=m}^\infty$是柯西序列,所以存在$N_0$使得$|a_j - a_k| \leq \epsilon/2$对所有的$j,k \geq N_0$均成立。

固定$a_n=j_k$,所以,
\begin{align*}
  |a_j - a_n|                  & \leq \epsilon/2                      \\
  \Rightarrow a_n - \epsilon/2 & \leq a_j       \leq a_n + \epsilon/2
\end{align*}
又因为,$|a_n - L| > \epsilon$所以$a_n > \epsilon + L$或$a_n < L - \epsilon$。

如果$a_n > \epsilon + L$,那么,
\begin{align*}
  a_n - \epsilon/2          & \leq a_j \\
  L + \epsilon - \epsilon/2 & < a_j    \\
  L + \epsilon/2            & < a_j
\end{align*}

如果$a_n < L - \epsilon$,那么,

\begin{align*}
  a_j & \leq a_n + \epsilon/2       \\
  a_j & < L - \epsilon + \epsilon/2 \\
  a_j & < L - \epsilon/2
\end{align*}

\section*{6.1.7}

证明:

证明方法与命题6.1.4的类似。

首先假设$(a_n)_{n=m}^\infty$是定义5.1.12 意义下的有界序列,那么存在有理数$M$,该序列以$M$为界,
由于有理数$M$也是实数,所以$(a_n)_{n=m}^\infty$是定义6.1.16 意义下的有界序列。

现在假设是定义6.1.16 下的有界序列,那么存在实数$M$,该序列以$M$为界,根据命题5.4.12 可知,
存在一个比$M$大的有理数$M^\prime$,由于$M^\prime$是有理数,且$M < M^\prime$,所有该序列也以$M^\prime$为界,
所以$(a_n)_{n=m}^\infty$是定义5.1.12 意义下的有界序列。

\end{document}