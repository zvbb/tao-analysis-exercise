\documentclass{article}
\usepackage{mathtools} 
\usepackage{fontspec}
\usepackage[UTF8]{ctex}
\usepackage{amsthm}
\usepackage{mdframed}
\usepackage{xcolor}
\usepackage{amssymb}
\usepackage{amsmath}

\newmdtheoremenv[
  backgroundcolor=gray!10,
  linewidth=0pt,
  innerleftmargin=10pt,
  innerrightmargin=10pt,
  innertopmargin=10pt,
  innerbottommargin=10pt
]{zgraytheorem}{}
% 定义说明环境样式
\newtheoremstyle{mystyle}% 说明环境样式的名称
  {1em}% 上方间距
  {1em}% 下方间距
  {\normalfont}% 说明内容的字体样式
  {}% 缩进量
  {\bfseries}% 说明标记的字体样式
  {.}% 说明标记和说明内容之间的标点
  {1em}% 说明标记后的水平空间
  {}% 说明标记后的垂直空间
% 使用新定义的样式创建说明环境
\theoremstyle{mystyle}
\newtheorem*{zremark}{说明}


\begin{document}
\title{6.5 习题}
\maketitle

\section*{6.5.1}

有理数$q > 0$,可以表示成正整数$a/b$的形式,其中$a,b > 0$。

由定义5.6.7 可知$n^q = (n^{1/b})^a$,所以$1/n^q = (1/n^{1/b})^a$。

由推论6.5.1 可知$\lim\limits_{n \rightarrow \infty}1/n^{1/b} = 0$。
由极限定律(定理6.1.19)可知
\begin{align*}
   & \lim\limits_{n \rightarrow \infty}1/n^q           \\
   & =\lim\limits_{n \rightarrow \infty}(1/n^{1/b})^a  \\
   & = (\lim\limits_{n \rightarrow \infty}1/n^{1/b})^a \\
   & = 0^a                                             \\
   & = 0
\end{align*}

\section*{6.5.2}

(1)$|x|<1$时。

如果$0 < x < 1$,则由命题6.3.10 可知 极限$\lim\limits_{n \rightarrow \infty}x^n = 0$。

如果$x=0$,则对任意$n$都有$x^n = 0^n = 0$,于是极限$\lim\limits_{n \rightarrow \infty}x^n = 0$。

如果$-1 < x < 0$,则 $-1(-x)^n \leq x^n \leq (-x)^n$, 则由极限定律可知
\begin{equation*}
  \begin{cases*}
    \lim\limits_{n \rightarrow \infty}-1(-x)^n = 0 \\
    \lim\limits_{n \rightarrow \infty}(-x)^n = 0   \\
  \end{cases*}
\end{equation*}

由夹逼定理可知 $\lim\limits_{n \rightarrow \infty}(x)^n = 0$。

(2)$x = 1$时。

对任意$n$都有$x^n = 1^n = 1$,于是极限$\lim\limits_{n \rightarrow \infty}x^n = 1$。

(3.1)$x=-1$时。

当$n$是偶数时,$x^n = (-1)^n = 1$,于是存在极限点$1$。

当$n$是奇数时,$x^n = (-1)^n = -1$,于是存在极限点$-1$。

极限点不是唯一的,由命题6.4.5 可知,此时$\lim\limits_{n \rightarrow \infty}x^n$不存在。

(3.2)$|x|>1$时。

(3.2.1)当$x > 1$时。

由习题6.3.4 可知 $x > 1$时,序列$x^n$是发散的。

(3.3.2)当$x < -1$时。

反证法,假设极限$\lim\limits_{n \rightarrow \infty}x^n$存在。

不妨设极限$\lim\limits_{n \rightarrow \infty}x^n$等于$c$,那么,
对任意$\epsilon > 0$,存在$n \geq N$都有$|x^n - c| \leq \epsilon$均成立。
又因为
\begin{align*}
   & |(-x)^n - |c||  \\
   & = ||x^n| - |c|| \\
   & \leq |x^n - c|  \\
   & \leq \epsilon
\end{align*}
由此可知,$\lim\limits_{n \rightarrow \infty}(-x)^n$收敛于$|c|$,这与其是发散的这一事实矛盾。

\section*{6.5.3}


\end{document}