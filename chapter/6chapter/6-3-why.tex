\documentclass{article}
\usepackage{mathtools} 
\usepackage{fontspec}
\usepackage[UTF8]{ctex}
\usepackage{amsthm}
\usepackage{mdframed}
\usepackage{xcolor}
\usepackage{amssymb}
\usepackage{amsmath}

\newmdtheoremenv[
  backgroundcolor=gray!10,
  linewidth=0pt,
  innerleftmargin=10pt,
  innerrightmargin=10pt,
  innertopmargin=10pt,
  innerbottommargin=10pt
]{zgraytheorem}{}
% 定义说明环境样式
\newtheoremstyle{mystyle}% 说明环境样式的名称
  {1em}% 上方间距
  {1em}% 下方间距
  {\normalfont}% 说明内容的字体样式
  {}% 缩进量
  {\bfseries}% 说明标记的字体样式
  {.}% 说明标记和说明内容之间的标点
  {1em}% 说明标记后的水平空间
  {}% 说明标记后的垂直空间
% 使用新定义的样式创建说明环境
\theoremstyle{mystyle}
\newtheorem*{zremark}{说明}


\begin{document}
\title{6.3 为什么}
\maketitle

\textbf{1.$0<x<1$,那么序列$(x^n)_{n=1}^\infty$是单调递减的。}

需要证明$x^n \geq x^{n+1}$:
\begin{align*}
   & x^n - x^{n+1}      \\
   & = x^n(1-x)     < 0 \\
\end{align*}
所以$x^n \geq x^{n+1}$

\textbf{2.定义5.2.6中定义的等价序列,如果有极限,则极限是相同的。}

设序列$(a_n)_{n=0}^\infty$和序列$(b_n)_{n=0}^\infty$是等价序列,
并且$(a_n)_{n=0}^\infty$收敛于$x$,现在需要证明:$(b_n)_{n=0}^\infty$也收敛与$x$。

任意实数$\epsilon > 0$,$\epsilon/2 > 0$,所以存在$N \geq 0$对任意$n \geq N$有
\begin{align*}
  |a_n - x| & \leq \epsilon / 2 \\
  d(a_n, x) & \leq \epsilon / 2 \\
\end{align*}
又因为序列$(a_n)_{n=0}^\infty$和序列$(b_n)_{n=0}^\infty$是等价序列,所以是最终
$\epsilon/2 -$接近的,即存在$N^\prime \geq n$使得,
\begin{align*}
  |b_n - a_n| & \leq \epsilon/2 \\
  d(b_n, a_n) & \leq \epsilon/2 \\
\end{align*}

由命题4.3.3(g)【准确的说是实数版本,并把$y$看做$a_n$】所以,
\begin{align*}
  d(b_n - x) & \leq d(a_n, x) + d(b_n, a_n) \\
             & \leq \epsilon                \\
\end{align*}
所以序列$(b_n)_{n=0}^\infty$最终$\epsilon -$接近于$x$。
由$\epsilon$的任意性可知,序列$(b_n)_{n=0}^\infty$收敛于$x$。

\end{document}