\documentclass{article}
\usepackage{fontspec}
\usepackage[UTF8]{ctex}
\usepackage{amsthm}
\usepackage{mdframed}
\usepackage{xcolor}
\usepackage{amssymb}
\usepackage{amsmath}

\newmdtheoremenv[
  backgroundcolor=gray!10,
  linewidth=0pt,
  innerleftmargin=10pt,
  innerrightmargin=10pt,
  innertopmargin=10pt,
  innerbottommargin=10pt
]{zgraytheorem}{Theorem}

% 定义说明环境样式
\newtheoremstyle{mystyle}% 说明环境样式的名称
  {1em}% 上方间距
  {1em}% 下方间距
  {\normalfont}% 说明内容的字体样式
  {}% 缩进量
  {\bfseries}% 说明标记的字体样式
  {.}% 说明标记和说明内容之间的标点
  {1em}% 说明标记后的水平空间
  {}% 说明标记后的垂直空间
% 使用新定义的样式创建说明环境
\theoremstyle{mystyle}
\newtheorem*{zremark}{说明}

% 定义证明环境样式
\newtheoremstyle{zproofstyle}% 证明环境样式的名称
  {0.5em}% 上方间距
  {0.5em}% 下方间距
  {\itshape}% 证明内容的字体样式
  {}% 缩进量
  {\bfseries}% 证明标记的字体样式
  {.\newline}% 证明标记和证明内容之间的标点
  {1em}% 证明标记后的水平空间
  {}% 证明标记后的垂直空间

% 使用新定义的样式创建证明环境
\theoremstyle{zproofstyle}
\newtheorem*{zproof}{证明}

\begin{document}
\title{2.2习题}
\maketitle
\subsection*{2.2.3}

(a)(序是自反的)$a \geq a$

\begin{zproof}
    因为$a = a + 0$,由定义2.2.11可知$a \geq a$
\end{zproof}

(b)(序是可传递的)如果$a \geq b$并且$b \geq $,那么$a \geq c$。

\begin{zproof}
    如果$a \geq b$并且$b \geq c$,那么存在自然数m,n,使得$a = b + m, b = c + n$,
    由相等公理(替换公理)可知$a= c + n + m$,所以$a = c + (n+m)$,
    而两自然数相加仍然是自然数,所以$n+m$也是自然数,由定义2.2.11可知$a \geq c$,
    至此,命题得证
\end{zproof}

(c)(序是反对称的)如果$a \geq b$并且$b \geq a$,那么a=b。

\begin{zproof}
    $a \geq b$并且$b \geq a$,可知存在m,n使得$a=b+m, b=a+m$,
    替换公理替换掉b,则$a=b+n \Rightarrow a=a+m+n$,
    由加法是可结合的(命题2.2.5)可知$a=a+m+n=a+(m+n)$
    这里m+n必须是0,假设$m+n \neq 0$,所以$m+n$是正数。

    这里要证明以下命题f:自然数a与正数c相加大于a。
    对z做归纳。

    z=1时,$a=a+(m+n)=a+(0++)=(a+0)++=a++>a$。

    归纳假设z=k时,$a+k > a$。

    当z=k++,$a+(k++)=(a+k)++>a+k$,所以$a+(k++) \neq (a+k)$
    由$(a+k)>a$,可知$(a+k) \neq a$,所以$a+(k++) \neq a$,由定义2.2.11可知$a+(k++)>a$

    那么$a >  a+m+n$,这与$a=a+m+n $矛盾。

    至此,命题f得证

    由命题f可知$m+n$不能是正数,否则与$a=a+(m+n)$矛盾。
    由命题2.2.8可知$m=0,n=0$,又$a=b+n$,所以$a=b+0=b$。

    至此,命题得证
\end{zproof}

(d)(加法保持序不变)$a \geq b$,当且仅当$a+c \geq b+c$。

\begin{zproof}
    $a \geq b$,可知存在自然数n,使得$a=b+n$。
    $a+c=b+n+c=b+c+n$,所以$a+c \geq b+c$
\end{zproof}

(e)a<b,当且仅当$a++ \leq b$
\begin{zproof}
    $\Rightarrow$

    a<b,可知存在自然数m,使得b=a+m,且$a \neq b$。
    由此可是$m \neq 0$,因为如果m=0,那么$b=a+0, b=a$这与a<b矛盾。

    对m进行归纳。

    m=1时,$b=a+1=a++=a++$,所以$a++ \leq b$

    归纳假设m=k时,$b=a+k$,$a \leq b$,即:$a \leq (a+k)$

    m=k++,$b=a+(k++)=(a+k)++ \geq a+k \geq a$,由(b)可知$b \geq a$

    综上所述,充分性得到证明

    $\Leftarrow$
    
    $a++ \leq b$,可知存在m,$b=(a++)+m=a+(m++)$(用到了加法的交换律和加法的结合律),
    自然数a与正数c相加大于a(在2-2-why.tex中有证明),所以$b > a$

    综上所述,必要性得到证明

    至此,命题得证
\end{zproof}

\begin{zproof}

    (f)a<b,当且仅当存在自然数d使得$b=a+d$

    $\Rightarrow$

    a<b,可知存在自然数m使得$b=a+m$,如果m=0,那么$b=a+m=a$,这与a<b矛盾,
    所以m是正数。

    $\Leftarrow$

    存在自然数d使得$b=a+d$,由自然数与正数相加大于该自然数(在2-2-why.tex中有证明),
    所以$a<b$

    至此,命题得证

\end{zproof}
\end{document}