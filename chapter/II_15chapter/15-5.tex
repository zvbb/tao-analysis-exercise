\documentclass{article}
\usepackage{mathtools} 
\usepackage{fontspec}
\usepackage[UTF8]{ctex}
\usepackage{amsthm}
\usepackage{mdframed}
\usepackage{xcolor}
\usepackage{amssymb}
\usepackage{amsmath}


% 定义新的带灰色背景的说明环境 zremark
\newmdtheoremenv[
  backgroundcolor=gray!10,
  % 边框与背景一致,边框线会消失
  linecolor=gray!10
]{zremark}{说明}


\begin{document}
\title{15.5 习题}
\author{张志聪}
\maketitle

\section*{15.5.1}

\begin{itemize}
  \item (a)

        (1)绝对收敛。

        任意$x \in \mathbb{R}$,都有
        \begin{align*}
          \lim\sup\limits_{n \to \infty} \frac{|\frac{x^{n + 1}}{(n + 1)!}|}{|\frac{x^{n}}{(n)!}|}
           & = \lim\sup\limits_{n \to \infty} |\frac{x}{n + 1}| \\
           & = 0 < 1
        \end{align*}

        由推论7.5.3(比值判别法)可知,$\sum\limits_{n = 0}^\infty \frac{x^n}{n!}$绝对收敛。

        (2)

        由命题7.2.9(绝对收敛判别法)可知,绝对收敛的级数,也是条件收敛的。

        (3)收敛半径是$\infty$

        \begin{align*}
          \lim\sup\limits_{n \to \infty} |\frac{1}{n!}| \leq \lim\sup\limits_{n \to \infty} |\frac{1}{n}| = 0
        \end{align*}

        于是可得,收敛半径$R = \infty$。

        (4)$\exp$是$(-\infty, \infty)$上的实解析函数。

        由习题15.2.8(f)可知直接得到。

  \item (b)

        由定理15.1.6(d)可知,$\exp$在$\mathbb{R}$上可微。
        又因为
        \begin{align*}
          (\exp(x))^\prime & = \sum\limits_{n = 1}^\infty n \frac{x^{n - 1}}{n!}   \\
                           & = \sum\limits_{n = 1}^\infty \frac{x^{n - 1}}{(n-1)!} \\
        \end{align*}
        设$n^\prime = n - 1$,于是
        \begin{align*}
           & \sum\limits_{n = 1}^\infty \frac{x^{n - 1}}{(n-1)!}                \\
           & = \sum\limits_{n^\prime = 0}^\infty \frac{x^{n^\prime}}{n^\prime!} \\
           & = \exp(x)
        \end{align*}
  \item (c)

        由定理15.1.6(c)可知,$\exp$在$\mathbb{R}$上连续。
        又由(b)可知,$\exp(x)$是$\exp(x)$的原函数。

        于是利用定理11.9.4(微积分第二基本定理)可得
        \begin{align*}
          \int_{[a, b]} \exp(x) dx = \exp(b) - \exp(a)
        \end{align*}


  \item (d)
        \begin{align*}
          \exp(x + y) = \sum\limits_{n = 0}^\infty \frac{(x + y)^n}{n!}
        \end{align*}
        利用二项式公式(习题7.1.4)可知直接得到
        \begin{align*}
          \exp(x + y)
           & =\sum\limits_{n = 0}^\infty \frac{(x + y)^n}{n!}                                                  \\
           & = \sum\limits_{n = 0}^\infty \frac{1}{n!}\sum \limits_{j = 0}^n \frac{n!}{j!(n - j)!}x^jy^{n - j}
        \end{align*}
        约分掉$n!$可得
        \begin{align*}
          \exp(x + y)
           & = \sum\limits_{n = 0}^\infty \sum \limits_{j = 0}^n \frac{x^jy^{n - j}}{j!(n - j)!}
        \end{align*}

        由于$\exp$函数在$\mathbb{R}$上绝对收敛的,所以可以使用定理8.2.2(富比尼定理)
        \begin{align*}
          \exp(x + y)
           & = \sum\limits_{j = 0}^\infty \sum \limits_{n = j}^\infty \frac{x^jy^{n - j}}{j!(n - j)!}
        \end{align*}
        令$m = n - j$,于是可得,
        \begin{align*}
          \exp(x + y)
           & = \sum\limits_{j = 0}^\infty \sum \limits_{m = 0}^\infty \frac{x^jy^m}{j!m!}
        \end{align*}

        分离成两个级数:
        \begin{align*}
          \exp(x + y)
           & = \sum\limits_{j = 0}^\infty \frac{x^j}{j!} \sum \limits_{m = 0}^\infty \frac{y^m}{m!} \\
           & = \exp(x) \exp(y)
        \end{align*}
  \item (e)

        (1)$\exp(0) = 1$

        \begin{align*}
          \exp(0) = \sum\limits_{n = 0}^\infty \frac{0^n}{n!} = 1
        \end{align*}

        (2)$\exp(-x) = \frac{1}{\exp(x)}$。

        因为
        \begin{align*}
           & \exp(0) = 1 = \exp(x - x) = \exp(x)\exp(-x) \\
           & \implies                                    \\
           & \exp(-x) = \frac{1}{\exp(x)}
        \end{align*}

        (3)$\exp(x) > 0$。

        由于(2)易得,任意$x$,都有$\exp(x) \neq 0$。

        于是我们有,
        \begin{align*}
          \exp(x) & = \exp(\frac{1}{2}x + \frac{1}{2}x )   \\
                  & = \exp(\frac{1}{2}x)\exp(\frac{1}{2}x) \\
                  & = \exp(\frac{1}{2}x)^2                 \\
                  & > 0
        \end{align*}

  \item (f)

        由(b)(e)和命题10.3.3可以得到该结论。

\end{itemize}

\section*{15.5.2}

(1)先按照书中提示,先证明对于所有的$k=1,2,3,...,$都有$(n + k)! > 2^kn!$。

对$k$进行归纳。

$k = 1$时,$(n + k)! = (n + 1)! = (n + 1)n!$,
因为$n \geq 3$,所以$n + 1 > 2^k = 2^1 = 2$,
所以$(n + k)! > 2^kn!$成立。

归纳假设$k = j$时,$(n + j)! > 2^jn!$成立。

$k = j + 1$时,
\begin{align*}
  (n + j + 1)! & = (n + j)!(n + j + 1) \\
\end{align*}
由归纳假设和$n + j + 1 > 2$可得,
\begin{align*}
  (n + j)!(n + j + 1) & > 2^jn! \times 2 = 2^{j + 1}n!
\end{align*}
即
\begin{align*}
  (n + j + 1)! & > 2^{j + 1}n!
\end{align*}

归纳完成。

(2)证明$\frac{1}{(n + 1)!} + \frac{1}{(n + 2)!} + ... < \frac{1}{n!}$。

由(1)可知,
\begin{align*}
  \frac{1}{(n + 1)!} + \frac{1}{(n + 2)!} + ...
   & < \sum\limits_{k = 1}^\infty \frac{1}{2^kn!} = \frac{1}{n!}\sum\limits_{k = 1}^\infty (\frac{1}{2})^k
\end{align*}

由引理7.3.3可知,
\begin{align*}
  \sum\limits_{k = 0}^\infty \frac{1}{2^kn!} = \frac{1}{1-\frac{1}{2}} = 2
\end{align*}
于是可得
\begin{align*}
  \sum\limits_{k = 1}^\infty (\frac{1}{2})^k
   & = 2 - \sum\limits_{k = 0}^0 (\frac{1}{2})^k \\
   & = 2 - 1                                     \\
   & = 1
\end{align*}

综上可得
\begin{align*}
  \frac{1}{(n + 1)!} + \frac{1}{(n + 2)!} + ...
   & < \frac{1}{n!}
\end{align*}

(3)任意的$n \geq 3$,$n!e$都不是整数。

\begin{align*}
  n!\sum\limits_{m = 0}^\infty \frac{1}{m!}
   & = n!\sum\limits_{m = 0}^n \frac{1}{m!} + n!\sum\limits_{m = n + 1}^\infty \frac{1}{m!}
\end{align*}

因为$m \leq n$,都有$\frac{n!}{m!}$是正整数。
所以
\begin{align*}
  n!\sum\limits_{m = 0}^n \frac{1}{m!}
\end{align*}
是正整数。

又由(2)可知,
\begin{align*}
  0 < n!\sum\limits_{m = n + 1}^\infty \frac{1}{m!} < 1
\end{align*}

综上,命题成立。

(4)推导出$e$是无理数。

反证法,假设$e$不是无理数,又因为$e > 0$,所以存在正整数$a, b$,使得$e = \frac{a}{b}$。

于是,我们有,
\begin{align*}
   & \frac{a}{b} = e  \\
   & \implies         \\
   & a = b e          \\
   & \implies         \\
   & a(b - 1)! = b! e
\end{align*}
因为$a(b - 1)!$是正整数,所以$b! e$也是正整数,这与(3)矛盾。

注意:这里有个细节,就是$b \geq 3$不一定成立的困惑,这个无需考虑,只需做扩分操作即可。

\section*{15.5.3}

\begin{itemize}
  \item (a) $x$是自然数。
        对$x$进行归纳。

        (1)$x = 0$时,由定理15.5.2(e)可知,$\exp(0) = 1$,又$e^0 = 1$,所以命题成立。

        (2)归纳假设,$x = k$时,$\exp(k) = e^k$成立。

        (3)$x = k + 1$时,结合归纳假设和定理15.5.2(d),得
        \begin{align*}
          \exp(k + 1) = \exp(k)\exp(1) = e^k e = e^{k + 1}
        \end{align*}

        归纳完成。

  \item (b) $x$是整数。

        由(a)可知,我们只需讨论$x$是负整数的情况。

        设$-x < 0$,于是$x > 0$,
        \begin{align*}
          \exp(-x) & = \frac{1}{\exp(x)} \\
                   & = \frac{1}{e^x}     \\
                   & = e^{-x}
        \end{align*}
        综上,命题成立。

  \item (c) $x$是有理数。

        $x$可以表示成$\frac{a}{b}$,其中$a, b$都是整数,且$b > 0$。
        \begin{align*}
           & \exp(x)^b = \exp(\frac{a}{b}b) = \exp(a) = e^a \\
           & \implies                                       \\
           & \exp(x) = e^{\frac{a}{b}} = e^x
        \end{align*}

        综上,命题成立。

  \item (d) $x$是实数。

        对任意实数$x$,存在序列$(a_n)_{n = 0}^\infty$,使得$x = \lim\limits_{n \to \infty} a_n$。

        由定义6.7.2可知,$e^x = \lim\limits_{n \to \infty} e^{a_n}$。

        又因为$e^{a_n} = \exp(a_n)$,对所有的$n$均成立,又因为$\exp$是连续的,利用命题9.4.7可知,
        \begin{align*}
          \lim\limits_{n \to \infty} \exp(a_n) = \exp(x)
        \end{align*}

        综上可得,
        \begin{align*}
          e^x & = \lim\limits_{n \to \infty} e^{a_n}   \\
              & = \lim\limits_{n \to \infty} \exp(a_n) \\
              & = \exp(x)
        \end{align*}
\end{itemize}

\section*{15.5.4}

证明$f$在$0$处是可微的,
\begin{align*}
  \lim\limits_{x \to 0} \frac{f(x) - f(0)}{x} = \lim\limits_{x \to 0} \frac{\exp(\frac{-1}{x}) - 0}{x}
\end{align*}




\end{document}