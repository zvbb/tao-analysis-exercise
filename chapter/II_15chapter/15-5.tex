\documentclass{article}
\usepackage{mathtools} 
\usepackage{fontspec}
\usepackage[UTF8]{ctex}
\usepackage{amsthm}
\usepackage{mdframed}
\usepackage{xcolor}
\usepackage{amssymb}
\usepackage{amsmath}


% 定义新的带灰色背景的说明环境 zremark
\newmdtheoremenv[
  backgroundcolor=gray!10,
  % 边框与背景一致,边框线会消失
  linecolor=gray!10
]{zremark}{说明}


\begin{document}
\title{15.5 习题}
\author{张志聪}
\maketitle

\section*{15.5.1}

\begin{itemize}
  \item (a)

        (1)绝对收敛。

        任意$x \in \mathbb{R}$,都有
        \begin{align*}
          \lim\sup\limits_{n \to \infty} \frac{|\frac{x^{n + 1}}{(n + 1)!}|}{|\frac{x^{n}}{(n)!}|}
           & = \lim\sup\limits_{n \to \infty} |\frac{x}{n + 1}| \\
           & = 0 < 1
        \end{align*}

        由推论7.5.3(比值判别法)可知,$\sum\limits_{n = 0}^\infty \frac{x^n}{n!}$绝对收敛。

        (2)

        由命题7.2.9(绝对收敛判别法)可知,绝对收敛的级数,也是条件收敛的。

        (3)收敛半径是$\infty$

        \begin{align*}
          \lim\sup\limits_{n \to \infty} |\frac{1}{n!}| \leq \lim\sup\limits_{n \to \infty} |\frac{1}{n}| = 0
        \end{align*}

        于是可得,收敛半径$R = \infty$。

        (4)$\exp$是$(-\infty, \infty)$上的实解析函数。

        由习题15.2.8(f)可知直接得到。

  \item (b)

        由定理15.1.6(d)可知,$\exp$在$\mathbb{R}$上可微。
        又因为
        \begin{align*}
          (\exp(x))^\prime & = \sum\limits_{n = 1}^\infty n \frac{x^{n - 1}}{n!}   \\
                           & = \sum\limits_{n = 1}^\infty \frac{x^{n - 1}}{(n-1)!} \\
        \end{align*}
        设$n^\prime = n - 1$,于是
        \begin{align*}
           & \sum\limits_{n = 1}^\infty \frac{x^{n - 1}}{(n-1)!}                \\
           & = \sum\limits_{n^\prime = 0}^\infty \frac{x^{n^\prime}}{n^\prime!} \\
           & = \exp(x)
        \end{align*}
  \item (c)

        由定理15.1.6(c)可知,$\exp$在$\mathbb{R}$上连续。
        又由(b)可知,$\exp(x)$是$\exp(x)$的原函数。

        于是利用定理11.9.4(微积分第二基本定理)可得
        \begin{align*}
          \int_{[a, b]} \exp(x) dx = \exp(b) - \exp(a)
        \end{align*}


  \item (d)
        \begin{align*}
          \exp(x + y) = \sum\limits_{n = 0}^\infty \frac{(x + y)^n}{n!}
        \end{align*}
        利用二项式公式(习题7.1.4)可知直接得到
        \begin{align*}
          \exp(x + y)
           & =\sum\limits_{n = 0}^\infty \frac{(x + y)^n}{n!}                                                  \\
           & = \sum\limits_{n = 0}^\infty \frac{1}{n!}\sum \limits_{j = 0}^n \frac{n!}{j!(n - j)!}x^jy^{n - j}
        \end{align*}
        约分掉$n!$可得
        \begin{align*}
          \exp(x + y)
           & = \sum\limits_{n = 0}^\infty \sum \limits_{j = 0}^n \frac{x^jy^{n - j}}{j!(n - j)!}
        \end{align*}

        由于$\exp$函数在$\mathbb{R}$上绝对收敛的,所以可以使用定理8.2.2(富比尼定理)
        \begin{align*}
          \exp(x + y)
           & = \sum\limits_{j = 0}^\infty \sum \limits_{n = j}^\infty \frac{x^jy^{n - j}}{j!(n - j)!}
        \end{align*}
        令$m = n - j$,于是可得,
        \begin{align*}
          \exp(x + y)
           & = \sum\limits_{j = 0}^\infty \sum \limits_{m = 0}^\infty \frac{x^jy^m}{j!m!}
        \end{align*}

        分离成两个级数:
        \begin{align*}
          \exp(x + y)
           & = \sum\limits_{j = 0}^\infty \frac{x^j}{j!} \sum \limits_{m = 0}^\infty \frac{y^m}{m!} \\
           & = \exp(x) \exp(y)
        \end{align*}
  \item (e)

        (1)$\exp(0) = 1$

        \begin{align*}
          \exp(0) = \sum\limits_{n = 0}^\infty \frac{0^n}{n!} = 1
        \end{align*}

        (2)$\exp(-x) = \frac{1}{\exp(x)}$。

        因为
        \begin{align*}
           & \exp(0) = 1 = \exp(x - x) = \exp(x)\exp(-x) \\
           & \implies                                    \\
           & \exp(-x) = \frac{1}{\exp(x)}
        \end{align*}

        (3)$\exp(x) > 0$。

        由于(2)易得,任意$x$,都有$\exp(x) \neq 0$。

        于是我们有,
        \begin{align*}
          \exp(x) & = \exp(\frac{1}{2}x + \frac{1}{2}x )   \\
                  & = \exp(\frac{1}{2}x)\exp(\frac{1}{2}x) \\
                  & = \exp(\frac{1}{2}x)^2                 \\
                  & > 0
        \end{align*}

  \item (f)

        由(b)(e)和命题10.3.3可以得到该结论。

\end{itemize}

\end{document}