\documentclass{article}
\usepackage{mathtools} 
\usepackage{fontspec}
\usepackage[UTF8]{ctex}
\usepackage{amsthm}
\usepackage{mdframed}
\usepackage{xcolor}
\usepackage{amssymb}
\usepackage{amsmath}


% 定义新的带灰色背景的说明环境 zremark
\newmdtheoremenv[
  backgroundcolor=gray!10,
  % 边框与背景一致,边框线会消失
  linecolor=gray!10
]{zremark}{说明}


\begin{document}
\title{15.5 习题}
\author{张志聪}
\maketitle

\section*{15.5.1}

\begin{itemize}
  \item (a)

        (1)绝对收敛。

        任意$x \in \mathbb{R}$,都有
        \begin{align*}
          \lim\sup\limits_{n \to \infty} \frac{|\frac{x^{n + 1}}{(n + 1)!}|}{|\frac{x^{n}}{(n)!}|}
           & = \lim\sup\limits_{n \to \infty} |\frac{x}{n + 1}| \\
           & = 0 < 1
        \end{align*}

        由推论7.5.3(比值判别法)可知,$\sum\limits_{n = 0}^\infty \frac{x^n}{n!}$绝对收敛。

        (2)

        由命题7.2.9(绝对收敛判别法)可知,绝对收敛的级数,也是条件收敛的。

        (3)收敛半径是$\infty$

        \begin{align*}
          \lim\sup\limits_{n \to \infty} |\frac{1}{n!}| \leq \lim\sup\limits_{n \to \infty} |\frac{1}{n}| = 0
        \end{align*}

        于是可得,收敛半径$R = \infty$。

        (4)$\exp$是$(-\infty, \infty)$上的实解析函数。

        由习题15.2.8(f)可知直接得到。

  \item (b)

        由定理15.1.6(d)可知,$\exp$在$\mathbb{R}$上可微。
        又因为
        \begin{align*}
          (\exp(x))^\prime & = \sum\limits_{n = 1}^\infty n \frac{x^{n - 1}}{n!}   \\
                           & = \sum\limits_{n = 1}^\infty \frac{x^{n - 1}}{(n-1)!} \\
        \end{align*}
        设$n^\prime = n - 1$,于是
        \begin{align*}
           & \sum\limits_{n = 1}^\infty \frac{x^{n - 1}}{(n-1)!}                \\
           & = \sum\limits_{n^\prime = 0}^\infty \frac{x^{n^\prime}}{n^\prime!} \\
           & = \exp(x)
        \end{align*}
  \item (c)

        由定理15.1.6(c)可知,$\exp$在$\mathbb{R}$上连续。
        又由(b)可知,$\exp(x)$是$\exp(x)$的原函数。

        于是利用定理11.9.4(微积分第二基本定理)可得
        \begin{align*}
          \int_{[a, b]} \exp(x) dx = \exp(b) - \exp(a)
        \end{align*}


  \item (d)
        \begin{align*}
          \exp(x + y) = \sum\limits_{n = 0}^\infty \frac{(x + y)^n}{n!}
        \end{align*}
        利用二项式公式(习题7.1.4)可知直接得到
        \begin{align*}
          \exp(x + y)
           & =\sum\limits_{n = 0}^\infty \frac{(x + y)^n}{n!}                                                  \\
           & = \sum\limits_{n = 0}^\infty \frac{1}{n!}\sum \limits_{j = 0}^n \frac{n!}{j!(n - j)!}x^jy^{n - j}
        \end{align*}
        约分掉$n!$可得
        \begin{align*}
          \exp(x + y)
           & = \sum\limits_{n = 0}^\infty \sum \limits_{j = 0}^n \frac{x^jy^{n - j}}{j!(n - j)!}
        \end{align*}

        由于$\exp$函数在$\mathbb{R}$上绝对收敛的,所以可以使用定理8.2.2(富比尼定理)
        \begin{align*}
          \exp(x + y)
           & = \sum\limits_{j = 0}^\infty \sum \limits_{n = j}^\infty \frac{x^jy^{n - j}}{j!(n - j)!}
        \end{align*}
        令$m = n - j$,于是可得,
        \begin{align*}
          \exp(x + y)
           & = \sum\limits_{j = 0}^\infty \sum \limits_{m = 0}^\infty \frac{x^jy^m}{j!m!}
        \end{align*}

        分离成两个级数:
        \begin{align*}
          \exp(x + y)
           & = \sum\limits_{j = 0}^\infty \frac{x^j}{j!} \sum \limits_{m = 0}^\infty \frac{y^m}{m!} \\
           & = \exp(x) \exp(y)
        \end{align*}
  \item (e)

        (1)$\exp(0) = 1$

        \begin{align*}
          \exp(0) = \sum\limits_{n = 0}^\infty \frac{0^n}{n!} = 1
        \end{align*}

        (2)$\exp(-x) = \frac{1}{\exp(x)}$。

        因为
        \begin{align*}
           & \exp(0) = 1 = \exp(x - x) = \exp(x)\exp(-x) \\
           & \implies                                    \\
           & \exp(-x) = \frac{1}{\exp(x)}
        \end{align*}

        (3)$\exp(x) > 0$。

        由于(2)易得,任意$x$,都有$\exp(x) \neq 0$。

        于是我们有,
        \begin{align*}
          \exp(x) & = \exp(\frac{1}{2}x + \frac{1}{2}x )   \\
                  & = \exp(\frac{1}{2}x)\exp(\frac{1}{2}x) \\
                  & = \exp(\frac{1}{2}x)^2                 \\
                  & > 0
        \end{align*}

  \item (f)

        由(b)(e)和命题10.3.3可以得到该结论。

\end{itemize}

\section*{15.5.2}

(1)先按照书中提示,先证明对于所有的$k=1,2,3,...,$都有$(n + k)! > 2^kn!$。

对$k$进行归纳。

$k = 1$时,$(n + k)! = (n + 1)! = (n + 1)n!$,
因为$n \geq 3$,所以$n + 1 > 2^k = 2^1 = 2$,
所以$(n + k)! > 2^kn!$成立。

归纳假设$k = j$时,$(n + j)! > 2^jn!$成立。

$k = j + 1$时,
\begin{align*}
  (n + j + 1)! & = (n + j)!(n + j + 1) \\
\end{align*}
由归纳假设和$n + j + 1 > 2$可得,
\begin{align*}
  (n + j)!(n + j + 1) & > 2^jn! \times 2 = 2^{j + 1}n!
\end{align*}
即
\begin{align*}
  (n + j + 1)! & > 2^{j + 1}n!
\end{align*}

归纳完成。

(2)证明$\frac{1}{(n + 1)!} + \frac{1}{(n + 2)!} + ... < \frac{1}{n!}$。

由(1)可知,
\begin{align*}
  \frac{1}{(n + 1)!} + \frac{1}{(n + 2)!} + ...
   & < \sum\limits_{k = 1}^\infty \frac{1}{2^kn!} = \frac{1}{n!}\sum\limits_{k = 1}^\infty (\frac{1}{2})^k
\end{align*}

由引理7.3.3可知,
\begin{align*}
  \sum\limits_{k = 0}^\infty \frac{1}{2^kn!} = \frac{1}{1-\frac{1}{2}} = 2
\end{align*}
于是可得
\begin{align*}
  \sum\limits_{k = 1}^\infty (\frac{1}{2})^k
   & = 2 - \sum\limits_{k = 0}^0 (\frac{1}{2})^k \\
   & = 2 - 1                                     \\
   & = 1
\end{align*}

综上可得
\begin{align*}
  \frac{1}{(n + 1)!} + \frac{1}{(n + 2)!} + ...
   & < \frac{1}{n!}
\end{align*}

(3)任意的$n \geq 3$,$n!e$都不是整数。

\begin{align*}
  n!\sum\limits_{m = 0}^\infty \frac{1}{m!}
   & = n!\sum\limits_{m = 0}^n \frac{1}{m!} + n!\sum\limits_{m = n + 1}^\infty \frac{1}{m!}
\end{align*}

因为$m \leq n$,都有$\frac{n!}{m!}$是正整数。
所以
\begin{align*}
  n!\sum\limits_{m = 0}^n \frac{1}{m!}
\end{align*}
是正整数。

又由(2)可知,
\begin{align*}
  0 < n!\sum\limits_{m = n + 1}^\infty \frac{1}{m!} < 1
\end{align*}

综上,命题成立。

(4)推导出$e$是无理数。

反证法,假设$e$不是无理数,又因为$e > 0$,所以存在正整数$a, b$,使得$e = \frac{a}{b}$。

于是,我们有,
\begin{align*}
   & \frac{a}{b} = e  \\
   & \implies         \\
   & a = b e          \\
   & \implies         \\
   & a(b - 1)! = b! e
\end{align*}
因为$a(b - 1)!$是正整数,所以$b! e$也是正整数,这与(3)矛盾。

注意:这里有个细节,就是$b \geq 3$不一定成立的困惑,这个无需考虑,只需做扩分操作即可。

\section*{15.5.3}

\begin{itemize}
  \item (a) $x$是自然数。
        对$x$进行归纳。

        (1)$x = 0$时,由定理15.5.2(e)可知,$\exp(0) = 1$,又$e^0 = 1$,所以命题成立。

        (2)归纳假设,$x = k$时,$\exp(k) = e^k$成立。

        (3)$x = k + 1$时,结合归纳假设和定理15.5.2(d),得
        \begin{align*}
          \exp(k + 1) = \exp(k)\exp(1) = e^k e = e^{k + 1}
        \end{align*}

        归纳完成。

  \item (b) $x$是整数。

        由(a)可知,我们只需讨论$x$是负整数的情况。

        设$-x < 0$,于是$x > 0$,
        \begin{align*}
          \exp(-x) & = \frac{1}{\exp(x)} \\
                   & = \frac{1}{e^x}     \\
                   & = e^{-x}
        \end{align*}
        综上,命题成立。

  \item (c) $x$是有理数。

        $x$可以表示成$\frac{a}{b}$,其中$a, b$都是整数,且$b > 0$。
        \begin{align*}
           & \exp(x)^b = \exp(\frac{a}{b}b) = \exp(a) = e^a \\
           & \implies                                       \\
           & \exp(x) = e^{\frac{a}{b}} = e^x
        \end{align*}

        综上,命题成立。

  \item (d) $x$是实数。

        对任意实数$x$,存在序列$(a_n)_{n = 0}^\infty$,使得$x = \lim\limits_{n \to \infty} a_n$。

        由定义6.7.2可知,$e^x = \lim\limits_{n \to \infty} e^{a_n}$。

        又因为$e^{a_n} = \exp(a_n)$,对所有的$n$均成立,又因为$\exp$是连续的,利用命题9.4.7可知,
        \begin{align*}
          \lim\limits_{n \to \infty} \exp(a_n) = \exp(x)
        \end{align*}

        综上可得,
        \begin{align*}
          e^x & = \lim\limits_{n \to \infty} e^{a_n}   \\
              & = \lim\limits_{n \to \infty} \exp(a_n) \\
              & = \exp(x)
        \end{align*}
\end{itemize}

\section*{15.5.4}

\begin{itemize}
  \item (a)$f$是无限可微的。

        \begin{itemize}
          \item $x < 0$

                $f^{(k)}(x) = 0^{(k)} = 0$。
          \item $x > 0$

                导数通项式:
                $f^{(k)}(x) = \frac{P_k(x)\exp(\frac{-1}{x})}{x^{2k}}$
                ,其中$P_k(x)$是一个关于$x$的多项式。

                可以通过归纳法证明,证明略。

          \item $x = 0$

                对$k$进行归纳。

                (1)$k = 1$时,
                左极限为$0$,我们主要讨论右极限。
                右极限:
                \begin{align*}
                  \lim\limits_{x \to 0; x \in (0, + \infty)} \frac{f(x) - f(0)}{x - 0}
                   & = \lim\limits_{x \to 0; x \in (0, + \infty)} \frac{f(x)}{x}                        \\
                   & = \lim\limits_{x \to 0; x \in (0, + \infty)} \frac{\exp(\frac{-1}{x})}{x}          \\
                   & = \lim\limits_{x \to 0; x \in (0, + \infty)} \frac{\frac{1}{\exp(\frac{1}{x})}}{x} \\
                   & = \lim\limits_{x \to 0; x \in (0, + \infty)} \frac{1}{\exp(\frac{1}{x})x}          \\
                   & = \lim\limits_{x \to 0; x \in (0, + \infty)} \frac{\frac{1}{x}}{\exp(\frac{1}{x})} \\
                \end{align*}

                定义函数$\phi: \mathbb{R} - \{0\} \to \mathbb{R}$为$\phi(x) = \frac{1}{x}$。

                于是我们有
                \begin{align*}
                  \lim\limits_{x \to 0; x \in (0, + \infty)} \phi(x) = +\infty
                \end{align*}

                于是可得,
                \begin{align*}
                   & \lim\limits_{x \to 0; x \in (0, + \infty)} \frac{\frac{1}{x}}{\exp(\frac{1}{x})}        \\
                   & =  \lim\limits_{\phi(x) \to +\infty; x \in (0, + \infty)} \frac{\phi(x)}{\exp(\phi(x))} \\
                   & = 0
                \end{align*}

                (注意:第一个等式使用了一个小命题,在10-5-comment.tex的"3定理"中有说明。第二个等式使用了习题15.5.8)。

                (2)归纳假设$k \leq n$时,命题成立。

                (3)$k = n + 1$时,

                \begin{align*}
                   & \lim\limits_{x \to 0; x \in (0, + \infty)} \frac{f^{(n)}(x) - f^{(n)}(0)}{x - 0}                 \\
                   & = \lim\limits_{x \to 0; x \in (0, + \infty)} \frac{f^{(n)}(x) - 0}{x - 0}                        \\
                   & = \lim\limits_{x \to 0; x \in (0, + \infty)} \frac{f^{(n)}(x)}{x}                                \\
                   & = \lim\limits_{x \to 0; x \in (0, + \infty)} \frac{\frac{P_n(x)\exp(\frac{-1}{x})}{x^{2n}}}{x}   \\
                   & = \lim\limits_{x \to 0; x \in (0, + \infty)} \frac{P_n(x)\exp(\frac{-1}{x})}{x^{2n+1}}           \\
                   & = \lim\limits_{x \to 0; x \in (0, + \infty)} P_n(x) \frac{\frac{1}{x}^{2n+1}}{\exp(\frac{1}{x})}
                \end{align*}

                因为$P_n(x)$是连续的,所以$\lim\limits_{x \to 0; x \in (0, + \infty)} P_n(x) = P_n(0)$,
                再次利用习题15.5.8可知,$\lim\limits_{x \to 0; x \in (0, + \infty)} \frac{\frac{1}{x}^{2n+1}}{\exp(\frac{1}{x})} = 0$。

                综上,
                \begin{align*}
                   & \lim\limits_{x \to 0; x \in (0, + \infty)} P_n(x) \frac{\frac{1}{x}^{2n+1}}{\exp(\frac{1}{x})}                                              \\
                   & = \lim\limits_{x \to 0; x \in (0, + \infty)} P_n(x) \lim\limits_{x \to 0; x \in (0, + \infty)} \frac{\frac{1}{x}^{2n+1}}{\exp(\frac{1}{x})} \\
                   & = P_n(0) \times 0                                                                                                                           \\
                   & = 0
                \end{align*}
                即$f^{(n + 1)}(x) = 0$。

                归纳完成。


          \item (b)$f$在$0$处不是实解析的。


                反证法,假设$f$在$0$处是实解析的。即存在收敛半径大于或等于$r> 0$的幂级数
                $\sum\limits_{n = 0}^\infty c_n(x - 0)^n = \sum\limits_{n = 0}^\infty c_nx^n$在
                $(-r, r)$上收敛于$f$。

                利用泰勒公式(推论15.2.10),我们有泰勒公式
                \begin{align*}
                  f(x) = \sum\limits_{n = 0}^\infty \frac{f^{(n)}(0)}{n!}(x - 0)^n, x \in (-r, r)
                \end{align*}

                因为任意$n \geq 0$,$f^{(n)}(0) = 0$,所以
                \begin{align*}
                  f(x) = \sum\limits_{n = 0}^\infty 0 = 0, x \in (-r, r)
                \end{align*}

                而任意$x \in (-r, r)$,利用定理15.5.2(e)可得
                \begin{align*}
                  f(x) = \exp(-1/x) > 0
                \end{align*}

                存在矛盾。


        \end{itemize}
\end{itemize}

\section*{15.5.5}

\begin{itemize}
  \item (a)

        对任意$x \in (0, + \infty)$,存在$y \in \mathbb{R}$使得$\exp(y) = x$,
        利用定理10.4.2可知,
        \begin{align*}
          \ln^\prime(x) & = \frac{1}{\exp^\prime(y)} \\
                        & = \frac{1}{x}
        \end{align*}

        因为$\ln(x)$是$\frac{1}{x}$的原函数,于是利用微积分基本定理可得,
        \begin{align*}
          \int_{[a, b]}\frac{1}{x} dx = \ln(b) - \ln(a)
        \end{align*}

  \item (b)

        因为$\exp(\ln (x)) = x, \exp(\ln (y)) = y$,于是
        \begin{align*}
          \ln(xy) & = \ln(\exp(\ln (x))\exp(\ln (y))) \\
                  & = \ln(\exp(\ln (x) + \ln (y)))    \\
                  & = \ln(x) + \ln(y)
        \end{align*}

  \item (c)

        (1)因为$\exp(0) = 1$,于是
        \begin{align*}
          \ln(1) = \ln(\exp(0)) = 0
        \end{align*}

        (2)设$\exp(y) = x$,于是
        \begin{align*}
          -\ln(x) = \ln(\exp(y)) = -y
        \end{align*}

        又因为
        \begin{align*}
          -y = \ln(\exp(-y)) & = \ln(\frac{1}{\exp(y)}) \\
                             & = \ln(\frac{1}{x})
        \end{align*}

  \item (d)

        设$\exp(z) = x$,于是
        \begin{align*}
          y\ln(x) = yz
        \end{align*}

        又因为
        \begin{align*}
          \ln(x^y) & = \ln(\exp(z)^y) \\
                   & = \ln(\exp(yz))  \\
                   & = yz
        \end{align*}

  \item (e)

        (1)

        利用引理7.3.3可得,$\sum\limits_{n = 0}^\infty x^{n}$在$(-1, 1)$上逐点收敛于
        $\frac{1}{1 - x}$。

        由定理15.1.6(c)可知,$\sum\limits_{n = 0}^\infty x^{n}$一致收敛于某个函数$f$。
        由习题14.2.2(a),和逐点收敛函数的唯一性可知,
        $\sum\limits_{n = 0}^\infty x^{n}$一致收敛于$\frac{1}{1 - x}$。

        于是由定理15.1.6(e)可知,
        \begin{align*}
          \int_{[0, x]} \frac{1}{1 - y} dy
           & = \sum\limits_{n = 0}^\infty \frac{x^{n + 1} - (0 - 0)^{n + 1}}{n + 1} \\
           & = \sum\limits_{n = 1}^\infty \frac{x^{n}}{n}
        \end{align*}

        又因为
        \begin{align*}
          \int_{[0, x]} \frac{1}{1 - y} dy
           & = -\ln(1 - y) |_0^x \\
           & = -\ln(1 - x)
        \end{align*}

        综上可得,
        \begin{align*}
          \ln(1 - x) = - \sum\limits_{n = 1}^\infty \frac{x^{n}}{n}
        \end{align*}

        (2)

        令$y = 1 - x, x \in (-1, 1)$,于是$y \in (0, 2)$。

        于是利用(1),我们有
        \begin{align*}
          \ln(y)
           & = \ln(1 - x)                                                   \\
           & = - \sum\limits_{n = 1}^\infty \frac{x^{n}}{n}                 \\
           & = - \sum\limits_{n = 1}^\infty \frac{(1 - y)^{n}}{n}           \\
           & = - \sum\limits_{n = 1}^\infty \frac{(-1)^n(y - 1)^{n}}{n}     \\
           & = \sum\limits_{n = 1}^\infty \frac{(-1)^{n + 1}(y - 1)^{n}}{n}
        \end{align*}
\end{itemize}

\section*{15.5.6}

任意实数$x_0 \in (0, + \infty)$都存在自然数$N$使得$N - 1 < x_0 < N + 1$
(可以利用命题4.4.1和命题5.4.12,完成证明)。

接下来证明,$\ln$在$x_0$处是解析的,并且存在幂级数展开式。

证明思路:与习题15.5.5(e)(2)类似,都是通过变换,满足已有定理的定义域。

令$y = \frac{x}{x_0}, x \in (0, 2x_0)$,于是$y \in (0, 2)$。

利用定理15.5.6(e)可得
\begin{align*}
  \ln(y) & = \ln(\frac{x}{x_0})                                                       \\
         & = \sum \limits_{n = 1}^\infty \frac{(-1)^{n + 1}}{n} (\frac{x}{x_0} - 1)^n \\
         & = \sum \limits_{n = 1}^\infty \frac{(-1)^{n + 1}}{nx_0^n} (x - x_0)^n
\end{align*}

又因为,
\begin{align*}
  \ln(\frac{x}{x_0}) = \ln(xx_0^{-1}) = \ln(x) - \ln(x_0)
\end{align*}

综上可得,
\begin{align*}
  \ln(x) & = \ln(x_0) + \sum \limits_{n = 1}^\infty \frac{(-1)^{n + 1}}{nx_0^n} (x - x_0)^n            \\
         & = \ln(x_0)(x - x_0)^0 + \sum \limits_{n = 1}^\infty \frac{(-1)^{n + 1}}{nx_0^n} (x - x_0)^n
\end{align*}

命题成立。

\section*{15.5.7}
\begin{align*}
  (f(x)e^{-x})^\prime
   & = f^\prime(x) e^{-x} + f(x)e^{-x} (-1) \\
   & = e^{-x} (f^\prime(x) - f(x) )         \\
   & = 0
\end{align*}

由此可得$f(x)e^{-x}$是常数,设为$C$。又因为$f(x)$,$e^{-x}$都是正实数,所以$C > 0$。

于是
\begin{align*}
  f(x)e^{-x} & = C    \\
  f(x)       & = Ce^x
\end{align*}

\section*{15.5.8}

直接使用洛必达定理的推广,本书中没有讲到这个推广。我在10-5-comment.tex中的“2.4”中添加了这个推广。

因为$x \to +\infty$,$e^x$及$x^m$都趋于无穷大,
所以,利用洛必达定理可得,
\begin{align*}
  \lim\limits_{x \to +\infty} \frac{e^x}{x^m}
  =
  \lim\limits_{x \to +\infty} \frac{e^x}{mx^{m - 1}}
\end{align*}

多次使用洛必达定理,可以得到
\begin{align*}
  \lim\limits_{x \to +\infty} \frac{e^x}{x^m}
  =
  \lim\limits_{x \to +\infty} \frac{e^x}{mx^{m - 1}}
  =
  ...
  =
  \lim\limits_{x \to +\infty} \frac{e^x}{m!}
  = +\infty
\end{align*}

\section*{15.5.9}


$P(x)$是多项式,所以我们可以找到一个正整数$k \geq 0$和实数$a_0, a_1, \dots, a_k$使得
\begin{align*}
  P(x) & = a_0 + a_1x + a_2x^2 + \dots + a_kx^k
\end{align*}

对$k$进行归纳。

(1)$k = 0$时,$P(x)$是常数或$P(x) = a_0$,
于是$|P(x)| = |a_0|$是一个常数,
由$x \to +\infty$,$e^{cx}$趋于无穷大可得,命题成立。

(2)归纳假设,$k = n$时,命题成立。

(3)$k = n + 1$时,$P(x) = a_0 + a_1x + a_2x^2 + \dots + a_nx^n + a_{n + 1}x^{n + 1}$。

\section*{15.5.10}

(1)利用$\exp, \ln$的连续性证明。

可以仿照推论13.2.3的证明。

我们有,
\begin{align*}
  \exp(y\ln(x)) = e^{y\ln(x)} = (e^{\ln(x)})^y = x^y
\end{align*}

先证明$y\ln(x)$是连续的。

对任意$(x_0, y_0) \in (0, + \infty) \times \mathbb{R}$,
设$(x^{(n)})_{n = 1}^\infty$是$(0, + \infty) \times \mathbb{R}$中收敛于
$(x_0, y_0)$的点,其中$x^{(n)} = (x_n, y_n)$,利用命题12.1.18(d)可知,
$(x_n)_{n = 1}^\infty$收敛于$x_0$,$(y_n)_{n = 1}^\infty$收敛于$y_0$。

于是利用定理6.1.19(极限定律)我们有
\begin{align*}
  \lim\limits_{n \to \infty} y_n \ln(x_n)
   & = \lim\limits_{n \to \infty} y_n \lim\limits_{n \to \infty} \ln(x_n) \\
   & = y_0 \ln(x_0)
\end{align*}

所以,$y\ln(x)$是连续的。

由于$\exp$函数是连续的,利用推论13.1.7可知,
$\exp(y\ln(x))$是连续的。

(2)挑战一下

todo 还没思路。

\end{document}