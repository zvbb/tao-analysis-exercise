\documentclass{article}
\usepackage{mathtools} 
\usepackage{fontspec}
\usepackage[UTF8]{ctex}
\usepackage{amsthm}
\usepackage{mdframed}
\usepackage{xcolor}
\usepackage{amssymb}
\usepackage{amsmath}


% 定义新的带灰色背景的说明环境 zremark
\newmdtheoremenv[
  backgroundcolor=gray!10,
  % 边框与背景一致,边框线会消失
  linecolor=gray!10
]{zremark}{说明}


\begin{document}
\title{15.7 习题}
\author{张志聪}
\maketitle

\section*{15.7.1}

\begin{itemize}
  \item (a)

        利用引理15.6.6和习题15.6.16中的$\exp(z + w) = \exp(z)\exp(w)$。
        \begin{align*}
          \sin(x)^2 + \cos(x)^2
           & = \left(\frac{e^{ix} - e^{-ix}}{2i}\right)^2 + \left(\frac{e^{ix} + e^{-ix}}{2}\right)^2       \\
           & = \frac{e^{2ix} + e^{-2ix} - 2 e^{ix - ix}}{-4} + \frac{e^{2ix} + e^{-2ix} + 2 e^{ix - ix}}{4} \\
           & = \frac{e^{2ix} + e^{-2ix}}{-4} + \frac{-2}{-4} + \frac{e^{2ix} + e^{-2ix}}{4} + \frac{2}{4}   \\
           & = 1
        \end{align*}
  \item (b)

        \begin{align*}
          sin^\prime(x) & = \left(\sum\limits_{n = 0}^\infty \frac{(-1)^nx^{2n + 1}}{(2n + 1)!} \right)^\prime \\
                        & = \sum\limits_{n = 0}^\infty \frac{(-1)^n(2n + 1)x^{2n}}{(2n + 1)!}                  \\
                        & = \sum\limits_{n = 0}^\infty \frac{(-1)^nx^{2n}}{(2n)!}                              \\
                        & = cos(x)
        \end{align*}

        \begin{align*}
          cos^\prime(x) & = \left(\sum\limits_{n = 0}^\infty \frac{(-1)^nx^{2n}}{(2n)!}\right)^\prime \\
                        & = \sum\limits_{n = 1}^\infty \frac{2n(-1)^nx^{2n-1}}{(2n)!}                 \\
                        & = \sum\limits_{n = 1}^\infty \frac{(-1)^nx^{2n-1}}{(2n-1)!}                 \\
        \end{align*}
        令$m = n - 1$,即$n = m + 1$,利用命题7.4.3(级数的重排序),
        \begin{align*}
           & \sum\limits_{n = 1}^\infty \frac{(-1)^nx^{2n-1}}{(2n-1)!}                     \\
           & = \sum\limits_{m = 0}^\infty \frac{(-1)^{m + 1}x^{2(m + 1)-1}}{(2(m + 1)-1)!} \\
           & = \sum\limits_{m = 0}^\infty \frac{(-1)^{m + 1}x^{2m + 1}}{(2m + 1)!}         \\
           & = - \sum\limits_{m = 0}^\infty \frac{(-1)^{m}x^{2m + 1}}{(2m + 1)!}           \\
           & = -sin(x)
        \end{align*}

  \item (c)

        \begin{align*}
          sin(-x) & = \frac{e^{-ix} - e^{ix}}{2i}   \\
                  & = - \frac{e^{ix} - e^{-ix}}{2i} \\
                  & = -sin(x)
        \end{align*}

        \begin{align*}
          cos(-x) & = \frac{e^{i(-x)} + e^{-i(-x)}}{2} \\
                  & = \frac{e^{-ix} + e^{ix}}{2}       \\
                  & = cos(x)
        \end{align*}

  \item (d)

        \begin{align*}
          cos(x)cos(y) - sin(x)sin(y) & = \frac{e^{ix} + e^{-ix}}{2} \frac{e^{iy} + e^{-iy}}{2}                    \\
                                      & - \frac{e^{ix} - e^{-ix}}{2i} \frac{e^{iy} - e^{-iy}}{2i}                  \\
                                      & = \frac{e^{ix}e^{iy} + e^{ix}e^{-iy} + e^{-ix}e^{iy} + e^{-ix}e^{-iy}}{4}  \\
                                      & - \frac{e^{ix}e^{iy} - e^{ix}e^{-iy} - e^{-ix}e^{iy} + e^{-ix}e^{-iy}}{-4} \\
                                      & = \frac{2e^{ix}e^{iy} + 2e^{-ix}e^{-iy}}{4}                                \\
                                      & = \frac{e^{ix}e^{iy} + e^{-ix}e^{-iy}}{2}                                  \\
                                      & = \frac{e^{i(x + y)} + e^{-i(x + y)}}{2}                                   \\
                                      & = cos(x + y)
        \end{align*}

        \begin{align*}
          sin(x)cos(y) + cos(x)sin(y) & = \frac{e^{ix} - e^{-ix}}{2i} \frac{e^{iy} + e^{-iy}}{2}                    \\
                                      & + \frac{e^{ix} + e^{-ix}}{2}\frac{e^{iy} - e^{-iy}}{2i}                     \\
                                      & = \frac{e^{ix}e^{iy} + e^{ix}e^{-iy} - e^{-ix}e^{iy} - e^{-ix}e^{-iy} }{4i} \\
                                      & + \frac{e^{ix}e^{iy} - e^{ix}e^{-iy} + e^{-ix}e^{iy} - e^{-ix}e^{-iy}}{4i}  \\
                                      & = \frac{2e^{ix}e^{iy} - 2e^{-ix}e^{-iy}}{4i}                                \\
                                      & = \frac{e^{i(x + y)} - e^{-i(x + y)}}{2i}                                   \\
                                      & = sin(x + y)
        \end{align*}


  \item (e)

        \begin{align*}
          cos(0) & = \frac{e^{i \times 0} + e^{-i \times 0}}{2i} \\
                 & = \frac{1 + 1}{2}                             \\
                 & = 1
        \end{align*}

        由(a)可知,$sin(0) = 1 - cos(0)^2 = 1 - 1 = 0$


  \item (f)

        \begin{align*}
          cos(x) + isin(x) & = \frac{e^{ix} + e^{-ix}}{2} + i\frac{e^{ix} - e^{-ix}}{2i}     \\
                           & = \frac{e^{ix} + e^{-ix}}{2} + i^2\frac{e^{ix} - e^{-ix}}{2i^2} \\
                           & = \frac{e^{ix} + e^{-ix}}{2} + \frac{e^{ix} - e^{-ix}}{2}       \\
                           & = \frac{2e^{ix}}{2}                                             \\
                           & = e^{ix}
        \end{align*}

        同理可得,
        \begin{align*}
          cos(x) - isin(x) & = \frac{e^{ix} + e^{-ix}}{2} - i\frac{e^{ix} - e^{-ix}}{2i}     \\
                           & = \frac{e^{ix} + e^{-ix}}{2} - i^2\frac{e^{ix} - e^{-ix}}{2i^2} \\
                           & = \frac{e^{ix} + e^{-ix}}{2} - \frac{e^{ix} - e^{-ix}}{2}       \\
                           & = \frac{2e^{-ix}}{2}                                            \\
                           & = e^{-ix}
        \end{align*}

\end{itemize}

\section*{15.7.2}

(1)

反证法,假设$c$不存在,即,对任意$c > 0$,存在$0 < |y - x_0| < c$,使得$f(y) = 0$。

因为$f$在$x_0$处是可微的,那么,
\begin{align*}
  \lim_{x \to x_0;x \in \mathbb{R} - {x_0}} \frac{f(x) - f(x_0)}{x - x_0} = f^\prime(x_0)
\end{align*}

于是,对$\epsilon = \frac{1}{2} f(x_0)$,存在$\delta > 0$,使得只要$|x - x_0| < \delta$,就有
\begin{align*}
  \left|\frac{f(x) - f(x_0)}{x - x_0} - f^\prime(x_0)\right| < \frac{1}{2} f^\prime(x_0) \\
\end{align*}

由假设可知,取$c = \delta$,那么,存在$0 < |y - x_0| < c$,使得$f(y) = 0$。
综上,我们有
\begin{align*}
  \left|\frac{f(y) - f(x_0)}{y - x_0} - f^\prime(x_0)\right| < \frac{1}{2} f^\prime(x_0) \\
  \left|\frac{0 - 0}{y - x_0} - f^\prime(x_0)\right| < \frac{1}{2} f^\prime(x_0)         \\
  f^\prime(x_0) < \frac{1}{2} f^\prime(x_0)                                              \\
\end{align*}
存在矛盾。

(2)

因为$sin(x) = 0, sin^\prime(0) = cos(0) = 1$,所以,由(1)可知,
存在$c > 0$使得只要$0 < |0 - x| = |x| < c$,$sin(x) \neq 0$。

即$-c < x < 0$或$0 < x < c$,都有$sin(x) \neq 0$。

\section*{15.7.3}

\begin{itemize}
  \item (a)

        \begin{align*}
          cos(x + \pi) & = cos(x)cos(\pi) - sin(x)sin(\pi) \\
                       & = cos(x)(-1) - sin(x)0            \\
                       & = -cos(x)
        \end{align*}

        \begin{align*}
          sin(x + \pi) & = sin(x)cos(\pi) + cos(x)sin(\pi) \\
                       & = sin(x)(-1) + cos(x)0            \\
                       & = -sin(x)
        \end{align*}

        特别地,
        \begin{align*}
          cos(x + 2\pi) & = cos((x + \pi) + \pi) \\
                        & = -cos(x + \pi)        \\
                        & = cos(x)
        \end{align*}

        \begin{align*}
          sin(x + 2\pi) & = sin((x + \pi) + \pi) \\
                        & = -sin(x + \pi)        \\
                        & = sin(x)
        \end{align*}

  \item (b)

        \begin{itemize}
          \item $\Rightarrow$

                由书中的讨论可知,$sin(0) = 0, sin(\pi) = 0$且$x \in (0, \pi), sin(x) \neq 0$。

                对任意$x$都可以表示成$x = n\pi + x_0$,其中$x_0 \in [0, \pi)$,$n$是整数。

                由(a)可知,
                \begin{align*}
                  sin(x) = (-1)^nsin(x_0)
                \end{align*}

                综上,只有$x_0 = 0$时,$sin(x) = 0$。此时,$x/\pi = n$是一个整数。


          \item $\Leftarrow$

                因为$x = n\pi$,其中$n$是整数,所以,
                \begin{align*}
                  sin(x) & = sin(n\pi)     \\
                         & = sin(0 + n\pi) \\
                         & = (-1)^{n} 0    \\
                         & = 0
                \end{align*}
        \end{itemize}
  \item (c)

        因为
        \begin{align*}
          sin(\pi) & = sin(\frac{1}{2}\pi + \frac{1}{2}\pi)                                            \\
                   & = sin(\frac{1}{2}\pi)cos(\frac{1}{2}\pi) + cos(\frac{1}{2}\pi)sin(\frac{1}{2}\pi) \\
                   & = 2sin(\frac{1}{2}\pi)sin(\frac{1}{2}\pi)                                         \\
                   & = 0
        \end{align*}

        因为$sin(\frac{1}{2}\pi) > 0$,所以,$cos(\frac{1}{2}\pi) = 0$。
        又由$sin(\frac{1}{2}\pi)^2 + cos(\frac{1}{2}\pi)^2 = 1$可得,$sin(\frac{1}{2}\pi) = 1$。

        又我们有,
        \begin{align*}
          sin(x + \frac{1}{2}\pi) & = sin(x)cos(\frac{1}{2}\pi) + cos(x)sin(\frac{1}{2}\pi) \\
                                  & = cos(x)
        \end{align*}

        综上,由(b)可知,(c)成立。

\end{itemize}

\section*{15.7.4}



\end{document}