\documentclass{article}
\usepackage{mathtools} 
\usepackage{fontspec}
\usepackage[UTF8]{ctex}
\usepackage{amsthm}
\usepackage{mdframed}
\usepackage{xcolor}
\usepackage{amssymb}
\usepackage{amsmath}


% 定义新的带灰色背景的说明环境 zremark
\newmdtheoremenv[
  backgroundcolor=gray!10,
  % 边框与背景一致,边框线会消失
  linecolor=gray!10
]{zremark}{说明}


\begin{document}
\title{15.7 习题}
\author{张志聪}
\maketitle

\section*{15.7.1}

\begin{itemize}
  \item (a)

        利用引理15.6.6和习题15.6.16中的$\exp(z + w) = \exp(z)\exp(w)$。
        \begin{align*}
          \sin(x)^2 + \cos(x)^2
           & = \left(\frac{e^{ix} - e^{-ix}}{2i}\right)^2 + \left(\frac{e^{ix} + e^{-ix}}{2}\right)^2       \\
           & = \frac{e^{2ix} + e^{-2ix} - 2 e^{ix - ix}}{-4} + \frac{e^{2ix} + e^{-2ix} + 2 e^{ix - ix}}{4} \\
           & = \frac{e^{2ix} + e^{-2ix}}{-4} + \frac{-2}{-4} + \frac{e^{2ix} + e^{-2ix}}{4} + \frac{2}{4}   \\
           & = 1
        \end{align*}
  \item (b)

        \begin{align*}
          sin^\prime(x) & = \left(\sum\limits_{n = 0}^\infty \frac{(-1)^nx^{2n + 1}}{(2n + 1)!} \right)^\prime \\
                        & = \sum\limits_{n = 0}^\infty \frac{(-1)^n(2n + 1)x^{2n}}{(2n + 1)!}                  \\
                        & = \sum\limits_{n = 0}^\infty \frac{(-1)^nx^{2n}}{(2n)!}                              \\
                        & = cos(x)
        \end{align*}

        \begin{align*}
          cos^\prime(x) & = \left(\sum\limits_{n = 0}^\infty \frac{(-1)^nx^{2n}}{(2n)!}\right)^\prime \\
                        & = \sum\limits_{n = 1}^\infty \frac{2n(-1)^nx^{2n-1}}{(2n)!}                 \\
                        & = \sum\limits_{n = 1}^\infty \frac{(-1)^nx^{2n-1}}{(2n-1)!}                 \\
        \end{align*}
        令$m = n - 1$,即$n = m + 1$,利用命题7.4.3(级数的重排序),
        \begin{align*}
           & \sum\limits_{n = 1}^\infty \frac{(-1)^nx^{2n-1}}{(2n-1)!}                     \\
           & = \sum\limits_{m = 0}^\infty \frac{(-1)^{m + 1}x^{2(m + 1)-1}}{(2(m + 1)-1)!} \\
           & = \sum\limits_{m = 0}^\infty \frac{(-1)^{m + 1}x^{2m + 1}}{(2m + 1)!}         \\
           & = - \sum\limits_{m = 0}^\infty \frac{(-1)^{m}x^{2m + 1}}{(2m + 1)!}           \\
           & = -sin(x)
        \end{align*}

  \item (c)

        \begin{align*}
          sin(-x) & = \frac{e^{-ix} - e^{ix}}{2i}   \\
                  & = - \frac{e^{ix} - e^{-ix}}{2i} \\
                  & = -sin(x)
        \end{align*}

        \begin{align*}
          cos(-x) & = \frac{e^{i(-x)} + e^{-i(-x)}}{2} \\
                  & = \frac{e^{-ix} + e^{ix}}{2}       \\
                  & = cos(x)
        \end{align*}

  \item (d)

        \begin{align*}
          cos(x)cos(y) - sin(x)sin(y) & = \frac{e^{ix} + e^{-ix}}{2} \frac{e^{iy} + e^{-iy}}{2}                    \\
                                      & - \frac{e^{ix} - e^{-ix}}{2i} \frac{e^{iy} - e^{-iy}}{2i}                  \\
                                      & = \frac{e^{ix}e^{iy} + e^{ix}e^{-iy} + e^{-ix}e^{iy} + e^{-ix}e^{-iy}}{4}  \\
                                      & - \frac{e^{ix}e^{iy} - e^{ix}e^{-iy} - e^{-ix}e^{iy} + e^{-ix}e^{-iy}}{-4} \\
                                      & = \frac{2e^{ix}e^{iy} + 2e^{-ix}e^{-iy}}{4}                                \\
                                      & = \frac{e^{ix}e^{iy} + e^{-ix}e^{-iy}}{2}                                  \\
                                      & = \frac{e^{i(x + y)} + e^{-i(x + y)}}{2}                                   \\
                                      & = cos(x + y)
        \end{align*}

        \begin{align*}
          sin(x)cos(y) + cos(x)sin(y) & = \frac{e^{ix} - e^{-ix}}{2i} \frac{e^{iy} + e^{-iy}}{2}                    \\
                                      & + \frac{e^{ix} + e^{-ix}}{2}\frac{e^{iy} - e^{-iy}}{2i}                     \\
                                      & = \frac{e^{ix}e^{iy} + e^{ix}e^{-iy} - e^{-ix}e^{iy} - e^{-ix}e^{-iy} }{4i} \\
                                      & + \frac{e^{ix}e^{iy} - e^{ix}e^{-iy} + e^{-ix}e^{iy} - e^{-ix}e^{-iy}}{4i}  \\
                                      & = \frac{2e^{ix}e^{iy} - 2e^{-ix}e^{-iy}}{4i}                                \\
                                      & = \frac{e^{i(x + y)} - e^{-i(x + y)}}{2i}                                   \\
                                      & = sin(x + y)
        \end{align*}


  \item (e)

        \begin{align*}
          cos(0) & = \frac{e^{i \times 0} + e^{-i \times 0}}{2i} \\
                 & = \frac{1 + 1}{2}                             \\
                 & = 1
        \end{align*}

        由(a)可知,$sin(0) = 1 - cos(0)^2 = 1 - 1 = 0$


  \item (f)

        \begin{align*}
          cos(x) + isin(x) & = \frac{e^{ix} + e^{-ix}}{2} + i\frac{e^{ix} - e^{-ix}}{2i}     \\
                           & = \frac{e^{ix} + e^{-ix}}{2} + i^2\frac{e^{ix} - e^{-ix}}{2i^2} \\
                           & = \frac{e^{ix} + e^{-ix}}{2} + \frac{e^{ix} - e^{-ix}}{2}       \\
                           & = \frac{2e^{ix}}{2}                                             \\
                           & = e^{ix}
        \end{align*}

        同理可得,
        \begin{align*}
          cos(x) - isin(x) & = \frac{e^{ix} + e^{-ix}}{2} - i\frac{e^{ix} - e^{-ix}}{2i}     \\
                           & = \frac{e^{ix} + e^{-ix}}{2} - i^2\frac{e^{ix} - e^{-ix}}{2i^2} \\
                           & = \frac{e^{ix} + e^{-ix}}{2} - \frac{e^{ix} - e^{-ix}}{2}       \\
                           & = \frac{2e^{-ix}}{2}                                            \\
                           & = e^{-ix}
        \end{align*}

\end{itemize}

\section*{15.7.2}

(1)

反证法,假设$c$不存在,即,对任意$c > 0$,存在$0 < |y - x_0| < c$,使得$f(y) = 0$。

因为$f$在$x_0$处是可微的,那么,
\begin{align*}
  \lim_{x \to x_0;x \in \mathbb{R} - {x_0}} \frac{f(x) - f(x_0)}{x - x_0} = f^\prime(x_0)
\end{align*}

于是,对$\epsilon = \frac{1}{2} f(x_0)$,存在$\delta > 0$,使得只要$|x - x_0| < \delta$,就有
\begin{align*}
  \left|\frac{f(x) - f(x_0)}{x - x_0} - f^\prime(x_0)\right| < \frac{1}{2} f^\prime(x_0) \\
\end{align*}

由假设可知,取$c = \delta$,那么,存在$0 < |y - x_0| < c$,使得$f(y) = 0$。
综上,我们有
\begin{align*}
  \left|\frac{f(y) - f(x_0)}{y - x_0} - f^\prime(x_0)\right| < \frac{1}{2} f^\prime(x_0) \\
  \left|\frac{0 - 0}{y - x_0} - f^\prime(x_0)\right| < \frac{1}{2} f^\prime(x_0)         \\
  f^\prime(x_0) < \frac{1}{2} f^\prime(x_0)                                              \\
\end{align*}
存在矛盾。

(2)

因为$sin(x) = 0, sin^\prime(0) = cos(0) = 1$,所以,由(1)可知,
存在$c > 0$使得只要$0 < |0 - x| = |x| < c$,$sin(x) \neq 0$。

即$-c < x < 0$或$0 < x < c$,都有$sin(x) \neq 0$。

\section*{15.7.3}

\begin{itemize}
  \item (a)

        \begin{align*}
          cos(x + \pi) & = cos(x)cos(\pi) - sin(x)sin(\pi) \\
                       & = cos(x)(-1) - sin(x)0            \\
                       & = -cos(x)
        \end{align*}

        \begin{align*}
          sin(x + \pi) & = sin(x)cos(\pi) + cos(x)sin(\pi) \\
                       & = sin(x)(-1) + cos(x)0            \\
                       & = -sin(x)
        \end{align*}

        特别地,
        \begin{align*}
          cos(x + 2\pi) & = cos((x + \pi) + \pi) \\
                        & = -cos(x + \pi)        \\
                        & = cos(x)
        \end{align*}

        \begin{align*}
          sin(x + 2\pi) & = sin((x + \pi) + \pi) \\
                        & = -sin(x + \pi)        \\
                        & = sin(x)
        \end{align*}

  \item (b)

        \begin{itemize}
          \item $\Rightarrow$

                由书中的讨论可知,$sin(0) = 0, sin(\pi) = 0$且$x \in (0, \pi), sin(x) \neq 0$。

                对任意$x$都可以表示成$x = n\pi + x_0$,其中$x_0 \in [0, \pi)$,$n$是整数。

                由(a)可知,
                \begin{align*}
                  sin(x) = (-1)^nsin(x_0)
                \end{align*}

                综上,只有$x_0 = 0$时,$sin(x) = 0$。此时,$x/\pi = n$是一个整数。


          \item $\Leftarrow$

                因为$x = n\pi$,其中$n$是整数,所以,
                \begin{align*}
                  sin(x) & = sin(n\pi)     \\
                         & = sin(0 + n\pi) \\
                         & = (-1)^{n} 0    \\
                         & = 0
                \end{align*}
        \end{itemize}
  \item (c)

        因为
        \begin{align*}
          sin(\pi) & = sin(\frac{1}{2}\pi + \frac{1}{2}\pi)                                            \\
                   & = sin(\frac{1}{2}\pi)cos(\frac{1}{2}\pi) + cos(\frac{1}{2}\pi)sin(\frac{1}{2}\pi) \\
                   & = 2sin(\frac{1}{2}\pi)sin(\frac{1}{2}\pi)                                         \\
                   & = 0
        \end{align*}

        因为$sin(\frac{1}{2}\pi) > 0$,所以,$cos(\frac{1}{2}\pi) = 0$。
        又由$sin(\frac{1}{2}\pi)^2 + cos(\frac{1}{2}\pi)^2 = 1$可得,$sin(\frac{1}{2}\pi) = 1$。

        又我们有,
        \begin{align*}
          sin(x + \frac{1}{2}\pi) & = sin(x)cos(\frac{1}{2}\pi) + cos(x)sin(\frac{1}{2}\pi) \\
                                  & = cos(x)
        \end{align*}

        综上,由(b)可知,(c)成立。

\end{itemize}

\section*{15.7.4}

(1)对$y$值进行讨论。
\begin{itemize}
  \item $y = 1$。

        于是$x = 0$,取$\theta = 0$,于是
        \begin{align*}
          x = 0 = sin(\theta) = sin(0) \\
          y = 1 = cos(\theta) = cos(0)
        \end{align*}

  \item $y = -1$。

        于是$x = 0$,取$\theta = \pi$,于是
        \begin{align*}
          x = 0 = sin(\theta) = sin(\pi) = -sin(0) \\
          y = -1 = cos(\theta) = cos(\pi) = -cos(0)
        \end{align*}

  \item $y \in (-1, 1)$。

        因为$cos^(z) = -sin(z)$,又
        所以在$z \in (0, \pi)$,$sin(z) > 0$,
        所以$cos(z)$在$(0, \pi)$中严格单调递减,
        由介质定理可得,在$(cos(0), cos(\pi)) = (-1, 1)$中,存在$\theta_0 \in (0, \pi)$,使得
        \begin{align*}
          cos(\theta_0) = y
        \end{align*}

        又因为
        \begin{align*}
          sin(\theta_0)^2 = 1 - cos(\theta_0)^2 = 1 - y^2 = x^2
        \end{align*}

        于是,$sin(\theta_0) = x$或$sin(\theta_0) = -x$。

        \begin{itemize}
          \item 如果$sin(\theta_0) = x$,直接取$\theta = \theta_0$即可。
          \item 如果$sin(\theta_0) = -x$。

                取$\theta = - \theta_0$,于是
                \begin{align*}
                  cos(\theta) = cos(-\theta_0) = cos(\theta_0) = y \\
                  sin(\theta) = sin(-\theta_0) = -sin(\theta_0) = -(-x) = x
                \end{align*}

                又因为$-\theta_0 \in (-\pi, 0) \subseteq (-\pi, \pi]$,满足题设。
        \end{itemize}
\end{itemize}

(2)唯一性证明。

反证法,假设存在$\theta^\prime \in (-\pi, \pi], \theta \neq \theta^\prime$,使得
\begin{align*}
  x = sin(\theta^\prime) = sin(\theta) \\
  y = cos(\theta^\prime) = cos(\theta)
\end{align*}

$y = 1$或$ y = -1$时,唯一性可以直接由定理15.7.5(b)推导出。我们主要考虑$y \in (-1, 1)$时。

因为$cos(x)$在$(0, \pi)$中严格单调递减,
所以在$(0, \pi)$中最多存在一个$\theta$使得$cos(\theta) = y$。

同理,$(-\pi, 0)$中最多存在一个$\theta^\prime$使得$cos(\theta^\prime) = y$。

又因为$cos(-x) = cos(x)$,于是可得$\theta = -\theta^\prime$。

而$sin(-x) = -sin(x)$,于是$sin(\theta^\prime) = -sin(\theta)$,
这与$sin(\theta) = sin(\theta^\prime)$矛盾。

\section*{15.7.5}

\begin{itemize}
  \item (a) $r = s$。

        由定理15.7.2(f)可知,
        \begin{align*}
          re^{i\theta} = r cos(\theta) + r\sin(\theta)i \\
          se^{i\alpha} = s cos(\alpha) + s\sin(\alpha)i
        \end{align*}

        因为$re^{i\theta} = se^{i\alpha}$,于是$|re^{i\theta}| = |se^{i\alpha}|$。
        又因为
        \begin{align*}
          |re^{i\theta}| = \sqrt{(r cos(\theta))^2 + (r\sin(\theta))^2} = r \\
          |se^{i\alpha}| = \sqrt{(r cos(\alpha))^2 + (r\sin(\alpha))^2} = s
        \end{align*}

        综上,$r = s$。

  \item (b)存在一个整数$k$使得$\theta = \alpha + 2\pi k$。

        结合(a)可知,$\theta, \alpha$要满足以下条件:
        \begin{equation*}
          \begin{cases*}
            cos(\theta) = cos(\alpha) \\
            sin(\theta) = sin(\alpha)
          \end{cases*}
        \end{equation*}

        如果$\theta = \alpha$,此时$k = 0$,命题成立。

        如果$\theta \neq \alpha$。
        因为$sin(x), cos(x)$都是周期函数,且周期为$2\pi$,
        所以,我们可以在$(-\pi, \pi]$上考虑该问题。

        令$x^2 = sin^2(\theta), y^2 = cos^2(\theta)$,
        于是$x^2 + y^2 = 1$。
        于是利用习题15.7.4可知,
        恰存在一个实数$\vartheta \in (-\pi, \pi]$使得
        $x = sin(\vartheta), y = cos(\vartheta)$。

        由$\vartheta$的唯一性可知,$\theta, \alpha$要满足:
        \begin{align*}
          \theta = \vartheta + 2\pi k_1 \\
          \alpha = \vartheta + 2\pi k_2
        \end{align*}
        (其中,$k_1, k_2 \in \mathbb{Z}$)

        如果不满足该条件,会导致$\vartheta$不唯一。
        因为存在$k^\prime \in \mathbb{Z}$使得
        \begin{align*}
          \alpha + 2\pi k^\prime \in (-\pi, \pi] \\
          \alpha + 2\pi k^\prime \neq \vartheta
        \end{align*}
        (这里以$\alpha$为例)

        于是
        $x = sin(\alpha + 2\pi k^\prime), y = cos(\alpha + 2\pi k^\prime)$,
        与$\vartheta$的唯一性矛盾。

        综上,命题成立。
\end{itemize}

\section*{15.7.6}

\begin{align*}
  re^{i\theta} & = r(cos(\theta) + sin(\theta)i) \\
               & = rcos(\theta) + rsin(\theta)i
\end{align*}

因为要满足$z = re^{i\theta}$,于是要保证,
\begin{align*}
  |z| & = |re^{i\theta}|                             \\
      & = \sqrt{r^2cos^2(\theta) + r^2sin^2(\theta)} \\
      & = \sqrt{r^2}                                 \\
      & = |r|
\end{align*}

因为,$r > 0$,所以,取$r = |z|$。
(注意,这里的$r$是唯一。因为如果$r \neq |z|$,会导致$z = re^{i\theta}$无法成立。)

另外,我们需要求出以下两个方程的解。
\begin{equation*}
  \begin{cases*}
    rcos(\theta) = \mathfrak{R}(z) \\
    rsin(\theta) = \mathfrak{I}(z)
  \end{cases*}
\end{equation*}

两等式分别平方,然后,相加:
\begin{align*}
  r^2cos^2(\theta) + r^2sin^2(\theta) = (\mathfrak{R}(z))^2 + (\mathfrak{I}(z))^2 = |z|^2 \\
  r^2(cos^2(\theta) + sin^2(\theta)) = |z|^2                                              \\
  |z|^2(cos^2(\theta) + sin^2(\theta)) = |z|^2                                            \\
  cos^2(\theta) + sin^2(\theta) = 1
\end{align*}

由习题15.7.4可知,$\theta$存在且唯一。

\section*{15.7.7}

\begin{align*}
  \mathfrak{R}((cos(\theta) + isin(\theta))^n) & = \mathfrak{R}((e^{i\theta})^n) \\
                                               & = \mathfrak{R}(e^{in\theta})    \\
                                               & = cos(n\theta)
\end{align*}

同理可得,
\begin{align*}
  \mathfrak{I}((cos(\theta) + isin(\theta))^n) & = \mathfrak{I}((e^{i\theta})^n) \\
                                               & = \mathfrak{I}(e^{in\theta})    \\
                                               & = sin(n\theta)
\end{align*}

\section*{15.7.8}

\begin{itemize}
  \item $\tan(x)$可微且单调递增,$\frac{d}{dx} tan(x) = 1 + tan(x)^2$。

        由于$sin(x), cos(x)$在$R$上可微,且$x \in (-\pi/2, \pi/2)$上,$con(x) \neq 0$,
        由定理10.1.13(h),$tan(x)$可微,且
        \begin{align*}
          (tan(x))^\prime & = (\frac{sin(x)}{con(x)})^\prime                  \\
                          & = \frac{cos(x)cos(x) - sin(x)(-sin(x))}{con(x)^2} \\
                          & = \frac{1}{cos(x)^2}
        \end{align*}

        因为$x \in (-\pi/2, \pi/2), cos(x)^2 > 0$,于是$(tan(x))^\prime > 0$,
        所以,$tan(x)$在$(-\pi/2, \pi/2)$上是严格单调递增的。

        我们有,
        \begin{align*}
          1 + tan(x)^2 & = 1 + \frac{sin(x)^2}{cos(x)^2}        \\
                       & = \frac{cos(x)^2 + sin(x)^2}{cos(x)^2} \\
                       & = \frac{1}{cos(x)^2}                   \\
                       & = (tan(x))^\prime
        \end{align*}

        所以,$\frac{d}{dx} tan(x) = 1 + tan(x)^2$。

  \item $\lim\limits_{x \to \pi/2} tan(x) = +\infty, \lim\limits_{x \to -\pi/2} tan(x) = -\infty$。

        由定理15.7.5(c)可知,$cos(\pi/2) = 0$,于是$sin(x) = 1 - cos(x)^2 = 1$。
        因为,$cos(x), sin(x)$都是连续的,所以,
        \begin{align*}
          \lim\limits_{x \to \pi/2} cos(x) = 0 \\
          \lim\limits_{x \to \pi/2} sin(x) = 1
        \end{align*}

        于是,
        \begin{align*}
          \lim\limits_{x \to \pi/2} tan(x) = +\infty
        \end{align*}

        注意:这里不能直接使用极限定理(定理6.1.19)得到,而是利用函数在一点处收敛的定义(定义9.3.6),具体证明略。

        类似地,
        \begin{align*}
          \lim\limits_{x \to -\pi/2} tan(x) = -\infty
        \end{align*}

  \item $tan(x)$实际上是$(-\pi/2, \pi/2) \to \mathbb{R}$的双射。

        由之前的讨论可知,$\tan(x)$在$(-\pi/2, \pi/2)$上是严格单调递增的,
        于是,$tan(x)$是$(-\pi/2, \pi/2) \to \mathbb{R}$的双射。

  \item $tan^{-1}$可微的,并且有$\frac{d}{dx} tan^{-1}(x) = \frac{1}{1 + x^2}$。

        由定理10.4.2(反函数定理)可知,$tan^{-1}$可微,并且对$y = tan(x)$有
        \begin{align*}
          (tan^{-1})^\prime(y) & = \frac{1}{tan^\prime(x)} \\
                               & = \frac{1}{1 + tan(x)^2}  \\
                               & = \frac{1}{1 + y^2}
        \end{align*}

        所以,$\frac{d}{dx} tan^{-1}(x) = \frac{1}{1 + x^2}$。
\end{itemize}

\section*{15.7.9}

(1)

因为,$x \in (-1, 1)$,所以,$|-x^2| < 1$。

所以,
\begin{align*}
  \frac{1}{1 - (- x^2)} & = \sum \limits_{n = 0}^{\infty} (- x^2)^n     \\
                        & = \sum \limits_{n = 0}^{\infty} (-1)^n x^{2n}
\end{align*}

$\sum \limits_{n = 0}^{\infty} (-1)^n x^{2n}$
这个级数与标准幂级数形式$\sum \limits_{n = 0}^{\infty} c_n(x - a)^n$不一致,
但它任然是一个合法的幂级数,但需要做如下改变:

设$m = 2n$,于是利用命题7.4.3(级数的重排序)
\begin{align*}
  \sum \limits_{n = 0}^{\infty} (-1)^n x^{2n}
  =
  \sum \limits_{m = 0}^{\infty} c_m x^{m}
\end{align*}
其中,
\begin{equation*}
  c_m = \begin{cases}
    (-1)^{m/2} & \text{if } m \text{ is 偶数} \\
    0          & \text{if } m \text{ is 奇数}
  \end{cases}
\end{equation*}
(其实这个幂级数缺少奇次项,书中定义的$sin(x), cos(x)$幂级数表示形式,也分别缺少奇次项和偶次项)。

利用定理15.1.6(e)可知,
\begin{align*}
  \int_{[0, x]} \frac{1}{1 + y^2} dy
   & = \sum \limits_{m = 0}^{\infty} c_m \frac{x^{m + 1}}{m + 1} \\
\end{align*}
令$n = \frac{1}{2}m$,于是再次利用命题7.4.3(级数的重排序)
\begin{align*}
  \sum \limits_{m = 0}^{\infty} c_m \frac{x^{m + 1}}{m + 1}
  =
  \sum \limits_{n = 0}^{\infty} (-1)^n \frac{x^{2n + 1}}{2n + 1}
\end{align*}

又因为,
\begin{align*}
  \int_{[0, x]} \frac{1}{1 + y^2} dy
   & = tan^{-1}(y)|_0^x          \\
   & = tan^{-1}(x) - tan^{-1}(0) \\
   & = tan^{-1}(x) - 0           \\
   & = tan^{-1}(x)
\end{align*}

综上可得,
\begin{align*}
  tan^{-1}(x) = \sum \limits_{n = 0}^{\infty} (-1)^n \frac{x^{2n + 1}}{2n + 1}
\end{align*}

(2)

\begin{itemize}
  \item 使用阿贝尔定理证明。(按照提示证明,但个人有一点问题:这个幂级数的收敛半径不是$1$)。

        因为
        \begin{align*}
          \sum \limits_{n = 0}^{\infty} \frac{(-1)^n}{2n + 1}
        \end{align*}

        所以,有定理15.3.1(阿贝尔定理)可得,
        \begin{align*}
          \lim\limits_{x \to 1} tan^{-1}(x) = \sum \limits_{n = 0}^{\infty} \frac{(-1)^n}{2n + 1} \\
          \implies                                                                                \\
          tan^{-1} (1) = \sum \limits_{n = 0}^{\infty} \frac{(-1)^n}{2n + 1}
        \end{align*}

        % \item 使用命题14.3.3证明。

        %       $tan^{-1}(x) = \sum \limits_{n = 0}^{\infty} (-1)^n \frac{x^{2n + 1}}{2n + 1}$,
        %       即部分和$\sum \limits_{n = 0}^{N} (-1)^n \frac{x^{2n + 1}}{2n + 1}$一致收敛于$tan^{-1}(x)$。

        %       又因为,对任意$N$都有,
        %       \begin{align*}
        %         \lim\limits_{x \to 1} \sum \limits_{n = 0}^{N} (-1)^n \frac{x^{2n + 1}}{2n + 1}
        %          & = \sum \limits_{n = 0}^{N} \frac{(-1)^n}{2n + 1}
        %       \end{align*}

        %       于是,利用命题14.3.3可知,
        %       \begin{align*}
        %         \lim\limits_{x \to 1} tan^{-1}(x)
        %          & = \lim\limits_{N \to +\infty} \lim \limits_{x \to 1} \sum \limits_{n = 0}^{N} (-1)^n \frac{x^{2n + 1}}{2n + 1} \\
        %          & = \lim\limits_{N \to +\infty} \sum \limits_{n = 0}^{N} \frac{(-1)^n}{2n + 1}                                   \\
        %          & = \sum \limits_{n = 0}^{\infty} \frac{(-1)^n}{2n + 1}
        %       \end{align*}
        %       因为,$tan^{-1}(x)$是连续的,所以,
        %       \begin{align*}
        %         \lim\limits_{x \to 1} tan^{-1}(x) = tan^{-1}(1)
        %       \end{align*}

        %       综上可得,
        %       \begin{align*}
        %         tan^{-1}(1) & = \sum \limits_{n = 0}^{\infty} \frac{(-1)^n}{2n + 1}
        %       \end{align*}
\end{itemize}

接下来,现在需要证明: $tan^{-1}(1) = \frac{\pi}{4}$。

\begin{align*}
  cos(\pi/2) & = cos(\pi/4 + \pi/4)                          \\
             & = cos(\pi/4)cos(\pi/4) - sin(\pi/4)sin(\pi/4) \\
             & = 0
\end{align*}
于是可得,
\begin{align*}
  cos(\pi/4)cos(\pi/4) = sin(\pi/4)sin(\pi/4)
\end{align*}
又因为,
\begin{align*}
  cos(\pi/4)cos(\pi/4) + sin(\pi/4)sin(\pi/4) = 1
\end{align*}
又因为$cos(\pi/4) > 0, sin(\pi/4) > 0$,于是
\begin{align*}
  cos(\pi/4) = \frac{1}{\sqrt{2}} \\
  sin(\pi/4) = \frac{1}{\sqrt{2}}
\end{align*}

因为,$tan(x) = \frac{sin(x)}{cos(x)}$,且在$(-\pi/2, \pi/2)$中严格单调递减,
所以,当且仅当$x = \pi/4$时,$tan(x) = 1$。

综上可得,
\begin{align*}
  tan^{-1}(1) = \frac{\pi}{4}
\end{align*}

所以,
\begin{align*}
  \pi = 4tan^{-1}(1) = 4\lim\limits_{n \to +\infty} \frac{(-1)^n}{2n + 1}
\end{align*}

(3)推导出$4 - \frac{4}{3} < \pi < 4$。

todo

\section*{15.7.19}

\begin{itemize}
  \item (a)

        利用定理14.5.8(威尔斯特拉斯M判别法)完成证明。

        令$f_(n) = 4^{-n} cos(32^n \pi x)$是有界且连续的,
        又因为$\sum \limits_{n = 1}^{\infty} ||f_(n)||_{\infty} = \sum \limits_{n = 1}^{\infty} 4^{-n}$,
        由引理7.3.3(几何级数)可知,$\sum \limits_{n = 1}^{\infty} 4^{-n}$收敛,
        于是利用定理14.5.8(威尔斯特拉斯M判别法),
        $\sum \limits_{n = 1}^{\infty} 4^{-n} cos(32^n \pi x)$
        一致收敛于连续的函数$f$。
  \item (b)

        直接利用书中的提示的小命题,不做证明了。

        \begin{align*}
          \left|f(\frac{j + 1}{32^m}) - f(\frac{j}{32^m})\right|
           & = \left|\sum\limits_{n = 1}^{\infty} 4^{-n} cos(32^n \pi \frac{j + 1}{32^m}) - \sum\limits_{n = 1}^{\infty} 4^{-n} cos(32^n \pi \frac{j}{32^m}) \right| \\
           & =  \left|\sum\limits_{n = 1}^{\infty} 4^{-n} cos(32^n \pi \frac{j + 1}{32^m}) - 4^{-n} cos(32^n \pi \frac{j}{32^m}) \right|
        \end{align*}

        $n > m$时,$32^n \pi \frac{j + 1}{32^m}$与$32^n \pi \frac{j}{32^m}$都是$2\pi$的整数倍,
        所以,
        \begin{align*}
          \left|\sum\limits_{n = m+1}^{\infty} 4^{-n} cos(32^n \pi \frac{j + 1}{32^m}) - 4^{-n} cos(32^n \pi \frac{j}{32^m}) \right|
           & = \left|\sum\limits_{n = m}^{\infty} 4^{-n} \times 1  - 4^{-n} \times 1  \right| \\
           & = 0
        \end{align*}

        $n = m$时,
        \begin{align*}
          \left|\sum\limits_{n = m}^{m} 4^{-n} cos(32^n \pi \frac{j + 1}{32^m}) - 4^{-n} cos(32^n \pi \frac{j}{32^m}) \right|
           & = \left|4^{-m} cos(32^m \pi \frac{j + 1}{32^m}) - 4^{-m} cos(32^m \pi \frac{j}{32^m}) \right| \\
           & = \left|4^{-m} cos(\pi (j + 1)) - 4^{-m} cos(\pi j) \right|                                   \\
           & = 2 \times 4^{-m}
        \end{align*}

        $n < m$时,
        \begin{align*}
          \left|\sum\limits_{n = 1}^{m-1} 4^{-n} cos(32^n \pi \frac{j + 1}{32^m}) - 4^{-n} cos(32^n \pi \frac{j}{32^m}) \right|
           & \leq \sum\limits_{n = 1}^{m-1} 4^{-n}\frac{\pi}{32^{m - n}}                         \\
           & = \sum\limits_{n = 1}^{m-1} 4^{-n}4^{n-m}\frac{\pi}{(8 \times 4)^{m - n} 4^{n - m}} \\
           & = \sum\limits_{n = 1}^{m-1} 4^{-m}\frac{\pi}{8^{m - n}}                             \\
           & = 4^{-m}\sum\limits_{n = 1}^{m-1} \frac{\pi}{(2 \times 4)^{m - n}}                  \\
           & < 4^{-m} \sum\limits_{n = 1}^{m-1} \frac{1}{2^{m - n}}                              \\
           & \leq 4^{-m} \sum\limits_{n = 1}^{\infty} \frac{1}{2^{n}}                            \\
           & = 4^{-m}
        \end{align*}

        综上可得,
        \begin{align*}
          \left|f(\frac{j + 1}{32^m}) - f(\frac{j}{32^m})\right|
           & = \left|\sum\limits_{n = 1}^{\infty} 4^{-n} cos(32^n \pi \frac{j + 1}{32^m}) - \sum\limits_{n = 1}^{\infty} 4^{-n} cos(32^n \pi \frac{j}{32^m}) \right| \\
           & =  \left|\sum\limits_{n = 1}^{\infty} 4^{-n} cos(32^n \pi \frac{j + 1}{32^m}) - 4^{-n} cos(32^n \pi \frac{j}{32^m}) \right|                             \\
           & \geq 2 \times 4^{-m} - 4^{-m}                                                                                                                           \\
           & = 4^{-m}
        \end{align*}
  \item (c)

        todo (没解出来)

        反证法,假设$f$是可微的,那么$f$是连续。那么,对任意$x_0$,$f$在$x_0$是连续的。

        对于,任意$\epsilon > 0$,都存在$\delta > 0$,使得只要$|x - x_0| < \delta$,
        就有,
        \begin{align*}
          \left|f(x) - f(x_0)\right| < \epsilon
        \end{align*}
        $x_1, x_2 \in (x_0 - \delta, x_0 + \delta)$,我们有
        \begin{align*}
          \left|f(x_1) - f(x_2)\right|
           & = \left|f(x_1) - f(x_0) + f(x_0) - f(x_2)\right|                 \\
           & \leq \left|f(x_1) - f(x_0)\right| + \left|f(x_0) - f(x_2)\right| \\
           & < 2\epsilon
        \end{align*}

  \item (d)

        令$f_n = 4^{-n} cos(32^n \pi x)$,$f_n$是一个可微函数,
        他的导函数$f_n^\prime = 4^{-n} (-32^n \pi) sin(32^n \pi x)$
        是连续的,其中,
        \begin{align}
          \sum \limits_{n = 1}^{\infty} ||f_n||_{\infty} = \sum \limits_{n = 1}^{\infty} 4^{-n} 32^n \pi
        \end{align}
        不收敛,无法满足14.7.3的前置条件。

\end{itemize}

\end{document}


