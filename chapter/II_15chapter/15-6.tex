\documentclass{article}
\usepackage{mathtools} 
\usepackage{fontspec}
\usepackage[UTF8]{ctex}
\usepackage{amsthm}
\usepackage{mdframed}
\usepackage{xcolor}
\usepackage{amssymb}
\usepackage{amsmath}


% 定义新的带灰色背景的说明环境 zremark
\newmdtheoremenv[
  backgroundcolor=gray!10,
  % 边框与背景一致,边框线会消失
  linecolor=gray!10
]{zremark}{说明}


\begin{document}
\title{15.6 习题}
\author{张志聪}
\maketitle

\section*{15.6.1}

设$z_1 = (a, b), z_2 = (c, d), z_3 = (e, f)$。

\begin{itemize}
  \item (a) 可交换性:$z_1 + z_2 = z_2 + z_1$。

        按照定义15.6.3(复数的加法运算)可知,
        \begin{align*}
          z_1 + z_2 & = (a + c, b + d) \\
          z_2 + z_1 & = (c + a, d + b) \\
        \end{align*}
        因为
        \begin{align*}
          a + c & = c + a \\
          b + d & = d + b
        \end{align*}
        于是,由定义15.6.2中关于相等的定义可知,
        \begin{align*}
          z_1 + z_2 = z_2 + z_1
        \end{align*}

  \item (b) 结合性:$(z_1 + z_2) + z_3 = z_1 + (z_2 + z_3)$。

        我们有
        \begin{align*}
          (z_1 + z_2) + z_2 & = (a + c, b + d) + (e, f) \\
                            & = (a + c + e, b + d + f)
        \end{align*}

        又因为

        \begin{align*}
          z_1 + (z_2 + z_3) & = (a, b) + (c + e, d + f) \\
                            & = (a + c + e, b + d + f)
        \end{align*}

        于是,由定义15.6.2中关于相等的定义可知,
        \begin{align*}
          (z_1 + z_2) + z_3 = z_1 + (z_2 + z_3)
        \end{align*}

  \item (c) 恒等性:$z_1 + 0_\mathbb{C} = 0_\mathbb{C} + z_1$。

        我们有,
        \begin{align*}
          z_1 + 0_\mathbb{C} & = (a, b) + (0, 0) \\
                             & = (a, b)
        \end{align*}

        又因为
        \begin{align*}
          0_\mathbb{C} + z_1 & = (0, 0) + (a, b) \\
                             & = (a, b)
        \end{align*}

        于是,由定义15.6.2中关于相等的定义可知,
        \begin{align*}
          z_1 + 0_\mathbb{C} = 0_\mathbb{C} + z_1
        \end{align*}

  \item (d) 逆元性:$z_1 + (-z_1) = (-z_1) + z_1 = 0_\mathbb{C}$。

        由(a)可交换性可知
        \begin{align*}
          z_1 + (-z_1) = (-z_1) + z_1
        \end{align*}

        我们有,
        \begin{align*}
          z_1 + (-z_1) & = (a, b) + (-a, -b) \\
                       & = (0, 0)            \\
                       & = 0_\mathbb{C}
        \end{align*}

\end{itemize}

\section*{15.6.2}

设$z_1 = (a, b), z_2 = (c, d), z_3 = (e, f)$。

\begin{itemize}
  \item (a) 可交换性:$z_1z_2 = z_2z_1$。

        由定义15.6.5可知,
        \begin{align*}
          z_1z_2 & = (a, b)(c, d)       \\
                 & = (ac - bd, ad + bc)
        \end{align*}
        \begin{align*}
          z_2z_1 & = (c, d)(a, b)       \\
                 & = (ca - db, cb + da)
        \end{align*}

        因为
        \begin{align*}
          ac - bd & = ca - db \\
          ad + bc & = cb + da
        \end{align*}

        于是,由定义15.6.2中关于相等的定义可知,
        \begin{align*}
          z_1z_2 = z_2z_1
        \end{align*}

  \item (b) 结合性:$(z_1z_2)z_3 = z_1(z_2z_3)$。

        因为
        \begin{align*}
          (z_1z_2)z_3 & = ((a, b)(c, d))(e, f)                             \\
                      & = (ac - bd, ad + bc)(e, f)                         \\
                      & = ((ac-bd)e - (ad + bc)f, (ac - bd)f + (ad + bc)e) \\
                      & = (ace - bde - adf - bcf, acf - bdf + ade + bce)
        \end{align*}
        \begin{align*}
          z_1(z_2z_3) & = (a, b)((c, d)(e, f))                               \\
                      & = (a, b)(ce - df, cf + de)                           \\
                      & = (a(ce - df) - b(cf + de), a(cf + de) + b(ce - df)) \\
                      & = (ace - adf - bcf - bde, acf + ade + bce - bdf)
        \end{align*}

        于是,由定义15.6.2中关于相等的定义可知,
        \begin{align*}
          (z_1z_2)z_3 = z_1(z_2z_3)
        \end{align*}

  \item (c) 恒等性:$z_11_\mathbb{C} = 1_\mathbb{C}z_1 = z_1$。

        由(a)可知
        \begin{align*}
          z_11_\mathbb{C} = 1_\mathbb{C}z_1
        \end{align*}

        又有
        \begin{align*}
          z_11_\mathbb{C} & = (a, b)(1, 0)                                    \\
                          & = (a\times 1 - b \times 0, a\times 0 + b\times 1) \\
                          & = (a, b)                                          \\
                          & = z_1
        \end{align*}

  \item (d) 分配性:$z_1(z_2 + z_3) = z_1z_2 + z_1z_3$和$(z_2 + z_3)z_1 = z_2z_1 + z_3z_1$。

        因为
        \begin{align*}
          z_1(z_2 + z_3) & = (a, b)((c, d) + (e, f))                    \\
                         & = (a, b)(c + e, d + f)                       \\
                         & = (a(c + e) - b(d + f), a(d + f) + b(c + e)) \\
                         & = (ac + ae - bd - bf, ad + af + bc + be)
        \end{align*}
        \begin{align*}
          z_1z_2 + z_1z_2 & = (a, b)(c, d) + (a, b)(e, f)             \\
                          & = (ac - bd, ad + bc) + (ae - bf, af + be) \\
                          & = (ac - bd + ae - bf, ad + bc + af + be)
        \end{align*}

        于是,由定义15.6.2中关于相等的定义可知,
        \begin{align*}
          z_1(z_2 + z_3) = z_1z_2 + z_1z_3
        \end{align*}

        同理可得,
        \begin{align*}
          (z_2 + z_3)z_1 = z_2z_1 + z_3z_1
        \end{align*}

\end{itemize}

\section*{15.6.3}

这个引理是想说明:形式符号$z = (a, b)$与$z = a + bi$是等价的。

因为
\begin{align*}
  a + bi & = (a, 0) + (b, 0)(0, 1) \\
         & = (a, 0) + (0, b)       \\
         & = (a, b)
\end{align*}
从而,$a + bi$与$(a, b)$就是一回事。

\section*{15.6.4}

设$z = a + bi, w = c + di$。

\begin{itemize}
  \item $\overline{z + w} = \overline{z} + \overline{w}$。

        因为
        \begin{align*}
          z + w = a + bi + c + di & = a + c + (b + d)i
        \end{align*}
        于是
        \begin{align*}
          \overline{z + w} = a + c - (b + d)i
        \end{align*}
\end{itemize}



\end{document}