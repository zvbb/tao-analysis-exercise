\documentclass{article}
\usepackage{mathtools} 
\usepackage{fontspec}
\usepackage[UTF8]{ctex}
\usepackage{amsthm}
\usepackage{mdframed}
\usepackage{xcolor}
\usepackage{amssymb}
\usepackage{amsmath}


% 定义新的带灰色背景的说明环境 zremark
\newmdtheoremenv[
  backgroundcolor=gray!10,
  % 边框与背景一致,边框线会消失
  linecolor=gray!10
]{zremark}{说明}


\begin{document}
\title{15.6 习题}
\author{张志聪}
\maketitle

\section*{15.6.1}

设$z_1 = (a, b), z_2 = (c, d), z_3 = (e, f)$。

\begin{itemize}
  \item (a) 可交换性:$z_1 + z_2 = z_2 + z_1$。

        按照定义15.6.3(复数的加法运算)可知,
        \begin{align*}
          z_1 + z_2 & = (a + c, b + d) \\
          z_2 + z_1 & = (c + a, d + b) \\
        \end{align*}
        因为
        \begin{align*}
          a + c & = c + a \\
          b + d & = d + b
        \end{align*}
        于是,由定义15.6.2中关于相等的定义可知,
        \begin{align*}
          z_1 + z_2 = z_2 + z_1
        \end{align*}

  \item (b) 结合性:$(z_1 + z_2) + z_3 = z_1 + (z_2 + z_3)$。

        我们有
        \begin{align*}
          (z_1 + z_2) + z_2 & = (a + c, b + d) + (e, f) \\
                            & = (a + c + e, b + d + f)
        \end{align*}

        又因为

        \begin{align*}
          z_1 + (z_2 + z_3) & = (a, b) + (c + e, d + f) \\
                            & = (a + c + e, b + d + f)
        \end{align*}

        于是,由定义15.6.2中关于相等的定义可知,
        \begin{align*}
          (z_1 + z_2) + z_3 = z_1 + (z_2 + z_3)
        \end{align*}

  \item (c) 恒等性:$z_1 + 0_\mathbb{C} = 0_\mathbb{C} + z_1$。

        我们有,
        \begin{align*}
          z_1 + 0_\mathbb{C} & = (a, b) + (0, 0) \\
                             & = (a, b)
        \end{align*}

        又因为
        \begin{align*}
          0_\mathbb{C} + z_1 & = (0, 0) + (a, b) \\
                             & = (a, b)
        \end{align*}

        于是,由定义15.6.2中关于相等的定义可知,
        \begin{align*}
          z_1 + 0_\mathbb{C} = 0_\mathbb{C} + z_1
        \end{align*}

  \item (d) 逆元性:$z_1 + (-z_1) = (-z_1) + z_1 = 0_\mathbb{C}$。

        由(a)可交换性可知
        \begin{align*}
          z_1 + (-z_1) = (-z_1) + z_1
        \end{align*}

        我们有,
        \begin{align*}
          z_1 + (-z_1) & = (a, b) + (-a, -b) \\
                       & = (0, 0)            \\
                       & = 0_\mathbb{C}
        \end{align*}

\end{itemize}

\section*{15.6.2}

设$z_1 = (a, b), z_2 = (c, d), z_3 = (e, f)$。

\begin{itemize}
  \item (a) 可交换性:$z_1z_2 = z_2z_1$。

        由定义15.6.5可知,
        \begin{align*}
          z_1z_2 & = (a, b)(c, d)       \\
                 & = (ac - bd, ad + bc)
        \end{align*}
        \begin{align*}
          z_2z_1 & = (c, d)(a, b)       \\
                 & = (ca - db, cb + da)
        \end{align*}

        因为
        \begin{align*}
          ac - bd & = ca - db \\
          ad + bc & = cb + da
        \end{align*}

        于是,由定义15.6.2中关于相等的定义可知,
        \begin{align*}
          z_1z_2 = z_2z_1
        \end{align*}

  \item (b) 结合性:$(z_1z_2)z_3 = z_1(z_2z_3)$。

        因为
        \begin{align*}
          (z_1z_2)z_3 & = ((a, b)(c, d))(e, f)                             \\
                      & = (ac - bd, ad + bc)(e, f)                         \\
                      & = ((ac-bd)e - (ad + bc)f, (ac - bd)f + (ad + bc)e) \\
                      & = (ace - bde - adf - bcf, acf - bdf + ade + bce)
        \end{align*}
        \begin{align*}
          z_1(z_2z_3) & = (a, b)((c, d)(e, f))                               \\
                      & = (a, b)(ce - df, cf + de)                           \\
                      & = (a(ce - df) - b(cf + de), a(cf + de) + b(ce - df)) \\
                      & = (ace - adf - bcf - bde, acf + ade + bce - bdf)
        \end{align*}

        于是,由定义15.6.2中关于相等的定义可知,
        \begin{align*}
          (z_1z_2)z_3 = z_1(z_2z_3)
        \end{align*}

  \item (c) 恒等性:$z_11_\mathbb{C} = 1_\mathbb{C}z_1 = z_1$。

        由(a)可知
        \begin{align*}
          z_11_\mathbb{C} = 1_\mathbb{C}z_1
        \end{align*}

        又有
        \begin{align*}
          z_11_\mathbb{C} & = (a, b)(1, 0)                                    \\
                          & = (a\times 1 - b \times 0, a\times 0 + b\times 1) \\
                          & = (a, b)                                          \\
                          & = z_1
        \end{align*}

  \item (d) 分配性:$z_1(z_2 + z_3) = z_1z_2 + z_1z_3$和$(z_2 + z_3)z_1 = z_2z_1 + z_3z_1$。

        因为
        \begin{align*}
          z_1(z_2 + z_3) & = (a, b)((c, d) + (e, f))                    \\
                         & = (a, b)(c + e, d + f)                       \\
                         & = (a(c + e) - b(d + f), a(d + f) + b(c + e)) \\
                         & = (ac + ae - bd - bf, ad + af + bc + be)
        \end{align*}
        \begin{align*}
          z_1z_2 + z_1z_2 & = (a, b)(c, d) + (a, b)(e, f)             \\
                          & = (ac - bd, ad + bc) + (ae - bf, af + be) \\
                          & = (ac - bd + ae - bf, ad + bc + af + be)
        \end{align*}

        于是,由定义15.6.2中关于相等的定义可知,
        \begin{align*}
          z_1(z_2 + z_3) = z_1z_2 + z_1z_3
        \end{align*}

        同理可得,
        \begin{align*}
          (z_2 + z_3)z_1 = z_2z_1 + z_3z_1
        \end{align*}

\end{itemize}

\section*{15.6.3}

这个引理是想说明:形式符号$z = (a, b)$与$z = a + bi$是等价的。

因为
\begin{align*}
  a + bi & = (a, 0) + (b, 0)(0, 1) \\
         & = (a, 0) + (0, b)       \\
         & = (a, b)
\end{align*}
从而,$a + bi$与$(a, b)$就是一回事。

\section*{15.6.4}

设$z = a + bi, w = c + di$。

\begin{itemize}
  \item $\overline{z + w} = \overline{z} + \overline{w}$。

        因为
        \begin{align*}
          z + w = a + bi + c + di & = a + c + (b + d)i
        \end{align*}
        于是
        \begin{align*}
          \overline{z + w} = a + c - (b + d)i
        \end{align*}

        又
        \begin{align*}
          \overline{z} + \overline{w} & = a - bi + c - di  \\
                                      & = a + c - (b + d)i
        \end{align*}

        所以,$\overline{z + w} = \overline{z} + \overline{w}$。

  \item $\overline{-z} = - \overline{z}$。

        \begin{align*}
          \overline{-z} & = \overline{-a - bi} \\
                        & = -a + bi
        \end{align*}
        \begin{align*}
          - \overline{z} & = -(a - bi) \\
                         & = -a + bi
        \end{align*}

        所以,$\overline{-z} = - \overline{z}$。

  \item $\overline{zw} = \overline{z} \; \overline{w}$。

        因为
        \begin{align*}
          \overline{zw} & = \overline{(a + bi)(c + di)}       \\
                        & = \overline{(ac - bd) + (ad + bc)i} \\
                        & = (ac - bd) - (ad + bc)i
        \end{align*}

        \begin{align*}
          \overline{z} \; \overline{w} & = \overline{a + bi}\overline{c + di} \\
                                       & = (a - bi)(c - di)                   \\
                                       & = ac - bd - (ad + bc)i
        \end{align*}

        所以,$\overline{zw} = \overline{z} \; \overline{w}$。

  \item $\overline{\overline{z}} = z$。

        \begin{align*}
          \overline{\overline{z}} & = \overline{\overline{a + bi}} \\
                                  & = \overline{a - bi}            \\
                                  & = a + bi                       \\
                                  & = z
        \end{align*}

  \item $\overline{z} = \overline{w}$当且仅当$z = w$。

        \begin{itemize}
          \item $\Rightarrow$

                \begin{align*}
                  \overline{z} & = \overline{w} \\
                  a - bi       & = c - di
                \end{align*}
                于是,$a = c$且$-b = -d$,即$a = c$且$b = d$。
                所以$z = w$。

          \item $\Leftarrow$

                $z = w$,所以$a = c$且$b = d$。
                \begin{align*}
                  \overline{z} & = a - bi \\
                  \overline{w} & = c - di
                \end{align*}
                于是$a = c$且$-b = -d$,
                所以$\overline{z} = \overline{w}$。
        \end{itemize}

  \item $\overline{z} = z$当且仅当$z$是一个实数。
        \begin{itemize}
          \item $\Rightarrow$

                $\overline{z} = z$,那么$-b = b$,所以$b = 0$,即$\mathfrak{I}(z) = 0$,
                所以$z$是一个实数。

          \item $\Leftarrow$

                $z$是一个实数,$\mathfrak{I}(z) = 0$,于是$b = 0$,
                所以$\overline{z} = z$。
        \end{itemize}

\end{itemize}

\section*{15.6.5}

设$z = a + bi$,
所以$\mathfrak{R} (z) = a, \mathfrak{I}(z) = b$。

又因为
\begin{align*}
  \frac{z + \overline{z}}{2} & = \frac{a + bi + a - bi}{2} \\
                             & = \frac{2a}{2}              \\
                             & = a                         \\
                             & = \mathfrak{R} (z)
\end{align*}

\begin{align*}
  \frac{z - \overline{z}}{2i} & = \frac{a + bi - (a - bi)}{2i} \\
                              & = \frac{b2i}{2i}               \\
                              & = b                            \\
                              & = \mathfrak{I}(z)
\end{align*}

\section*{15.6.6}

设$z = a + bi, w = c + di$。

\begin{itemize}
  \item 恒等式$z\overline{z} = |z|^2$,从而有$|z| = \sqrt{z \overline{z}}$。

        \begin{align*}
          z \overline{z} & = (a + bi)(a - bi)         \\
                         & = a^2 - b^2i^2 - abi + abi \\
                         & = a^2 - b^2i^2             \\
                         & = a^2 + b^2
        \end{align*}

        又因为
        \begin{align*}
          |z|^2 & = \sqrt{a^2 + b^2}^2 \\
                & = a^2 + b^2
        \end{align*}

        于是,$z\overline{z} = |z|^2$,从而有$|z| = \sqrt{z \overline{z}}$。

  \item $|zw| = |z||w|$且$|\overline{z}| = |z|$。

        由之前的讨论可得,
        \begin{align*}
          |zw| & = \sqrt{zw\overline{zw}}              \\
               & = \sqrt{zw\overline{z}\;\overline{w}} \\
               & = \sqrt{z\overline{z}w\overline{w}}   \\
               & = \sqrt{|z|^2|w|^2}                   \\
               & = |z||w|
        \end{align*}

        $|\overline{z}| = |z|$直接可以从复数绝对值定义中得到。

  \item 不等式$-|z| \leq \mathfrak{R}(z) \leq |z|$。

        因为
        \begin{align*}
          |z|^2 = a^2 + b^2 \geq a^2
        \end{align*}

        由引理5.6.9(d)(更准确的说是实数版本)可得
        \begin{align*}
          |z| \geq |a| \geq \mathfrak{R}(z) = a
        \end{align*}

        于是,
        \begin{align*}
          -|z| \leq -|a| \leq a = \mathfrak{R}(z)
        \end{align*}

  \item 不等式$-|z| \leq \mathfrak{I}(z) \leq |z|$。

        与上一个不等式证明方式一致。

  \item 不等式$|z| \leq |\mathfrak{R}| + |\mathfrak{I}|$。

        因为
        \begin{align*}
          (|\mathfrak{R}| + |\mathfrak{I}|)^2
           & = (|a| + |b|)^2        \\
           & = a^2 + b^2 + 2|ab|    \\
           & \geq a^2 + b^2 = |z|^2
        \end{align*}

        由引理5.6.9(d)(更准确的说是实数版本)可得
        \begin{align*}
          |z| \leq |\mathfrak{R}| + |\mathfrak{I}|
        \end{align*}

  \item 三角不等式$|z + w| \leq |z| + |w|$。

        按照书中的提示进行证明。

        由之前的命题可得,
        \begin{align*}
          \mathfrak{R}(z\overline{w}) \leq |z\overline{w}| = |z||\overline{w}| = |z||w|
        \end{align*}

        于是,
        \begin{align*}
          \mathfrak{R}(z\overline{w}) \leq |z||w|
        \end{align*}

        利用习题15.6.5可得,
        \begin{align*}
          \mathfrak{R}(z\overline{w}) & = \frac{z\overline{w} + \overline{z\overline{w}}}{2} \\
                                      & = \frac{z\overline{w} + \overline{z}w}{2}
        \end{align*}

        于是,
        \begin{align*}
          \frac{z\overline{w} + \overline{z}w}{2} & \leq |z||w|  \\
          z\overline{w} + \overline{z}w           & \leq 2|z||w|
        \end{align*}

        然后,不等式两端加上$|z|^2 + |w|^2$,
        \begin{align*}
          z\overline{w} + \overline{z}w + |z|^2 + |w|^2 \leq 2|z||w| + |z|^2 + |w|^2        \\
          z\overline{w} + \overline{z}w + z\overline{z} + w\overline{w}  \leq (|z| + |w|)^2 \\
          (z + w)(\overline{z + w}) \leq (|z| + |w|)^2                                      \\
          |z + w|^2 \leq (|z| + |w|)^2                                                      \\
          |z + w| \leq |z| + |w|
        \end{align*}
\end{itemize}

\section*{15.6.7}

注意:实数也是复数,所以复数的相关性质,实数也具备。

因为
\begin{align*}
  |z/w| & = |zw^{-1}|                                \\
        & = \left|z|w|^{-2}\overline{w}\right|       \\
        & = \left|\frac{z\overline{w}}{|w|^2}\right| \\
        & = |\frac{1}{|w|^2}| |z\overline{w}|        \\
        & = \frac{1}{|w|^2} |z| |\overline{w}|       \\
        & = \frac{|\overline{w}|}{|w|^2} |z|         \\
        & = \frac{|z|}{|w|}
\end{align*}

\section*{15.6.8}

注意:实数也是复数,所以复数的相关性质,实数也具备。

\begin{itemize}
  \item $\Rightarrow$




  \item $\Leftarrow$

        \begin{align*}
          |z + w| & = |cw + w|    \\
                  & = |(c + 1) w| \\
                  & = |c + 1| |w|
        \end{align*}
        \begin{align*}
          |z| + |w| & = |cw| + |w|    \\
                    & = |c| |w| + |w| \\
                    & = |c + 1| |w|
        \end{align*}

        所以$|z + w| = |z| + |w|$。
\end{itemize}

\section*{15.6.9}

$z = \mathfrak{R}(z) + \mathfrak{I}(z) i$。

\begin{itemize}
  \item $\Rightarrow$

        (1)$\lim\limits_{n \to \infty} \mathfrak{R} (z_n) = \mathfrak{R}(z)$。

        反证法,假设$\lim\limits_{n \to \infty} \mathfrak{R} (z_n)$不收敛于$\mathfrak{R}(z)$。

        于是存在$\epsilon$,不存在$N > 0$,使得只要$n \geq N$,就有
        \begin{align*}
          |\mathfrak{R} (z_n) - \mathfrak{R} (z) | < \epsilon
        \end{align*}
        即对所有的$n$都有
        \begin{align*}
          |\mathfrak{R} (z_n) - \mathfrak{R} (z) | \geq \epsilon
        \end{align*}

        又因为$\lim\limits_{n \to \infty} z_n = z$,所以对
        $\epsilon > 0$,存在$N > 0$,使得只要$n \geq N$,都有
        \begin{align*}
           & |z_n - z| < \epsilon                                                                                  \\
           & \implies                                                                                              \\
           & \sqrt{(\mathfrak{R} (z_n) - \mathfrak{R}(z))^2 + (\mathfrak{I} (z_n) - \mathfrak{I}(z))^2} < \epsilon
        \end{align*}

        因为
        \begin{align*}
          \sqrt{(\mathfrak{R} (z_n) - \mathfrak{R}(z))^2 + (\mathfrak{I} (z_n) - \mathfrak{I}(z))^2}
          > |\mathfrak{R} (z_n) - \mathfrak{R} (z) | \geq \epsilon
        \end{align*}

        存在矛盾。

        所以,$\lim\limits_{n \to \infty} \mathfrak{R} (z_n) = \mathfrak{R}(z)$。


        (2)$\lim\limits_{n \to \infty} \mathfrak{I} (z_n) = \mathfrak{I}(z)$。

        证明类似,不做赘述。

  \item $\Leftarrow$

        对任意$\epsilon > 0$。

        因为$\lim\limits_{n \to \infty} \mathfrak{R} (z_n) = \mathfrak{R}(z)$,
        存在$N_1 > 0$,使得只要$n \geq N_1$,都有
        \begin{align*}
          |\mathfrak{R} (z_n) - \mathfrak{R} (z) | < \frac{1}{\sqrt{2}}\epsilon
        \end{align*}

        同理,因为$\lim\limits_{n \to \infty} \mathfrak{I} (z_n) = \mathfrak{I}(z)$,
        存在$N_2 > 0$,使得只要$n \geq N_2$,都有
        \begin{align*}
          |\mathfrak{I} (z_n) - \mathfrak{I} (z) | < \frac{1}{\sqrt{2}}\epsilon
        \end{align*}

        取$N = max(N_1, N_2)$,使得只要$n \geq N$,都有
        \begin{align*}
          |z_n - z|
           & =  \sqrt{(\mathfrak{R} (z_n) - \mathfrak{R}(z))^2 + (\mathfrak{I} (z_n) - \mathfrak{I}(z))^2} \\
           & < \epsilon
        \end{align*}

        由$\epsilon$的任意性可得,$\lim\limits_{n \to \infty} z_n = z$。

\end{itemize}

\section*{15.6.10}

由定义12.4.9(完备度量空间)可知,
我们需要证明,度量空间$(\mathbb{C}, d)$中的任意$(z_n)_{n = 1}^\infty$柯西序列都有极限。

对任意$\epsilon > 0$,存在$N > 0$,使得只要$p, q \geq N$,都有
\begin{align*}
  |z_p - z_q| < \epsilon \\
  \implies               \\
  \sqrt{(\mathfrak{R} (z_p) - \mathfrak{R}(z_q))^2 + (\mathfrak{I} (z_p) - \mathfrak{I}(z_q))^2} < \epsilon
\end{align*}
所以,
\begin{align*}
  \sqrt{(\mathfrak{R} (z_p) - \mathfrak{R}(z_q))^2} < \epsilon \\
  \implies                                                     \\
  |\mathfrak{R} (z_p) - \mathfrak{R}(z_q)| < \epsilon
\end{align*}

所以$(\mathfrak{R}(z_n))_{n = 1}^\infty$是柯西序列,
由于$\mathfrak{z_n}$都是实数,实数度量空间是完备的,
所以,$(\mathfrak{R}(z_n))_{n = 1}^\infty$是收敛序列。

同理可得,$(\mathfrak{I}(z_n))_{n = 1}^\infty$是收敛序列。

综上,利用引理15.6.13可得,$(z_n)_{n = 1}^\infty$收敛。

\section*{15.6.11}

\begin{itemize}
  \item $f$是双射。

        证明$f$是单射和双射即可,证明略。

  \item $f, f^{-1}$都是连续函数。

        (1)$f$是连续函数。

        任意$(x_0, y_0) \in \mathbb{R}^2$,
        设$(x^{(n)})_{n = 1}^\infty$是$\mathbb{R}^2$中的序列,
        其中$x^{(n)} := (x_n, y_n)$,
        并且序列收敛于$(x_0, y_0)$,
        我们需要证明$(f(x^{n}))_{n = 1}^\infty$即
        $(z_n)_{n = 1}^\infty$其中$z_n = x_n + y_n i$收敛于$f(x_0, y_0) = x_0 + y_0 i$。

        由命题12.1.18可知,
        \begin{align*}
          \lim\limits_{n \to \infty} x_n = x_0 \\
          \lim\limits_{n \to \infty} y_n = y_0
        \end{align*}

        由引理15.6.13可知,
        $(z_n)_{n = 1}^\infty$收敛于$f(x_0, y_0)$。

        (2)$f^{-1}$是连续函数。

        证明方式与(1)类似,证明略。

\end{itemize}


\section*{15.6.12}

我们构造出这样一个通路:

对任意$z_0, z_1 \in \mathbb{C}$,定义$\gamma (x) = (1 - x)z_0 + z_1 x$,其中$x \in [0, 1]$。

以上定义的通道满足:

\begin{itemize}
  \item $\gamma (0) = z_0, \gamma (1) = z_1$。
  \item $\gamma$是连续的。(可以直接使用引理15.6.14证明。)
\end{itemize}

道路连通蕴含连通性,所以,$\mathbb{C}$是连通的。

\section*{15.6.13}


\begin{itemize}
  \item (1)证明:$E$是紧致的,当且仅当$E$既是闭的又是有界的。
        \begin{itemize}
          \item $\Rightarrow$

                利用推论 12.5.6 可证。

          \item $\Leftarrow$
                设$(z_n)_{n = 1}^\infty$是$E$中的任意序列。

                按照习题15.6.11中定义的$f^{-1}$,我们有$E^\prime = f^{-1}(E), E^\prime \subseteq \mathbb{R}$。

                因为$\mathbb{C}$的通常度量$d$是与欧几里得度量是一致的,
                所以,$E$是闭的又是有界的,那么,$E^\prime$也是闭的又是有界的。

                所以,$(f^{-1}(z_n))_{n = 1}^\infty$存在收敛的$R^2$上的子序列$(a_n)_{n = 1}^\infty$,
                其对应的复数序列为$(f(a_n))_{n = 1}^\infty$。

                由命题12.1.18(d)可知,我们有以下极限存在,
                \begin{align*}
                  \lim\limits_{n \to \infty} \mathfrak{R} (f(a_n)) \\
                  \lim\limits_{n \to \infty} \mathfrak{I} (f(a_n))
                \end{align*}

                所以,由引理15.6.13可知,$\lim\limits_{n \to \infty} f(a_n)$极限存在。

                综上,由$(z_n)_{n = 1}^\infty$的任意性可得, $E$是紧致的。
        \end{itemize}

  \item (2)$\mathbb{C}$不是紧致的。

        $\mathbb{C}$的子集$\mathbb{R}$都不是紧致的,$\mathbb{C}$不可能是紧致的。

        或者直接举一个反例$(a_n)_{n = 1}^\infty$,其中$a_n = n$,这个序列就没有收敛的子序列。

\end{itemize}

\section*{15.6.14}

只证明其中的某几个。

\begin{itemize}
  \item $(z_n + w_n)_{n = 1}^\infty$收敛,并且$\lim\limits_{n \to \infty} z_n + w_n = \lim\limits_{n \to \infty} z_n + \lim\limits_{n \to \infty} w_n$。

        因为$(z_n)_{n = 1}^\infty$是收敛的复数序列,由引理15.6.13可知,以下极限存在
        \begin{align*}
          \lim\limits_{n \to \infty} \mathfrak{R}(z_n) \\
          \lim\limits_{n \to \infty} \mathfrak{I}(z_n)
        \end{align*}

        同理可得,以下极限存在
        \begin{align*}
          \lim\limits_{n \to \infty} \mathfrak{R}(w_n) \\
          \lim\limits_{n \to \infty} \mathfrak{I}(w_n)
        \end{align*}

        又因为,对任意$n$都有,
        \begin{align*}
          \mathfrak{R}(z_n + w_n) = \mathfrak{R}(z_n) + \mathfrak{R}(w_n)
        \end{align*}

        于是由极限定理(引理6.1.19)可知,
        \begin{align*}
          \lim\limits_{n \to \infty} \mathfrak{R}(z_n + w_n) = \lim\limits_{n \to \infty} \mathfrak{R}(z_n) + \lim\limits_{n \to \infty} \mathfrak{R}(w_n)
        \end{align*}

        同理可得,
        \begin{align*}
          \lim\limits_{n \to \infty} \mathfrak{I}(z_n + w_n) = \lim\limits_{n \to \infty} \mathfrak{I}(z_n) + \lim\limits_{n \to \infty} \mathfrak{I}(w_n)
        \end{align*}

        再次利用15.6.13可知,$(z_n + w_n)_{n = 1}^\infty$收敛,并且
        \begin{align*}
          \lim\limits_{n \to \infty} z_n + w_n = z
        \end{align*}
        其中,$\mathfrak{R}(z) = \lim\limits_{n \to \infty} \mathfrak{R}(z_n) + \lim\limits_{n \to \infty} \mathfrak{R}(w_n),
          \mathfrak{I}(z) = \lim\limits_{n \to \infty} \mathfrak{I}(z_n) + \lim\limits_{n \to \infty} \mathfrak{I}(w_n)$,
        所以,
        \begin{align*}
          \lim\limits_{n \to \infty} z_n + w_n = \lim\limits_{n \to \infty} z_n + \lim\limits_{n \to \infty} w_n
        \end{align*}


  \item $(\overline{z_n})_{n = 1}^\infty$收敛,并且$\lim\limits_{n \to \infty} \overline{z_n} = \overline{\lim\limits_{n \to \infty} z_n}$。

        不妨设$(z_n)_{n = 1}^\infty$收敛于$z$,即$\overline{\lim\limits_{n \to \infty} z_n} = z$,
        于是
        \begin{align*}
          \lim\limits_{n \to \infty} \mathfrak{R}(z_n) = \mathfrak{R}(z) \\
          \lim\limits_{n \to \infty} \mathfrak{I}(z_n) = \mathfrak{I}(z)
        \end{align*}

        对每一个$n$,都有
        \begin{align*}
          \overline{z_n} = \mathfrak{R}(z_n) - \mathfrak{I}(z_n) i
        \end{align*}

        于是由极限定理(引理6.1.19)可知,
        \begin{align*}
          \lim\limits_{n \to \infty} -\mathfrak{I}(z_n) = -\mathfrak{I}(z)
        \end{align*}

        综上,利用引理15.6.13可知,
        \begin{align*}
          \lim\limits_{n \to \infty} \overline{z_n} = \overline{\lim\limits_{n \to \infty} z_n}
        \end{align*}
\end{itemize}

\section*{15.6.15}

按照书中的提示进行证明。

(1)首先,$1 \neq 0$。假设$1$是负数。

由负运算公理可知$-1$是正的,那么,由可乘性可得$-1 \times -1 = 1$是正数,
与假设矛盾。

综上可得,$1$是正数,从而$-1$是负数。

(2)不一致的证明(由公理集合推演得到一个矛盾的结果)。



$z = i$,$z \neq 0$,由三歧性可知,$i$要么是负数,要么是正数。

如果$i$是正数,那么,由可乘性公理得$i^2 = -1$是正数,这与(1)矛盾。

如果$i$是负数,那么,由负运算可得$-i$是正数,由可乘性公理得$(-i)^2 = -1$是正数,这与(1)矛盾。

综上,公理间存在矛盾。

\section*{15.6.16}

\begin{itemize}
  \item (1)叙述并证明关于复数的比值判别法。

        \begin{itemize}
          \item 比值判别法:设$\sum\limits_{n = m}^\infty z_n$是一个所有项都不为零的复数级数
                (不为零的假设是为了保证下文中的比值$|z_{n + 1}|/|z_n|$是有意义的)。

                \begin{itemize}
                  \item 如果$\lim\sup\limits_{n \to \infty} \frac{|z_{n + 1}|}{|z_n|} < 1$,那么
                        级数$\sum\limits_{n = m}^\infty z_n$是绝对收敛的(从而是条件收敛的。)
                  \item 如果$\lim\sup\limits_{n \to \infty} \frac{|z_{n + 1}|}{|z_n|} > 1$,那么
                        级数$\sum\limits_{n = m}^\infty z_n$不是条件收敛的(从而不可能是绝对收敛的)。
                  \item 在其他情况下,我们无法给出任何结论。
                \end{itemize}

          \item 证明

                (1)如果$\lim\sup\limits_{n \to \infty} \frac{|z_{n + 1}|}{|z_n|} < 1$。

                因为$|z_n|$是实数,
                于是利用推论7.5.3(比值判别法)可知,级数$\sum\limits_{n = m}^\infty z_n$是绝对收敛的。

                接下来,证明绝对收敛的复数序列必定条件收敛。

                设$(a_n)_{n = 1}^\infty$是绝对收敛的复数序列,其中$a_n = x_n + y_n i$。
                由于$|a_n| = \sqrt{x_n^2 + y_n^2}$,于是
                \begin{align*}
                  |x_n| \leq |a_n| \\
                  |y_n| \leq |a_n|
                \end{align*}

                于是可得,$(x_n)_{n = 1}^\infty$和$(y_n)_{n = 1}^\infty$这两个实数序列都是绝对收敛的,
                所以,$\lim\limits_{n \to \infty} x_n$和$\lim\limits_{n \to \infty} y_n$都存在,
                由引理15.6.13可知,以下极限存在,
                \begin{align*}
                  \lim\limits_{n \to \infty} a_n
                \end{align*}

                综上可得,绝对收敛的复数序列必定条件收敛。

                所以,级数$\sum\limits_{n = m}^\infty z_n$是绝对收敛的,从而是条件收敛的。


                (2)如果$\lim\sup\limits_{n \to \infty} \frac{|z_{n + 1}|}{|z_n|} > 1$。

                所以,存在$c > 1$和对应的$N \geq m$,使得$n \geq N$,就有
                \begin{align*}
                  |z_{n + 1}| > |z_N|c^{n + 1 - N}
                \end{align*}

                所以,$n \to +\infty$时,$|z_n| \to +\infty$,
                所以$(z_n)_{n = m}^\infty$是发散的,
                利用7.2.6(零判别法)(更准确的说法是复数版本)
                级数$\sum\limits_{n = m}^\infty z_n$是发散的。

                (3)直接利用习题7.5.3即可,把实数级数看做复数级数的特例。
        \end{itemize}

  \item (2)利用比值定理证明对任意的$z$,$\exp(z)$都是收敛的。

        因为,对任意复数$z$,都有
        \begin{align*}
          \lim\sup\limits_{n \to \infty}\frac{\frac{|z^{n+1}|}{(n+1)!}}{\frac{|z^{n}|}{n!}}
           & = \lim\sup\limits_{n \to \infty}\frac{|z^{n+1}|}{|z^{n}|} \frac{1}{n+1} \\
           & = \lim\sup\limits_{n \to \infty}\frac{|z^{n}||z|}{|z^{n}|}\frac{1}{n+1} \\
           & = \lim\sup\limits_{n \to \infty}\frac{|z|}{n + 1}                       \\
           & = 0 < 1
        \end{align*}

        于是利用复数级数的比值判别法可知,$\exp(z)$绝对收敛,从而是条件收敛的。

  \item (3)$\exp(z + w) = \exp(z) \exp(w)$。

  
\end{itemize}






\end{document}