\documentclass{article}
\usepackage{mathtools} 
\usepackage{fontspec}
\usepackage[UTF8]{ctex}
\usepackage{amsthm}
\usepackage{mdframed}
\usepackage{xcolor}
\usepackage{amssymb}
\usepackage{amsmath}


% 定义新的带灰色背景的说明环境 zremark
\newmdtheoremenv[
  backgroundcolor=gray!10,
  % 边框与背景一致,边框线会消失
  linecolor=gray!10
]{zremark}{说明}


\begin{document}
\title{15.7 注释}
\author{张志聪}
\maketitle

\section*{1}

\begin{zremark}
  对所有的$x > 0$,都有$cot^\prime(x) \leq -1$,那么由微积分基本定理可知,对所有的$x > 0$和$s > 0$,都有$cot(x + s) \leq cot(x) - s$。
  以上命题是如何证明。
\end{zremark}
\textbf{证明:}

利用定理11.9.4(微积分第二基本定理)
\begin{align*}
  \int_{x, x+s} cot^\prime(t) dt \leq \int_{[x, x+s]} -1 dt \\
  \implies                                                  \\
  cot(x + s) - cot(x) \leq x + s - x                        \\
  \implies                                                  \\
  cot(x + s) \leq cot(x) - s
\end{align*}

\section*{2}

\begin{zremark}
  设$E$是集合$E := \{x \in (0, +\infty) : sin(x) = 0\}$,因为$sin$在$[c, +\infty)$上是连续的,
  所以$E$在$[c, +\infty)$上是闭的。
\end{zremark}
\textbf{证明:}

函数$sin : [c, +\infty) \to sin([c, +\infty))$。
因为$\{0\}$是$sin([c, +\infty))$中的闭集,那么集合
$sin^{-1}(0) := \{x \in [c, +\infty) : sin(x) = 0\}$(这里$E = sin^{-1}(0)$)
就是$[c, +\infty)$中的闭集。

\section*{3}

\begin{zremark}
  $E$包含了它的全体附着点,从而就包含了$inf(E)$。
\end{zremark}
\textbf{证明:}

因为$inf(E)$就是$E$的附着点。

\end{document}