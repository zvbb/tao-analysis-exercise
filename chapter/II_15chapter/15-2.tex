\documentclass{article}
\usepackage{mathtools} 
\usepackage{fontspec}
\usepackage[UTF8]{ctex}
\usepackage{amsthm}
\usepackage{mdframed}
\usepackage{xcolor}
\usepackage{amssymb}
\usepackage{amsmath}


% 定义新的带灰色背景的说明环境 zremark
\newmdtheoremenv[
  backgroundcolor=gray!10,
  % 边框与背景一致,边框线会消失
  linecolor=gray!10
]{zremark}{说明}


\begin{document}
\title{15.2 习题}
\author{张志聪}
\maketitle

\section*{15.2.1}

\begin{itemize}
  \item $0 \leq k \leq n$时。

        对$k$采用归纳法。

        归纳基始$k = 0$,$c\frac{n!}{(n - 0)!}(x - a)^{n - 0} = c(x - a)^n = f(x)$,
        这与$f^{(0)}(x) = f(x)$一致。

        归纳假设$k = j$是命题成立。

        $k = j + 1$时,
        \begin{align*}
          f^{(j + 1)}(x)
           & = (f^{(j)}(x))^\prime                                    \\
           & = (c\frac{n!}{(n - j)!}(x - a)^{n - j})^\prime           \\
           & = c\frac{n!}{(n - j)!}(n - j)(x - a)^{n - j - 1}         \\
           & = c\frac{n!}{(n - (j + 1))!}(n - j)(x - a)^{n - (j + 1)} \\
        \end{align*}

  \item $k > n$时。

        由之前的讨论可知,
        \begin{align*}
          f^{(n)}(x)
           & = c\frac{n!}{(n - n)!}(x - a)^{n - n} \\
           & = cn!
        \end{align*}

        于是可得$k > n$时,$f^{(k)}(x) = 0$。

\end{itemize}

\section*{15.2.2}
对任意$a \in \mathbb{R} \setminus \{1\}$,
定义级数如下
\begin{align*}
  \sum\limits_{n = 0}^\infty (\frac{1}{1-a})^{n + 1} (x - a)^n
\end{align*}

接下来我们证明该级数一致收敛于某个函数,并逐点收敛于$f$,由一致收敛与逐点收敛的函数相同,证明该级数一致收敛于$f$。

先计算收敛半径:
\begin{align*}
  \lim\limits_{n \to \infty} |c_n|^\frac{1}{n}
   & = \lim\limits_{n \to \infty} \left(|(\frac{1}{1-a})^{n + 1}|\right)^\frac{1}{n} \\
   & = \lim\limits_{n \to \infty} |\frac{1}{1-a}| (|\frac{1}{1-a}|)^{\frac{1}{n}}    \\
   & = |\frac{1}{1-a}|\lim\limits_{n \to \infty} (|\frac{1}{1-a}|)^{\frac{1}{n}}     \\
   & = |\frac{1}{1-a}| \times 1                                                      \\
   & = |\frac{1}{1-a}|
\end{align*}
$\lim\limits_{n \to \infty} (|\frac{1}{1-a}|)^{\frac{1}{n}} = 1$利用了引理6.5.2。

于是由命题6.4.12(f)可得,
\begin{align*}
  \lim\sup\limits_{n \to \infty} |c_n|^\frac{1}{n} = |\frac{1}{1-a}|
\end{align*}
所以
\begin{align*}
  R = |1 - a|
\end{align*}
于是令$r = \frac{1}{2}|1 - a|$,只需$|x - a| < r$,
即$x \in (a - r, a + r)$时,由定理15.1.6(c)可知,
级数$\sum\limits_{n = 0}^\infty (\frac{1}{1-a})^{n + 1} (x - a)^n$在$[a - r, a + r]$上一致收敛
于某个函数$g$,因为$(a - r, a + r) \subseteq [a - r, a + r]$,
所以级数$\sum\limits_{n = 0}^\infty (\frac{1}{1-a})^{n + 1} (x - a)^n$
在$(a - r, a + r)$上也一致收敛于$g$。

对任意$x \in (a - r, a + r)$,于是可得
\begin{align*}
  |\frac{x - a}{1-a}| < 1
\end{align*}
于是利用引理7.3.3可得
\begin{align*}
  \sum\limits_{n = 0}^\infty (\frac{1}{1-a})^{n + 1} (x - a)^n
   & = \frac{1}{1-a} \sum\limits_{n = 0}^\infty (\frac{x - a}{1-a})^{n} \\
   & = \frac{1}{1-a} \frac{1}{1 - \frac{x - a}{1-a}}                    \\
   & = \frac{1}{1 - x}
\end{align*}

于是可得级数逐点收敛于$f$。

综上,级数$\sum\limits_{n = 0}^\infty (\frac{1}{1-a})^{n + 1} (x - a)^n$在$(a - 1, a + 1)$上也一致收敛
于$f$(注14.2.8中有阐述)。

由$a$的任意性可知,命题成立。

\section*{15.2.3}
注意:命题中的$r \leq R$($R$为级数$\sum\limits_{n = 1}^\infty nc_n (x - a)^{n - 1}$的收敛半径)

对$k$进行归纳。

(1)归纳基始$k = 1$,由定理15.1.6(d)可知,
级数$\sum\limits_{n = 1}^\infty nc_n (x - a)^{n - 1}$
在区间$(a - r, a + r)$上一致收敛$f^\prime$,
即
\begin{align*}
  f^\prime(x) & = \sum\limits_{n = 1}^\infty nc_n (x - a)^{n - 1}         \\
              & = \sum\limits_{n = 0}^\infty c_{n + 1}(n + 1) (x - a)^{n}
\end{align*}
命题成立。

(2)归纳假设$k = j$时,命题成立,函数$f$在$(a - r, a + r)$上都是$j$次可微的,
并且$j$次导函数由下式给出:
\begin{align*}
  f^{(j)}(x) = \sum\limits_{n = 0}^\infty c_{n + j}(n+1)(n+2)...(n + j)(x - a)^n
   & = \sum\limits_{n = 0}^\infty c_{n+j}\frac{(n+j)!}{n!}(x - a)^n
\end{align*}

(3)$k = j+1$时。
\begin{align*}
  \lim\sup\limits_{n \to \infty} |c_{n+j}\frac{(n+j)!}{n!}|^{\frac{1}{n}}
   & = \lim\sup\limits_{n \to \infty} |c_{n+j}|^{\frac{1}{n}}|\frac{(n+j)!}{n!}|^{\frac{1}{n}}                                 \\
   & = \lim\sup\limits_{n \to \infty} |c_{n+j}|^{\frac{1}{n}} \lim\sup\limits_{n \to \infty} |\frac{(n+j)!}{n!}|^{\frac{1}{n}} \\
   & = \lim\sup\limits_{n \to \infty} |c_{n+j}|^{\frac{1}{n}} \times 1                                                         \\
   & = \lim\sup\limits_{n \to \infty} |c_{n+j}|^{\frac{1}{n}} = R
\end{align*}

再次利用定理15.1.6(d)可知,
函数$f^{(j)}$在$(a - R, a + R)$上可微,
级数$\sum\limits_{n = 1}^\infty nc_{n+j}\frac{(n+j)!}{n!}(x - a)^{n - 1}$在
区间$(a - r, a + r)$上一致收敛于$(f^{(j)})^\prime = f^{(j+1)}$。

又因为
\begin{align*}
   & \sum\limits_{n = 1}^\infty nc_{n+j}\frac{(n+j)!}{n!}(x - a)^{n - 1}   \\
   & = \sum\limits_{n = 0}^\infty nc_{n+1+j}\frac{(n+1+j)!}{n!}(x - a)^{n}
\end{align*}

综上可得
\begin{align*}
  f^{(j+1)} = \sum\limits_{n = 0}^\infty nc_{n+1+j}\frac{(n+1+j)!}{n!}(x - a)^{n}
\end{align*}

命题成立,归纳完成。

\section*{15.2.4}

由命题15.2.6可知,
\begin{align*}
  f^{(k)}(x) = \sum\limits_{n = 0}^\infty c_{n + k} \frac{(n + k)!}{n!}(x - a)^{n}
\end{align*}
当$x = a$时,如果$n > 0$,则$(x - a)^n = 0^n = 0$,
于是级数
$\sum\limits_{n = 0}^\infty c_{n + k} \frac{(n + k)!}{n!}(x - a)^{n}$
只剩第一项,即
\begin{align*}
  f^{(k)}(a)
   & = \sum\limits_{n = 0}^\infty c_{n + k} \frac{(n + k)!}{n!}0^{n} \\
   & = c_{0 + k}\frac{(0 + k)!}{0!}0^{0}                             \\
   & = c_{k}\frac{k!}{1} \times 1                                    \\
   & = c_{k}k!
\end{align*}

于是可得,级数$\sum\limits_{n = 0}^\infty c_n(x - a)^n$的系数为
\begin{align*}
  c_{n} = \frac{f^{(n)}(a)}{n!}
\end{align*}

综上可得
\begin{align*}
  f(x) = \sum\limits_{n = 0}^\infty \frac{f^{(n)}(a)}{n!}(x - a)^n
\end{align*}

\section*{15.2.5}
(1)证明恒等式。

\begin{align*}
  (x - a)^n & = ((x - b) + (b - a))^n                                                \\
            & = \sum\limits_{m = 0}^n \frac{n!}{m!(n - m)!} (b - a)^{n - m}(x - b)^m
\end{align*}
注意:第二个等式使用了二项式公式,即习题7.1.4。

(2)解释这个恒等式为什么与泰勒公式以及习题15.2.1是一致的。(即彼此之间不存在矛盾)

令$f(x) = (x - a)^n$,并设$f$是在$b$处实解析的函数(习题15.2.6保证了这个假设是成立的),
由泰勒公式可知,
\begin{align*}
  f(x) = \sum\limits_{m = 0}^\infty \frac{f^{(m)}(b)}{m!}(x - b)^m
\end{align*}

由习题15.2.1可知,
\begin{align*}
  f^{(m)}(b) = \frac{n!}{(n - m)!}(b - a)^{n - m}
\end{align*}

综上可得,
\begin{align*}
  f(x) & = \sum\limits_{m = 0}^\infty \frac{f^{(m)}(b)}{m!}(x - b)^m            \\
       & = \sum\limits_{m = 0}^n \frac{n!}{m!(n - m)!} (b - a)^{n - m}(x - b)^m
\end{align*}

\section*{15.2.6}

设$g$是一元多项式,所以我们可以找到一个正整数$m \geq 0$和实数$a_0, a_1, \dots, a_k$使得
\begin{align*}
  g(x) & = a_0 + a_1x + a_2x^2 + \dots + a_kx^k
\end{align*}

对任意$b \in \mathbb{R}$,结合习题15.2.5,我们有
\begin{align*}
  x^n = \sum\limits_{m = 0}^n \frac{n!}{m!(n - m)!} b^{n - m}(x - b)^m
\end{align*}

$j \leq k$时,令系数为
\begin{align*}
  c_j = \sum\limits_{n = 0}^k \frac{n!}{m!(n - m)!} b^{n - m}, \;\; m = j \text{且} m \leq n
\end{align*}

于是可以把$g(x)$表示为
\begin{align*}
  g(x) & = a_0 + a_1x + a_2x^2 + \dots + a_mx^m    \\
       & = \sum\limits_{n = 0}^\infty c_n(x - b)^n
\end{align*}
其中,$n > k$时,$c_n = 0$。于是可得$g$在$b$处是实解析的。

由$b$的任意性可知,$g$在$\mathbb{R}$上是实解析的,命题成立。

\section*{15.2.7}

(1)证明恒等式$\frac{r}{r - x} = \sum\limits_{n = 0}^\infty x^nr^{-n}, \;\; r \in (-r, r)$。

因为
\begin{align*}
  x^nr^{-n} = (xr^{-1})^n = (\frac{x}{r})^n
\end{align*}

又
\begin{align*}
  |\frac{x}{r}| < 1
\end{align*}

于是利用引理7.3.3可知,任意$x \in (-r, r)$
\begin{align*}
  \sum\limits_{n = 0}^\infty x^nr^{-n} = \frac{1}{1 - \frac{x}{r}} = \frac{r}{r - x}
\end{align*}

(2)证明恒等式

令$f(x) = \frac{r}{r - x}, x \in (-r, r)$,于是由(1)可得
\begin{align*}
  f(x) = \sum\limits_{n = 0}^\infty x^nr^{-n}, \;\; r \in (-r, r)
\end{align*}
其中幂级数第$n$个系数为$c_n = r^{-n}$。

对$m \geq 0$,我们有(可以通过归纳法证明),
\begin{align*}
  f^{(m)}(x) = \frac{m!r}{(r - x)^{m + 1}}
\end{align*}

又
\begin{align*}
  (\sum\limits_{n = 0}^\infty x^nr^{-n})^{(m)}
   & = \sum\limits_{n = 0}^\infty \frac{(n + m)!}{n!}x^nr^{-(n + m)} \\
   & = \sum\limits_{n = m}^\infty \frac{n!}{(n - m)!}x^{n - m}r^{-n}
\end{align*}

利用命题15.2.6可知,
\begin{align*}
  f^{(m)}(x)                  & = \sum\limits_{n = m}^\infty \frac{n!}{(n - m)!}x^{n - m}r^{-n} \\
                              & \implies                                                        \\
  \frac{m!r}{(r - x)^{m + 1}} & = \sum\limits_{n = m}^\infty \frac{n!}{(n - m)!}x^{n - m}r^{-n}
\end{align*}

即:
\begin{align*}
  \frac{r}{(r - x)^{m + 1}} = \sum\limits_{n = m}^\infty \frac{n!}{m!(n - m)!}x^{n - m}r^{-n}
\end{align*}

(3)证明绝对收敛。

使用推论7.5.3(比值判别法)
\begin{align*}
   & \lim\sup\limits_{n \to \infty} \frac{\frac{(n + 1)!}{m!(n + 1 - m)!}x^{n+1-m}r^{-(n+1)}}{\frac{n!}{m!(n - m)!}x^{n - m}r^{-n}} \\
   & = \lim\sup\limits_{n \to \infty} \frac{(n + 1)}{(n + 1 - m)} xr^{-1}                                                           \\
   & =  \lim\sup\limits_{n \to \infty} (1 + \frac{m}{(n + 1 - m)}) \frac{x}{r}
\end{align*}
由于$\lim\limits_{n \to \infty} 1 + \frac{m}{(n + 1 - m)} = 1$且$\frac{x}{r} < 1$可知,
\begin{align*}
  \lim\sup\limits_{n \to \infty} (1 + \frac{m}{(n + 1 - m)}) \frac{x}{r} < 1
\end{align*}

综上可得
\begin{align*}
  \sum\limits_{n = m}^{\infty} \frac{n!}{m!(n - m)!}x^{n - m}r^{-n}
\end{align*}
绝对收敛。


\section*{15.2.8}

\begin{itemize}
  \item (a)

        因为
        \begin{align*}
          |a - b|     \leq r - s \\
          |a - b| + s  \leq r
        \end{align*}

        反证法,假设$|a - b| \geq r$,于是
        \begin{align*}
          |a - b| + s \geq r + s
        \end{align*}
        因为$s > 0$,所以$|a - b| + s  \leq r$与$|a - b| + s \geq r + s$矛盾,
        假设不成立。

  \item (b)

        收敛半径为:
        \begin{align*}
          R = \frac{1}{\lim\sup\limits_{n \to \infty} |c_n|^\frac{1}{n}}
        \end{align*}

        于是可得
        \begin{align*}
          \lim\sup\limits_{n \to \infty} |c_n|^\frac{1}{n} = \frac{1}{R}
        \end{align*}

        因为$r - \epsilon < R$,所以
        \begin{align*}
          \frac{1}{r - \epsilon} > \frac{1}{R}
        \end{align*}

        于是由上极限的定义可知,存在$N$使得
        \begin{align*}
          C_N^+ \geq \frac{1}{r - \epsilon}
        \end{align*}
        其中$C_N^+ = \sup(|c_n|^\frac{1}{n})_{n = N}^\infty$。

        由上确界的定义可知,对所有的$n \geq N$,都有
        \begin{align*}
          |c_n|^\frac{1}{n} \leq \frac{1}{r - \epsilon} \\
          \implies                                      \\
          |c_n| \leq (r - \epsilon)^{-n}
        \end{align*}

        因为$n < N$是有限的,且由推论5.4.13(阿基米德性质)可知,
        对每一个$n$都存在,正整数$M_n$使得
        \begin{align*}
          |c_n| < M_n (r - \epsilon)^{-n}
        \end{align*}

        取$C := max(M_0, M_2, \cdots, M_N)$(其中$M_N = 1$),我们有
        \begin{align*}
          |c_n| \leq C(r - \epsilon)^{-n}
        \end{align*}

  \item (c)

        由(a)可知$|a - b| < r$,所以存在$0 < \epsilon < 1$使得$|a - b| < r - \epsilon$,
        又由(b)可知,存在$C > 0$使得对所有的正整数$n \geq 0$都有
        \begin{align*}
          |c_n| \leq C(r - \epsilon)^{-n}
        \end{align*}

        于是可得对任意的$n \geq m$都有
        \begin{align*}
          \left| \frac{n!}{m!(n - m)!}(b - a)^{n - m}c_n \right|
           & \leq C \left| \frac{n!}{m!(n - m)!}(b - a)^{n - m}(r - \epsilon)^{-n} \right| \\
        \end{align*}

        \begin{align*}
          \sum_{n = m}^\infty \left| \frac{n!}{m!(n - m)!}(b - a)^{n - m}c_n \right|
           & \leq C \sum_{n = m}^\infty\left| \frac{n!}{m!(n - m)!}(b - a)^{n - m}(r - \epsilon)^{-n} \right| \\
        \end{align*}

        因为$|b - a| \in (-r + \epsilon, r - \epsilon)$,由习题15.2.7可知,
        $C \sum_{n = m}^\infty\left| \frac{n!}{m!(n - m)!}(b - a)^{n - m}(r - \epsilon)^{-n} \right|$收敛,
        利用推论7.3.2(比较判别法)可知,
        $\sum_{n = m}^\infty \left| \frac{n!}{m!(n - m)!}(b - a)^{n - m}c_n \right|$绝对收敛,
        所以$\sum_{n = m}^\infty \frac{n!}{m!(n - m)!}(b - a)^{n - m}c_n$收敛,于是
        $d_m$是某个实数,所以是有意义的。

  \item (d)

        由(c)中的讨论可知,
        \begin{align*}
          |d_m| & \leq C \sum_{n = m}^\infty\left| \frac{n!}{m!(n - m)!}(b - a)^{n - m}(r - \epsilon)^{-n} \right| \\
                & = C \frac{r - \epsilon}{(r - \epsilon - |b - a|)^{m + 1}}                                        \\
                & \leq C \frac{r - \epsilon}{(r - \epsilon)^{m + 1}}                                               \\
                & = C \frac{1}{(r - \epsilon)^{m}}
        \end{align*}

        题设$(b - s, b + s)$是$(a - r, a + r)$的子集可知,
        \begin{align*}
          r \geq s
        \end{align*}

        综上可得
        \begin{align*}
          |d_m| \leq C \frac{1}{(r - \epsilon)^{m}} \leq C \frac{1}{(s - \epsilon)^{m}}
        \end{align*}

  \item (e)

        (1)绝对收敛。

        因为$x \in (b - s, b + s)$,所以
        \begin{align*}
          -s < x - b < s
        \end{align*}

        利用(d),对任意$m \geq 0$都有
        \begin{align*}
          |d_m||x - b|^m \leq C(s - \epsilon)^{-m} |x - b|^m \\
          |d_m||x - b|^m \leq C \left|\frac{x - b}{s - \epsilon}\right|^m
        \end{align*}

        由(d)可知,上式对任意的$\epsilon$都成立,
        令$|x - b| < |s - \epsilon|, \left|\frac{x - b}{s - \epsilon}\right| < 1$,
        于是可得
        \begin{align*}
          \sum \limits_{m = 0}^\infty C \left|\frac{x - b}{s - \epsilon}\right|^m
        \end{align*}
        收敛。由比较判别法可知,
        \begin{align*}
          \sum \limits_{m = 0}^\infty |d_m||x - b|^m
        \end{align*}
        收敛。

        (2)一致收敛于$f(x)$。
        \begin{align*}
          \sum \limits_{m = 0}^\infty d_m (x - b)^m
           & = \sum \limits_{m = 0}^\infty \sum \limits_{n = m}^\infty \frac{n!}{m!(n - m)!}(b - a)^{n - m}c_n(x - b)^m \\
           & = \sum \limits_{n = 0}^\infty \sum \limits_{m = 0}^n \frac{n!}{m!(n - m)!}(b - a)^{n - m}(x - b)^m         \\
           & = \sum \limits_{n = 0}^\infty c_n (x - b)^n
        \end{align*}
        (注意:其中第二个等式,求和上下限的变换十分关键,建议绘制一张坐标图,$m, n$分别作为$x$轴和$y$轴)。

        由上可得,
        在$x \in (b - s, b + s)$上,
        无限级数$\sum \limits_{m = 0}^\infty d_m (x - b)^m$
        与无限级数$\sum \limits_{n = 0}^\infty c_n (x - b)^n$等价。

        由题设可知,在$(a - r, a + r)$上$\sum \limits_{n = 0}^\infty c_n (x - b)^n$一致收敛于$f(x)$,
        且$(b - s, b + s)$是$(a - r, a + r)$的子区间,
        所以,$\sum \limits_{m = 0}^\infty d_m (x - b)^m$也一致收敛于$f(x)$。



  \item (f)

        对任意$b \in (a -r, a + r)$,存在$s > 0$使得$(b - s, b + s)$是$(a -r, a + r)$的子集,
        由(e)可知,对于所有的$x \in (b - s, b + s)$,
        存在幂级数$\sum\limits_{m = 0}^\infty d_m (x - b)^m$收敛于$f$,命题得证。

\end{itemize}


\end{document}