\documentclass{article}
\usepackage{mathtools} 
\usepackage{fontspec}
\usepackage[UTF8]{ctex}
\usepackage{amsthm}
\usepackage{mdframed}
\usepackage{xcolor}
\usepackage{amssymb}
\usepackage{amsmath}


% 定义新的带灰色背景的说明环境 zremark
\newmdtheoremenv[
  backgroundcolor=gray!10,
  % 边框与背景一致,边框线会消失
  linecolor=gray!10
]{zremark}{说明}


\begin{document}
\title{15.1 习题}
\author{张志聪}
\maketitle

\section*{15.1.1}
\begin{align*}
  \lim\sup\limits_{n \to \infty} |c_n(x -a)^n|^\frac{1}{n}
   & = \lim\sup\limits_{n \to \infty} |c_n|^\frac{1}{n}|x -a|
\end{align*}

\begin{itemize}
  \item (a)

        由定义15.1.3(收敛半径)可知,

        (1)$R = + \infty$,那么,$|x - a| > R$这个前置条件无法成立。

        (2)$R = 0$,那么,$\lim\sup\limits_{n \to \infty} |c_n|^\frac{1}{n} = +\infty$,且
        $|x - a| > 0$,所以,$\lim\sup\limits_{n \to \infty} |c_n|^\frac{1}{n}|x -a| > 1$,
        由定理7.5.1(b)可知,级数发散。

        (3)$R > 0$($R$是实数)。那么,$\lim\sup\limits_{n \to \infty} |c_n|^\frac{1}{n} = \frac{1}{R}$,
        于是,$|x - a| > R$,
        那么,$\lim\sup\limits_{n \to \infty} |c_n|^\frac{1}{n}|x - a| = \frac{1}{R} |x - a| > 1$,
        (注意:$|x - a|$此时可以看做常数,
        $\lim\sup\limits_{n \to \infty} |c_n|^\frac{1}{n}|x - a| = \frac{1}{R} |x - a|$可以通过反证法证明)
        由定理7.5.1(b)可知,级数发散。
  \item (b)

        由定义15.1.3(收敛半径)可知,

        (1)$R = + \infty$,那么,$\lim\sup\limits_{n \to \infty} |c_n|^\frac{1}{n} = 0$且$|x - a| < R$总能成立,
        于是$\lim\sup\limits_{n \to \infty} |c_n|^\frac{1}{n}|x -a| = 0$,
        由定理7.5.1(a)可知,级数绝对收敛。

        (2)$R = 0$,那么,$|x - a| < R$这个前置条件无法成立。

        (3)$R > 0$($R$是实数)。那么,$\lim\sup\limits_{n \to \infty} |c_n|^\frac{1}{n} = \frac{1}{R}$,
        于是,$|x - a| < R$,
        那么,$\lim\sup\limits_{n \to \infty} |c_n|^\frac{1}{n}|x - a| = \frac{1}{R} |x - a| < 1$,
        由定理7.5.1(b)可知,级数绝对收敛。
  \item (c)

        (1)一致收敛于$f$。

        由定义14.5.2(无限级数)可知,
        我们需要证明$N \to \infty$时,
        部分和$\sum\limits_{n = 0}^N f^{(n)}$,
        其中$f^{(n)} := c_n(x - a)^n$,沿着$[a -r, a + r]$一致收敛于$f$。

        于是,利用14.5.7(威尔斯特拉斯M判别法),我们需要证明
        \begin{align*}
          \sum\limits_{n = 0}^\infty ||f^{(n)}||_\infty
        \end{align*}
        是收敛的。

        因为$|c_n(x - a)^n|$在$[a, a+r]$上是单增的,
        所以$x_0 = a + r$时,$f^{(n)} = c_n(x_0 - a)^n$取最大值,
        即$||f^{(n)}||_\infty = |c_n(x_0 − a)n|$。

        同理,$x_0 = a - r$时,$f^{(n)} = c_n(x_0 - a)^n$取最大值。

        由(b)可得,对任意$x_0 \in [a - r, a + r], \sum\limits_{n = 0}^\infty c_n(x_0 - a)^n$是绝对收敛的,
        即$\sum\limits_{n = 0}^\infty |c_n(x_0 - a)^n|$是收敛的。

        特别地,$x_0 = a + r$或$x_0 = a - r$,级数也是收敛的,即$\sum\limits_{n = 0}^\infty ||f^{(n)}||_\infty$
        收敛。

        于是可知,
        级数$\sum\limits_{n = 0}^\infty f^{(n)}$一致收敛于某个函数,
        即级数$\sum\limits_{n = 0}^\infty c_n(x - a)^n$一致收敛于某个函数。
        (注意:这里的函数用$\sum\limits_{n = 0}^\infty c_n(x - a)^n$本身表示,它代表一致收敛的函数$f$)。


        (2)$f$是连续的。

        对于每一个$N$,函数$\sum\limits_{n = 0}^N f^{(n)}$都是连续的
        (这里其实有借助定义14.5.2),由推论14.3.2可知,$f$是连续的。

  \item (d)

        令$f_n = c_n(x - a)^n$,于是$f_n$是连续且可微的,
        且导函数$f_n^\prime = nc_n(x - a)^{n - 1}$也是连续的。

        我们有($r < R$)
        \begin{align*}
          \sum\limits_{n = 1}^\infty ||f_n^\prime||_\infty
           & = \sum\limits_{n = 1}^\infty nc_nr^{n - 1} \\
        \end{align*}

        结合例15.1.15可知:
        \begin{align*}
          \lim\sup\limits_{n \to \infty} (nc_nr^n)^\frac{1}{n}
           & = \lim\sup\limits_{n \to \infty} n^\frac{1}{n} \lim\sup\limits_{n \to \infty} (c_nr^n)^\frac{1}{n} \\
           & < 1
        \end{align*}
        于是可得$\sum\limits_{n = 1}^\infty nc_nr^n$收敛。

        因为
        \begin{align*}
          \sum\limits_{n = 1}^\infty nc_nr^{n - 1} = r^{-1} \sum\limits_{n = 1}^\infty nc_nr^{n}
        \end{align*}
        可得,$\sum\limits_{n = 1}^\infty nc_nr^{n - 1}$收敛。

        综上,由威尔斯特拉斯M判别法可知,$\sum \limits_{n = 1}^\infty f_n^\prime$一致收敛于某个函数$g$。

        又$x_0 = a$时,
        \begin{align*}
          \sum \limits_{n = 1}^\infty c_n(x_0 - a)^n = 0
        \end{align*}

        $F_N = \sum \limits_{n = 1}^N c_n(x - a)^n$是一个可微函数,
        并且其倒数$F_N^\prime = \sum \limits_{n = 1}^N nc_n(x - a)^{n - 1} = \sum \limits_{n = 1}^N f_n^\prime$是连续的。
        又由之前的讨论可知,导函数序列$F_N^\prime$一致收敛于某个函数$g$,
        并且存在一点$x_0 = a$使得极限$\lim\limits_{n \to \infty} F_N(x_0) = \sum \limits_{n = 1}^\infty c_n(x_0 - a)^n = 0$,
        由定理14.7.1可知,函数序列$F_N$一致收敛于一个可微函数,由该函数的唯一性可知,$F_N$一致收敛于$f$,并且$f$的导函数等于$g$。
        所以$g = f^\prime$。

  \item (e)

        由推论14.6.2可知,
        \begin{align*}
          \int_{[y, z]} f
           & = \sum\limits_{n = 0}^\infty \int_{[y, z]} c_n(x - a)^n                                          \\
           & = \sum\limits_{n = 0}^\infty \frac{c_n(x - a)^{n + 1}}{n + 1}|_y^z                               \\
           & = \sum\limits_{n = 0}^\infty \frac{c_n(z - a)^{n + 1}}{n + 1} - \frac{c_n(y - a)^{n + 1}}{n + 1} \\
           & = \sum\limits_{n = 0}^\infty c_n\frac{(z - a)^{n + 1} - (y - a)^{n + 1}}{n + 1}                  \\
        \end{align*}
\end{itemize}

\section*{15.1.2}

\begin{itemize}
  \item (a)
        \begin{align*}
          \sum\limits_{n = 0}^\infty x^n
        \end{align*}

  \item (b)
        \begin{align*}
          \sum\limits_{n = 0}^\infty \frac{x^n}{n}
        \end{align*}

        使用推论7.3.7、命题7.2.12(交错级数判别法)可以验证是否正确。

  \item (c)

        \begin{align*}
          \sum\limits_{n = 0}^\infty \frac{(-1)^n}{n} x^n
        \end{align*}

        使用推论7.3.7、命题7.2.12(交错级数判别法)可以验证是否正确。

  \item (d)

        \begin{align*}
          \sum\limits_{n = 0}^\infty \frac{x^n}{n^2}
        \end{align*}

        使用推论7.3.7、命题7.2.12(交错级数判别法)可以验证是否正确。

  \item (d)

        \begin{align*}
          \sum\limits_{n = 0}^\infty x^n
        \end{align*}

        例14.5.8中有说明。

\end{itemize}

\end{document}