\documentclass{article}
\usepackage{mathtools} 
\usepackage{fontspec}
\usepackage[UTF8]{ctex}
\usepackage{amsthm}
\usepackage{mdframed}
\usepackage{xcolor}
\usepackage{amssymb}
\usepackage{amsmath}


% 定义新的带灰色背景的说明环境 zremark
\newmdtheoremenv[
  backgroundcolor=gray!10,
  % 边框与背景一致,边框线会消失
  linecolor=gray!10
]{zremark}{说明}

% 通用矩阵命令: \flexmatrix{矩阵名}{元素符号}{行数}{列数}
\newcommand{\flexmatrix}[4]{
  \[
  #1 = \begin{pmatrix}
    #2_{11}     & #2_{12}     & \cdots & #2_{1#4}   \\
    #2_{21}     & #2_{22}     & \cdots & #2_{2#4}   \\
    \vdots      & \vdots      & \ddots & \vdots     \\
    #2_{#31}    & #2_{#32}    & \cdots & #2_{#3#4}
  \end{pmatrix}
  \]
}

% 简化版命令(默认矩阵名为A,元素符号为a): \quickmatrix{行数}{列数}
\newcommand{\quickmatrix}[2]{\flexmatrix{A}{a}{#1}{#2}}


\begin{document}
\title{3.1 注释}
\author{张志聪}
\maketitle

\begin{zremark}
  德摩根定律的无限版本:
  \begin{align*}
    \bigcap\limits_{k = 1}^\infty A_{k}
    = \mathbb{R}^n \setminus \bigcup\limits_{k = 1}^\infty (\mathbb{R}^n \setminus A{k})
  \end{align*}
\end{zremark}

\textbf{证明:}

\begin{itemize}
  \item $\Rightarrow$
  
  任意$x \in \bigcap\limits_{k = 1}^\infty A_{k}$,
  则对任意$k$,都有$x \in A_k$,
  所以$x \notin \mathbb{R}^n \setminus A{k}$,
  进而$x \notin \bigcup\limits_{k = 1}^\infty (\mathbb{R}^n \setminus A{k})$,
  (否则会产生矛盾,存在$k$使得$x \in \mathbb{R}^n \setminus A{k}$)。
  从而$x \in \mathbb{R}^n \setminus \bigcup\limits_{k = 1}^\infty (\mathbb{R}^n \setminus A{k})$。

  \item $\Leftarrow$

  任意$x \in \mathbb{R}^n \setminus \bigcup\limits_{k = 1}^\infty (\mathbb{R}^n \setminus A{k})$,
  则对任意$k$,都有$x \notin \bigcup\limits_{k = 1}^\infty (\mathbb{R}^n \setminus A{k})$,
  即对任意的$k$都有$x \in A_k$,从而
  $x \in \bigcap\limits_{k = 1}^\infty A_{k}$。
  
\end{itemize}


\end{document}