\documentclass{article}
\usepackage{mathtools} 
\usepackage{fontspec}
\usepackage[UTF8]{ctex}
\usepackage{amsthm}
\usepackage{mdframed}
\usepackage{xcolor}
\usepackage{amssymb}
\usepackage{amsmath}


% 定义新的带灰色背景的说明环境 zremark
\newmdtheoremenv[
  backgroundcolor=gray!10,
  % 边框与背景一致,边框线会消失
  linecolor=gray!10
]{zremark}{说明}


\begin{document}
\title{3.4 习题}
\author{张志聪}
\maketitle

\section*{3.4.8}

\begin{zremark}
      这里要先理解题意了。是要求我们利用“并集公理”与“双元素集公理”构造出“两集合公理”。
      怎么才算构造成功呢?假设$X,Y$是集合,那么需要构造出一个集合$W$,满足对任意对象$x$,
      \begin{align*}
            x \in W \Leftrightarrow (x \in X \text{ 或 } x \in Y)
      \end{align*}
\end{zremark}
假设$X,Y$是集合,由公理3.1(集合是对象)和公理3.3(单元素集与双元素集)可知,
存在一个集合$A := \{X, Y\}$。

由公理3.11我们可以构造$W := \cup A$,于是,
\begin{align*}
      x \in  W  \Leftrightarrow (\text{存在}S \in A \text{使得} x \in S)
\end{align*}
由公理3.3可知,“$\text{存在}S \in A \text{使得} x \in S$”即:$S=X \text{ 或 } S=Y$使得$x \in S$。
所以构造的$W$是满足要求的。


\end{document}
