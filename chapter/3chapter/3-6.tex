\documentclass{article}
\usepackage{fontspec}
\usepackage[UTF8]{ctex}
\usepackage{amsthm}
\usepackage{mdframed}
\usepackage{xcolor}
\usepackage{amssymb}
\usepackage{amsmath}

\newmdtheoremenv[
  backgroundcolor=gray!10,
  linewidth=0pt,
  innerleftmargin=10pt,
  innerrightmargin=10pt,
  innertopmargin=10pt,
  innerbottommargin=10pt
]{zgraytheorem}{Theorem}

% 定义说明环境样式
\newtheoremstyle{mystyle}% 说明环境样式的名称
  {1em}% 上方间距
  {1em}% 下方间距
  {\normalfont}% 说明内容的字体样式
  {}% 缩进量
  {\bfseries}% 说明标记的字体样式
  {.}% 说明标记和说明内容之间的标点
  {1em}% 说明标记后的水平空间
  {}% 说明标记后的垂直空间
% 使用新定义的样式创建说明环境
\theoremstyle{mystyle}
\newtheorem*{zremark}{说明}

% 定义证明环境样式
\newtheoremstyle{zproofstyle}% 证明环境样式的名称
  {0.5em}% 上方间距
  {0.5em}% 下方间距
  {\itshape}% 证明内容的字体样式
  {}% 缩进量
  {\bfseries}% 证明标记的字体样式
  {.\newline}% 证明标记和证明内容之间的标点
  {1em}% 证明标记后的水平空间
  {}% 证明标记后的垂直空间

% 使用新定义的样式创建证明环境
\theoremstyle{zproofstyle}
\newtheorem*{zproof}{证明}

\begin{document}
\title{3.6 习题}
\maketitle

\section*{3.6.1}
\begin{zproof}
  \textcircled{1}X和X有相等的基数。

  构造一个从X到X的函数f,使得f(x)=x($\{x \in X\}$)。函数f是双射函数,是显而易见的,
  这里不做证明了。

  \textcircled{2}如果X和Y有相等的基数,那么Y和X有相等的基数。

  有X和Y有相等的基数,可知存在一个双射:$f: X \rightarrow Y$。
  那么存在f的逆$f^{-1}: Y \rightarrow X$,由逆的定义可知$f^{-1}$是双射函数。

  \textcircled{3}如果X和Y有相等的基数且Y和Z有相等的基数,那么X和Z有相等的基数。

  由X和Y有相等的基数,可知存在一个双射:$f: X \rightarrow Y$。
  由Y和Z有相等的基数,可知存在一个双射:$g: Y \rightarrow Z$。
  那么g和f的复合函数为$g \circ f: X \rightarrow Z$。

  由习题3.3.7可知$g \circ f$是双射函数。
  由此可知存在一个双射:$g \circ f: X \rightarrow Z$,所以X和Z有相等的基数。
\end{zproof}

\subsection*{3.6.2}
\begin{zproof}
  \textcircled{1}充分性:一个集合X的基数为0,则X是空集。

  那么存在从X到$\{i \in N: 1 \leq i \leq 0\}$的双射:$f: X \rightarrow \{i \in N: 1 \leq i \leq 0\}$。
  而$\{i \in N: 1 \leq i \leq 0\}$是$\varnothing$,即$f: X \rightarrow \varnothing$。
  如果X不是空集,那么则存在一个$x \in X$使得$f(x) \in \varnothing$,
  这显然是不成立的,所以X是空集

  \textcircled{2}必要性:X是空集,则X的基数为0。

  若X是空集,由习题3.3.3知$f:\varnothing \rightarrow \varnothing$为双射,
  而$\{i \in N : 1 \leq i \leq 0\}=\varnothing$,
  即存在双射函数$f:\varnothing \rightarrow \{i \in N : 1 \leq i \leq 0\}$,
  由定义3.6.5知集合X基数为0.
\end{zproof}

\end{document}