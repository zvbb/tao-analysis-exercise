\documentclass{article}
\usepackage{fontspec}
\usepackage[UTF8]{ctex}
\usepackage{amsthm}
\usepackage{mdframed}
\usepackage{xcolor}
\usepackage{amssymb}
\usepackage{amsmath}

\newmdtheoremenv[
  backgroundcolor=gray!10,
  linewidth=0pt,
  innerleftmargin=10pt,
  innerrightmargin=10pt,
  innertopmargin=10pt,
  innerbottommargin=10pt
]{zgraytheorem}{Theorem}

% 定义说明环境样式
\newtheoremstyle{mystyle}% 说明环境样式的名称
  {1em}% 上方间距
  {1em}% 下方间距
  {\normalfont}% 说明内容的字体样式
  {}% 缩进量
  {\bfseries}% 说明标记的字体样式
  {.}% 说明标记和说明内容之间的标点
  {1em}% 说明标记后的水平空间
  {}% 说明标记后的垂直空间
% 使用新定义的样式创建说明环境
\theoremstyle{mystyle}
\newtheorem*{zremark}{说明}

% 定义证明环境样式
\newtheoremstyle{zproofstyle}% 证明环境样式的名称
  {0.5em}% 上方间距
  {0.5em}% 下方间距
  {\itshape}% 证明内容的字体样式
  {}% 缩进量
  {\bfseries}% 证明标记的字体样式
  {.\newline}% 证明标记和证明内容之间的标点
  {1em}% 证明标记后的水平空间
  {}% 证明标记后的垂直空间

% 使用新定义的样式创建证明环境
\theoremstyle{zproofstyle}
\newtheorem*{zproof}{证明}

\begin{document}
\title{3.6 习题}
\maketitle

\section*{3.6.1}
\begin{zproof}
  \textcircled{1}X和X有相等的基数。

  构造一个从X到X的函数f,使得f(x)=x($\{x \in X\}$)。函数f是双射函数,是显而易见的,
  这里不做证明了。

  \textcircled{2}如果X和Y有相等的基数,那么Y和X有相等的基数。

  有X和Y有相等的基数,可知存在一个双射:$f: X \rightarrow Y$。
  那么存在f的逆$f^{-1}: Y \rightarrow X$,由逆的定义可知$f^{-1}$是双射函数。

  \textcircled{3}如果X和Y有相等的基数且Y和Z有相等的基数,那么X和Z有相等的基数。

  由X和Y有相等的基数,可知存在一个双射:$f: X \rightarrow Y$。
  由Y和Z有相等的基数,可知存在一个双射:$g: Y \rightarrow Z$。
  那么g和f的复合函数为$g \circ f: X \rightarrow Z$。

  由习题3.3.7可知$g \circ f$是双射函数。
  由此可知存在一个双射:$g \circ f: X \rightarrow Z$,所以X和Z有相等的基数。
\end{zproof}

\subsection*{3.6.2}
\begin{zproof}
  \textcircled{1}充分性:一个集合X的基数为0,则X是空集。

  那么存在从X到$\{i \in N: 1 \leq i \leq 0\}$的双射:$f: X \rightarrow \{i \in N: 1 \leq i \leq 0\}$。
  而$\{i \in N: 1 \leq i \leq 0\}$是$\varnothing$,即$f: X \rightarrow \varnothing$。
  如果X不是空集,那么则存在一个$x \in X$使得$f(x) \in \varnothing$,
  这显然是不成立的,所以X是空集

  \textcircled{2}必要性:X是空集,则X的基数为0。

  若X是空集,由习题3.3.3知$f:\varnothing \rightarrow \varnothing$为双射,
  而$\{i \in N : 1 \leq i \leq 0\}=\varnothing$,
  即存在双射函数$f:\varnothing \rightarrow \{i \in N : 1 \leq i \leq 0\}$,
  由定义3.6.5可知集合X基数为0.
\end{zproof}

\subsection*{3.6.3}
\begin{zproof}
  对n进行归纳:

  n=0时,f是空函数,命题空成立。

  归纳假设n=k时,命题成立。

  下面我们证明该命题对于k++也为真。
  设集合$N_k = \{i \in \mathsf{N}: 1 \leq i \leq k\}, N_{k++} = \{i \in \mathsf{N}: 1 \leq i \leq k++\}$。
  函数$f_{k++}: N_{k++} \rightarrow N$是一个函数,
  我们可以由$f_{k++}$定义出一个函数$f_k: N_k \rightarrow N$,对任意$i \in N_k, f_k(i) = f_{k++}(i)$。
  由归纳假设可知,存在一个自然数M使得$f_k(i) \leq M, i \in N_k$,即$f_{k++}(i) \leq M, i \in N_k$,
  此时我们可以取$f_{k++}(k++), M$中的较大值为$M^\prime$,
  由此可知该$M^\prime$使得$f_{k++}(i) \leq M^\prime, i \in N_{k++}$。归纳法完成。
\end{zproof}

\subsection*{3.6.4}

(a)设$X$是一个有限集,设$x$是一个对象并且$x$不是$X$中的元素。
那么$X \cup \{x\}$是有限的,且$\# (X \cup \{x\}) = \# (X) + 1$
\begin{zproof}
  $X$是有限集,不妨设$X$的基数是自然数n。因此存在从$X$到$\{i \in N: 1 \leq i \leq n\}$的双射函数$f$。
  定义出一个函数$g: X \cup \{x\} \rightarrow \{i \in N: 1 \leq i \leq n+1\}$,
  使得$g(x) = n+1$,$g(i) = f(i), i \in X$。
  由g的定义可知其是双射函数,且$X \cup \{x\}$的基数是n+1,
  所以$X \cup \{x\}$是有限的,且$\# (X \cup \{x\}) = \# (X) + 1$
\end{zproof}

(b)设$X$和$Y$都是有限集,那么$X \cup Y$是有限的,且$\# (X \cup Y) \leq \# (X) + \# (Y)$。
另外,如果$X$和$Y$是不相交的(即$X \cap Y = \varnothing$),那么$\# (X \cup Y) = \# (X) + \# (Y)$
\begin{zproof}
  $X$和$Y$都是有限集,不妨设$X$和$Y$的基数分别为m和n。
  通过对n进行归纳,完成证明;

  n=0时,即$Y$的基数是0,也就是说$Y = \varnothing$,$X \cup Y = X \cup \varnothing = X$,
  此时(b)命题显然是成立的。

  归纳假设n=k时,(b)命题成立。

  现在需证明n=k++,任取$x \in Y, Z=Y \backslash \{x\}$,由引理3.6.9可知,$Z$的基数为k,
  由归纳假设可知,$X$与$Z$满足命题(b),由此可知$X \cup Z$是有限的;

  $X \cup Y = X \cup Z \cup \{x\}$。

  \textcircled{1} $X \cap Y = \varnothing$,由此可知$x \not\in X \cup Z$,且由归纳假设知$\#(X \cup Z) = \#(X)+\#(Z)$。
  由命题(a)可知$X \cup Z \cup \{x\}$是有限的,且$\#(X \cup Z \cup \{x\})=\#(X \cup Z)+1$,
  即$X \cup Y$是有限的,且$\#(X \cup Y) = \#(X \cup Z)+1 = \#(X) + \#(Z) + 1 = \#(X) + \#(Y)$,
  即$\#(X \cup Y) = \#(X) + \#(Y)$;

  \textcircled{2} $X \cap Y \neq \varnothing$

  如果$x \in X \cup Z$则$X \cup Y=X \cup Z \cup \{x\}=X \cup Z$,即$X \cup Y=X \cup Z$
  由于同一集合只有一个基数,所以$\#(X \cup Y)=\#(X \cup Z)$,
  又由归纳假设可知 $\#(X \cup Z) \leq \#(X) + \#(Z)$,所以$\#(X \cup Y) \leq \#(X)+ \#(Y)$。

  如果$x \not\in X \cup Z$,(由$X \cap Y \neq \varnothing$,则必须$X \cap Z \neq \varnothing$否则与假设矛盾,
  所以$\#(X \cup Z) \leq \#(X) + \#(Z)$)由命题(a)可知$\#(X \cup Z \cup \{x\})=\#(X \cup Z)+1$,
  即$X \cup Y$是有限的,且$\#(X \cup Y) = \#(X \cup Z)+1 \leq \#(X) + \#(Z) + 1 = \#(X) + \#(Y)$,
  即$\#(X \cup Y) \leq \#(X) + \#(Y)$;

  综上,n=k++情况也成立,至此,(b)命题成立。
\end{zproof}

(c)设$X$是一个有限集,$Y$是$X$的一个子集。那么$Y$是有限的,且$\# (Y) \leq \# (X)$。
另外,如果$Y \neq X$(即$Y$是$X$的一个真子集),那么我们有$\#(Y) < \#(X)$。

\begin{zproof}
  对$X$的基数进行归纳。

  $X$的基数为0,即$X = \varnothing$,此时$Y$是$X$的子集,则$Y = \varnothing$,
  很明显$Y$是有限的(基数是0),且$\# (Y) \leq \# (X)$。而命题的后半部分,因为空集不存在真子集,所以空成立。

  归纳假设n=k时,$X$的基数为k,命题(c)成立。

  现需证明n=k++,命题(c)成立。若$Y=X$显然$\#(Y) \leq \#(X)$;
  若$Y \neq X$,则存在$x \in X$,使得$Y \subseteq (X \backslash {x})$,
  由归纳假设可知$\#(Y) \leq \#(X \backslash x)$,由引理3.6.9可知$\#(Y) < \#(X)$。

  综上命题(c)成立。
\end{zproof}

(d)如果$X$是一个有限集,并且$f: X \rightarrow Y$是一个函数,那么$f(X)$是一个有限集并且满足$\#(f(X)) \leq \#(X)$。
另外,如果$f$是一对一的,那么$\#(f(X)) = \#(X)$。

\begin{zproof}
  对$X$的基数n进行归纳;

  归纳基始n=0,即$X = \varnothing$,由定义3.4.1(集合的像)可知$f(X) = \varnothing$,
  即$\#(f(X)) = 0$,此时命题(d)成立

  n=k++时,设$X^\prime = X \backslash \{x\}$,
  由归纳假设可知$\#(f(X^\prime)) \leq \#(X^\prime)$,

  \textcircled{1} $f(X^\prime) = f(X)$,则$\#(f(X)) =\#(f(X^\prime)) \leq \#(X^\prime) < \#(X)$。
  此时f不是双射,命题后半部分空成立。

  \textcircled{2} $f(X^\prime) \subsetneq  f(X)$
  则$f(x) \not \in f(X^\prime)$,且$f(X) = f(X^\prime) \cup f(x)$,
  由命题(a)可知$\#(f(X))=\#(f(X^\prime)) + 1$,有$\#(X) = \#(X^\prime) + 1$,
  所以由归纳假设$\#(f(X^\prime)) \leq \#(X^\prime)$可知$\#(f(X^\prime)) + 1 \leq \#(X^\prime) + 1$,
  即$\#(f(X)) \leq \#(X)$;若$f$是一对一的,则$X^\prime \backslash \{x\} \rightarrow Y \backslash \{f(x)\}$也是一对一,
  由归纳假设知$\#(f(X^\prime)) = \#(X^\prime)$,由此可知$\#(f(X^\prime)) + 1 = \#(X^\prime) + 1$,
  $\#(f(X)) = \#(X)$。综上,n=k++时命题(d)成立。

  至此,命题成立
\end{zproof}

(e)设$X$和$Y$都是有限集,那么笛卡尔积$X \times Y$是有限的并且$\#(X \times Y) = \#(X) \times \#(Y)$。
\begin{zproof}
  设$X,Y$的基数分别为n、m,对n进行归纳。

  归纳基始n=0,即$X$是空集,有笛卡尔积的定义可知,$X \times Y = \varnothing$,
  由此可知$\#(X \times Y) = 0$,且$X \times Y$是有限的。又$\#(X \times Y) = \#(X) \times \#(Y) = 0$,
  所以n=0时,命题(e)成立。

  n=k++时,设对任意$x \in X$,构造$X^\prime = X \backslash \{x\}$,
  由习题3.5.5可知$X \times Y = (X^\prime \cup \{x\}) \times (Y \cup Y) = (X^\prime \times Y) \cup (\{x\} \times Y)$,
  由笛卡尔积的定义可知$(X^\prime \times Y) \cap (\{x\} \times Y) = \varnothing$,
  由(b)可知,$\#((X^\prime \times Y) \cup (\{x\} \times Y)) = \#(X^\prime \times Y) + \#(\{x\} \times Y)$
  由归纳假设可知$\#(X^\prime \times Y) = \#(X^\prime) \times \#(Y)$,现在只需证明$\#(\{x\} \times Y)=\#(Y)$,
  命题就能完成证明。
  (在直觉上是显然的,但为了严谨性,还是需要证明),要想证明基数相同,
  按照定义3.6.1只需找到从$(\{x\} \times Y)$到$Y$的一个双射函数$f: (\{x\} \times Y) \rightarrow Y$。
  可以定义$f$如下:$f((x,y))=y, (x,y) \in (\{x\} \times Y)$,这里的f是双射性是显然的,为了简洁不做说明了。
  由此可知$\#(X \times Y) = \#(X^\prime \times Y) + \#(\{x\} \times Y)=\#(X^\prime \times Y) + \#(Y)$
  $=\#(X^\prime) \times \#(Y) + \#(Y)=(\#(X^\prime)++) \times \#(Y) = \#(X) \times \#(Y)$,
  n=k++时命题(e)成立。

  至此归纳完成,命题(e)得到证明。

\end{zproof}

(f)设$X$和$Y$都是有限集,那么集合$Y^X$(在公理3.10中被定义)是有限的,并且$\#(Y^X)=\#(Y)^{\#(X)}$
\begin{zproof}
  公理3.10中对幂集公理的定义,很难定量分析,我们使用其他公理对幂集公理重新定义。


  $I$为一个集合,并对每一个元素$y_0 \in I$均有一个集合$A_{y_0}$,
  $A_{y_0} = \{ \{f\text{是X到Y的函数}, f(x_0)=y_0\}: y_0 \in Y\}$
  幂集定义如下:
  $W=\bigcup\limits_{y \in I}A_{y} = \cup\{A_y : y \in I\}$

  现在需要证明该定义和幂集公理的等价性。

  $f \in W \Leftrightarrow \text{存在} y \in I \text{使得} f \in A_y$,
  由此可知$f\text{是X到Y的函数}$,所以$f \in Y^X$。

  $f \in Y^X$,由于$f$是X到Y的函数,则对$x_0 \in X$有$y=f(x_0)$,$y \in Y$,所以$f \in A_y$,
  所以$f \in W$。

  综上可证该定义和幂集公理的等价性。


  设$X,Y$的基数分别为n、m,通过对n进行归纳,证明该命题。

  归纳基始n=0,即$X = \varnothing$,而$f: \varnothing \rightarrow Y$的函数,由函数相等的定义可知是唯一的,
  所以$\#(Y^X)=1,\#(Y)^{\#(X)}=m^0=1$,由此可知$\#(Y^X)=\#(Y)^{\#(X)}$,在n=0时命题(f)成立

  n=k++时,设$X^\prime=X \backslash \{x_0\}, x_0 \in X$,
  证明$\#(A_y)=\#(Y^{X^\prime})$,
  函数$G: A_y \rightarrow Y^{X^\prime}$,
  定义如下:$ g=G(f), x \in X^\prime, f(x)=g(x)$。

  证明函数G的定义是合法,即证明g的唯一性,假设存在g满足定义,
  即对任意$f \in A_y$,存在$g^\prime(x)=f(x)=g(x)$,由函数相等的定义可知$g=g^\prime$,g的唯一性得证。

  证明G是双射的,先证明单射,如果G不是单射,则存在$f_1 \neq f_2$,有相同的函数值$g$,
  由于$f_1 \neq f_2$所以存在$x \in X^\prime, f_1(x) \neq f_2(x)$,有G的定义可知$g(x)=f_1(x)=f_2(x)$,
  这与$f_1(x) \neq f_2(x)$矛盾,所以G是单射。

  证明G是满射的,对任意函数值$g \in Y^{X^\prime}$,可以定义出一个函数$f: X \rightarrow Y, f(x_0)=y, f(x)=g(x)$,
  该函数$f \in A_y$,所以G是满射的。

  由此可知$\#(A_y)=\#(Y^{X^\prime})=m^k$

  由$A_y$的定义方式可知是不相交的,即对任意$y_0 \neq y_1, A_{y_0} \cap A_{y_1}=\varnothing$,
  由(b)可知$\#(W)=\sum_{y \in I} \#(A_y) = m \times \#(Y^{X^\prime}) = m \times (m^k) = m^{k++} $,
  由此可知n=k++命题(f)也成立。

  至此命题(f)成立。
\end{zproof}

\section*{3.6.5}
\begin{zproof}
  \begin{align}
    A \times B & = \{(a,b): a \in A, b \in B \} \\
    B \times A & = \{(b,a): b \in B, a \in A \}
  \end{align}
  现在定义函数$f: A \times B \rightarrow B \times A, f(a, b):= (b, a)$。

  接下来要证明f的双射性(为了简洁不做说明了)。

  由命题3.6.14可知
  \begin{align}
    \#(A \times B) & = \#(A) \times \#(B) \\
    \#(B \times A) & = \#(B) \times \#(A)
  \end{align}
  又因为$A \times B, B \times A$之间存在一个双射f,所以两个集合之间有相同的基数,
  由此通过(3)(4)可知$\#(A) \times \#(B) = \#(B) \times \#(A)$
\end{zproof}

\section*{3.6.6}
\begin{zproof}
  \textcircled{1}构造双射

  由公理3.10(幂集公理)可知,
  \begin{align}
    (A^B)^C &= \{f: f\text{是一个定义域为}C,\text{值域为}A^B\text{的函数}\} \\
    A^{B \times C} &= \{g: g\text{是一个定义域为}B \times C,\text{值域为}A\text{的函数}\} 
  \end{align}
    定义函数$G: (A^B)^C \rightarrow A^{B \times C}$如下: 
    $G(f):=g$,$f,g$满足以下性质:对任意$b \in B, c \in C$有$[f(c)](b)=g(b,c)$

    现需证明G是满足函数定义的,对相同的输入只存在唯一的函数值。
    函数f对任意$b \in B, c \in C$存在$g, g^\prime$使得$[f(c)](b)=g(b,c)=g^\prime(b,c)$,
    对任意$b \in B, c \in C$,$g(b,c)=g^\prime(b,c)$通过函数相等的定义可知$g=g^\prime$。

    现需证明G是双射函数。对任意$f \neq f^\prime$, 
    \begin{align}
      [f(c)](b) &= g(b,c)  \\
      [f^\prime(c)](b) &= g^\prime(b,c) 
    \end{align}
    由于$f \neq f^\prime$所以存在$(b^\prime, c^\prime)$使得$f(b^\prime, c^\prime) \neq f^\prime(b^\prime, c^\prime)$,
    可得$g(b^\prime, c^\prime) \neq g^\prime(b^\prime, c^\prime)$,所以$g \neq g^\prime$,所以G是单射函数。

    $g \in A^{B \times C}$,对某个$c_0\in C$我们定义函数$h_0: B\rightarrow A, h_0(b):=g(b, c_0);$
    再对每个$c\in C$我们定义函数$f: C\rightarrow A^B, f(c):=h_c$。 
    此时对任意$c\in C,b\in B$有$[f(c)](b)=h_c(b)=g(b,c)$。 故$G(f)=g$。 故G是满射的

    \textcircled{2} $(a^b)^c = a^{bc}$

    设A、B、C的基数分别为a、b、c,由命题3.6.14可知
    \begin{align}
      \#((A^B)^C) &= \#(A^B)^{\#(C)} = (\#(A)^{\#B})^{\#(C)} = (a^b)^c \\
      \#(A^{B \times C}) &= \#A^{\#(B \times C)} = \#A^{\#(B) \times \#(C)} = a^{bc}
    \end{align}
    又$(A^B)^C,A^{B \times C}$基数相同,所以$(a^b)^c = a^{bc}$

    \textcircled{3} $a^b \times a^c = a^{b+c}$

    通过构造明确的双射来证明:集合$A^B \times A^C$和集合$A^{B \cup C}$有相同的基数($B \cap C = \varnothing$)。
    由公理3.10(幂集公理)和定义3.5.4(笛卡尔积)可知
    \begin{align}
      A^B \times A^C &= \{(f, f^\prime): f \in A^B, f^\prime \in A^C\} \\ 
      A^{B \cup C} &= \{g: g\text{是定义域为}B \cup C \text{值域为} A \text{的函数} \}
    \end{align}
    定义函数$G: A^B \times A^C \rightarrow A^{B \cup C}$如下:

    $G(f,f^\prime):=g$,$f,f^\prime,g$满足如下性质:
    对任意$b \in B, c \in C$有$f(b)=g(b),f^\prime(c)=g(c)$

    现需证明G是满足函数定义的,对相同的输入只存在唯一的函数值。
    函数$f,f^\prime$对任意$b \in B, c \in C$存在$g, g^\prime$
    使得$f(b)=g(b),f^\prime(c)=g(c); f(b)=g^\prime(b),f^\prime(c)=g^\prime(c);$,
    对任意$b \in B, c \in C$,$g(b)=g^\prime(b);g(c)=g^\prime(c)$通过函数相等的定义可知$g=g^\prime$。

    现需证明G是双射函数。对任意$(f1,f2) \neq (f1^\prime,f2^\prime)$,
    $g=G(f1,f2),g^\prime=G(f1^\prime,f2^\prime)$,
    如果$g = g^\prime$,那么对任意$b \in B, c \in C$
    有$g(b)=g^\prime(b)=f1(b)=f1\prime(b),g(c)=g^\prime(c)=f2(c)=f2\prime(c)$,
    由此可知$f1=f1^\prime,f2=f2^\prime$,那么$(f1,f2) = (f1^\prime,f2^\prime)$,
    这与前提$(f1,f2) \neq (f1^\prime,f2^\prime)$矛盾,$g \neq g^\prime$,所以G是单射。

    

\end{zproof}
\end{document}