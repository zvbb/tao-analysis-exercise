\documentclass{article}
\usepackage{fontspec}
\usepackage[UTF8]{ctex}
\usepackage{amsthm}
\usepackage{mdframed}
\usepackage{xcolor}
\usepackage{amssymb}
\usepackage{amsmath}

\newmdtheoremenv[
  backgroundcolor=gray!10,
  linewidth=0pt,
  innerleftmargin=10pt,
  innerrightmargin=10pt,
  innertopmargin=10pt,
  innerbottommargin=10pt
]{zgraytheorem}{Theorem}

% 定义说明环境样式
\newtheoremstyle{mystyle}% 说明环境样式的名称
  {1em}% 上方间距
  {1em}% 下方间距
  {\normalfont}% 说明内容的字体样式
  {}% 缩进量
  {\bfseries}% 说明标记的字体样式
  {.}% 说明标记和说明内容之间的标点
  {1em}% 说明标记后的水平空间
  {}% 说明标记后的垂直空间
% 使用新定义的样式创建说明环境
\theoremstyle{mystyle}
\newtheorem*{zremark}{说明}

% 定义证明环境样式
\newtheoremstyle{zproofstyle}% 证明环境样式的名称
  {0.5em}% 上方间距
  {0.5em}% 下方间距
  {\itshape}% 证明内容的字体样式
  {}% 缩进量
  {\bfseries}% 证明标记的字体样式
  {.\newline}% 证明标记和证明内容之间的标点
  {1em}% 证明标记后的水平空间
  {}% 证明标记后的垂直空间

% 使用新定义的样式创建证明环境
\theoremstyle{zproofstyle}
\newtheorem*{zproof}{证明}

\begin{document}
\title{3.6 习题}
\maketitle

\section*{3.6.1}
\begin{zproof}
  \textcircled{1}X和X有相等的基数。

  构造一个从X到X的函数f,使得f(x)=x($\{x \in X\}$)。函数f是双射函数,是显而易见的,
  这里不做证明了。

  \textcircled{2}如果X和Y有相等的基数,那么Y和X有相等的基数。

  有X和Y有相等的基数,可知存在一个双射:$f: X \rightarrow Y$。
  那么存在f的逆$f^{-1}: Y \rightarrow X$,由逆的定义可知$f^{-1}$是双射函数。

  \textcircled{3}如果X和Y有相等的基数且Y和Z有相等的基数,那么X和Z有相等的基数。

  由X和Y有相等的基数,可知存在一个双射:$f: X \rightarrow Y$。
  由Y和Z有相等的基数,可知存在一个双射:$g: Y \rightarrow Z$。
  那么g和f的复合函数为$g \circ f: X \rightarrow Z$。

  由习题3.3.7可知$g \circ f$是双射函数。
  由此可知存在一个双射:$g \circ f: X \rightarrow Z$,所以X和Z有相等的基数。
\end{zproof}

\subsection*{3.6.2}
\begin{zproof}
  \textcircled{1}充分性:一个集合X的基数为0,则X是空集。

  那么存在从X到$\{i \in N: 1 \leq i \leq 0\}$的双射:$f: X \rightarrow \{i \in N: 1 \leq i \leq 0\}$。
  而$\{i \in N: 1 \leq i \leq 0\}$是$\varnothing$,即$f: X \rightarrow \varnothing$。
  如果X不是空集,那么则存在一个$x \in X$使得$f(x) \in \varnothing$,
  这显然是不成立的,所以X是空集

  \textcircled{2}必要性:X是空集,则X的基数为0。

  若X是空集,由习题3.3.3知$f:\varnothing \rightarrow \varnothing$为双射,
  而$\{i \in N : 1 \leq i \leq 0\}=\varnothing$,
  即存在双射函数$f:\varnothing \rightarrow \{i \in N : 1 \leq i \leq 0\}$,
  由定义3.6.5可知集合X基数为0.
\end{zproof}

\subsection*{3.6.3}
\begin{zproof}
  对n进行归纳:

  n=0时,f是空函数,命题空成立。

  归纳假设n=k时,命题成立。

  下面我们证明该命题对于k++也为真。
  设集合$N_k = \{i \in \mathsf{N}: 1 \leq i \leq k\}, N_{k++} = \{i \in \mathsf{N}: 1 \leq i \leq k++\}$。
  函数$f_{k++}: N_{k++} \rightarrow N$是一个函数,
  我们可以由$f_{k++}$定义出一个函数$f_k: N_k \rightarrow N$,对任意$i \in N_k, f_k(i) = f_{k++}(i)$。
  由归纳假设可知,存在一个自然数M使得$f_k(i) \leq M, i \in N_k$,即$f_{k++}(i) \leq M, i \in N_k$,
  此时我们可以取$f_{k++}(k++), M$中的较大值为$M^\prime$,
  由此可知该$M^\prime$使得$f_{k++}(i) \leq M^\prime, i \in N_{k++}$。归纳法完成。
\end{zproof}

\subsection*{3.6.4}

(a)设$X$是一个有限集,设$x$是一个对象并且$x$不是$X$中的元素。
那么$X \cup \{x\}$是有限的,且$\# (X \cup \{x\}) = \# (X) + 1$
\begin{zproof}
  $X$是有限集,不妨设$X$的基数是自然数n。因此存在从$X$到$\{i \in N: 1 \leq i \leq n\}$的双射函数$f$。
  定义出一个函数$g: X \cup \{x\} \rightarrow \{i \in N: 1 \leq i \leq n+1\}$,
  使得$g(x) = n+1$,$g(i) = f(i), i \in X$。
  g是双射函数,由此可知$X \cup \{x\}$的基数是n+1,
  所以$X \cup \{x\}$是有限的,且$\# (X \cup \{x\}) = \# (X) + 1$
\end{zproof}

(b)设$X$和$Y$都是有限集,那么$X \cup Y$是有限的,且$\# (X \cup Y) \leq \# (X) + \# (Y)$。
另外,如果$X$和$Y$是不相交的(即$X \cap Y = \varnothing$),那么$\# (X \cup Y) = \# (X) + \# (Y)$
\begin{zproof}
  $X$和$Y$都是有限集,不妨设$X$和$Y$的基数分别为n和m。
  通过对n进行归纳,完成证明;

  n=0时,即$X$的基数是0,也就是说$X = \varnothing$,$X \cup Y = \varnothing \cup Y = Y$,
  此时(b)命题显然是成立的。

  归纳假设n=k时,(b)命题成立。 

  现在需证明n=k++,任取$x \in Y, Z=Y \backslash \{x\}$,由引理3.6.9可知,$Z$的基数为k,
  由归纳假设可知,$X$与$Z$满足命题(b),由此可知$X \cup Z$是有限的;

  $X \cup Y = X \cup Z \cup \{x\}$,
  \textcircled{1} $X \cap Y = \varnothing$,由此可知$x \not\in X \cup Z$,且$\#(X \cup Z) = \#(X)+\#(Z)$。
  由命题(a)可知$X \cup Z \cup \{x\}$是有限的,且$\#(X \cup Z \cup \{x\})=\#(X \cup Z)+1$,
  即$X \cup Y$是有限的,且$\#(X \cup Y) = \#(X \cup Z)+1 = \#(X) + \#(Z) + 1 = \#(X) + \#(Y)$,
  即$\#(X \cup Y) = \#(X) + \#(Y)$;

  \textcircled{2} $X \cap Y \neq \varnothing$

  如果$x \in X \cup Z$则$X \cup Y=X \cup Z \cup \{x\}=X \cup Z$,即$X \cup Y=X \cup Z$
  由于同一集合只有几个基数,所以$\#(X \cup Y)=\#(X \cup Z) \leq \#(X)+ \#(Y)$。

  如果$x \not\in X \cup Z$,(由$X \cap Y \neq \varnothing$,则$X \cap Z \neq \varnothing$,所以$\#(X \cup Z) \leq \#(X) + \#(Z)$)由命题(a)可知$\#(X \cup Z \cup \{x\})=\#(X \cup Z)+1$,
  即$X \cup Y$是有限的,且$\#(X \cup Y) = \#(X \cup Z)+1 \leq \#(X) + \#(Z) + 1 = \#(X) + \#(Y)$,
  即$\#(X \cup Y) \leq \#(X) + \#(Y)$;

  综上,n=k++情况也成立,至此,(b)命题成立。
\end{zproof}

(c)设$X$是一个有限集,$Y$是$X$的一个子集。那么$Y$是有限的,且$\# (Y) \leq \# (X)$。
另外,如果$Y \neq X$(即$Y$是$X$的一个真子集),那么我们有$\#(Y) < \#(X)$。

\end{document}