\documentclass{article}
\usepackage{fontspec}
\usepackage[UTF8]{ctex}
\usepackage{amsthm}
\usepackage{mdframed}
\usepackage{xcolor}
\usepackage{amssymb}
\usepackage{amsmath}

\newmdtheoremenv[
  backgroundcolor=gray!10,
  linewidth=0pt,
  innerleftmargin=10pt,
  innerrightmargin=10pt,
  innertopmargin=10pt,
  innerbottommargin=10pt
]{zgraytheorem}{Theorem}

% 定义说明环境样式
\newtheoremstyle{mystyle}% 说明环境样式的名称
  {1em}% 上方间距
  {1em}% 下方间距
  {\normalfont}% 说明内容的字体样式
  {}% 缩进量
  {\bfseries}% 说明标记的字体样式
  {.}% 说明标记和说明内容之间的标点
  {1em}% 说明标记后的水平空间
  {}% 说明标记后的垂直空间
% 使用新定义的样式创建说明环境
\theoremstyle{mystyle}
\newtheorem*{zremark}{说明}

% 定义证明环境样式
\newtheoremstyle{zproofstyle}% 证明环境样式的名称
  {0.5em}% 上方间距
  {0.5em}% 下方间距
  {\itshape}% 证明内容的字体样式
  {}% 缩进量
  {\bfseries}% 证明标记的字体样式
  {.\newline}% 证明标记和证明内容之间的标点
  {1em}% 证明标记后的水平空间
  {}% 证明标记后的垂直空间

% 使用新定义的样式创建证明环境
\theoremstyle{zproofstyle}
\newtheorem*{zproof}{证明}

\begin{document}
\title{3.6 为什么}
\maketitle

\section*{注3.6.3}
\textcircled{1}单射

对任意$x_1 \in X, x_2 \in X, x_1 \neq x_2, f(x_1)=2x_1,f(x_2)=2x_2$,
乘法是交换的(引理2.3.2)如果$f(x_1)=f(x_2)$则$2x_1=2x_2$,
由乘法的消去律(推论2.3.7)可知,$x_1=x_2$,与题设矛盾,所以f是单射的。

\textcircled{2}满射

对任意$y \in Y$,由于Y是偶数集,所以Y总的元素都需要符合偶数的定义,
即:对任意的Y中元素y,当且仅当y=2n,n是自然数。由此可得f是满射。

\section*{注3.6.6}
需要找到$X=\{i \in N: i<n\} \rightarrow Y=\{i \in N: 1\leq i \leq n \}$的双射函数f.
我们定义$f: X \rightarrow Y, \{f(x): x \in X, f(x) = x ++\}$

现在证明f是双射函数。

\textcircled{1}单射

对任意$i_1 \in X, i_2 \in X, i_1 \neq i_2, f(i_1) = i_1++, f(i_2) = i_2++$,
若$f(i_1) = f(i_2)$,则$i_1++ = i_2++$,由洛必达公理2.4可知$i_1 = x_2$,与$i_1 \neq i_2$矛盾,
所以$f(i_1) \neq f(i_2)$,所以f是单射的

\textcircled{2}满射

对任意$y \in Y$,可知y是正数,而正数可以由一个自然数加1得到,假设$y = b++$,
又$y \leq n$,所以$b < n$,所以$b \in X$,所以f是满射

至此,命题得证


\section*{引理3.6.8}
\textbf{证明:不存在从空集到一个非空集合的双射}

由函数的满射定义可知,对值域中的任意元素y,定义域中都存在一个元素x,使得函数y=f(x),
而如果定义域是空集,那么x是不存在的,所以无法满足满射定义。

\section*{引理3.6.9}
\textbf{证明:定义的g函数是双射函数}

\begin{zgraytheorem}
  \begin{zremark}
    g构造书中说的不够直观,其实就是比$f(x)$小的,保持不变,比$f(x)$大的,向左平移1下,也就是$f(x)-1$。
    这样就能保证值域不超过n,且为$\{i \in N: 1 \leq i \leq n-1\}$
  \end{zremark}
\end{zgraytheorem}

\begin{zproof}

  不妨设$Z=X - \{x\}, Y_{n-1}=\{i \in N: 1 \leq i \leq n-1\},Y_{n}=\{i \in N: 1 \leq i \leq n\} $。

  \textcircled{1}g是单射

  对任意$x_1 \in Z, x_2 \in Z, x_1 \neq x_2$,假设$g(x_1) = g(x_2)$,
  由函数g的构造方式可知,要么$g(x_1)=g(x_2)=f(x_1)=f(x_2)$,
  要么$g(x_1)=g(x_2)=f(x_1)-1=f(x_2)-1$,这都与f是单射函数矛盾,所以$g(x_1) \neq g(x_2)$,
  所以g是单射的。

  \textcircled{2}g是满射

  对任意$y \in Y_{n-1}$,$y \in Y_n$,由于f是满射的,则存在$a \in X$使得$y=f(a)$,
  若$f(a) < f(x)$,那么$y=g(a)=f(a)$;
  若$f(a) \geq f(x)$,则由于f是单射,所以存在$b \in X$使得$y++=f(b)$,
  由$f(b) > f(a) \geq f(x)$可知$f(b) > f(x)$,所以$g(b)=f(b)-1=(y++)-1=y$;
  由此可知,对任意y都会存在Z中的元素i,使得$y=g(i)$。
  (特别说明下,y++不可能取到n++的,因为$y \in Y_{n-1}$)
  至此,g是满射得证。

  综上,命题得证。
\end{zproof}
\end{document}