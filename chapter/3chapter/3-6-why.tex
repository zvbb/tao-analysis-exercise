\documentclass{article}
\usepackage{fontspec}
\usepackage[UTF8]{ctex}
\usepackage{amsthm}
\usepackage{mdframed}
\usepackage{xcolor}
\usepackage{amssymb}
\usepackage{amsmath}

\newmdtheoremenv[
  backgroundcolor=gray!10,
  linewidth=0pt,
  innerleftmargin=10pt,
  innerrightmargin=10pt,
  innertopmargin=10pt,
  innerbottommargin=10pt
]{zgraytheorem}{Theorem}

% 定义说明环境样式
\newtheoremstyle{mystyle}% 说明环境样式的名称
  {1em}% 上方间距
  {1em}% 下方间距
  {\normalfont}% 说明内容的字体样式
  {}% 缩进量
  {\bfseries}% 说明标记的字体样式
  {.}% 说明标记和说明内容之间的标点
  {1em}% 说明标记后的水平空间
  {}% 说明标记后的垂直空间
% 使用新定义的样式创建说明环境
\theoremstyle{mystyle}
\newtheorem*{zremark}{说明}

% 定义证明环境样式
\newtheoremstyle{zproofstyle}% 证明环境样式的名称
  {0.5em}% 上方间距
  {0.5em}% 下方间距
  {\itshape}% 证明内容的字体样式
  {}% 缩进量
  {\bfseries}% 证明标记的字体样式
  {.\newline}% 证明标记和证明内容之间的标点
  {1em}% 证明标记后的水平空间
  {}% 证明标记后的垂直空间

% 使用新定义的样式创建证明环境
\theoremstyle{zproofstyle}
\newtheorem*{zproof}{证明}

\begin{document}
\section{3.6 为什么}

\section*{注3.6.3}
\textcircled{1}单射

对任意$x_1 \in X, x_2 \in X, x_1 \neq x_2, f(x_1)=2x_1,f(x_2)=2x_2$,
乘法是交换的(引理2.3.2)如果$f(x_1)=f(x_2)$则$2x_1=2x_2$,
由乘法的消去律(推论2.3.7)可知,$x_1=x_2$,与题设矛盾,所以f是单射的。

\textcircled{2}满射

对任意$y \in Y$,由于Y是偶数集,所以Y总的元素都需要符合偶数的定义,
即:对任意的Y中元素y,当且仅当y=2n,n是自然数。由此可得f是满射。

\section*{注3.6.6}
需要找到$X=\{i \in N: i<n\} \rightarrow Y=\{i \in N: 1\leq i \leq n \}$的双射函数f.
我们定义$f: X \rightarrow Y, \{f(x): x \in X, f(x) = x ++\}$

现在证明f值域是Y, f是双射函数。

(1)f的值域是Y

若n=0,则X与Y都是空集,无需说明。

若n>0时,对任意$i \in X$,有$i < n$,由自然数序的定义(定义2.2.11)可知$i++ \leq n$
(其实通过定义无法直接获得该结论,习题2.2.3中有证明),
若i的最小值是0,有$f(0) = 1,1 \leq 1$,即:$1 \leq f(i) \leq n$,所以f的值域为Y。

(2)f是双射函数

\textcircled{1}单射

对任意$i_1 \in X, i_2 \in X, i_1 \neq i_2, f(i_1) = i_1++, f(i_2) = i_2++$,
若$f(i_1) = f(i_2)$,则$i_1++ = i_2++$,由洛必达公理2.4可知$i_1 = x_2$,与$i_1 \neq i_2$,
所以$f(i_1) \neq f(i_2)$,所以f是单射的

\textcircled{2}满射

对任意$y \in Y$,可知y是正数,而正数可以由一个自然数加1得到,假设$y = b++$,
又$y \leq n$,所以$b < n$,所以$b \in X$,所以f是满射

至此,命题得证

\section*{补充}

\end{document}
