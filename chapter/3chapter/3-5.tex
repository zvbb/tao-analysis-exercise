\documentclass{article}
\usepackage{fontspec}
\usepackage[UTF8]{ctex}
\usepackage{amsthm}
\usepackage{mdframed}
\usepackage{xcolor}
\usepackage{amssymb}
\usepackage{amsmath}

\newmdtheoremenv[
  backgroundcolor=gray!10,
  linewidth=0pt,
  innerleftmargin=10pt,
  innerrightmargin=10pt,
  innertopmargin=10pt,
  innerbottommargin=10pt
]{zgraytheorem}{Theorem}

% 定义说明环境样式
\newtheoremstyle{mystyle}% 说明环境样式的名称
  {1em}% 上方间距
  {1em}% 下方间距
  {\normalfont}% 说明内容的字体样式
  {}% 缩进量
  {\bfseries}% 说明标记的字体样式
  {.}% 说明标记和说明内容之间的标点
  {1em}% 说明标记后的水平空间
  {}% 说明标记后的垂直空间
% 使用新定义的样式创建说明环境
\theoremstyle{mystyle}
\newtheorem*{zremark}{说明}

% 定义证明环境样式
\newtheoremstyle{zproofstyle}% 证明环境样式的名称
  {0.5em}% 上方间距
  {0.5em}% 下方间距
  {\itshape}% 证明内容的字体样式
  {}% 缩进量
  {\bfseries}% 证明标记的字体样式
  {.\newline}% 证明标记和证明内容之间的标点
  {1em}% 证明标记后的水平空间
  {}% 证明标记后的垂直空间

% 使用新定义的样式创建证明环境
\theoremstyle{zproofstyle}
\newtheorem*{zproof}{证明}

\begin{document}
\title{3.5习题}
\maketitle
\subsection*{3.5.5}

\begin{zgraytheorem}
  \begin{zremark}
    按照定义证明即可
  \end{zremark}
\end{zgraytheorem}
\begin{zproof}
  % 证明1
  \textcircled{1}$(A \times B) \cap (C \times D) = (A \cap C) \times (B \cap D)$ 证明

  令$Z = (A \times B) \cap (C \times D)$,$Z^\prime=(A \cap C) \times (B \cap D)$
  现在我们只需证明属于Z中的元素也属于$Z^\prime$,反之亦然。

  对任意$(x,y) \in Z$ 那么 $(x,y)\in A \times B$ 且 $(x,y)\in C \times D$,所以
  $x \in A , x \in C$,$y \in B , y \in D$,由定义可知$(x,y) \in Z^\prime$。

  反之,对任意$(x,y) \in Z^\prime$ 那么 $(x,y)\in A \cap C$ 且 $(x,y)\in B \cap D$,所以
  $x \in A , x \in C$,$y \in B , y \in D$,由定义可知$(x,y) \in Z$。

  % 证明2
  剩下的证明类似,故略
\end{zproof}

\subsection*{3.5.6}
\begin{zproof}
  \textcircled{1}$A \times B \subseteq C \times D$ 当且仅当 $A \subseteq C$且$B \subseteq D$ 证明

  先证明 $A \times B \subseteq C \times D \implies A \subseteq C,B \subseteq D$
  任意$x \in A, y \in B \implies (x,y) \in A \times B$ 又 $A \times B \subseteq C \times D$
  所以 $(x,y) \in C \times D$,所以 $x \in C , y \in D$,由此可知对任意$x \in A \implies x \in C$,
  $y \in B \implies y \in D$,所以 $A \subseteq C$且$B \subseteq D$。

  再证明 $A \subseteq C,B \subseteq D \implies A \times B \subseteq C \times D$ 任意$(x,y) \in A \times B$
  所以 $x \in A , y \in B$,由$A \subseteq C,B \subseteq D$ 知 $x \in C , y \in D$ 那么 $(x,y) \in C \times D$,
  所以 $A \times B \subseteq C \times D$

  \textcircled{2}$A \times B = C \times D$ 当且仅当 $A = C$且$B = D$ 证明

  先证明 $A \times B = C \times D \implies A = C,B = D$。
  因 $A \times B = C \times D$ 由 \textcircled{1}知$A \subseteq C,B \subseteq D$,
  由集合相等的对称性可知$C \times D = A \times B$,所以$B \subseteq A,D \subseteq B$,
  综上$A = C$且$B = D$

  类似证明$A = C,B = D \implies A \times B = C \times D$。

  \textcircled{3}去掉空集的限制,空集和自然数0的效果很类似,
  上面的\textcircled{1}\textcircled{2}都不再成立
\end{zproof}

\subsection*{3.5.7}
\begin{zgraytheorem}
  \begin{zremark}
    证明唯一性,常见思路是先定义出目标对象,再证明其唯一性,
    即证明其他满足条件的对象,都与目标对象相等
  \end{zremark}
\end{zgraytheorem}
\begin{zproof}
  定义$h : Z \rightarrow X \times Y, h(z) := (f(z),g(z))$
  \newline
  由h的定义,显然$\pi_{X \times Y \rightarrow X} \circ h = f$且$\pi_{X \times Y \rightarrow Y} \circ h = g$
  现在证明其唯一性。假设存在另一个函数$h^\prime$满足$\pi_{X \times Y \rightarrow X} \circ h^\prime = f$且$\pi_{X \times Y \rightarrow Y} \circ h^\prime = g$,
  现需证明$h=h^\prime$,我们要说明对任意z有$h(z)=h^\prime(z)$。
  设$h(z) = (f(z), g(z)) = (x, y)$ $h^\prime(z) = (x^\prime, y^\prime)$
  由$\pi_{X \times Y \rightarrow X} \circ h = f$ 和 $\pi_{X \times Y \rightarrow X}(x,y):=x$知
  $\pi_{X \times Y \rightarrow X} \circ h(z) = x = f(z)$
  同理 $\pi_{X \times Y \rightarrow X} \circ h^\prime(z) = x^\prime = f(z)$,
  所以 $x=x^\prime$
  同理 $y=y^\prime$,
  综上对任意z有$h(z)=h^\prime(z)$
  那么由函数的相等定义,有$h^\prime = h$,唯一性得到证明
\end{zproof}

\subsection*{3.5.8}
\begin{zproof}
  如果每一个$X_i$都是非空集合,由引理3.5.12可知,集合$\prod\limits_{1\leqslant i \leqslant n}X_i$ 也是非空的,
  所以$\prod\limits_{1\leqslant i \leqslant n}X_i$为空至少有一个$X_i$为空。

  如果有一个$X_i$为空,由笛卡尔积的定义,$1\leqslant i \leqslant n$的$x_i$不存在,所以$\prod\limits_{1\leqslant i \leqslant n}X_i$为空。

  综上,命题得证
\end{zproof}

\subsection*{3.5.9}
\begin{zgraytheorem}
  \begin{zremark}
    按照集合相等的定义证明即可
  \end{zremark}
\end{zgraytheorem}
\begin{zproof}
  任意$x \in [(\bigcup \limits_{\alpha \in I}A_\alpha) \bigcap (\bigcup \limits_{\beta \in J}B_\beta)]$
  $\Rightarrow$ 存在 $\alpha \in I$ 使得 $x \in A_\alpha$ 且 存在 $\beta \in J$ 使得 $x \in B_\beta$,
  由此可知$(\alpha,\beta) \in I \times J, x \in (A_\alpha \bigcap B_\beta)$,
  所以$x \in \bigcup \limits_{(\alpha,\beta) \in I \times J}(A_\alpha \bigcap B_\beta)$

  任意$x \in \bigcup \limits_{(\alpha,\beta) \in I \times J}(A_\alpha \bigcap B_\beta)$
  $\Rightarrow$ 存在 $(\alpha,\beta) \in I \times J, x \in (A_\alpha \bigcap B_\beta)$,
  由此可知存在 $\alpha \in I$ 使得 $x \in A_\alpha$ 且 存在 $\beta \in J$ 使得 $x \in B_\beta$,
  所以$x \in [(\bigcup \limits_{\alpha \in I}A_\alpha) \bigcap (\bigcup \limits_{\beta \in J}B_\beta)]$

  综上,命题得证
\end{zproof}

\subsection*{3.5.10}
\begin{zgraytheorem}
  \begin{zremark}

  \end{zremark}
\end{zgraytheorem}
\begin{zproof}
  \textcircled{1}先证明函数相等$\Rightarrow$图相等

  假设两个函数$f: X \rightarrow Y$ 和 $\tilde{f}  : X \rightarrow Y$相等,
  那么由函数的相等定义,有任意$x \in X, f(x) = \tilde{f}(x)$,由图的定义可知图是一个集合,
  又$(x, f(x)) \in f$的图,$(x, \tilde{f}(x)) \in \tilde{f}$的图,且$f(x) = \tilde{f}(x)$,
  所以两函数的图相等。

  证明图相等$\Rightarrow$函数相等。

  假设两函数$f , \tilde{f}$的图相等。
  对任意$x \in X$,有 $(x, f(x)) \in f$的图,由图相等可知$(x, f(x)) \in \tilde{f}$的图。

  同理:$(x, \tilde{f}(x)) \in \tilde{f}$的图,$(x, \tilde{f}(x)) \in f$的图,

  假设两个函数不相等,应该存在$x_{0}, f(x_{0}) \neq \tilde{f}(x_{0})$,但由之前的说明可知,
  $(x_0, f(x_0)) \in \tilde{f}$的图,所以存在$(x_1, \tilde{f}(x_1))=(x_0,f(x_0)) \in \tilde{f}$的图,
  有有序对相等的定义可知$x_0=x_1, f(x_0) = \tilde{f}(x_1)$,
  而由$x_0=x_1$,可以得到$f(x_0) = \tilde{f}(x_1) = \tilde{f}(x_0)$
  这与$f(x_{0}) \neq \tilde{f}(x_{0})$矛盾,所以假设不成立

  综上,命题得证

  \textcircled{2}
  先定义函数f: $X \rightarrow Y$,其性质为$(x,y) \in G$。
  由题设“子集G具有下述性质:对每一个$x \in X$,集合{$y \in Y:(x,y) \in G$}中恰好有一个元素”,
  可知这里定义的y是存在且唯一,满足函数定义。由f的构造方式知,f的图与G相等(这里不做证明了)。

  现在证明f的唯一性。

  假设存在另外一个函数$f^\prime: X \rightarrow Y$,它的图与G相等。
  那么对任意$x \in X$,由图的定义可知$(x, f^\prime(x)) \in f^\prime$的图,
  因为$f^\prime$的图与G相等,所以$(x, f^\prime(x)) \in G$,
  由题设“子集G具有下述性质:对每一个$x \in X$,集合{$y \in Y:(x,y) \in G$}中恰好有一个元素”,
  所以这里的$y=f^\prime(x)$,有由f的定义可知$(x,f(x)) \in G$,由y的唯一性可知,$f^\prime(x)=f(x)$,
  所以$f=f^\prime$

  综上所述,函数f唯一。
\end{zproof}

\subsection*{3.5.11}
\begin{zgraytheorem}
  \begin{zremark}
    题目中的提示已经说明了证明思路
  \end{zremark}
\end{zgraytheorem}

\begin{zproof}
  \textcircled{1}对任意两个集合X和Y,利用引理 3.4.9 和分类公理构造出由$X \times Y$的一切子集组成的集合,它满足垂线测试。

  由引理3.4.9知存在集合$\{a : a \in X \times Y\}$,即$X\times Y$的所有子集构成的集合A,
  有分类公里得到$\{b \in A : b\text{满足垂线测试} \}$集合B。

  \textcircled{2}利用3.5.10和替代公理构造出一个集合,该集合与公理3.10相同。如下:
  \newline
  $f_G := \{f: \alpha \in B, \text{函数f是定义域为X值域为Y,f的图与}\alpha \text{相同}\}$
  \newline
  现在我们只需证明$f_G$集合与公理3.10描述的集合$f_{ps} := \{ f: \text{f是一个定义域X且值域为Y的函数}\}$相等。

  若$f_x \in f_G$,那么函数$f_x$的定义域是$X$,值域是$Y$,所以$f_x \in f_{ps}$,

  若$f_x \in f_{ps}$,那么函数$f_x$的定义域是$X$,值域是$Y$,
  又$f_x \text{的图}\{(x, f_x(x)), x \in X\}$为$\beta$,由函数的定义可知$\beta$满足垂线测试,
  所以$\beta \in B$。由此可以得到一个函数$f_x^\prime: X \rightarrow Y$它的图为$\beta$,
  由3.5.10可知两函数$f_x$和$f_x^\prime$相等,又$f_x^\prime \in f_G$,
  所以$f_x \in f_G$。

  综上所述,命题得证。
\end{zproof}

\subsection*{3.5.12}
\begin{zgraytheorem}
  \begin{zremark}
    证明思路:先证明提示中的命题,然后证明3.5.12
  \end{zremark}
\end{zgraytheorem}

\begin{zproof}
  \textcircled{1}证明提示中的命题

  以归纳法证明该命题。

  归纳基始,当N=0时 ,存在唯一的函数$a_0: \{0\} \rightarrow N$,$a_0(0)=c$。
  因为小于0的自然数不存在,
  所以$a_0(n++) = f(n,a(n)) \text{对所有满足n < 0 的 n} \in \mathsf{N} \text{均成立}$这一点空成立
  。由$a_0$的定义可知,$a_0$是唯一的。

  归纳假设N=k时,提示中的命题成立。

  现需证明N=k++时,提示中的命题成立。
  先尝试定义出函数$a_{k++}: \{n \in \mathsf{N}: n \leq k \} \rightarrow \mathsf{N}$k++ = f(k, a(k))
  当n<k时,$a_k(n) = a_{k++}(n), a_{k++}(k++) = f(k, a(k))$。

  现要证明函数$a_{k++}$的唯一性。假设存在另一个函数$a_j$,则存在一个x,使得$a_{k++}(x) \neq a_j(x)$,
  若x属于函数$a_k$的定义域,那么与归纳假设$a_k$的唯一性矛盾。若x=k++,由函数的定义可知,函数f对同一个对象,
  不可能有多个函数值,所以$x = k++$的情形不存在。由上述可知$a_j$是不存在的,所以$a_{k++}$是唯一的。

  综上所述,提示中的命题成立。(有一点说明,上面提及的所有函数的存在性是由定义保证的)

  \textcircled{2}证明3.5.12
  利用提示中的命题,定义出函数a,并证明其唯一性。

  函数$a: \mathsf{N} \rightarrow \mathsf{N}$,值域为$\{a_x(x): x \in \mathsf{N}\}$。
  由$a(0)=a_0(0)=c, a(n++) = a_{n++}(n++) = f(n, a_{n})$,可知定义的函数a是满足要求的。

  现在证明a的唯一性。

  假设存在一个函数$a^\prime$满足题设,又$a \neq a^\prime$。
  由函数的相等定义可知,若两个函数不相等,则存在一个自然数x,使得$a(x) \neq a^\prime(x)$,
  当x=0,因为$a, a^\prime $都满足题设,所以$a(0)=a^\prime(0)=c$,
  当$x \neq 0$时,即x是正数时,由于对任意正数都可以由一个自然数加1得到,我们假设$x = \alpha ++$,
  那么$a(\alpha ++) = f(\alpha, a(\alpha)) = a^\prime(\alpha ++) = f(\alpha, a^\prime (\alpha))$
  (对n进行归纳,可以证明该式子的正确性,这里不做证明了),所以这样的x不存在,到这里唯一性得到了证明。

  综上,命题得证。


  另一个挑战,不证明了,看着就让人emo

\end{zproof}

\subsection*{3.5.13}
\begin{zgraytheorem}
  \begin{zremark}
    证明存在性,有两种常见思路:

    一种是按照定义定义出目标对象;另一种是构造出目标对象,然后证明其符合定义。
    在本题中,我们使用后一种方法。
  \end{zremark}
\end{zgraytheorem}

\begin{zproof}
  定义函数$f: \mathsf{N} \rightarrow \mathsf{N}^\prime, f(0)=0^\prime$
  且对任意的$n \in \mathsf{N}, n^\prime \in \mathsf{N}^\prime, f(n) = n^\prime$,
  当且仅当$f(n++)=n^\prime++^\prime$

  \textcircled{1}证明定义的函数,符合函数定义。
  首先$\mathsf{N},\mathsf{N}^\prime$都是集合,
  我们要说明对任意定义域中的元素,在值域中可以找到唯一的元素与之对应。
  这里我们对n进行归纳,来证明这一点。

  归纳基始n=0, $f(0) = 0^\prime$,因$\mathsf{N}^\prime$也满足皮亚洛公理,由此可知$0^\prime$的唯一性,所以n=0满足定义;

  归纳假设$n=x,f(x) = x^\prime$,$x^\prime$是唯一的。

  现需$n=x++$满足定义,由函数f可知$f(x++) = x^\prime ++$,而2.4公理保证了$x^\prime ++$的唯一性。

  至此可知,函数f是满足函数定义的。

  \textcircled{2}证明函数f是双射的。

  \textcircled{2.1}单射

  归纳证明$f(x_1) = f(x_2)$,则$x_1 = x_2$

  归纳基始,假设$f(x_1)=f(x_2)=0^\prime$,那么$x_1=x_2=0$
  (
  这里简单说下原因,
  如果$x_1$不等于0,说明$x_1$是正数,而正数可以由一个自然数加1得到,这里假设$x_1=a++$
  由此可得到$f(a) = a^\prime,f(a++)=a^\prime++^\prime = 0^\prime$,
  这与洛必达公理2.3矛盾,所以$x_1$=0是必须的。
  )

  归纳假设$f(x_1) = f(x_2) = n^\prime$,则$x_1 = x_2$

  现需证明$f(x_1) = f(x_2) = n^\prime++\prime$,则$x_1 = x_2$;
  因为$x_1,x_2$都是正数(归纳基始中已说明原因),
  所以存在a,b使得$x_1 = a++, x_2 = b++$,由归纳假设可知$f(a) = f(b) = n^\prime$,
  所以a=b,
  所以$x_1 = x_2$。

  由此可知: f单射

  \textcircled{2.2}满射
  归纳证明对任意$n^\prime \in \mathsf{N}^\prime$,有$n \in \mathsf{N}$使得$f(n)=n^\prime$

  归纳基始,当$n^\prime=0^\prime$时,有$f(0) = 0^\prime$;

  归纳假设,当$n^\prime=x^\prime$时,有$f(x)=x^\prime$;

  现需证明,当$n^\prime=x^\prime++^\prime$时,有$n \in \mathsf{N}$使得$f(n)=x^\prime++^\prime$;
  由归纳假设可知$f(x)=x^\prime$,所以$x++ \in \mathsf{N}$,由f的定义可知$f(x++)=x^\prime++^\prime$。

  由此可知: f满射

  综上,命题得证

\end{zproof}
\end{document}
