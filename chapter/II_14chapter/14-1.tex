\documentclass{article}
\usepackage{mathtools} 
\usepackage{fontspec}
\usepackage[UTF8]{ctex}
\usepackage{amsthm}
\usepackage{mdframed}
\usepackage{xcolor}
\usepackage{amssymb}
\usepackage{amsmath}


% 定义新的带灰色背景的说明环境 zremark
\newmdtheoremenv[
  backgroundcolor=gray!10,
  % 边框与背景一致,边框线会消失
  linecolor=gray!10
]{zremark}{说明}


\begin{document}
\title{14.1 习题}
\author{张志聪}
\maketitle

\section*{14.1.1}
修改下证明顺序。

(1)如果极限$\lim\limits_{x \to x_0; x \in E} f(x)$存在,那么它一定等于$f(x_0)$。

反证法,假设$\lim\limits_{x \to x_0; x \in E} f(x) = L, L \neq f(x_0)$。
因为极限$\lim\limits_{x \to x_0; x \in E} f(x) = L$可知,
设$0 < \epsilon < d_Y(f(x_0), L)$,存在$\delta > 0$,使得只要$x \in E$满足$d_X(x, x_0) < \delta$,
就有$d_Y(f(x), L) < \epsilon$。
因为$x_0 \in E, d_X(x_0, x_0) = 0 < \delta$,
即$x = x_0$时$d_Y(f(x), L) > \epsilon$,存在矛盾。



(2)证明:极限$\lim\limits_{x \to x_0; x \in E} f(x)$存在,
当且仅当极限$\lim\limits_{x \to x_0; x \in E \setminus \{x_0\}} f(x)$存在且等于$f(x_0)$。

\begin{itemize}
  \item $\Rightarrow$

        极限$\lim\limits_{x \to x_0; x \in E} f(x)$存在,按照定义14.1.1可知,
        极限$\lim\limits_{x \to x_0; x \in E \setminus \{x_0\}} f(x)$存在。接下来,
        需要证明$\lim\limits_{x \to x_0; x \in E \setminus \{x_0\}} f(x) = f(x_0)$。

        反证法,假设$\lim\limits_{x \to x_0; x \in E \setminus \{x_0\}} f(x) = L, f(x_0) \neq L$。

        由(1)可知,$\lim\limits_{x \to x_0; x \in E} f(x) = f(x_0)$。

        那么,设$\epsilon = \frac{1}{2} d_Y(f(x_0), L)$,存在$\delta^\prime > 0$,
        使得只要$x \in E$满足$d_X(x, x_0) < \delta^\prime$, 就有$d_Y(f(x), f(x_0)) < \epsilon$。

        存在$\delta^{\prime\prime} > 0$,
        使得只要$x \in E$满足$0 < d_X(x, x_0) < \delta^{\prime\prime}$, 就有$d_Y(f(x), L) < \epsilon$

        综上,取$\delta = min(\delta^\prime, \delta^{\prime\prime})$,
        使得只要$x \in E$满足$0 < d_X(x, x_0) < \delta$,就有
        \begin{equation*}
          \begin{cases*}
            d_Y(f(x), f(x_0)) < \epsilon \\
            d_Y(f(x), L) < \epsilon
          \end{cases*}
        \end{equation*}
        于是可得
        \begin{align*}
          d_Y(f(x_0), L) \leq d_Y(f(x), f(x_0)) + d_Y(f(x), L) < d_Y(f(x_0), L)
        \end{align*}
        存在矛盾。

  \item $\Leftarrow$

        按照定义14.1.1可以直接证明,具体过程略。
\end{itemize}

\section*{14.1.2}

\begin{itemize}
  \item $(a) \Leftrightarrow (b)$

        与定理13.1.4证明相似,不做赘述。


  \item $(a) \implies (c)$

        因为$V$是开集且$L \in V$,所以存在$r > 0$使得$B_{(Y, d_Y)}(L, r) \subseteq V$。
        因为(a)成立,所以存在$\delta > 0$使得只要$x \in E$满足$d_X(x, x_0) < \delta$,
        就有$d_Y(f(x), L) < r$。

        令$U := B_{(X, d_X)}(x_0, \delta), U \subset X$。
        对任意$x \in U \cap E$,都有$x \in E$且$d_X(x, x_0) < \delta$,
        于是$d_Y(f(x), L) < r$,即$f(x) \in B_{(Y, d_Y)}(L, r) \subseteq V$。
        所以,$f(U \cap E) \subseteq V$。

  \item $(c) \implies (a)$

        设$(x^{(n)})_{n = 1}^\infty$是$E$中依度量$d_X$收敛于$x_0$的序列。
        任意$\epsilon > 0$,
        令$V := B_{(Y, d_Y)}(L, \epsilon)$,由(c)可知,存在一个包含$x_0$的开集$U \subset X$,
        使得$f(U \cap E) \subseteq V$。

        因为$U$是开集,所以存在$\delta > 0$使得$B_{(X, d_X)}(x_0, \delta) \subseteq U$。
        序列$(x^{(n)})_{n = 1}^\infty$是$E$中依度量$d_X$收敛于$x_0$的序列,
        所以存在$N \geq 1$使得
        \begin{align*}
          d_X(x_0, x^{(n)}) < \delta
        \end{align*}
        对所有的$n \geq N$均成立。

        于是对$n \geq N$,有$x^{(n)} \in B_{(X, d_X)}(x_0, \delta) \subseteq U, x^{(n)} \in E$,
        此时,
        \begin{align*}
          f(x^{(n)}) \in V
        \end{align*}
        即,对任意$n \geq N$都有
        \begin{align*}
          d_Y(f(x^{(n)}), L) < \epsilon
        \end{align*}
        由$\epsilon$的任意性可知,$(f(x^{(n)}))_{n = 1}^\infty$收敛于$L$。

  \item $(a) \implies (d)$

        (a)成立,那么,对任意$\epsilon > 0$,都存在$\delta > 0$使得只要
        $x \in E$满足$d_X(x, x_0) < \delta$,就有$d_Y(f(x), L) < \epsilon$。
        因为$x \in E \setminus \{x_0\}$时$g(x) = f(x)$,所以,以上性质函数$g$也成立。

        现在只需再额外考虑$x = x_0$是否满足定义要求即可。
        $d_X(x_0, x_0) = 0 < \delta$,此时
        \begin{align*}
          d_Y(g(x), L) = d_Y(g(x_0), L) = d_Y(L, L) = 0 < \epsilon
        \end{align*}
        于是可得
        \begin{align*}
          \lim\limits_{x \to x_0; x \in E \cup \{x_0\}} g(x) = L = g(x_0)
        \end{align*}
        所以,$g$在$x_0$处是连续的。

        特别地,$x \in E$,由习题14.1.1可知$f(x_0) = L$。

  \item $(d) \implies (a)$

        如果$x_0 \notin E$,则$E \setminus \{x_0\} = E$,由$g$在$x_0$处连续,我们有
        \begin{align*}
          \lim\limits_{x \to x_0; x \in E \setminus \{x_0\}} g(x) = \lim\limits_{x \to x_0; x \in E} g(x) = \lim\limits_{x \to x_0; x \in E} f(x) = g(x_0) = L
        \end{align*}

        如果$x_0 \in E$,由$g$在$x_0$处连续,我们有
        \begin{align*}
          \lim\limits_{x \to x_0; x \in E \setminus \{x_0\}} g(x) = \lim\limits_{x \to x_0; x \in E \setminus \{x_0\}} f(x) = g(x_0) = f(x_0)
        \end{align*}
        利用习题14.1.1可知,
        \begin{align*}
          \lim\limits_{x \to x_0; x \in E} f(x) = f(x_0)
        \end{align*}




\end{itemize}


\end{document}