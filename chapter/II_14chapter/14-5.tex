\documentclass{article}
\usepackage{mathtools} 
\usepackage{fontspec}
\usepackage[UTF8]{ctex}
\usepackage{amsthm}
\usepackage{mdframed}
\usepackage{xcolor}
\usepackage{amssymb}
\usepackage{amsmath}


% 定义新的带灰色背景的说明环境 zremark
\newmdtheoremenv[
  backgroundcolor=gray!10,
  % 边框与背景一致,边框线会消失
  linecolor=gray!10
]{zremark}{说明}


\begin{document}
\title{14.5 习题}
\author{张志聪}
\maketitle

\section*{14.5.1}

\begin{itemize}
  \item (a) 有界函数的有限和是有界的。

        成立;

        对有限和的个数$n$进行归纳。

        $n = 0$时,空虚的为真;

        $n = 1$时,$\sum\limits_{n=1}^1 f^{(n)} = f^{(1)}$,因为$f^{(1)}$是有界的,
        于是命题为真;

        归纳假设$n = k$时,$\sum\limits_{n=1}^k f^{(n)}$是有界的;

        $n = k + 1$时,
        \begin{align*}
          \sum\limits_{n=1}^{k + 1} f^{(n)} = \sum\limits_{n=1}^k f^{(n)} + f^{(k + 1)}
        \end{align*}
        由归纳假设可知$\sum\limits_{n=1}^k f^{(n)}$是有界的,又因为$f^{(k + 1)}$是有界的,
        于是存在$M, M^\prime > 0$,使得只要$x \in X$,都有
        \begin{align*}
          |\sum\limits_{n=1}^k f^{(n)} (x)| < M \\
          |f(x)| < M^\prime
        \end{align*}
        于是可得
        \begin{align*}
          |\sum\limits_{n=1}^{k + 1} f^{(n)}| \leq M + M^\prime
        \end{align*}
        综上可得,$\sum\limits_{n=1}^{k + 1} f^{(n)}$有界;

        归纳完成,命题成立。

  \item (b) 连续函数的有限和是连续的。

        成立;证明与(a)类似,略。

  \item (c) 一致连续函数的有限和是一致连续的。

        成立;证明与(a)类似,略。
\end{itemize}

\section*{14.5.2}
按照书中提示证明。

由题设可知,任意$n$都有$f^{(n)} \in C(X \to \mathbb{R})$,
由因为$\sum \limits_{n=1}^{\infty} ||f^{(n)}||_\infty$收敛,
于是可知$\sum \limits_{n=1}^{\infty} ||f^{(n)}||_\infty$是柯西序列,
对任意$\epsilon > 0$,存在$N \geq 1$,使得只要$p, q  \geq N, p \leq q$都有
\begin{align*}
   & \sum \limits_{n=1}^{q} ||f^{(n)}||_\infty - \sum \limits_{n=1}^{p} ||f^{(n)}||_\infty \\
   & = ||f^{(p)}||_\infty + ||f^{(p+1)}||_\infty + \cdots + ||f^{(q)}||_\infty             \\
   & < \epsilon
\end{align*}
又我们有
\begin{align*}
   & d_{B(X \to \mathbb{R})}(\sum \limits_{n=1}^{p} f^{(n)}, \sum \limits_{n=1}^{q} f^{(n)}) \\
   & = \sup \{|\sum \limits_{n=p+1}^{q} f^{(n)}(x)|: x \in X\}                               \\
   & \leq ||f^{(p)}||_\infty + ||f^{(p+1)}||_\infty + \cdots + ||f^{(q)}||_\infty            \\
   & < \epsilon
\end{align*}
于是可得,
部分和构成的序列$(\sum \limits_{n=1}^{N} f^{(n)})_{N = 1}^\infty$是$C(X \to \mathbb{R})$中的柯西序列,
利用定理14.4.5可知,$(\sum \limits_{n=1}^{N} f^{(n)})_{N = 1}^\infty$收敛于$C(X \to \mathbb{R})$中的一个函数$f$,
由命题14.4.4可知,$(\sum \limits_{n=1}^{N} f^{(n)})_{N = 1}^\infty$一致收敛于$f$,
所以,无限级数$\sum \limits_{n=1}^{\infty} f^{(n)}$一致收敛于$f$。

\end{document}