\documentclass{article}
\usepackage{mathtools} 
\usepackage{fontspec}
\usepackage[UTF8]{ctex}
\usepackage{amsthm}
\usepackage{mdframed}
\usepackage{xcolor}
\usepackage{amssymb}
\usepackage{amsmath}


% 定义新的带灰色背景的说明环境 zremark
\newmdtheoremenv[
  backgroundcolor=gray!10,
  % 边框与背景一致,边框线会消失
  linecolor=gray!10
]{zremark}{说明}


\begin{document}
\title{14.6 习题}
\author{张志聪}
\maketitle

\section*{14.6.1}

设$\sum\limits_{n=1}^{\infty} f^{(n)}$一致收敛于$f$,
即部分和$\sum\limits_{n=1}^{N} f^{(n)}$一致收敛于$f$。
于是由定理14.6.1可知,
\begin{align*}
   & \lim\limits_{N \to \infty} \int_{[a, b]} \sum\limits_{n=1}^{N} f^{(n)}   \\
   & = \int_{[a, b]} \lim\limits_{N \to \infty} \sum\limits_{n=1}^{N} f^{(n)} \\
   & = \int_{[a, b]} f                                                        \\
   & = \int_{[a, b]} \sum\limits_{n=1}^{\infty} f^{(n)}                       \\
\end{align*}
(注意:最后一个等式,仔细观察定义14.5.2即可得到。)

又对任意$\epsilon > 0$,
存在$N \geq 1$使得对于所有的$k > N$和所有的$x \in [a, b]$都有
\begin{align*}
   & |\sum\limits_{n=1}^{k} f^{(n)} - f(x)| < \epsilon                                         \\
   & \sum\limits_{n=1}^{k} f^{(n)} - \epsilon < f(x) < \sum\limits_{n=1}^{k} f^{(n)} +\epsilon
\end{align*}
上式两端在$[a, b]$上求积分可得,
\begin{align*}
   & \int_{[a, b]} (\sum\limits_{n=1}^{k} f^{(n)} - \epsilon) \leq \int_{[a, b]} f \leq \int_{[a, b]} (\sum\limits_{n=1}^{k} f^{(n)} + \epsilon)           \\
   & \sum\limits_{n=1}^{k} \int_{[a, b]} f^{(n)} - \epsilon(b - a) \leq \int_{[a, b]} f \leq \sum\limits_{n=1}^{k} \int_{[a, b]} f^{(n)} + \epsilon(b - a) \\
\end{align*}
上面的论述证明了对任意的$\epsilon > 0$,存在一个$N \geq 1$使得对所有的$k \geq N$都有
\begin{align*}
  \left|\sum\limits_{n=1}^{k} \int_{[a, b]} f^{(n)} - \int_{[a, b]} f \right| \leq 2\epsilon(b - c)
\end{align*}
由于$\epsilon$是任意的,因此$\sum\limits_{n=1}^{N} \int_{[a, b]} f^{(n)}$收敛于$\int_{[a, b]} f$。
综上可得,
\begin{align*}
  \sum\limits_{n=1}^{\infty} \int_{[a, b]} f^{(n)} = \int_{[a, b]} f = \int_{[a, b]} \sum\limits_{n=1}^{\infty} f^{(n)}
\end{align*}


\end{document}