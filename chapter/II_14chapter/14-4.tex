\documentclass{article}
\usepackage{mathtools} 
\usepackage{fontspec}
\usepackage[UTF8]{ctex}
\usepackage{amsthm}
\usepackage{mdframed}
\usepackage{xcolor}
\usepackage{amssymb}
\usepackage{amsmath}


% 定义新的带灰色背景的说明环境 zremark
\newmdtheoremenv[
  backgroundcolor=gray!10,
  % 边框与背景一致,边框线会消失
  linecolor=gray!10
]{zremark}{说明}


\begin{document}
\title{14.4 习题}
\author{张志聪}
\maketitle

\section*{14.4.1}

按度量空间的定义(定义12.1.2),我们需证明$d_{B(X \to Y)}$满足下面四个公理:

\begin{itemize}
  \item (a) 对任意的$f \in B(X \to Y)$,我们有$d_{B(X \to Y)}(f, f) = 0$。

        由定义14.4.2可知,$d_{B(X \to Y)}(f, f) = sup\{d_Y(f(x), f(x)): x \in X\}$,
        因为任意$x \in X, d_Y(f(x), f(x)) = 0$,
        所以$sup\{d_Y(f(x), f(x)): x \in X\} = sup\{0\} = 0$,即$d_{B(X \to Y)}(f, f) = 0$

  \item (b) (正性)对任意两个不同的$f, g \in B(X \to Y)$,我们有$d_{B(X \to Y)}(f, g) > 0$。

        因为$f \neq g$,那么,存在$x_0 \in X$使得$f(x_0) \neq g(x_0)$,
        于是$d_Y(f(x_0), g(x_0)) > 0$,
        所以$sup\{d_Y(f(x), g(x)): x \in X\} > 0$,
        即$d_{B(X \to Y)}(f, g) > 0$

  \item (c)(对称性)对任意的$f, g \in B(X \to Y)$,我们有$d_{B(X \to Y)}(f, g) = d_{B(X \to Y)}(g, f)$

        由定义14.4.2可知,
        \begin{align*}
          d_{B(X \to Y)}(f, g) = sup\{d_Y(f(x), g(x)): x \in X\} \\
          d_{B(X \to Y)}(g, f) = sup\{d_Y(g(x), f(x)): x \in X\}
        \end{align*}

        令
        \begin{align*}
          A := \{d_Y(f(x), g(x)): x \in X\} \\
          B := \{d_Y(g(x), f(x)): x \in X\}
        \end{align*}
        容易证明$A = B$,所以$sup A = sup B$,
        即$d_{B(X \to Y)}(f, g) = d_{B(X \to Y)}(g, f)$

  \item (d) (三角不等式)对任意的$f, g, h \in B(X \to Y)$,
        我们有$d_{B(X \to Y)}(f, h) \leq d_{B(X \to Y)}(f, g) + d_{B(X \to Y)}(g, h)$。

        由定义14.4.2可知,我们需证明:
        \begin{align*}
          sup\{d_Y(f(x), h(x)): x \in X\} \leq sup\{d_Y(f(x), g(x)): x \in X\} + sup\{d_Y(g(x), h(x)): x \in X\}
        \end{align*}

        令
        \begin{align*}
          A := \{d_Y(f(x), h(x)): x \in X\} \\
          B := \{d_Y(f(x), g(x)): x \in X\} \\
          C := \{d_Y(g(x), h(x)): x \in X\}
        \end{align*}

        任意$a_0 \in A$,存在$x \in X$使得
        \begin{align*}
          a_0 = d_Y(f(x), h(x))
        \end{align*}
        我们有
        \begin{align*}
          d_Y(f(x), h(x)) \leq d_Y(f(x), g(x)) + d_Y(g(x), h(x))
        \end{align*}
        又因为
        \begin{align*}
          d_Y(f(x), g(x)) \in B \\
          d_Y(g(x), h(x)) \in C
        \end{align*}
        综上可得,$sup A \leq sup B + sup C$,命题得证。

        \begin{zremark}
          $sup A \leq sup B + sup C$这个结论可用反证法证明,
          假设$sup A > sup B + sup C$,那么存在$a \in A$使得$sup A > a > sup B + sup C$,
          因为$a \in A$,所有存在$x \in X$使得
          \begin{align*}
            a = d_Y(f(x^\prime), h(x^\prime))
          \end{align*}
          由上面的讨论可知,存在$b \in B, c \in C$使得
          \begin{align*}
            a \leq b + c
          \end{align*}
          这会导致以下矛盾
          \begin{align*}
            b + c > sup B + sup C
          \end{align*}
        \end{zremark}
\end{itemize}

\section*{14.4.2}

\begin{itemize}
  \item $\Rightarrow$

        对任意的$\epsilon > 0$,
        因为$(f^{(n)})_{n = 1}^\infty$是依度量$d_{B(X \to Y)}$收敛于$f$,
        所以存在$N \geq 1$,使得只要$n \geq N$,就有
        \begin{align*}
          d_{B(X \to Y)}(f^{(n)}, f) < \epsilon
        \end{align*}
        即
        \begin{align*}
          sup\{d_Y(f^{(n)}(x), f(x)) : x \in X\} < \epsilon
        \end{align*}
        综上可得,对任意$\epsilon > 0$,存在$N \geq 1$,使得只要$n \geq N$和$x \in X$,就有
        \begin{align*}
          d_Y(f^{(n)}(x), f(x)) < \epsilon
        \end{align*}
        所以,$(f^{(n)})_{n = 1}^\infty$一致收敛于$f$。

  \item $\Leftarrow$

        对任意$\epsilon > 0$,
        因为$(f^{(n)})_{n = 1}^\infty$一致收敛于$f$,
        所以存在$N \geq 1$,使得只要$n \geq N$和$x \in X$,就有
        \begin{align*}
          d_Y(f^{(n)}(x), f(x)) < \epsilon
        \end{align*}
        对每一个$n$,令
        \begin{align*}
          A_n := \{d_Y(f^{(n)}(x), f(x)): x \in X\}
        \end{align*}
        由于$A_n$是实数集合,且存在上界$\epsilon$,所以其上确界小于$\epsilon$,
        即$sup A_n < \epsilon$。

        综上可得,对任意$\epsilon > 0$,存在$N \geq 1$,使得只要$n \geq N$,就有
        \begin{align*}
          d_{B(X \to Y)}(f^{(n)}, f) < \epsilon
        \end{align*}
        所以,$(f^{(n)})_{n = 1}^\infty$是依度量$d_{B(X \to Y)}$收敛于$f$。

\end{itemize}

\end{document}