\documentclass{article}
\usepackage{mathtools} 
\usepackage{fontspec}
\usepackage[UTF8]{ctex}
\usepackage{amsthm}
\usepackage{mdframed}
\usepackage{xcolor}
\usepackage{amssymb}
\usepackage{amsmath}


% 定义新的带灰色背景的说明环境 zremark
\newmdtheoremenv[
  backgroundcolor=gray!10,
  % 边框与背景一致,边框线会消失
  linecolor=gray!10
]{zremark}{说明}


\begin{document}
\title{14.8 习题}
\author{张志聪}
\maketitle

\section*{14.8.1}

令$u := max(a, c), v := min(b, d)$。

所以$u \geq a$,有以下情况:

$u > a, u = c$,任意$x \in [a, u), x \notin [c, d], f(x) = 0$,所以$\int_{[a, u)} f = 0$。

$u = a$,则$\int_{[a, u)} f = 0$。

        又$v \leq b$,有以下情况:

      $v < b, v = d$,任意$x \in (v, b], x \notin [c, d], f(x) = 0$,所以$\int_{(v, b]} f = 0$。

$v = b$,则$\int_{(v, b]} f = 0$。

综上,
\begin{align*}
  \int_{[a, b]} f = \int_{[a, u)} f + \int_{[u, v]} f + \int_{(v, b]} f = \int_{[u, v]} f
\end{align*}

同理可得
\begin{align*}
  \int_{[c, d]} f = \int_{[c, u)} f + \int_{[u, v]} f + \int_{(v, d]} f = \int_{[u, v]} f
\end{align*}

所以$\int_{[a, b]} f = \int_{[c, d]} f$

\section*{14.8.2}

\begin{itemize}
  \item (a)

        $n = 0$时,$(1 - y)^0 = 1, 1 - 0 y = 1$,命题成立。

        $n \geq 1$时,
        令$f : = (1-y)^{n}$,对$y$进行求导,
        \begin{align*}
          f^\prime = n (1 - y)^{n - 1} (-1)
        \end{align*}
        又$f(0) = 1$,对任意$y \in [0, 1]$,由推论10.2.9(中值定理)可得,
        存在$x \in (0, 1)$,使得
        \begin{align*}
          f^\prime(x)             & = \frac{f(y) - f(0)}{y - 0} \\
          n (1 - x)^{n - 1} (-1)  & = \frac{(1-y)^{n} - 1}{y}   \\
          1 - n (1 - x)^{n - 1} y & = (1-y)^{n}
        \end{align*}
        因为$x \in (0, 1)$,所以$(1 - x)^{n - 1} \leq 1$,所以
        \begin{align*}
          1 - n (1 - x)^{n - 1} y & \geq 1 - n y
        \end{align*}
        综上可得
        \begin{align*}
          (1-y)^{n} = 1 - n (1 - x)^{n - 1} y & \geq 1 - n y
        \end{align*}

  \item (b)

        $n = 0$时,$\frac{1}{\sqrt{0}}$是没有意义的,不做讨论。

        $n \geq 1$时,$\frac{1}{\sqrt{n}} \leq 1$,于是,我们有
        \begin{align*}
          \int_{-1}^{1} (1 - x^2)^n dx & = \int_{-1}^{-\frac{1}{\sqrt{n}}} (1 - x^2)^n dx + \int_{-\frac{1}{\sqrt{n}}}^{\frac{1}{\sqrt{n}}} (1 - x^2)^n dx + \int_{\frac{1}{\sqrt{n}}}^{1} (1 - x^2)^n dx
        \end{align*}
        当$x \in [-\frac{1}{\sqrt{n}}, \frac{1}{\sqrt{n}}]$,即$|x| \leq \frac{1}{\sqrt{n}}$时,
        由(a)可知$(1 - x^2)^n \geq 1 - nx^2 \geq 0$;

        当$x \in [-1, -\frac{1}{\sqrt{n}}]$或$x \in [\frac{1}{\sqrt{n}}, 1]$即$|x| \geq \frac{1}{\sqrt{n}}$时,
        $1 - x^2 \geq 0$。

        于是可得
        \begin{align*}
          \int_{-1}^{1} (1 - x^2)^n dx  \geq \int_{-\frac{1}{\sqrt{n}}}^{\frac{1}{\sqrt{n}}} (1 - x^2)^n dx
           & \geq \int_{-\frac{1}{\sqrt{n}}}^{\frac{1}{\sqrt{n}}} 1 - nx^2 dx                                   \\
           & = (x - \frac{nx^3}{3})|_{x = \frac{1}{\sqrt{n}}} - (x - \frac{nx^3}{3})|_{x = -\frac{1}{\sqrt{n}}} \\
           & = \frac{2}{3\sqrt{n}} + \frac{2}{3\sqrt{n}}                                                        \\
           & = \frac{4}{3\sqrt{n}}                                                                              \\
           & \geq \frac{1}{\sqrt{n}}
        \end{align*}

  \item (c)

        令函数$f$,当$x \in [-1, 1]$时,$f(x) := c(1-x^2)^N$,当$x \notin [-1, 1]$时,$f(x) = 0$;
        于是定义14.8.6(a)已经满足。

        现在我们要满足14.8.6(b),因为
        \begin{align*}
          \int_{-\infty}^{+\infty} f = \int_{-1}^{1} c(1-x^2)^N = c \int_{-1}^{1} (1-x^2)^N
        \end{align*}

        所以令$c := \frac{1}{\int_{-1}^{1} (1-x^2)^N}$,那么,14.8.6(b)也已满足。

        现在我们要满足14.8.6(c),因为
        \begin{align*}
          |f(x)| = |c(1-x^2)^N|
        \end{align*}

        由(b)可知,$0 \leq c \leq \sqrt{N}$。

        于是
        \begin{align*}
          |f(x)| = |c(1-x^2)^N| \leq \sqrt{N} (1-x^2)^N \leq \sqrt{N} (1-\delta^2)^N
        \end{align*}

        我们需要使得以下成立:
        \begin{align*}
          \sqrt{N} (1-\delta^2)^N \leq \epsilon
        \end{align*}

        问题最终是变成:确定满足以上不等式的$N$是否存在。

        令$y := 1 - \delta^2$,根据14.8.6(c)的前置条件可知,$0 < y < 1$。
        接下来,我们只需证明$\lim \limits_{N \to \infty} \sqrt{N} y^N = 0$,即可完成证明。

        使用推论7.5.3(比值判别法)
        \begin{align*}
          \lim \sup \limits_{N \to \infty} \frac{\sqrt{N+1} y^{N + 1}}{\sqrt{N} y^N}
          = \lim \sup \limits_{N \to \infty} y \sqrt{1 + \frac{1}{N}}
        \end{align*}
        因为$\lim\limits_{N \to \infty} \sqrt{1 + \frac{1}{N}} = 1$,且$0 < y < 1$,可得
        \begin{align*}
          \lim \sup \limits_{N \to \infty} \frac{\sqrt{N+1} y^{N + 1}}{\sqrt{N} y^N}
          = \lim \sup \limits_{N \to \infty} y \sqrt{1 + \frac{1}{N}}
          < 1
        \end{align*}
        于是可得,级数$\sum \limits_{N = 1}^{\infty} \sqrt{N} y^N$是绝对收敛的,因此其通项$\sqrt{N} y^N$收敛。

\end{itemize}

\section*{14.8.3}

按照书中提示的思路进行证明。

因为$f: \mathbb{R} \to \mathbb{R}$是连续的紧支撑函数,那么,存在一个区间$[a, b]$,对所有的$x \notin [a,b]$都有
$f(x) = 0, f(a) = f(b) = 0$。因为$[a,b]$是$\mathbb{R}$的子集,且是闭区间,由定理9.1.24(直线上的海涅-博雷尔定理)(b)可知,
$[a, b]$是紧致的(符合定义12.5.1(紧致性))。

利用命题13.3.2可知$f|_{[a,b]}$是有界的。又$x \notin [a, b], f(x) = 0$可知,$f$是有界的。

利用定理13.3.5可知$f|_{[a,b]}$是一致连续的,
又$x \notin [a, b], f(x) = 0, f(a)=f(b)=0$可知,
$f$是一致连续的。

\section*{14.8.4}

\begin{itemize}
  \item (a)

        \begin{itemize}
          \item $f \ast g$有支撑区间。

                $f,g$都是连续的紧支撑函数,不妨设$f$支撑在$[a,b]$上,$g$支撑在$[c,d]$上,于是函数$g(x - y)$支撑在$[x - d, x - c]$
                (把$y$看做自变量,$x$是常量,$c \leq x - y \leq d$,所以$x - d \leq y \leq x - c$)。

                如果$[a,b] \cap [x - d, x - c] = \varnothing$,
                此时,$b < x - d$或$a > x - c$,即$x > b + d$或$x < a + c$,
                则无论$y$如何取值,总有$f(y)g(x-y) = 0$,因为$y \notin [a,b] \cap [x - d, x - c] = \varnothing$ 。
                于是可得,$[a + c, b + d]$是$f \ast g$的支撑区间。

          \item $f \ast g$是连续函数。

                由习题$14.8.3$可知,存在$M > 0$使得任意$x \in \mathbb{R}$都有$|f(x)| \leq M$;
                对任意$\epsilon > 0$,存在$\delta > 0$,只要$x, x^\prime \in \mathbb{R}, |x - x^\prime| < \delta$,
                就有
                \begin{align*}
                  |g(x) - g(x^\prime)| < \epsilon
                \end{align*}

                对任意$x_0 \in \mathbb{R}$,只要$|x - x_0| < \delta$,
                我们有
                \begin{align*}
                  f \ast g(x) - f \ast g(x_0)
                   & = \int_{-\infty}^{\infty} f(y)g(x - y) dy - \int_{-\infty}^{\infty} f(y)g(x_0 - y) dy                         \\
                   & = \int_{[a, b] \cap [x - d, x - c]} f(y)g(x - y) dy - \int_{[a, b] \cap [x_0 - d, x_0 - c]} f(y)g(x_0 - y) dy \\
                   & = \int_{[a, b]} f(y)g(x - y) dy - \int_{[a, b]} f(y)g(x_0 - y) dy                                             \\
                   & = \int_{[a, b]} f(y)(g(x - y) - g(x_0 - y)) dy                                                                \\
                   & \leq M \int_{[a, b]} (g(x - y) - g(x_0 - y)) dy                                                               \\
                   & \leq M |b-a|\epsilon
                \end{align*}
                所以$f \ast g$在$x_0$处连续,由$x_0$的任意性值,$f \ast g$连续。
        \end{itemize}
  \item (b)

        利用命题11.10.6(变量替换公式II),解决自变量不同的问题,但用的是它的推论,即$\phi$是单减的,注意证明过程中符号的变化。

        定义$\phi (y) := x - y$函数$\phi : [c, d] \cap [x - b, x - a] \to [a, b] \cap [x - d, x - c]$,
        以区间的方式表示$\phi: [max(c, x - b), min(d, x - a)] \to [min(b, x - c), max(a, x - d)]$。

        \begin{align*}
          f \ast g(x) & = \int_{[a, b] \cap [x - d, x - c]} f(y)g(x - y) dy                       \\
                      & = \int_{[max(a, x - d), min(b, x - c)]} f(y)g(x - y)                      \\
                      & = - \int_{[max(c, x - b), min(d, x - a)]} f(\phi(y))g(x - \phi(y)) d_\phi \\
                      & = - \int_{[max(c, x - b), min(d, x - a)]} f(x - y)g(y) d_{x - y}          \\
                      & = - \int_{[max(c, x - b), min(d, x - a)]} -f(x - y)g(y)                   \\
                      & = \int_{[max(c, x - b), min(d, x - a)]} f(x - y)g(y)                      \\
                      & = \int_{[max(c, x - b), min(d, x - a)]} f(x - y)g(y) dy                   \\
                      & = g \ast f(x)
        \end{align*}
        (注意第二个等式,使用了推论11.10.3)
  \item (c)
        \begin{align*}
          f \ast (g + h)(x) & = \int_{-\infty}^{\infty} f(y)(g + h)((x - y)) dy                               \\
                            & = \int_{-\infty}^{\infty} f(y)(g(x - y) + h(x - y)) dy                          \\
                            & = \int_{-\infty}^{\infty} f(y)g(x - y) + f(y)h(x - y) dy                        \\
                            & = \int_{-\infty}^{\infty} f(y)g(x - y) + \int_{-\infty}^{\infty}f(y)h(x - y) dy \\
                            & = f \ast g(x) + f \ast h(x)
        \end{align*}
        \begin{align*}
          f * (cg)(x) & = \int_{-\infty}^{\infty} f(y)(cg)(x - y)  dy \\
                      & = \int_{-\infty}^{\infty} f(y)cg(x - y)  dy   \\
                      & = c\int_{-\infty}^{\infty} f(y)g(x - y)  dy   \\
                      & = c(f \ast g(x))
        \end{align*}
\end{itemize}

\section*{14.8.5}

$y \notin [0, 1], f(y) = 0$,这部分的积分无需考虑,只需考虑$y \in [0, 1]$这部分。
$y \in [0, 1], x \in [1, 2]$时,$x - y \in [0, 1]$,此时$g(x - y) - c$。

于是对任意$x_0 \in [1, 2]$,我们有
\begin{align*}
  f \ast g(x_0) & = \int_{-\infty}^{\infty} f(y)g(x_0 - y) dy \\
                & = \int_{[0, 1]} f(y) c dy
\end{align*}

上式的结果与$x_0$的取值无关,命题得证。

\section*{14.8.6}

\begin{itemize}
  \item (a)
        因为$g$支撑在$[-1, 1]$上,所以
        \begin{align*}
          \int_{-\infty}^{\infty} g = \int_{-1}^{1} g = 1
        \end{align*}
        又因为$-1 \leq x \leq 1$都有$g(x) \geq 0$,
        所以,
        \begin{equation*}
          \begin{cases*}
            \int_{[-1, \delta)} g \geq 0 \\
            \int_{(\delta, 1]} g \geq 0
          \end{cases*}
        \end{equation*}
        我们有
        \begin{align*}
          \int_{-1}^{1} g = \int_{[-1, \delta)} g + \int_{[-\delta, \delta]} g + \int_{(\delta, 1]} g = 1
        \end{align*}
        于是可得
        \begin{align*}
          \int_{[-\delta, \delta]} g \leq 1
        \end{align*}

        又由$\delta \leq |x| \leq 1$均有$|g(x)| \leq \epsilon$,我们有
        \begin{align*}
           & 1 = \int_{[-1, \delta)} g + \int_{[-\delta, \delta]} g + \int_{(\delta, 1]} g \leq \int_{[-\delta, \delta]} g + 2(1 - \delta)\epsilon \\
           & \implies                                                                                                                              \\
           & 1 - 2(1 - \delta) \epsilon \leq \int_{[-\delta, \delta]} g
        \end{align*}
        因为$0 < \delta < 1$,可得
        \begin{align*}
          1 - 2\epsilon < 1 - 2(1 - \delta) \epsilon \leq \int_{[-\delta, \delta]} g
        \end{align*}

  \item (b)

        按照书中提示的思路进行证明。

        \begin{itemize}
          \item 第一个积分接近于$f(x)$。

                因为$y \in [-\delta, \delta]$,所以$|x - y - x| = |y| \leq \delta$,
                有题设可知$|f(x - y) - f(x)| \leq \epsilon$即$f(x) - \epsilon \leq f(x-y) \leq f(x) + \epsilon$。

                结合(a)可知
                \begin{align*}
                   & \int_{[-\delta, \delta]} (f(x) - \epsilon) g(y) dy \leq \int_{[-\delta, \delta]} f(x - y) g(y) dy \leq \int_{[-\delta, \delta]} (f(x) + \epsilon) g(y) dy   \\
                   & \implies                                                                                                                                                    \\
                   & (f(x) - \epsilon) \int_{[-\delta, \delta]}  g(y) dy \leq \int_{[-\delta, \delta]} f(x - y) g(y) dy \leq (f(x) + \epsilon) \int_{[-\delta, \delta]}  g(y) dy \\
                   & \implies                                                                                                                                                    \\
                   & (f(x) - \epsilon)(1 - 2\epsilon) \leq \int_{[-\delta, \delta]} f(x - y) g(y) dy \leq (f(x) + \epsilon)
                \end{align*}
                于是结合$|f(x)| \leq M$,
                \begin{align*}
                  -2f(x)\epsilon - \epsilon + 2\epsilon^2 \leq \int_{[-\delta, \delta]} f(x - y) g(y) dy - f(x) \leq \epsilon \\
                  -2M\epsilon - \epsilon + 2\epsilon^2 \leq \int_{[-\delta, \delta]} f(x - y) g(y) dy - f(x) \leq \epsilon
                \end{align*}
          \item 第二个和第三个积分的取值范围

                由定义14.8.6(恒等逼近)的性质(c)和$|f(x)| \leq M$可得,
                \begin{align*}
                   & \int_{[\delta, 1]} f(x - y)g(y)dy                                                    \\
                   & \implies                                                                             \\
                   & -M(1-\delta)\epsilon \leq \int_{[\delta, 1]} f(x - y)g(y)dy \leq M(1-\delta)\epsilon
                \end{align*}
                同理可得,
                \begin{align*}
                   & -M(1-\delta)\epsilon \leq \int_{[-1, -\delta]} f(x - y)g(y)dy \leq M(1-\delta)\epsilon
                \end{align*}
        \end{itemize}

        综上可得,
        \begin{align*}
           & -2M\epsilon - \epsilon + 2\epsilon^2 -2M(1-\delta)\epsilon \leq |f \ast g(x) - f(x)| \leq 2M(1-\delta)\epsilon + \epsilon \\
           & \implies                                                                                                                  \\
           & -4M\epsilon - \epsilon \leq |f \ast g(x) - f(x)| < 4M\epsilon + \epsilon                                                  \\
           & \implies                                                                                                                  \\
           & |f \ast g(x) - f(x)| \leq (1 + 4M)\epsilon
        \end{align*}
        (注意,以上用到了$1 - \delta < 1, - \epsilon + 2\epsilon^2 > -\epsilon$)

\end{itemize}

\section*{14.8.7}

因为$f$是支撑在$[0,1]$上的连续函数,于是由习题14.8.3可知,$f$是有界的,并且是一致连续的。
于是可得存在$M > 0$,$f$以$M$为界。任意$\epsilon^\prime > 0$,
因为$f$是一致连续的,那么,存在$\delta^\prime > 0$,
使得只要$x, y \in \mathbb{R}, |x - y| < \delta^\prime$,
就有$|f(x) - f(y)| < \epsilon^\prime$。
取$min(1, \delta^\prime) > \delta > 0$,那么,
$x, y \in \mathbb{R}, |x - y| < \delta$,
就有$|f(x) - f(y)| < \epsilon^\prime$。

由引理14.8.8可知,存在一个$[-1, 1]$上的多项式$g$,并且它是一个$(\epsilon^\prime, \delta)$恒等逼近。

综上,由引理14.8.14可知,对所有的$x \in [0, 1]$都有,
\begin{align*}
  |f \ast g(x) - f(x)| \leq (1 + 4M)\epsilon^\prime
\end{align*}

又由引理14.8.13可知,$f \ast g$是$[0, 1]$上的多项式,
$P(x) := f \ast g(x)$就是我们需要的多项式。

以上讨论对任意$\epsilon^\prime$都成立,于是对任意的$\epsilon > 0$,
设$\epsilon^\prime = \frac{1}{1+4M}\epsilon$,那么,对所有的$x \in [0, 1]$都有,
\begin{align*}
  |P(x) - f(x)| \leq (1 + 4M)\epsilon^\prime = \epsilon
\end{align*}

命题得证。

\section*{14.8.8}

任意多项式$P$,我们可以找到一个整数$n \geq 0$和实数$c_0, c_1, \dots, c_n$使得
\begin{align*}
  P(x) = \sum\limits_{j = 0}^n c_j x^j, \;\; x \in [0, 1]
\end{align*}

于是我们有,
\begin{align*}
  \int_{[0, 1]} f(x) P(x) dx
   & = \int_{[0, 1]} f(x) \sum\limits_{j = 0}^n c_j x^j    \\
   & = \sum\limits_{j = 0}^n c_j \int_{[0, 1]} f(x) x^j dx \\
   & = 0
\end{align*}

由推论14.8.16可知,对任意$\epsilon > 0$,存在$P_0: \mathbb{R} \to \mathbb{R}$,它是$[0, 1]$上的多项式,
并使得$|P_0(x) - f(x)| \leq \epsilon$。于是可得$P_0(x) = f(x) + c\epsilon$,其中$-1 \leq c \leq 1$。
于是
\begin{align*}
  \int_{[0, 1]} f(x) P_0(x) dx = 0
   & = \int_{[0, 1]} f(x) (f(x) + c\epsilon)dx                        \\
   & = \int_{[0, 1]} f(x) f(x) dx + \int_{[0, 1]} + c\epsilon f(x) dx
\end{align*}

因为
\begin{align*}
  \int_{[0, 1]} c \epsilon f(x) dx = c \epsilon \int_{[0, 1]} f(x) x^0 dx = 0
\end{align*}
于是可得
\begin{align*}
  \int_{[0, 1]} f(x) f(x) dx = 0
\end{align*}

反证法,假设$f \not \equiv 0$,即存在$x_0 \in [0, 1]$使得$f(x_0) \neq 0$。

因为$f(x)$是一个连续函数,那么$f(x)f(x)$也是连续函数,且是非负的。于是对$\epsilon = \frac{1}{2}f(x_0)f(x_0) > 0$,
存在$\delta > 0$使得$x \in [0, 1], |x - x_0| < \delta$,
就有$|f(x)f(x) - f(x_0)f(x_0) < \epsilon$,
于是可得
\begin{align*}
  f(x)f(x) \geq f(x_0)f(x_0) - \epsilon = \frac{1}{2}f(x_0)f(x_0)
\end{align*}

我们有
\begin{align*}
  \int_{[0, 1]} f(x) f(x) dx
   & \geq \int_{[x_0 - \delta, x_0 + \delta]} f(x)f(x) dx \\
   & \geq 2\delta \frac{1}{2} f(x_0)f(x_0)                \\
   & = \delta f(x_0)f(x_0)                                \\
   & > 0
\end{align*}
存在矛盾。

\section*{14.8.9}

如果$x_0 = 0$,对任意$\epsilon > 0$,由$f$在$[0, 1]$上连续可知,存在$\delta > 0$,使得只要
$x \in [0, 1], 0 \leq x - x_0 < \delta$,就有$|f(x) - f(0)| < \epsilon$。又由$x \notin [0, 1], F(x) = 0$可知,
当$ -\delta < x - x_0 < 0$,就有$|F(x) - F(0)| = 0 < \epsilon$。
所以,只要$|x - x_0| < \delta$,就有$|F(x) - F(0)| < \epsilon$,
于是$F$在$0$处是连续的。

同理可得,$F$在$1$处是连续的。

$x_0 \in \mathbb{R} \setminus (0, 1)$时, $F$在$x_0$上的连续性是易证的,这里不做说明。

综上,$F$是连续的。


\end{document}