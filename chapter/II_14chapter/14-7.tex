\documentclass{article}
\usepackage{mathtools} 
\usepackage{fontspec}
\usepackage[UTF8]{ctex}
\usepackage{amsthm}
\usepackage{mdframed}
\usepackage{xcolor}
\usepackage{amssymb}
\usepackage{amsmath}


% 定义新的带灰色背景的说明环境 zremark
\newmdtheoremenv[
  backgroundcolor=gray!10,
  % 边框与背景一致,边框线会消失
  linecolor=gray!10
]{zremark}{说明}


\begin{document}
\title{14.7 习题}
\author{张志聪}
\maketitle

\section*{14.7.1}
(1)$f_n$一致收敛于$f$。

\begin{align*}
   & f_n(x) - f(x)                                                                     \\
   & = f_n(x) - L + \int_{[a, x_0]} g - \int_{[a, x]} g                                \\
   & = f_n(x) - L - f_n(x_0) + f_n(x_0) + \int_{[a, x_0]} g - \int_{[a, x]} g          \\
   & = f_n(x_0) - L + \int_{[x_0, x]} f^\prime_n + \int_{[a, x_0]} g - \int_{[a, x]} g \\
   & = f_n(x_0) - L + \int_{[x_0, x]} f^\prime_n - g                                   
\end{align*}

由$\lim\limits_{n \to \infty} f_n(x_0) = L$可得,
对任意$\epsilon > 0$,存在$N_1 > 0$,使得只要$n \geq N_1$,就有
\begin{align*}
  |f_n(x_0) - L| < \frac{1}{2}\epsilon
\end{align*}

又由$f_n^\prime$一致收敛与$g$,那么,存在$N_2 > 0$,使得只要$n \geq N_2$,就有
\begin{align*}
  f^\prime_n - g < \frac{1}{2}\frac{\epsilon}{|b - a|}
\end{align*}

综上可得,对任意的$\epsilon > 0$,存在$N > max(N_1, N_2)$,使得只要$n \geq N$,就有
\begin{align*}
  |f_n(x) - f(x)| = |f_n(x_0) - L + \int_{[x_0, x]} f^\prime_n - g| < \frac{1}{2}\epsilon + \frac{1}{2}\epsilon = \epsilon
\end{align*}
命题得证。

(2)$f$是可微的,它的导函数是$g$。

$L - \int_{[a, x_0]} g$是常数,所以,导数是$0$;$g$是连续的,由推论11.5.2可知,
\begin{align*}
  \int_{[a, x]} g
\end{align*}
是黎曼可积的。

又由定理11.9.1可知,$(\int_{[a, x]} g)^\prime = g(x)$。

综上,命题得证。

(3)例1.2.10与定理14.7.1不矛盾的原因。

把$\epsilon$看做$\frac{1}{n}$,例1.2.10的操作没有按照定理14.7.1的操作,所以,定理14.7.1与例1.2.10不矛盾。

\section*{14.7.2}


\end{document}