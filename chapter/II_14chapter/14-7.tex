\documentclass{article}
\usepackage{mathtools} 
\usepackage{fontspec}
\usepackage[UTF8]{ctex}
\usepackage{amsthm}
\usepackage{mdframed}
\usepackage{xcolor}
\usepackage{amssymb}
\usepackage{amsmath}


% 定义新的带灰色背景的说明环境 zremark
\newmdtheoremenv[
  backgroundcolor=gray!10,
  % 边框与背景一致,边框线会消失
  linecolor=gray!10
]{zremark}{说明}


\begin{document}
\title{14.7 习题}
\author{张志聪}
\maketitle

\section*{14.7.1}
(1)$f_n$一致收敛于$f$。

\begin{align*}
   & f_n(x) - f(x)                                                                     \\
   & = f_n(x) - L + \int_{[a, x_0]} g - \int_{[a, x]} g                                \\
   & = f_n(x) - L - f_n(x_0) + f_n(x_0) + \int_{[a, x_0]} g - \int_{[a, x]} g          \\
   & = f_n(x_0) - L + \int_{[x_0, x]} f^\prime_n + \int_{[a, x_0]} g - \int_{[a, x]} g \\
   & = f_n(x_0) - L + \int_{[x_0, x]} f^\prime_n - g
\end{align*}

由$\lim\limits_{n \to \infty} f_n(x_0) = L$可得,
对任意$\epsilon > 0$,存在$N_1 > 0$,使得只要$n \geq N_1$,就有
\begin{align*}
  |f_n(x_0) - L| < \frac{1}{2}\epsilon
\end{align*}

又由$f_n^\prime$一致收敛与$g$,那么,存在$N_2 > 0$,使得只要$n \geq N_2$,就有
\begin{align*}
  f^\prime_n - g < \frac{1}{2}\frac{\epsilon}{|b - a|}
\end{align*}

综上可得,对任意的$\epsilon > 0$,存在$N > max(N_1, N_2)$,使得只要$n \geq N$,就有
\begin{align*}
  |f_n(x) - f(x)| = |f_n(x_0) - L + \int_{[x_0, x]} f^\prime_n - g| < \frac{1}{2}\epsilon + \frac{1}{2}\epsilon = \epsilon
\end{align*}
命题得证。

(2)$f$是可微的,它的导函数是$g$。

$L - \int_{[a, x_0]} g$是常数,所以,导数是$0$;$g$是连续的,由推论11.5.2可知,
\begin{align*}
  \int_{[a, x]} g
\end{align*}
是黎曼可积的。

又由定理11.9.1可知,$(\int_{[a, x]} g)^\prime = g(x)$。

综上,命题得证。

(3)例1.2.10与定理14.7.1不矛盾的原因。

把$\epsilon$看做$\frac{1}{n}$,例1.2.10的操作没有按照定理14.7.1的操作,所以,定理14.7.1与例1.2.10不矛盾。

\section*{14.7.2}

如果
\begin{align*}
  d_{\infty}(f_n^\prime, f_m^\prime) \leq \epsilon
\end{align*}

即
\begin{align*}
  \sup\{|f_n^\prime(x) - f_m^\prime(x)|: x \in [a, b]\} \leq \epsilon
\end{align*}

所以对任意$x \in [a, b]$都有$|f_n^\prime(x) - f_m^\prime(x)| \leq \epsilon$。

定义$h := (f_n + f_m)(x)$的函数$h : [a, b] \to \mathbb{R}$,因为$f_n, f_m$连续可微,
由定理10.1.13(c)可知,$h$连续可微。

利用推论10.2.9可知,对任意$x \in [a, b]$,在区间$[x, x_0]$,存在$z \in [x, x_0]$,使得
\begin{align*}
  h^\prime(z) = \frac{h(x) - h(x_0)}{x - x_0}
\end{align*}

即
\begin{align*}
  h^\prime(z) = f_n^\prime(z) - f_m^\prime(z) = \frac{(f_n(x) - f_m(x)) - (f_n(x_0) - f_m(x_0))}{x - x_0}
\end{align*}

综上可得,
\begin{align*}
  \left| (f_n(x) - f_m(x)) - (f_n(x_0) - f_m(x_0))\right|  \leq \epsilon|x - x_0|
\end{align*}

(1)$f_n$一致收敛于某个函数$f$。

因为$\lim\limits_{n \to \infty} f_n^\prime = g$,
那么,对任意$\epsilon > 0$,存在$N_1 > 0$使得只要$m, n \geq N_1$和$x \in [a, b]$,都有
\begin{align*}
  |f_n^\prime(x) - g(x)| \leq \frac{1}{2} \epsilon \\
  |f_m^\prime(x) - g(x)| \leq \frac{1}{2} \epsilon
\end{align*}

于是可得
\begin{align*}
  |f_n^\prime(x) - f_m^\prime(x)| & = |f_n^\prime(x) - g(x) + g(x) - f_m^\prime(x)|      \\
                                  & \leq |f_n^\prime(x) - g(x)| + |f_m^\prime(x) - g(x)| \\
                                  & \leq \epsilon
\end{align*}

所以,我们有
\begin{align*}
  d_\infty(f_n^\prime, f_m^\prime) \leq \epsilon
\end{align*}

由之前的讨论
\begin{align*}
  |f_n(x) - f_m(x)| \leq \epsilon|x - x_0| + |f_n(x_0) - f_m(x_0)| \leq \epsilon|b - a| + |f_n(x_0) - f_m(x_0)|
\end{align*}

由$\lim\limits_{n \to \infty} f_n(x_0)$极限存在,于是$(f_n(x_0))_{n = 1}^\infty$是柯西序列,
那么,存在$N_2 \geq 1$,使得只要$n, m \geq N_2$,就有
\begin{align*}
  |f_n(x_0) - f_m(x_0)| \leq \epsilon|b - a|
\end{align*}

综上可得,只要$n, m \geq max(N_1, N_2)$,对任意$x \in [a, b]$都有
\begin{align*}
  |f_n(x) - f_m(x)| & \leq \epsilon|b - a| + |f_n(x_0) - f_m(x_0)|              \\
                    & \leq \epsilon|b - a| + \epsilon|b - a| = 2\epsilon|b - a|
\end{align*}

即
\begin{align*}
  d_\infty(f_n, f_m) \leq 2\epsilon|b - a|
\end{align*}

由$\epsilon$的任意性可知,
$(f_n)_{n = 1}^\infty$是度量空间$(C([a, b] \to \mathbb{R}), d_{\infty})$中的柯西序列,
由定理14.4.5可知,该序列收敛于$C([a, b] \to \mathbb{R})$中的一个函数$f$,由命题14.4.4可知,
$(f_n)_{n = 1}^\infty$一致收敛$f$。

(2)$f$是可微的,它的导函数是$g$。

todo 还没证明完,卡住了!

利用命题14.3.3进行证明。

对任意$c \in [a, b], E := [a, b] \setminus \{c\}$,
需证明$\lim\limits_{x \to c; x \in E} \frac{f(x) - f(c)}{x - c}$是存在的,
且等于$g(x)$。

定义函数如下:
\begin{equation*}
  \begin{cases*}
    F(x)    := \frac{f(x) - f(c)}{x - c},\;\;\;\; x \in E \\
    F_n(x)  : = \frac{f_n(x) - f_n(c)}{x - c}, \;\;\;\; x \in E
  \end{cases*}
\end{equation*}

接下来,证明$(F_n)_{n = 1}^\infty$一致收敛于$F$。
\begin{align*}
  F_n(x) - F(x) & = \frac{(f_n(x) - f_n(c)) - (f(x) - f(c))}{x - c} \\
                & = \frac{(f_n(x) - f(x)) + (f(c) - f_n(c))}{x - c}
\end{align*}

由$(f_n)_{n = 1}^\infty$一致收敛$f$可得,存在$N$,使得只要$n \geq N$和$x \in E$都有
\begin{align*}
  |f_n(x) - f(x)| \leq \epsilon
\end{align*}

综上可得
\begin{align*}
 |F_n(x) - F(x)| = |\frac{(f_n(x) - f(x)) + (f(c) - f_n(c))}{x - c}| \leq \frac{2\epsilon}{|x - c|}  
\end{align*}


\section*{14.7.3}

利用定理14.5.7(威尔斯特拉斯M判别法)可知
\begin{align*}
  \sum \limits_{n = 1}^\infty f_n^\prime
\end{align*}
是一致收敛的,不妨设一致收敛于连续函数$g$。

定义$F_N := \sum\limits_{n = 1}^N f_n$函数$F_N: [a, b] \to \mathbb{R}$,
因为$f_n$是可微函数,由10.1.13(c)可知,$F_N$是可微函数,且
导函数为
\begin{align*}
  F_N^\prime := \sum\limits_{n = 1}^N f_n^\prime
\end{align*}
由于$f_n^\prime$是连续的,那么导函数$F_N^\prime$也是连续的。

由之前的讨论可得$\lim\limits_{N \to \infty} F_N^\prime = g$,又因为
存在某个$x_0 \in [a, b]$,使得级数$\sum\limits_{n = 1}^\infty f_n(x_0)$收敛,
即$\lim\limits_{N \to \infty} F_N(x_0)$存在,于是利用定理14.7.1可知,
函数数列$F_N$一致收敛于可微函数$f$,并且$f$的导函数等于$g$。

\end{document}