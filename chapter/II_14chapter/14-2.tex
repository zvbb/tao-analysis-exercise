\documentclass{article}
\usepackage{mathtools} 
\usepackage{fontspec}
\usepackage[UTF8]{ctex}
\usepackage{amsthm}
\usepackage{mdframed}
\usepackage{xcolor}
\usepackage{amssymb}
\usepackage{amsmath}


% 定义新的带灰色背景的说明环境 zremark
\newmdtheoremenv[
  backgroundcolor=gray!10,
  % 边框与背景一致,边框线会消失
  linecolor=gray!10
]{zremark}{说明}


\begin{document}
\title{14.2 习题}
\author{张志聪}
\maketitle

\section*{14.2.1}

(a)
\begin{itemize}
  \item $\Rightarrow$

        任意$x_0 \in \mathbb{R}$,
        因为$(a_n)_{n = 0}^\infty$是收敛于$0$的实数列,那么,
        $(x_0 - a_n)_{n = 0}^\infty$是收敛于$x_0$的实数列。

        因为$f$是连续的,那么$f$在$x_0$处也是连续的,
        由定理13.1.4(b)可知,序列$(f(x_0 - a_n))_{n = 0}^\infty$收敛于$f(x_0)$,即
        \begin{align*}
          \lim_{n \to \infty}f(x_0 - a_n) = f(x_0)
        \end{align*}

        由$x_0$的任意性可知,$f_{a_n}$逐点收敛于$f$。

  \item $\Leftarrow$

        反证法,假设$f$不是连续的,那么,存在$f$在$x_0$处不连续。
        由定理13.1.4(b)可知,存在$(x_n)_{n = 0}^\infty$收敛于$x_0$的序列,
        使得$(f(x_n))_{n = 0}^\infty$不收敛于$f(x_0)$。

        构造$(a_n)_{n = 0}^\infty$实数列,其中$a_n = x_0 - x_n$,
        由极限定律可知$\lim\limits_{n \to \infty}a_n = 0$。
        我们有
        \begin{align*}
          f_{a_n}(x_0) = f(x_0 - a_n) = f(x_0 - (x_0 - x_n)) = f(x_n)
        \end{align*}
        综上,由$f_{a_n}$逐点收敛于$f$可知,
        \begin{align*}
          \lim_{n \to \infty}f(x_0 - a_n) = \lim_{n \to \infty}f(x_n) = f(x_0)
        \end{align*}
        这与$(f(x_n))_{n = 0}^\infty$不收敛于$f(x_0)$矛盾。
\end{itemize}

(b)

\begin{itemize}
  \item $\Rightarrow$

        任意$x \in \mathbb{R}, \epsilon > 0$。
        $f$是一致连续的,由定义13.3.4(一致连续性)可知,
        存在$\delta > 0$使得只要$x, x^\prime \in \mathbb{R}$满足$d(x, x^\prime) = |x - x^\prime| < \delta$,
        就有$d(f(x), f(x^\prime)) = |f(x) - f(x^\prime)|< \epsilon$。

        序列$(a_n)_{n = 0}^\infty$收敛于$0$,所以,存在$N$使得只要$n \geq N$就有
        \begin{align*}
          |a_n| < \delta
        \end{align*}
        又我们有
        \begin{align*}
          f_{a_n}(x) = f(x - a_n)
        \end{align*}
        综上可得,任意$x \in \mathbb{R}, n \geq N$,此时$d(x, x - a_n) < \delta$,我们有
        \begin{align*}
          d(f(x), f(x - a_n)) < \epsilon
        \end{align*}
        所以$f_{a_n}$一致收敛于$f$。

  \item $\Leftarrow$

        反证法,假设$f$不是一致连续的,那么,存在$\epsilon_0 > 0$,
        对任意$n \in \mathbb{N}$,存在$x_n, x_n^\prime \in \mathbb{R}, d(x_n, x_n^\prime) < \frac{1}{n}$,
        有$d(f(x_n), f(x_n^\prime)) \geq \epsilon_0$。

        利用选择公理,构造$(a_n)_{n = 1}^\infty$收敛于$0$的实数列,其中 $a_n = x_n - x_n^\prime$。
        我们有
        \begin{align*}
          f_{a_n}(x_n) = f(x_n - (x_n - x_n^\prime)) = f(x_n^\prime)
        \end{align*}
        因为序列$f_{a_n}$一致收敛于$f$,所以存在$N > 0$使得对所有的$n \geq N$都有
        \begin{align*}
          d(f_{a_n}(x_n), f(x_n)) = d(f(x_n^\prime), f(x_n)) < \epsilon_0
        \end{align*}
        存在矛盾。

\end{itemize}

\section*{14.2.2}

\begin{itemize}
  \item (a)

        任意$x_0 \in X$,因为$f^{(n)}$一致收敛于$f$,
        所以对任意的$\epsilon > 0$,存在$N > 0$使得
        对所有的$n \geq N$都有
        \begin{align*}
          d_Y(f^{(n)}(x_0), f(x_0)) < \epsilon
        \end{align*}
        于是可得
        \begin{align*}
          \lim_{n \to \infty} f^{(n)}(x_0) = f(x_0)
        \end{align*}
        由$x_0$的任意性可知,$f^{(n)}$逐点收敛于$f$。

  \item (b)

        \begin{itemize}
          \item 证明逐点收敛于零函数$0$。

                任意$x_0 \in (-1, 1)$, 由引理6.5.2可知,
                \begin{align*}
                  \lim \limits_{n \to \infty} f^{(n)}(x_0) = \lim \limits_{n \to \infty} x_0^n = 0
                \end{align*}

          \item 不一致收敛于任意函数$f: (-1, 1) \to \mathbb{R}$。

                反证法,假设$f^{(n)}$一致收敛于$f: (-1, 1) \to \mathbb{R}$,
                由(a)和逐点极限$f$的唯一性可得,$f$是零函数。

                因为$(\frac{1}{2})^{\frac{1}{N}} \in (-1, 1)$,我们有
                \begin{align*}
                  \left|f^{{(N)}}((\frac{1}{2})^{\frac{1}{N}}) - 0\right| = \frac{1}{2} > \epsilon
                \end{align*}
                所以,不存在满足要求的$N$。
        \end{itemize}
  \item (c)

        \begin{itemize}
          \item 逐点收敛于$g$。

                利用引理7.3.3,证明略

          \item 不一致收敛于$g$。

                任意$x \in X$,我们有
                \begin{align*}
                  \sum\limits_{n = 1}^N  f(x) = \frac{x(1 - x^N)}{1 - x}
                \end{align*}
                $c = \frac{1}{2}^{\frac{1}{N + 1}}$
                于是
                \begin{align*}
                  |\sum\limits_{n = 1}^N  f(x) - g(x)| & = |\frac{c(1 - c^N)}{1 - c} - \frac{c}{1 - c}|    \\
                                                       & = \frac{c^{N + 1}}{1 - c} > c^{N+1} = \frac{1}{2}
                \end{align*}
                所以,不存在满足要求的$N$。

          \item 换成闭区间$[-1, 1]$

                $x = 1$时,$\sum\limits_{n = 1}^\infty  f^{(n)}(x) = \sum\limits_{n = 1}^\infty  1$不是收敛的,
                所以$\lim\limits_{N \to \infty}\sum\limits_{n = 1}^N  f^{(n)}(x)$
                不会是逐点收敛的,进一步,也不会是一致收敛的。

        \end{itemize}


\end{itemize}

\section*{14.4.3}

任意$x_0 \in X$,因为$h: \mathbb{R} \to \mathbb{R}$是一个连续函数,所以对任意$\epsilon > 0$,
存在$\delta > 0$,使得只要$y \in \mathbb{R}, |f(x_0), y| < \delta$就有
\begin{align*}
    |g(f(x_0)) - g(y)| < \epsilon
\end{align*}
因为$f^{(n)}$在$X$上逐点收敛于函数$f: X \to \mathbb{R}$,所以,存在一个$N > 0$使得对所有的$n > N$都有
\begin{align*}
  |f^{(n)}(x_0) - f(x_0)| < \delta
\end{align*}
综上可得,对任意的$x_0 \in X$,存在一个$N > 0$使得对所有的$n > N$都有
\begin{align*}
  |g(f(x_0)) - g(f^{(n)}(x_0))| < \epsilon
\end{align*}
即
\begin{align*}
  \lim \limits_{n \to \infty} g(f^{(n)}(x_0)) = g(f(x_0))
\end{align*}
由$x_0$的任意性可知,函数$h \circ f^{(n)}$在$X$上逐点收敛于$h \circ f$。

\section*{14.2.4}

因为$f: X \to Y$是一个有界函数,所以在$Y$中存在一个球$B(Y,d_Y)(y_0, R)$使得对所有的$x \in X$都有$f(x) \in B(Y,d_Y)(y_0, R)$。
因为$f_n$一致收敛于函数$f$,所以对$\epsilon = 1 > 0$,存在$N > 0$,使得只要$x \in X, n \geq N$都有
\begin{align*}
  d_Y(f_n(x), f(x)) < \epsilon
\end{align*}
于是可得
\begin{align*}
  d_Y(f_n(x), y_0) < R + \epsilon
\end{align*}
对于$n < N$,因为$f_n$是度量空间$(Y, d_Y)$有界函数序列,
所以在$Y$中存在球$B(Y, d_Y)(y_1, R_1), B(Y, d_Y)(y_2, R_2), \dots, B(Y, d_Y)(y_{N - 1}, R_{N - 1})$,对所有的$x \in X$都有
\begin{align*}
  f_n(x) \in B(Y, d_Y)(y_i, R_i), i = 1, 2, \dots, N - 1
\end{align*}
令
\begin{align*}
  r = \max\limits_{i = 1, 2, \dots, N - 1} d_Y(y_i, y_0) + \max\limits_{i = 1, 2, \dots, N - 1} R_i + R + \epsilon \geq R + \epsilon
\end{align*}
所以对任意的$x \in X$,$n < N$有
\begin{align*}
  d_Y(f_n(x), y_0) \leq d_Y(f_n(x), y_n) + d_Y(y_n, y_0) < R_n + d_Y(y_n, y_0) < r
\end{align*}
综上,对所有的$x \in X$和所有的正整数$n$都有
\begin{align*}
  d_Y(f_n(x), y_0) \leq r
\end{align*}
即
\begin{align*}
  f_n(x) \in B(Y, d_Y)(y_0, r)
\end{align*}
至此,命题得证。


\end{document}