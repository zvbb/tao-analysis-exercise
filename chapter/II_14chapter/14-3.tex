\documentclass{article}
\usepackage{mathtools} 
\usepackage{fontspec}
\usepackage[UTF8]{ctex}
\usepackage{amsthm}
\usepackage{mdframed}
\usepackage{xcolor}
\usepackage{amssymb}
\usepackage{amsmath}


% 定义新的带灰色背景的说明环境 zremark
\newmdtheoremenv[
  backgroundcolor=gray!10,
  % 边框与背景一致,边框线会消失
  linecolor=gray!10
]{zremark}{说明}


\begin{document}
\title{14.3 习题}
\author{张志聪}
\maketitle

\section*{14.3.1}

反证法,假设$f$不在$x_0$处连续。
那么,存在$\epsilon_0 > 0$,任意$x \in X$都有$d_Y(f(x), f(x_0)) \geq \epsilon_0$。

因为序列一致收敛于$f$,所以存在$N > 0$使得只要$n \geq N, x \in X$就有$d_Y(f^{(n)}(x), f(x)) < \frac{1}{4} \epsilon_0$。

有每一个$n$,函数$f^{(n)}$都在$x_0$处连续,那么对$n \geq N$, 都存在$\delta > 0$使得只要$d_X(x, x_0) < \delta$,
就有$d_Y(f^{(n)}(x), f^{(n)}(x_0)) < \frac{1}{4} \epsilon_0$。

综上,$n \geq N$和$x \in X$且$d_X(x, x_0) < \delta$,就有
\begin{align*}
  d_Y(f(x), f(x_0)) & \leq d_Y(f(x), f^{(n)}(x)) + d_Y(f^{(n)}(x), f(x_0))                                                   \\
                    & \leq d_Y(f(x), f^{(n)}(x)) + d_Y(f^{(n)}(x), f^{(n)}(x_0)) + d_Y(f^{(n)}(x_0), f(x_0))                 \\
                    & \leq \frac{1}{4} \epsilon_0 + \frac{1}{4} \epsilon_0 + \frac{1}{4} \epsilon_0 = \frac{3}{4} \epsilon_0 \\
                    & < \epsilon_0
\end{align*}
(注意以上没有考虑$n < N$,因为我们只是想说明$\epsilon_0$是$d_Y(f(x), f(x_0))$的上界,是否有更小的上界或更大的上界,这里我们不用关心。)

存在矛盾。

\section*{14.3.2}

(1)先证明$\lim\limits_{x \to x_0; x \in E} f(x)$的存在性。

利用$Y$的完备性进行证明。
设$(x_m)_{m = 1}^\infty$是$E$上收敛于$x_0$的序列,
我们要证明$(f(x_m))_{m = 1}^\infty$是$Y$上的柯西序列即可完成证明。

对任意$\epsilon > 0$,由对每一个$n$,极限$\lim\limits_{x \to x_0; x \in E} f^{(n)}(x)$都存在,
不妨设收敛于$L_n$。
那么,存在$\delta > 0$,使得只要$d_X(x, x_0) < \delta$,就有$d_Y(f^{(n)}(x), L_n) < \frac{1}{4} \epsilon$。

因为$(x_m)_{m = 1}^\infty$是$E$上收敛于$x_0$的序列,所以存在$M > 1$,使得对所有的$p, q \geq M$都有
$d_X(x_p, x_q) < \delta$。

于是,对所有的$p, q \geq M$我们有
\begin{align*}
  d_Y(f^{(n)}(x_p), f^{(n)}(x_q)) \leq  d_Y(f^{(n)}(x_p), L_n) + d_Y(f^{(n)}(x_q), L_n) < \frac{1}{8} \epsilon + \frac{1}{8} \epsilon = \frac{1}{4} \epsilon
\end{align*}

因为$(f^{(n)})_{n = 1}^\infty$一致收敛于$f$,所以存在$N > 0$使得对所有的$n \geq N$和$x \in E$都有
$d_Y(f^{(n)}(x), f(x)) < \frac{1}{4}\epsilon$,那么对每一个$n$都有

综上,对所有的$p, q \geq M,n \geq N$,此时$d_X(x_p, x_q) < \delta$,于是我们有
\begin{align*}
  d_Y(f(x_p), f(x_q)) & \leq d_Y(f(x_p), f^{(n)}(x_p)) + d_Y(f^{(n)}(x_p), f(x_q))                                    \\
                      & \leq d_Y(f(x_p), f^{(n)}(x_p)) + d_Y(f^{(n)}(x_p), f^{(n)}(x_q)) + d_Y(f^{(n)}(x_q), f(x_p))  \\
                      & \leq \frac{1}{4} \epsilon + \frac{1}{4} \epsilon + \frac{1}{4} \epsilon = \frac{3}{4}\epsilon \\
                      & < \epsilon
\end{align*}
(注意以上没有考虑$n < N$,因为我们只是想说明$\epsilon$是$d_Y(f(x_p), f(x_q))$的上界,是否有更小的上界或更大的上界,这里我们不用关心。)

于是可得,$(f(x_m))_{m = 1}^\infty$是$Y$上的柯西序列。

(2)证明$(\lim\limits_{x \to x_0; x \in E} f^{(n)}(x))_{n = 1}^\infty$的极限等于$\lim\limits_{x \to x_0; x \in E} f(x)$。

不妨设$\lim\limits_{x \to x_0; x \in E} f(x) = L$,$\lim\limits_{x \to x_0; x \in E} f^{(n)}(x) = L_n$,
于是,我们需要证明:$\lim\limits_{n \to \infty} L_n = L$。

对任意$\epsilon > 0$,
因为$\lim\limits_{x \to x_0; x \in E} f(x) = L$,那么,存在$\delta > 0$,使得只要$x \in E$且$d_X(x, x_0) < \delta$,
就有$d_Y(f(x), L) < \frac{1}{3} \epsilon$。

因为$(f^{(n)})_{n = 1}^\infty$一致收敛于$f$,所以存在$N > 0$使得对所有的$n \geq N$和$x \in E$都有
$d_Y(f^{(n)}(x), f(x)) < \frac{1}{3} \epsilon$。

又因为对每一个$n$,$\lim\limits_{x \to x_0; x \in E} f^{(n)}(x) = L_n$,所以存在$\delta_n > 0$使得只要$x \in E$且$d_X(x, x_0) < \delta_n$,
就有$d_Y(f^{(n)}(x), L_n) < \frac{1}{3} \epsilon$。

综上,存在$N > 0$使得对每一个$n \geq N$和$d_X(x, x_0) < min(\delta, \delta_n)$,我们有
\begin{align*}
  d_Y(L_n, L) & \leq d_Y(L_n, f(x)) + d_Y(f(x), L)                                \\
              & \leq d_Y(f^{(n)}(x), L_n) + d_Y(f^{(n)}(x), f(x)) +  d_Y(f(x), L) \\
              & < \epsilon
\end{align*}

由$\epsilon$的任意性可知,$\lim\limits_{n \to \infty} L_n = L$。

注意:以上证明除了要求$n \geq N$,还要求$d_X(x, x_0) < min(\delta, \delta_n)$,可能会感到疑惑,
使得$\lim\limits_{n \to \infty} L_n$的收敛性不仅和$n$有关,还和$x$的值有关,其实这里的$x$是可以任取的,
其不会影响$\epsilon$是$d_Y(L_n, L)$的上界。

\section*{14.3.3}

因为在例1.2.8中,函数$f = x^n$是逐点收敛的,而不是一致收敛的。

\section*{14.3.4}

由推论$14.3.2$可知,$f$在$X$上连续。对任意$\epsilon > 0, y \in X$,存在$\delta > 0$,
使得只要$d_X(x, y) < \delta$,就有
\begin{align*}
  d_Y(f(x), f(y)) < \frac{1}{2} \epsilon
\end{align*}

因为$f^{(n)}$一致收敛于$f$,那么,存在$N > 0$,使得只要$n \geq N$和$y \in X$,就有
\begin{align*}
  d_Y(f^{(n)}(y), f(y)) < \frac{1}{2} \epsilon
\end{align*}

因为$x^{(n)}$是$X$中收敛于$x$的点列。所以存在$N^\prime > 0$使得只要$n \geq N^\prime$就有
\begin{align*}
  d_X(x^{(n)}, x) < \delta
\end{align*}

综上,$n > max(N, N^\prime)$,就有
\begin{align*}
  d_Y(f^{n}(x^{(n)}), f(x)) & \leq d_Y(f^{(n)}(y), f(y)) + d_Y(f(x), f(y))             \\
                            & < \frac{1}{2} \epsilon + \frac{1}{2} \epsilon = \epsilon
\end{align*}
于是可得
\begin{align*}
  \lim\limits_{n \to \infty} f^{(n)}(x^{(n)}) = f(x)
\end{align*}

\section*{14.3.5}

例14.2.4 中的例子就能说明此事。

$(\frac{1}{2})^{\frac{1}{n}}$收敛于$1$。我们有
\begin{align*}
  \lim\limits_{n \to \infty} f^{(n)}(x^{(n)}) = \lim\limits_{n \to \infty} \frac{1}{2} \neq f(1) = 1
\end{align*}

\section*{14.3.6}

序列$f^{(n)}$一致收敛于$f$。那么,对$\epsilon = 1 > 0$,存在$N > 0$使得对所有的$n \geq N$和$x \in X$都有
\begin{align*}
  d_Y(f^{(n)}(x), f(x)) < \epsilon
\end{align*}

因为对每一个$n$,函数$f^{(n)}$在$X$上都是有界的,所以对每一个$n$,$x \in X$,都有
\begin{align*}
  f^{(n)}(x) \in B(Y,d_Y) (y_n, R_n)
\end{align*}
即
\begin{align*}
  d_Y(f^{(n)}(x), y_n) < R_n
\end{align*}
其中$y_n \in Y, R_n \in \mathbb{R}$。

综上,对$n > N$使得对所有的$n \geq N$和$x \in X$都有
\begin{align*}
  d_Y(f(x), y_n) \leq d_Y(f^{(n)}(x), f(x)) + d_Y(f^{(n)}(x), y_n) 
\end{align*}

特别地$n = N$
\begin{align*}
  d_Y(f(x), y_N) \leq d_Y(f^{(N)}(x), f(x)) + d_Y(f^{(N)}(x), y_N) < R_N + \epsilon
\end{align*}

定义$r = R_N + \epsilon$,对所有的$x \in X$都有$f(x) \in B(Y, d_Y) (y_N, r)$,命题得证。

\section*{14.3.7}

习题14.2.2(c)就能说明此时。在$(-1, 1)$上,$f(x) = x^n$是有界的,$g(x) = x/(1-x)$在$(-1, 1)$上却是无界的,
因为$\lim\limits_{x \to -1} x/(1-x) = \infty$。

\section*{14.3.8}




\end{document}