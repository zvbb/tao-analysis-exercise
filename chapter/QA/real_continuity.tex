\documentclass{article}
\usepackage{mathtools} 
\usepackage{fontspec}
\usepackage[UTF8]{ctex}
\usepackage{amsthm}
\usepackage{mdframed}
\usepackage{xcolor}
\usepackage{amssymb}
\usepackage{amsmath}


% 定义新的带灰色背景的说明环境 zremark
\newmdtheoremenv[
  backgroundcolor=gray!10,
  % 边框与背景一致,边框线会消失
  linecolor=gray!10
]{zremark}{说明}


\begin{document}
\title{实数的连续性}
\author{张志聪}
\maketitle

\begin{zremark}
    在《陶哲轩实分析》(第三版)这个本书中,
    在第9章开头$\textcircled{1}$中说明本书没有严格定义“连续统”,
    给人的感觉是实数$\mathbb{R}$的连续性陶哲轩没有证明。
    但我个人觉得陶哲轩已经完成了证明。
\end{zremark}

\section*{1.直观}

离散集合与连续统是相对的概念。粗略的说,如果集合中的每一个元素与剩余元素之间有一段非零的距离,
那么这个集合就是离散的;如果一个集合是连通的并且没有“洞”,那么这个集合就是一个连续统。

陶哲轩用定理6.4.18(实数的完备性)已经隐含了实数的连续性,
实数的完备性就是实数“连续性”的一种正式数学表达。而且证明完备性的方法也有多种
比如戴德金分割(《数学分析八讲-辛钦》的方法)、柯西序列是收敛序列(《陶正轩实分析》的方法)等。

如果实数有空洞$l \notin \mathbb{R}$,就可以构造一个实数序列不断趋近$l$,
但因为$l \notin \mathbb{R}$,导致这个序列无法收敛到某个实数。

特别地,陶哲轩是通过上极限与下极限相等来证明定理6.4.18的,
也就依赖了柯西序列一定存在上极限和下极限(极限点的定义说明其本身就是实数),
但书中缺少这个说明,接下来我们证明下柯西序列一定具有上极限(下极限类似)。

设$(a_n)_{n = 1}^\infty$是任意实数柯西序列。

因为$(a_n)_{n = 1}^\infty$是实数柯西序列,由引理5.1.15可知,
序列$(a_n)_{n = 1}^\infty$是有界的,
即存在一个实数$M > 0$,使得$|a_n| \leq M$。

我们有
\begin{align*}
  \lim\sup\limits_{n \to \infty} a_n = \inf(a_N^+)_{N = 1}^\infty
\end{align*}
其中
\begin{align*}
  a_N^+ = \sup(a_n)_{n = N}^\infty
\end{align*}
于是可知$(a_N^+)_{N = 1}^\infty$是一个单调递减的序列,
由命题6.3.8(单调有界序列收敛于实数,这个命题的证明过程会依赖定理5.5.9)可知,且
\begin{align*}
  \lim\limits_{N \to \infty} a_N^+ = \inf(a_N^+)_{N = 1}^\infty \leq M
\end{align*}
命题得证。

\end{document}
