\documentclass{article}
\usepackage{mathtools} 
\usepackage{fontspec}
\usepackage[UTF8]{ctex}
\usepackage{amsthm}
\usepackage{mdframed}
\usepackage{xcolor}
\usepackage{amssymb}
\usepackage{amsmath}


% 定义新的带灰色背景的说明环境 zremark
\newmdtheoremenv[
  backgroundcolor=gray!10,
  % 边框与背景一致,边框线会消失
  linecolor=gray!10
]{zremark}{说明}


\begin{document}
\title{实数的连续性}
\author{张志聪}
\maketitle

\begin{zremark}
    在《陶哲轩实分析》(第三版)这个本书中,没有证明实数是“连续统”的,
    在第9章开头,作者将实数定义为“连续统”,并且没有给出证明。
\end{zremark}

\section*{1.直观}

离散集合与连续统是相对的概念。粗略的说,如果集合中的每一个元素与剩余元素之间有一段非零的距离,
那么这个集合就是离散的;如果一个集合是连通的并且没有“洞”,那么这个集合就是一个连续统。

\end{document}
