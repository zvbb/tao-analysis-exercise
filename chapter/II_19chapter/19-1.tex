\documentclass{article}
\usepackage{mathtools} 
\usepackage{fontspec}
\usepackage[UTF8]{ctex}
\usepackage{amsthm}
\usepackage{mdframed}
\usepackage{xcolor}
\usepackage{amssymb}
\usepackage{amsmath}


% 定义新的带灰色背景的说明环境 zremark
\newmdtheoremenv[
  backgroundcolor=gray!10,
  % 边框与背景一致,边框线会消失
  linecolor=gray!10
]{zremark}{说明}

% 通用矩阵命令: \flexmatrix{矩阵名}{元素符号}{行数}{列数}
\newcommand{\flexmatrix}[4]{
  \[
  #1 = \begin{pmatrix}
    #2_{11}     & #2_{12}     & \cdots & #2_{1#4}   \\
    #2_{21}     & #2_{22}     & \cdots & #2_{2#4}   \\
    \vdots      & \vdots      & \ddots & \vdots     \\
    #2_{#31}    & #2_{#32}    & \cdots & #2_{#3#4}
  \end{pmatrix}
  \]
}

% 简化版命令(默认矩阵名为A,元素符号为a): \quickmatrix{行数}{列数}
\newcommand{\quickmatrix}[2]{\flexmatrix{A}{a}{#1}{#2}}


\begin{document}
\title{19.1 习题}
\author{张志聪}
\maketitle

\section*{19.1.1}

 (1)

因为 $f$ 是简单函数,设 $f(\Omega) = \{c_1, c_2, \cdots, c_N\}$,定义:
\[
  E_j := \{x \in \Omega : f(x) = c_j\}, \quad 1 \leq j \leq N
\]
这些集合两两不交,且 $\bigcup_{j=1}^N E_j = \Omega$。

同理,设 $g(\Omega) = \{d_1, d_2, \cdots, d_M\}$,定义:
\[
  X_k := \{x \in \Omega : g(x) = d_k\}, \quad 1 \leq k \leq M
\]
这些集合也两两不交,且 $\bigcup_{k=1}^M X_k = \Omega$。

考虑集合:
\[
  W := \{W_{j,k} := E_j \cap X_k : 1 \leq j \leq N,\ 1 \leq k \leq M\}
\]
由于交集运算,$W_{j,k}$ 两两不交,且它们的并集仍为 $\Omega$,因此 $W$ 构成了 $\Omega$ 的一个有限划分。

对任意 $x \in \Omega$,存在唯一的 $j,k$ 使得 $x \in W_{j,k}$,此时有:
\[
  (f + g)(x) = f(x) + g(x) = c_j + d_k
\]
因为 $c_j + d_k$ 的可能取值至多为 $N \times M$ 个有限个数,所以 $f + g$ 的取值是有限的。

因此,$f + g$ 是简单函数。

(2)

特别地$cf$也是简单函数。此时$cf(\Omega) = \{c \times c_1, c \times c_2, \cdots,c \times c_N\}$,
满足简单函数的定义。

\section*{19.1.2}

因为 $f$ 是简单函数,设 $f(\Omega) = \{c_1, c_2, \cdots, c_N\}$,定义:
\[
  E_j := \{x \in \Omega : f(x) = c_j\}, \quad 1 \leq j \leq N
\]
这些集合两两不交,且 $\bigcup_{j=1}^N E_j = \Omega$。

因为$f$是可测函数,所以对任意$1 \leq j \leq N$,$f^{-1}(c_j) = E_j$都是可测集合。

对于任意$x \in \Omega$,存在$E_j$使得$x \in E_j$,此时$f(x) = c_j$,
又因为对
\begin{align*}
  \sum \limits_{i = 1}^N c_i \chi_{E_i} (x)
\end{align*}
当$i \neq j$时,由特征函数的定义可知$c_i \chi_{E_i} = 0 $;

当$i = j$时,由特征函数的定义可知$c_i \chi_{E_i} = c_j$;

所以
\begin{align*}
  \sum \limits_{i = 1}^N c_i \chi_{E_i}(x) = c_j
\end{align*}

综上可得
\begin{align*}
  f(x) = \sum \limits_{i = 1}^N c_i \chi_{E_i}(x)
\end{align*}

由$x$的任意性可知,$f =  \sum \limits_{i = 1}^N c_i \chi_{E_i}$

\section*{19.1.3}

设$(f_n)_{n = 1}^\infty$是函数序列,其中$f_n:  \Omega \to \mathbb{R}$,
\begin{align*}
  f_n(x) = \sup\{\frac{j}{2^n}: j \in \mathbb{Z}, \frac{j}{2^n} \leq min(f(x), 2^n)\}
\end{align*}
接下来证明该序列是否满足所需的性质。

(1) 证明$f_1(x) \geq 0$。
\begin{align*}
  f_1(x) & = \sup\{\frac{j}{2^1}: j \in \mathbb{Z}, \frac{j}{2^1} \leq min(f(x), 2^1)\} \\
         & = \sup\{\frac{j}{2}: j \in \mathbb{Z}, \frac{j}{2} \leq min(f(x), 2)\}       \\
\end{align*}
因为$f$是非负的,所以$min(f(x), 2) \geq 0$。

现在证明$f_1(x) \geq 0$,反证法,假设$f_1(x) < 0$,
即
\begin{align*}
  f_1(x) = \sup\{\frac{j}{2}: j \in \mathbb{Z}, \frac{j}{2} \leq min(f(x), 2)\} < 0
\end{align*}
当$j = 0$时,$0 \in \{\frac{j}{2}: j \in \mathbb{Z}, \frac{j}{2} \leq min(f(x), 2)\}$,
这与上确界$f_1(x) < 0$矛盾,假设不成立。

(2) 证明$(f_n)_{n = 1}^\infty$序列是递增的,即$f_n(x) \leq f_{n + 1}(x)$。

我们有
\begin{align*}
  f_n(x)       & = \sup\{\frac{j}{2^n}: j \in \mathbb{Z}, \frac{j}{2^n} \leq min(f(x), 2^n)\}                   \\
  f_{n + 1}(x) & = \sup\{\frac{j}{2^{n + 1}}: j \in \mathbb{Z}, \frac{j}{2^{n + 1}} \leq min(f(x), 2^{n + 1})\}
\end{align*}

因为$2^n < 2^{n + 1}$,所有
\begin{align*}
  min(f(x), 2^n) & \leq min(f(x), 2^{n + 1})
\end{align*}

反证法,假设$f_n(x) > f_{n + 1}(x)$,即存在
$y \in \{\frac{j}{2^n}: j \in \mathbb{Z}, \frac{j}{2^n} \leq min(f(x), 2^n)\}$使得
\begin{align*}
  y > f_{n + 1}(x)
\end{align*}
$y$可表示成$\frac{j_0}{2^n}$的形式,即$y = \frac{j_0}{2^n}$,
于是我们有
\begin{align*}
  \frac{j_0}{2^n} > f_{n + 1}(x)
\end{align*}
于是可得
\begin{align*}
  \frac{j_0}{2^n} \leq min(f(x), 2^n) \leq min(f(x), 2^{n + 1})
\end{align*}
这表明$\frac{j_0}{2^n} = \frac{2j_0}{2^{n + 1}} \in \{\frac{j}{2^{n + 1}}: j \in \mathbb{Z}, \frac{j}{2^{n + 1}} \leq min(f(x), 2^{n + 1})\}$,
于是存在矛盾。

(3) 证明$(f_n)_{n = 1}^\infty$序列逐点收敛于$f$。

对任意$x_0 \in \Omega, \epsilon > 0$。
我们有
\begin{align*}
  \lim\limits_{n \to \infty} 2^n = +\infty
\end{align*}
因为$f(x_0)$是定值,所以存在$N \geq 1$使得只要$n \geq N$,就有
\begin{align*}
  2^n \geq f(x_0)
\end{align*}
于是可得,当$n \geq N$时,
\begin{align*}
  min(f(x_0), 2^n) = f(x_0)
\end{align*}

存在$k \in \mathbb{Z}$使得$\frac{k}{2^n} \leq f_n(x_0) < \frac{k+1}{2^n}$(参考习题5.5.2),
又有
\begin{align*}
  \frac{k}{2^n}   & \in \{\frac{j}{2^n}: j \in \mathbb{Z}, \frac{j}{2^n} \leq min(f(x_0), 2^n) = = f(x_0)\} \\
  \frac{k+1}{2^n} & > min(f(x_0), 2^n) = f(x_0)
\end{align*}
综上可得,存在$n \geq N$使得
\begin{align*}
  |f_n(x_0) - f(x_0)| < \frac{1}{2^n}
\end{align*}
因为
\begin{align*}
  \lim\limits_{n \to \infty} \frac{1}{2^n} = 0
\end{align*}
所以存在$N^\prime \geq N$,使得
\begin{align*}
  \frac{1}{2^n} < \epsilon
\end{align*}

综上,对任意$x_0 \in \Omega, \epsilon > 0$,存在$N^\prime$,使得只要$n \geq N^\prime$,就有
\begin{align*}
  |f_n(x_0) - f(x_0)| < \epsilon
\end{align*}
所以$(f_n)_{n = 1}^\infty$序列逐点收敛于$f$。

\end{document}