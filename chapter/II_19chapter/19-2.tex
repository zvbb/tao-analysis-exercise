\documentclass{article}
\usepackage{mathtools} 
\usepackage{fontspec}
\usepackage[UTF8]{ctex}
\usepackage{amsthm}
\usepackage{mdframed}
\usepackage{xcolor}
\usepackage{amssymb}
\usepackage{amsmath}


% 定义新的带灰色背景的说明环境 zremark
\newmdtheoremenv[
  backgroundcolor=gray!10,
  % 边框与背景一致,边框线会消失
  linecolor=gray!10
]{zremark}{说明}

% 通用矩阵命令: \flexmatrix{矩阵名}{元素符号}{行数}{列数}
\newcommand{\flexmatrix}[4]{
  \[
  #1 = \begin{pmatrix}
    #2_{11}     & #2_{12}     & \cdots & #2_{1#4}   \\
    #2_{21}     & #2_{22}     & \cdots & #2_{2#4}   \\
    \vdots      & \vdots      & \ddots & \vdots     \\
    #2_{#31}    & #2_{#32}    & \cdots & #2_{#3#4}
  \end{pmatrix}
  \]
}

% 简化版命令(默认矩阵名为A,元素符号为a): \quickmatrix{行数}{列数}
\newcommand{\quickmatrix}[2]{\flexmatrix{A}{a}{#1}{#2}}


\begin{document}
\title{19.2 习题}
\author{张志聪}
\maketitle

\section*{19.2.1}

\begin{itemize}
  \item (a)

        (1)

        因为$0$是从下方控制$f$的非负简单函数,因此可得
        \begin{align*}
          0 \leq \int_{\Omega} f \leq +\infty
        \end{align*}

        (2)
        \begin{itemize}
          \item $\Rightarrow$

                考虑集合
                \begin{align*}
                  A : = \{x \in \Omega : f(x) > 0\}
                \end{align*}
                我们要证明这个集合的测度为零。

                对于任意的$n \in \mathbb{N}^+$,定义
                \begin{align*}
                  A_n : = \{x \in \Omega : f(x) > \frac{1}{n}\}
                \end{align*}
                显然$A_n \subseteq A$,并且
                \begin{align*}
                  A = \bigcup_{n = 1}^{\infty} A_n
                \end{align*}
                因为在$A_n$上,$f(x) \geq \frac{1}{n}$,于是我们有
                \begin{align*}
                  \int_{\Omega} f \geq \int_{A_n} f \geq \int_{A_n} \frac{1}{n} = \frac{1}{n} m(A_n) \geq 0
                \end{align*}
                (注意没有用到(c),因为这个可以通过定义19.2.2和命题19.1.10推出)

                通过题设可知,$\int_{\Omega} f = 0$,所以
                \begin{align*}
                  \frac{1}{n} m(A_n) \leq 0 \\
                  \implies                  \\
                  m(A_n) = 0
                \end{align*}
                于是,我们有
                \begin{align*}
                  m(A) = m(\bigcup_{n = 1}^{\infty} A_n) \leq \sum_{n = 1}^{\infty} m(A_n) = 0
                \end{align*}
                命题得证。

          \item $\Leftarrow$

                设
                \begin{align*}
                  A = \{x \in \Omega : f(x) \neq 0\}
                \end{align*}
                由题设可知
                \begin{align*}
                  m(A) = 0
                \end{align*}
                于是由命题19.1.10(a)可知,
                对所有$s$是一个非负简单函数,并且$s$从下方控制$f$的函数,我们有
                \begin{align*}
                  \int_{\Omega} s = 0
                \end{align*}
                所以
                \begin{align*}
                  \int_{\Omega} f = 0
                \end{align*}
        \end{itemize}

  \item (b)

        考虑集合
        \begin{align*}
          A = \{\int_{\Omega} s: s \text{是一个非负简单函数,并且$s$是从下方控制$f$}\} \\
          B = \{\int_{\Omega} s: s \text{是一个非负简单函数,并且$s$是从下方控制$cf$}\}
        \end{align*}

        对任意$s$从下方控制$f$时,$cs$也从下方控制$cf$(反之亦成立)。
        而且由命题19.1.10(c)可知,
        \begin{align*}
          \int_{\Omega} cs = c \int_{\Omega} s
        \end{align*}
        于是可得
        \begin{align*}
          x \in A         \\
          \Leftrightarrow \\
          cx \in B
        \end{align*}
        所以
        \begin{align*}
          \sup(B) = c \sup(A)
        \end{align*}
        即:
        \begin{align*}
          \int_{\Omega} cf = c \int_{\Omega} f
        \end{align*}

  \item (c)

        考虑集合
        \begin{align*}
          A = \{\int_{\Omega} s: s \text{是一个非负简单函数,并且$s$是从下方控制$f$}\} \\
          B = \{\int_{\Omega} s: s \text{是一个非负简单函数,并且$s$是从下方控制$g$}\}
        \end{align*}

        由题设可知,对任意$s$从下方控制$f$时,$s$也从下方控制$g$。
        于是可得
        \begin{align*}
          x \in A
          \implies
          x \in B
          \implies
          A \subseteq B
        \end{align*}
        于是我们有
        \begin{align*}
          sup(A) \leq sup(B)
        \end{align*}
        即
        \begin{align*}
          \int_{\Omega} f \leq \int_{\Omega} g
        \end{align*}
  \item (d)

        考虑集合
        \begin{align*}
          A : = \{x \in \Omega : f(x) \neq g(x)\}
        \end{align*}
        有题设可知$m(A) = 0$。

        对任意$\epsilon > 0$,由定义19.2.2(通过上确界的方式定义的)可知,
        存在一个非负简单函数$s$,使得
        \begin{align*}
          \int_{\Omega} f - \epsilon < \int_{\Omega} s
        \end{align*}
        定义一个$s^\prime$从下方控制$g$,
        \begin{equation*}
          s^\prime(x) = \begin{cases}
            0    & \text{if } x \in A    \\
            s(x) & \text{if } x \notin A
          \end{cases}
        \end{equation*}
        于是可得
        \begin{align*}
          \int_{\Omega} s^\prime \leq \int_{\Omega} g
        \end{align*}
        令$h = s - s^\prime$,于是由命题19.1.10(a)可知
        \begin{align*}
          \int_{\Omega} h = 0
        \end{align*}
        因为$s = h + s^\prime$,于是由命题19.1.10(b)可知
        \begin{align*}
          \int_{\Omega} s = \int_{\Omega} h + \int_{\Omega} s^\prime = \int_{\Omega} s^\prime
        \end{align*}
        综上可得
        \begin{align*}
          \int_{\Omega} f - \epsilon < \int_{\Omega} s^\prime \leq \int_{\Omega} g
        \end{align*}
        由$\epsilon$的任意性可得
        \begin{align*}
          \int_{\Omega} f \leq \int_{\Omega} g
        \end{align*}
        类似地,可得
        \begin{align*}
          \int_{\Omega} g \leq \int_{\Omega} f
        \end{align*}
        所以
        \begin{align*}
          \int_{\Omega} g  = \int_{\Omega} f
        \end{align*}

  \item (e)

        命题有些错误,应该是:

        \textbf{如果$\Omega^\prime \subseteq \Omega$是一个可测集,
          那么$\int_{\Omega^\prime} f = \int_{\Omega} f_{\chi_{\Omega^\prime}} \leq \int_{\Omega} f$,
          其中$f_{\chi_{\Omega^\prime}}$表示只在$\Omega^\prime$上保留$f$的值,其它地方为0。
        }

        (1)先证明$f$是非负简单函数时,命题成立。

        因为在$\Omega$上,$f \geq f_{\chi_{\Omega^\prime}}$,由命题19.1.10(d)可得
        \begin{align*}
          \int_{\Omega} f_{\chi_{\Omega^\prime}} \leq \int_{\Omega} f
        \end{align*}

        因为 $f$ 是非负简单函数,设 $f(\Omega^\prime) = \{c_1, c_2, \cdots, c_N\}$,定义:
        \[
          E_j := \{x \in \Omega^\prime : f(x) = c_j\}, \quad 1 \leq j \leq N
        \]
        这些集合两两不交,且 $\bigcup_{j=1}^N E_j = \Omega^\prime$。
        由引理19.1.9,$\int_{\Omega^\prime} f$可表示为
        \begin{align*}
          \int_{\Omega^\prime} f = \sum\limits_{j = 1}^N c_j m(E_j)
        \end{align*}

        记$E_0 = \Omega \setminus \Omega^\prime$,由题设可知$E_0$是可测集(因为可以被$f^{-1}$表示出来,
        而且$f$是可测函数。),
        且与$E_j, 1 \leq j \leq N$不相交,
        于是由引理19.1.9,$\int_{\Omega} f_{\chi_{\Omega^\prime}}$可表示为
        \begin{align*}
          \int_{\Omega} f_{\chi_{\Omega^\prime}}
           & = \sum\limits_{j = 0}^N c_j m(E_j)                   \\
           & = 0 \times m(E_0) + \sum\limits_{j = 1}^N c_j m(E_j) \\
           & = \sum\limits_{j = 1}^N c_j m(E_j)                   \\
           & = \int_{\Omega^\prime} f
        \end{align*}

        (2) $f$是非负可测函数,命题成立。

        因为在$\Omega$上,$f \geq f_{\chi_{\Omega^\prime}}$,由(c)可得
        \begin{align*}
          \int_{\Omega} f_{\chi_{\Omega^\prime}} \leq \int_{\Omega} f
        \end{align*}

        对任意$\epsilon > 0$,存在一个非负简单函数$s$,并且从下方控制$f$,使得
        \begin{align*}
          \int_{\Omega^\prime} f - \epsilon < \int_{\Omega^\prime} s
        \end{align*}
        令
        \begin{equation*}
          s^\prime(x) = \begin{cases}
            0    & \text{if } x \in \Omega \setminus \Omega^\prime \\
            s(x) & \text{if } x \notin \Omega^\prime
          \end{cases}
        \end{equation*}
        于是$s^\prime$是一个非负简单函数,并且从下方控制$f_{\chi_{\Omega^\prime}}$,
        所以
        \begin{align*}
          \int_{\Omega} s^\prime \leq \int_{\Omega} f_{\chi_{\Omega^\prime}}
        \end{align*}
        又由(1)可知
        \begin{align*}
          \int_{\Omega^\prime} s = \int_{\Omega} s^\prime
        \end{align*}
        综上可得
        \begin{align*}
          \int_{\Omega^\prime} f - \epsilon < \int_{\Omega} f_{\chi_{\Omega^\prime}} \\
          \implies                                                                   \\
          \int_{\Omega^\prime} f \leq \int_{\Omega} f_{\chi_{\Omega^\prime}}
        \end{align*}

        类似的,存在一个非负简单函数$s$,并且从下方控制$f_{\chi_{\Omega^\prime}}$,使得
        \begin{align*}
          \int_{\Omega} f_{\chi_{\Omega^\prime}} - \epsilon < \int_{\Omega} s
        \end{align*}
        令
        \begin{equation*}
          s^\prime(x) = s(x) \;\;\; x \in \Omega^\prime
        \end{equation*}
        由于$s$是从下方控制$f_{\chi_{\Omega^\prime}}$,
        于是$\Omega \setminus \Omega^\prime$上$s(x) = 0$,
        于是由(1),我们有
        \begin{align*}
          \int_{\Omega^\prime} s^\prime = \int_{\Omega} s
        \end{align*}
        又因为$s^\prime$从下方控制$f$,所以
        \begin{align*}
          \int_{\Omega^\prime} s^\prime \leq \int_{\Omega^\prime} f
        \end{align*}
        综上可得
        \begin{align*}
          \int_{\Omega} f_{\chi_{\Omega^\prime}} - \epsilon < \int_{\Omega^\prime} f \\
          \implies                                                                   \\
          \int_{\Omega} f_{\chi_{\Omega^\prime}} \leq \int_{\Omega^\prime} f
        \end{align*}

        综上,我们有
        \begin{align*}
          \int_{\Omega} f_{\chi_{\Omega^\prime}} = \int_{\Omega^\prime} f
        \end{align*}

\end{itemize}

\section*{19.2.2}

对任意$\epsilon > 0$,存在非负简单函数$s_f, s_g$,分别从下方控制$f, g$,使得
\begin{align*}
  \int_{\Omega} f - \frac{1}{2}\epsilon < \int_{\Omega} s_f \\
  \int_{\Omega} g - \frac{1}{2}\epsilon < \int_{\Omega} s_g
\end{align*}
因为
\begin{align*}
  s_f(x) \leq f(x) \\
  s_g(x) \leq g(x)
\end{align*}
于是,我们有
\begin{align*}
  (s_f + s_g)(x) \leq (f + g)(x)
\end{align*}
即非负简单函数$s_f + s_g$从下方控制$f + g$,于是我们有
\begin{align*}
  \int_{\Omega} s_f + s_g \leq \int_{\Omega} (f + g)
\end{align*}
由命题19.1.10可知,
\begin{align*}
  \int_{\Omega} s_f + s_g =  \int_{\Omega} f + \int_{\Omega} g
\end{align*}
综上可得
\begin{align*}
  \int_{\Omega} f + \int_{\Omega} g - \epsilon < \int_{\Omega} s_f + \int_{\Omega} s_g = \int_{\Omega} s_f + s_g \leq \int_{\Omega} (f + g) \\
  \implies \\
  \int_{\Omega} f + \int_{\Omega} g \leq \int_{\Omega} (f + g)
\end{align*}

\section*{19.2.3}

考虑序列$(F_N)_{N = 1}^\infty$,其中
\begin{align*}
  F_N : = \sum \limits_{n = 1}^N g_n
\end{align*}
因为$f_n$都是非负可测函数,于是我们有
\begin{align*}
  0 \leq F_1 \leq F_2 \leq \cdots 
\end{align*}
且有推论18.5.7可知,$F_N$都是非负可测函数。
于是利用定理19.2.9,我们有
\begin{align*}
  \int_{\Omega} \sup\limits_{N} F_N = \sup\limits_{N} \int_{\Omega} F_N
\end{align*}

因为$(F_N)_{N = 1}^\infty$是单调的递增序列,
于是有
\begin{align*}
  \sum \limits_{n = 1}^\infty g_n = \lim\limits_{N \to \infty} F_N = \sup\limits_{N} F_N
\end{align*}
(第二个等式的证明在19-2-comment.tex中有)\\
于是
\begin{align*}
  \int_{\Omega} \sup\limits_{N} F_N  = \int_{\Omega} \sum \limits_{n = 1}^\infty g_n 
\end{align*}

同理可得(考虑$\int_{\Omega} F_N = \sum \limits_{n = 1}^N \int_{\Omega} g_n $),
\begin{align*}
  \sup\limits_{N} \int_{\Omega} F_N = \sum \limits_{n = 1}^\infty \int_{\Omega} g_n 
\end{align*}

综上,我们有
\begin{align*}
  \int_{\Omega} \sum \limits_{n = 1}^\infty g_n = \sum \limits_{n = 1}^\infty \int_{\Omega} g_n
\end{align*}

\end{document}