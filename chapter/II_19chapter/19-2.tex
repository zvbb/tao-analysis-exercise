\documentclass{article}
\usepackage{mathtools} 
\usepackage{fontspec}
\usepackage[UTF8]{ctex}
\usepackage{amsthm}
\usepackage{mdframed}
\usepackage{xcolor}
\usepackage{amssymb}
\usepackage{amsmath}


% 定义新的带灰色背景的说明环境 zremark
\newmdtheoremenv[
  backgroundcolor=gray!10,
  % 边框与背景一致,边框线会消失
  linecolor=gray!10
]{zremark}{说明}

% 通用矩阵命令: \flexmatrix{矩阵名}{元素符号}{行数}{列数}
\newcommand{\flexmatrix}[4]{
  \[
  #1 = \begin{pmatrix}
    #2_{11}     & #2_{12}     & \cdots & #2_{1#4}   \\
    #2_{21}     & #2_{22}     & \cdots & #2_{2#4}   \\
    \vdots      & \vdots      & \ddots & \vdots     \\
    #2_{#31}    & #2_{#32}    & \cdots & #2_{#3#4}
  \end{pmatrix}
  \]
}

% 简化版命令(默认矩阵名为A,元素符号为a): \quickmatrix{行数}{列数}
\newcommand{\quickmatrix}[2]{\flexmatrix{A}{a}{#1}{#2}}


\begin{document}
\title{19.2 习题}
\author{张志聪}
\maketitle

\section*{19.2.1}

\begin{itemize}
  \item (a)

        (1)

        因为$0$是从下方控制$f$的非负简单函数,因此可得
        \begin{align*}
          0 \leq \int_{\Omega} f \leq +\infty
        \end{align*}

        (2)
        \begin{itemize}
          \item $\Rightarrow$

                考虑集合
                \begin{align*}
                  A : = \{x \in \Omega : f(x) > 0\}
                \end{align*}
                我们要证明这个集合的测度为零。

                对于任意的$n \in \mathbb{N}^+$,定义
                \begin{align*}
                  A_n : = \{x \in \Omega : f(x) > \frac{1}{n}\}
                \end{align*}
                显然$A_n \subseteq A$,并且
                \begin{align*}
                  A = \bigcup_{n = 1}^{\infty} A_n
                \end{align*}
                因为在$A_n$上,$f(x) \geq \frac{1}{n}$,于是我们有
                \begin{align*}
                  \int_{\Omega} f \geq \int_{A_n} f \geq \int_{A_n} \frac{1}{n} = \frac{1}{n} m(A_n) \geq 0
                \end{align*}
                (注意没有用到(c),因为这个可以通过定义19.2.2和命题19.1.10推出)

                通过题设可知,$\int_{\Omega} f = 0$,所以
                \begin{align*}
                  \frac{1}{n} m(A_n) \leq 0 \\
                  \implies                  \\
                  m(A_n) = 0
                \end{align*}
                于是,我们有
                \begin{align*}
                  m(A) = m(\bigcup_{n = 1}^{\infty} A_n) \leq \sum_{n = 1}^{\infty} m(A_n) = 0
                \end{align*}
                命题得证。

          \item $\Leftarrow$

                设
                \begin{align*}
                  A = \{x \in \Omega : f(x) \neq 0\}
                \end{align*}
                由题设可知
                \begin{align*}
                  m(A) = 0
                \end{align*}
                于是由命题19.1.10(a)可知,
                对所有$s$是一个非负简单函数,并且$s$从下方控制$f$的函数,我们有
                \begin{align*}
                  \int_{\Omega} s = 0
                \end{align*}
                所以
                \begin{align*}
                  \int_{\Omega} f = 0
                \end{align*}
        \end{itemize}

  \item (b)

        考虑集合
        \begin{align*}
          A = \{\int_{\Omega} s: s \text{是一个非负简单函数,并且$s$是从下方控制$f$}\} \\
          B = \{\int_{\Omega} s: s \text{是一个非负简单函数,并且$s$是从下方控制$cf$}\}
        \end{align*}

        对任意$s$从下方控制$f$时,$cs$也从下方控制$cf$(反之亦成立)。
        而且由命题19.1.10(c)可知,
        \begin{align*}
          \int_{\Omega} cs = c \int_{\Omega} s
        \end{align*}
        于是可得
        \begin{align*}
          x \in A         \\
          \Leftrightarrow \\
          cx \in B
        \end{align*}
        所以
        \begin{align*}
          \sup(B) = c \sup(A)
        \end{align*}
        即:
        \begin{align*}
          \int_{\Omega} cf = c \int_{\Omega} f
        \end{align*}

  \item (c)

        考虑集合
        \begin{align*}
          A = \{\int_{\Omega} s: s \text{是一个非负简单函数,并且$s$是从下方控制$f$}\} \\
          B = \{\int_{\Omega} s: s \text{是一个非负简单函数,并且$s$是从下方控制$g$}\}
        \end{align*}

        由题设可知,对任意$s$从下方控制$f$时,$s$也从下方控制$g$。
        于是可得
        \begin{align*}
          x \in A
          \implies
          x \in B
          \implies
          A \subseteq B
        \end{align*}
        于是我们有
        \begin{align*}
          sup(A) \leq sup(B)
        \end{align*}
        即
        \begin{align*}
          \int_{\Omega} f \leq \int_{\Omega} g
        \end{align*}
  \item (d)

        考虑集合
        \begin{align*}
          A : = \{x \in \Omega : f(x) \neq g(x)\}
        \end{align*}
        有题设可知$m(A) = 0$。

        对任意$\epsilon > 0$,由定义19.2.2(通过上确界的方式定义的)可知,
        存在一个非负简单函数$s$,使得
        \begin{align*}
          \int_{\Omega} f - \epsilon < \int_{\Omega} s
        \end{align*}
        定义一个$s^\prime$从下方控制$g$,
        \begin{equation*}
          s^\prime(x) = \begin{cases}
            0    & \text{if } x \in A    \\
            s(x) & \text{if } x \notin A
          \end{cases}
        \end{equation*}
        于是可得
        \begin{align*}
          \int_{\Omega} s^\prime \leq \int_{\Omega} g
        \end{align*}
        令$h = s - s^\prime$,于是由命题19.1.10(a)可知
        \begin{align*}
          \int_{\Omega} h = 0
        \end{align*}
        因为$s = h + s^\prime$,于是由命题19.1.10(b)可知
        \begin{align*}
          \int_{\Omega} s = \int_{\Omega} h + \int_{\Omega} s^\prime = \int_{\Omega} s^\prime
        \end{align*}
        综上可得
        \begin{align*}
          \int_{\Omega} f - \epsilon < \int_{\Omega} s^\prime \leq \int_{\Omega} g
        \end{align*}
        由$\epsilon$的任意性可得
        \begin{align*}
          \int_{\Omega} f \leq \int_{\Omega} g
        \end{align*}
        类似地,可得
        \begin{align*}
          \int_{\Omega} g \leq \int_{\Omega} f
        \end{align*}
        所以
        \begin{align*}
          \int_{\Omega} g  = \int_{\Omega} f
        \end{align*}

  \item (e)

        命题有些错误,应该是:

        \textbf{如果$\Omega^\prime \subseteq \Omega$是一个可测集,
          那么$\int_{\Omega^\prime} f = \int_{\Omega} f_{\chi_{\Omega^\prime}} \leq \int_{\Omega} f$,
          其中$f_{\chi_{\Omega^\prime}}$表示只在$\Omega^\prime$上保留$f$的值,其它地方为0。
        }

        (1)先证明$f$是非负简单函数时,命题成立。

        因为在$\Omega$上,$f \geq f_{\chi_{\Omega^\prime}}$,由命题19.1.10(d)可得
        \begin{align*}
          \int_{\Omega} f_{\chi_{\Omega^\prime}} \leq \int_{\Omega} f
        \end{align*}

        因为 $f$ 是非负简单函数,设 $f(\Omega^\prime) = \{c_1, c_2, \cdots, c_N\}$,定义:
        \[
          E_j := \{x \in \Omega^\prime : f(x) = c_j\}, \quad 1 \leq j \leq N
        \]
        这些集合两两不交,且 $\bigcup_{j=1}^N E_j = \Omega^\prime$。
        由引理19.1.9,$\int_{\Omega^\prime} f$可表示为
        \begin{align*}
          \int_{\Omega^\prime} f = \sum\limits_{j = 1}^N c_j m(E_j)
        \end{align*}

        记$E_0 = \Omega \setminus \Omega^\prime$,由题设可知$E_0$是可测集(因为可以被$f^{-1}$表示出来,
        而且$f$是可测函数。),
        且与$E_j, 1 \leq j \leq N$不相交,
        于是由引理19.1.9,$\int_{\Omega} f_{\chi_{\Omega^\prime}}$可表示为
        \begin{align*}
          \int_{\Omega} f_{\chi_{\Omega^\prime}}
           & = \sum\limits_{j = 0}^N c_j m(E_j)                   \\
           & = 0 \times m(E_0) + \sum\limits_{j = 1}^N c_j m(E_j) \\
           & = \sum\limits_{j = 1}^N c_j m(E_j)                   \\
           & = \int_{\Omega^\prime} f
        \end{align*}

        (2) $f$是非负可测函数,命题成立。

        因为在$\Omega$上,$f \geq f_{\chi_{\Omega^\prime}}$,由(c)可得
        \begin{align*}
          \int_{\Omega} f_{\chi_{\Omega^\prime}} \leq \int_{\Omega} f
        \end{align*}

        对任意$\epsilon > 0$,存在一个非负简单函数$s$,并且从下方控制$f$,使得
        \begin{align*}
          \int_{\Omega^\prime} f - \epsilon < \int_{\Omega^\prime} s
        \end{align*}
        令
        \begin{equation*}
          s^\prime(x) = \begin{cases}
            0    & \text{if } x \in \Omega \setminus \Omega^\prime \\
            s(x) & \text{if } x \notin \Omega^\prime
          \end{cases}
        \end{equation*}
        于是$s^\prime$是一个非负简单函数,并且从下方控制$f_{\chi_{\Omega^\prime}}$,
        所以
        \begin{align*}
          \int_{\Omega} s^\prime \leq \int_{\Omega} f_{\chi_{\Omega^\prime}}
        \end{align*}
        又由(1)可知
        \begin{align*}
          \int_{\Omega^\prime} s = \int_{\Omega} s^\prime
        \end{align*}
        综上可得
        \begin{align*}
          \int_{\Omega^\prime} f - \epsilon < \int_{\Omega} f_{\chi_{\Omega^\prime}} \\
          \implies                                                                   \\
          \int_{\Omega^\prime} f \leq \int_{\Omega} f_{\chi_{\Omega^\prime}}
        \end{align*}

        类似的,存在一个非负简单函数$s$,并且从下方控制$f_{\chi_{\Omega^\prime}}$,使得
        \begin{align*}
          \int_{\Omega} f_{\chi_{\Omega^\prime}} - \epsilon < \int_{\Omega} s
        \end{align*}
        令
        \begin{equation*}
          s^\prime(x) = s(x) \;\;\; x \in \Omega^\prime
        \end{equation*}
        由于$s$是从下方控制$f_{\chi_{\Omega^\prime}}$,
        于是$\Omega \setminus \Omega^\prime$上$s(x) = 0$,
        于是由(1),我们有
        \begin{align*}
          \int_{\Omega^\prime} s^\prime = \int_{\Omega} s
        \end{align*}
        又因为$s^\prime$从下方控制$f$,所以
        \begin{align*}
          \int_{\Omega^\prime} s^\prime \leq \int_{\Omega^\prime} f
        \end{align*}
        综上可得
        \begin{align*}
          \int_{\Omega} f_{\chi_{\Omega^\prime}} - \epsilon < \int_{\Omega^\prime} f \\
          \implies                                                                   \\
          \int_{\Omega} f_{\chi_{\Omega^\prime}} \leq \int_{\Omega^\prime} f
        \end{align*}

        综上,我们有
        \begin{align*}
          \int_{\Omega} f_{\chi_{\Omega^\prime}} = \int_{\Omega^\prime} f
        \end{align*}

\end{itemize}

\section*{19.2.2}

对任意$\epsilon > 0$,存在非负简单函数$s_f, s_g$,分别从下方控制$f, g$,使得
\begin{align*}
  \int_{\Omega} f - \frac{1}{2}\epsilon < \int_{\Omega} s_f \\
  \int_{\Omega} g - \frac{1}{2}\epsilon < \int_{\Omega} s_g
\end{align*}
因为
\begin{align*}
  s_f(x) \leq f(x) \\
  s_g(x) \leq g(x)
\end{align*}
于是,我们有
\begin{align*}
  (s_f + s_g)(x) \leq (f + g)(x)
\end{align*}
即非负简单函数$s_f + s_g$从下方控制$f + g$,于是我们有
\begin{align*}
  \int_{\Omega} s_f + s_g \leq \int_{\Omega} (f + g)
\end{align*}
由命题19.1.10可知,
\begin{align*}
  \int_{\Omega} s_f + s_g =  \int_{\Omega} f + \int_{\Omega} g
\end{align*}
综上可得
\begin{align*}
  \int_{\Omega} f + \int_{\Omega} g - \epsilon < \int_{\Omega} s_f + \int_{\Omega} s_g = \int_{\Omega} s_f + s_g \leq \int_{\Omega} (f + g) \\
  \implies                                                                                                                                  \\
  \int_{\Omega} f + \int_{\Omega} g \leq \int_{\Omega} (f + g)
\end{align*}

\section*{19.2.3}

考虑序列$(F_N)_{N = 1}^\infty$,其中
\begin{align*}
  F_N : = \sum \limits_{n = 1}^N g_n
\end{align*}
因为$g_n$都是非负可测函数,于是我们有
\begin{align*}
  0 \leq F_1 \leq F_2 \leq \cdots
\end{align*}
且有推论18.5.7可知,$F_N$都是非负可测函数。
于是利用定理19.2.9,我们有
\begin{align*}
  \int_{\Omega} \sup\limits_{N} F_N = \sup\limits_{N} \int_{\Omega} F_N
\end{align*}

因为$(F_N)_{N = 1}^\infty$是单调的递增序列,
那么它逐点收敛于可测函数$f$(可测性由引理18.5.10保证),允许$f(x) = +\infty$。
于是有
\begin{align*}
  \sum \limits_{n = 1}^\infty g_n = f = \sup\limits_{N} F_N
\end{align*}
(第二个等式的证明在19-2-comment.tex中有)\\
于是
\begin{align*}
  \int_{\Omega} \sup\limits_{N} F_N  = \int_{\Omega} \sum \limits_{n = 1}^\infty g_n
\end{align*}

同理可得(考虑$\int_{\Omega} F_N = \sum \limits_{n = 1}^N \int_{\Omega} g_n $),
\begin{align*}
  \sup\limits_{N} \int_{\Omega} F_N = \sum \limits_{n = 1}^\infty \int_{\Omega} g_n
\end{align*}

综上,我们有
\begin{align*}
  \int_{\Omega} \sum \limits_{n = 1}^\infty g_n = \sum \limits_{n = 1}^\infty \int_{\Omega} g_n
\end{align*}

\section*{19.2.4}

这道题有些反直觉:不等式的左侧不等于$0$,希望我能描述清楚。

右侧:\\
对每一个$n = 1, 2, 3, \cdots$,
$f_n$都是简单函数,我们有
\begin{align*}
  \int_{\mathbb{R}} f_n = 0
\end{align*}
于是
\begin{align*}
  \sum \limits_{n = 1}^\infty \int_{\mathbb{R}} f_n = \sum \limits_{n = 1}^\infty 0 = 0
\end{align*}

左侧:\\
考虑序列$(F_N)_{N = 1}^\infty$,其中
\begin{align*}
  F_N : = \sum \limits_{n = 1}^N f_n
\end{align*}

构造函数$f: \mathbb{R} \to \mathbb{R}$,
\begin{equation*}
  f(x) = \begin{cases}
    1 & \;\; x \in [1, 2)    \\
    0 & \;\; x \notin [1, 2)
  \end{cases}
\end{equation*}

接下来证明$(F_N)_{N = 1}^\infty$逐点收敛于函数$f$。

\begin{itemize}
  \item $x < 1$。

        任意$N \geq 1$都有
        \begin{align*}
          F_N(x) = 0 = f(x)
        \end{align*}

  \item $x \in [1, 2)$。

        任意$N \geq 1$都有
        \begin{align*}
          f_1(x) = 1 - 0 = 1
        \end{align*}
        其他$f_n(x) = 0$,于是可得
        \begin{align*}
          F_N(x) = 1 = f(x)
        \end{align*}

  \item $x > 2$时。

        令$m = \left\lfloor x \right\rfloor$
        于是存在$N \geq m + 1$,就有
        \begin{align*}
          f_m(x) = 0 - 1 = -1 \\
          f_{m + 1}(x) = 1 - 0 = 1
        \end{align*}
        其他$f_n(x) = 0$,
        于是可得
        \begin{align*}
          F_N(x) = 0 = f(x)
        \end{align*}
\end{itemize}

综上可得,$(F_N)_{N = 1}^\infty$逐点收敛于函数$f$,
于是我们有
\begin{align*}
  \int_{\mathbb{R}} \sum\limits_{n=1}^\infty f_n = \int_{\mathbb{R}} f = 1
\end{align*}

所以$\int_{\mathbb{R}} \sum\limits_{n=1}^\infty f_n \neq \sum \limits_{n = 1}^\infty \int_{\mathbb{R}} f_n$

\section*{19.2.5}

考虑集合
\begin{align*}
  A := \{x \in \Omega: f(x) = +\infty\}
\end{align*}
证明$m(A) = 0$。

反证法,假设$m(A) \neq 0$,即$m(A) > 0$。

构造简单函数序列$(t_n)_{n = 1}^\infty$,其中
\begin{equation*}
  t_n(x) =
  \begin{cases*}
    0 & $x \notin A$ \\
    n & $x \in A$
  \end{cases*}
\end{equation*}
对每一个$t_n$都是简单函数,并在下方控制$f$。

由引理19.1.9可知,对任意$n$,我们有
\begin{align*}
  \int_{\Omega} t_n = n \times m(A)
\end{align*}
于是
\begin{align*}
  \lim\limits_{n \to \infty} n \times m(A) = +\infty
\end{align*}
又因为,对任意$n$,我们有
\begin{align*}
  \int_{\Omega} f \geq \int_{\Omega} t_n
\end{align*}
所以,我们有
\begin{align*}
  \int_{\Omega} f \geq \lim\limits_{n \to \infty} \int_{\Omega} t_n = +\infty
\end{align*}
这与题设$\int_{\Omega} f$是有限的相互矛盾,假设不成立,命题得证。

\section*{19.2.6}
先设$\Omega = \bigcup \limits_{n = 1}^\infty \Omega_n$,
由题设可知,$\Omega$是可测的。

构造函数序列$(g_n)_{n = 1}^\infty$,其中
\begin{align*}
  g_n = \chi_{\Omega_n}
\end{align*}
于是我们有
\begin{align*}
  \int_{\Omega} g_n = m(\Omega_n)
\end{align*}
利用推论19.2.11可知,
\begin{align*}
  \int_{\Omega} \sum \limits_{n = 1}^\infty g_n
  = \sum \limits_{n = 1}^\infty \int_{\Omega} g_n
  = \sum \limits_{n = 1}^\infty m(\Omega_n)
\end{align*}
以为$\sum \limits_{n = 1}^\infty m(\Omega_n)$是有限的,
那么由引理19.2.14可知$\sum \limits_{n = 1}^\infty g_n$是几乎处处有限。
即集合
\begin{align*}
  A := \{x \in \Omega: \left(\sum \limits_{n = 1}^\infty g_n\right)(x) = +\infty\}
\end{align*}
的测度为零($m(A) = 0$)。

接下来证明集合$B := \{x \in \Omega: \text{存在无限个$n$使得$x \in \Omega_n$}\}$的测得为零(即$m(B) \neq 0$)。
反证法,假设集合$B$的测度不为零,即$m(B) > 0$。
为了完成证明,我们只需证明
\begin{align*}
  B \subseteq A
\end{align*}
对任意$x \in B$,那么集合$C := \{m \in \mathbb{N}^+ : x \in \Omega_m\}$是无限集,
又因为对任意$m \in C$,我们有
\begin{align*}
  g_m(x) = 1
\end{align*}
进而
\begin{align*}
  \sum\limits_{m \in C} g_m(x) = +\infty
\end{align*}
于是可得
\begin{align*}
  \left(\sum \limits_{n = 1}^\infty g_n\right)(x) \geq \sum\limits_{m \in C} g_m(x) = +\infty
\end{align*}
所以$x \in A$,从而$B \subseteq A$,
于是$m(B) \leq m(A)$,与$m(A) = 0, m(B) > 0$矛盾。
假设不成立,命题得证。

\section*{19.2.7 $\circledast$}
设
\begin{align*}
  A := \{x \in [0, 1]: \text{存在无限多个正整数$a$和$q$使得$|x \ a/q| \leq c/q^p$}\}
\end{align*}

设$\Omega_1, \Omega_2, \cdots$是$[0, 1]$的一列可测子集,
并且对任意$q \in \mathbb{N}^+$,
\begin{align*}
  \Omega_q = \bigcup \limits_{a = 1}^q [\frac{a}{q} - \frac{c}{q^p}, \frac{a}{q} + \frac{c}{q^p}] \cap [0, 1]
\end{align*}
这里我们只需考虑$1 \leq a \leq q$的情况(放在本节最后,不放在主体中,对理解整体思路有帮助。)
(为什么这样构造,这是因为它有一个关键性质:如果$x \in \Omega_q$,能够满足$|x - \frac{a}{q}| \leq \frac{c}{q^p}$)。

对任意$\Omega_q$,我们有
\begin{align*}
  m(\Omega_q) & \leq ((\frac{a}{q} + \frac{c}{q^p}) - (\frac{a}{q} - \frac{c}{q^p})) \times q \\
              & = \frac{2cq}{q^p}                                                             \\
              & = \frac{2c}{q^{p - 1}}
\end{align*}
由推论7.5.3(比值判别法)易得
\begin{align*}
  \sum\limits_{q = 1}^\infty m(\Omega_q)
\end{align*}
有限收敛。

由Borel-Cantelli引理可知,集合
\begin{align*}
  B := \{x \in [0, 1]: \text{存在无限多个$q$使得$x \in \Omega_q$}\}
\end{align*}
的测度为零(即$m(B) = 0$)。

到这一步,证明还未完成,需要证明$A \subseteq B$。\\
对任意$x \in A$,那么存在无限多个正整数$a$和$q$使得$|x - a/q| \leq c/q^p$,
对每一个具体的$a, q$(其中$1 \leq a \leq q$),有
$x \in [\frac{a}{q} - \frac{c}{q^p}, \frac{a}{q} + \frac{c}{q^p}]$,
所以,$x \in \Omega_q$,进而可得$x \in B$,于是我们有
\begin{align*}
  A \subseteq B
\end{align*}
我们有
\begin{align*}
  0 \leq m(A) \leq m(B) = 0 \\
  \implies                  \\
  m(A) = 0
\end{align*}
命题得证。

\section*{19.2.8}

% 考虑集合
% \begin{align*}
%   A : = \{x \in \mathbb{R}: x \text{不是丢番图数}\}
% \end{align*}
% 我们只需证明$m(A)= 0$。

% A等价于以下集合

todo (没思路)

\section*{19.2.9}

任意$\epsilon > 0$,对所有的$n = 1, 2, 3 \cdots$,考虑集合
\begin{align*}
  A_n = \{x \in \mathbb{R}: f_n(x) > \frac{1}{\epsilon 2^n}\}
\end{align*}
先证明: $m(A_n) \leq \frac{\epsilon}{2^n}$。
反证法,假设$m(A_n) > \frac{\epsilon}{2^n}$。
我们构造一个简单函数$s$,它从下方控制$f_n$:
\begin{equation*}
  s(x) = \begin{cases}
    0                      & x \notin A_n \\
    \frac{1}{\epsilon 2^n} & x \in A_n
  \end{cases}
\end{equation*}
于是,我们有
\begin{align*}
  \int_{\mathbb{R}} f_n \geq \int_{\mathbb{R}} s > \frac{1}{\epsilon 2^n} \frac{\epsilon}{2^n} = \frac{1}{4^n}
\end{align*}
这与题设$\int_{\mathbb{R}} f_n \leq \frac{1}{4^n}$矛盾,
假设不成立,$m(A_n) \leq \frac{\epsilon}{2^n}$得证。

令
\begin{align*}
  E := \bigcup\limits_{n = 1}^\infty A_n
\end{align*}
利用引理7.3.3,我们有
\begin{align*}
  m(E) \leq \sum\limits_{n = 1}^\infty m(A_n) \leq \sum\limits_{n = 1}^\infty \frac{\epsilon}{2^n} = \epsilon
\end{align*}

对任意$x \in \mathbb{R} \setminus E$,
那么,对任意$n$,因为$x \notin A_n$,我们有
\begin{align*}
  0 \leq f_n(x) \leq \frac{1}{\epsilon 2^n}
\end{align*}
于是
\begin{align*}
  0 \leq \lim\limits_{n \to \infty} f_n(x) \leq \lim\limits_{n \to \infty} \frac{1}{\epsilon 2^n} = 0
\end{align*}
由夹逼定理可得,$\lim\limits_{n \to \infty} f_n(x) = 0$。

\section*{19.2.10 $\circledast$}

 (1)

按照书中提示,先证明对于任意的正整数$m$,我们能够找到一个$N > 0$使得
$m(\{x \in [0, 1]: f_n(x) > \frac{1}{m}\}) \leq \epsilon/2^m$
对所有的$n \geq N$都成立。

对每一个$m \in \mathbb{N}^+$,
我们都构造$F_n^{(m)} := \bigcup\limits_{i = n}^\infty E_i^{(m)}$,其中
$E_n^{(m)} := \{x \in [0, 1]: f_n(x) > \frac{1}{m}\}$。
可见,$F_1^{(m)} \supseteq F_2^{(m)} \supseteq F_3^{(m)} \supseteq \cdots$,
而且$F_1^{(m)} \subseteq [0, 1]$。

注意,上面对每个$m$都都构造了一套$F_n^{(m)}$集合序列。

对任意每一个$m$,由习题18.2.3(b),我们有
\begin{align*}
  m(\bigcap\limits_{n = 1}^\infty F_n^{(m)}) = \lim\limits_{n \to \infty} m(F_n^{(m)})
\end{align*}
接下来证明
\begin{align*}
  \bigcap\limits_{n = 1}^\infty F_n^{(m)} = \varnothing
\end{align*}
反证法,假设$\bigcap\limits_{n = 1}^\infty F_n^{(m)} \neq \varnothing$,
那么存在$x \in \bigcap\limits_{n = 1}^\infty F_n^{(m)}$,
即对每一个$n$都有$x \in F_n^{(m)}$,从而$x \in \bigcup\limits_{i = n}^\infty E_i^{(m)}$,
我们可得,对任意$n$,都有$i \geq n$使得
\begin{align*}
  x \in E_i^{(m)} \\
  \implies        \\
  f_i(x) > \frac{1}{m}
\end{align*}
这与函数序列$f_n$逐点收敛于零矛盾。

因为对任意的$m$都有$\bigcap\limits_{n = 1}^\infty F_n^{(m)} = \varnothing$,于是我们有
\begin{align*}
  m(\bigcap\limits_{n = 1}^\infty F_n^{(m)}) = \lim\limits_{n \to \infty} m(F_n^{(m)}) = 0
\end{align*}
于是对任意$\epsilon > 0$和任意正整数$m$,存在$N_m$使得只要$n \geq N_m$,就有
\begin{align*}
  m(F_n^{(m)}) \leq \epsilon/2^m
\end{align*}
构造集合$E$如下:
\begin{align*}
  E : = \bigcup\limits_{m = 1}^\infty F_{N_m}^{(m)}
\end{align*}
(形象化的理解,这一步就像剥洋葱,剥掉不要的。)\\
于是,我们有
\begin{align*}
  m(E) \leq \sum\limits_{m=1}^\infty m(F_{N_m}^{(m)}) \leq \sum\limits_{m = 1}^\infty \epsilon/2^m = \epsilon
\end{align*}

接下来证明:$f_n(x)$在$[0, 1] \setminus E$上一致收敛于$0$。

对任意$x \in [0, 1] \setminus E, \epsilon^\prime > 0$。
存在正整数$M$使得$\frac{1}{M} \leq \epsilon^\prime$。
$x \in [0, 1] \setminus E$可得$x \notin E$,
从而$x \notin F_{N_M}^{(M)}$,于是当$n \geq N_M$,我们有
\begin{align*}
  f_n(x) \leq 1/M \leq \epsilon^\prime
\end{align*}
命题得证。

(2)换成$\mathbb{R}$,命题成立么?

不成立,举一个反例
\begin{equation*}
  f_n(x) = \begin{cases}
    1 & x \in [n, n+1]    \\
    0 & x \notin [n, n+1]
  \end{cases}
\end{equation*}
对每一个$n$,$f_n$都会在$[n, n+1]$上等于1,其他地方都等于$0$,
所以函数序列$f_n$逐点收敛于$0$。

但由于这个等于1的区间$[n, n+1]$,随着$n \to \infty$一直向无穷处移动,
我们就无法通过抠出一个有限的集合$E$,来拦截这种行为。

\section*{19.2.11}
定义函数$f: \mathbb{N} \times \mathbb{N} \to \mathbb{R}+$如下:
\begin{equation*}
  f(n, m) = \begin{cases}
    1 & n = m    \\
    0 & n \neq m
  \end{cases}
\end{equation*}

\begin{itemize}
  \item (1)证明对每一个$n$,$\sum\limits_{m=1}^\infty  f(n, m)$都是收敛的。

        如果$n \geq 1$,于是存在$n = m$使得$f(n, m) = 1$,其他$f(n, m) = 0$,
        于是可得
        \begin{align*}
          \sum\limits_{m=1}^\infty  f(n, m) = 1
        \end{align*}

        如果$n < 1$,则
        \begin{align*}
          \sum\limits_{m=1}^\infty  f(n, m) = 0
        \end{align*}
        命题得证。

  \item (2) 对于每一个$m$,$\lim\limits_{n \to \infty} f(n, m)$。

        存在$N > m$使得只要$n  \geq N$,就有
        \begin{align*}
          f(n, m) = 0
        \end{align*}
        于是可得
        \begin{align*}
          \lim\limits_{n \to \infty} f(n, m) = 0
        \end{align*}

  \item (3)$f$满足下面这个不等式:
        \begin{align*}
          \lim\limits_{n \to \infty} \sum\limits_{m = 1}^{\infty} f(n,m)
          \neq \sum\limits_{m = 1}^{\infty} \lim\limits_{n \to \infty} f(n, m)
        \end{align*}

        由(1)可知
        \begin{align*}
          \lim\limits_{n \to \infty} \sum\limits_{m = 1}^{\infty} f(n,m) = 1 \\
          \sum\limits_{m = 1}^{\infty} \lim\limits_{n \to \infty} f(n, m) = 0
         \end{align*}
         综上,不等式成立。
\end{itemize}

\textbf{补充部分:}

\begin{zremark}
  习题19.2.7中:“只需考虑$1 \leq a \leq q$的情况”合理性证明。
\end{zremark}

\textbf{证明:}

$a, q \in \mathbb{N}^+$,需要满足题设:
\begin{align*}
  |x - \frac{a}{q}| \leq \frac{c}{q^p} \\
  x - \frac{c}{q^p} \leq \frac{a}{q} \leq x + \frac{c}{q^p}
\end{align*}
即$\frac{a}{q} \in [x - \frac{c}{q^p}, x + \frac{c}{q^p}]$可满足题设。
又因为$x \in [0, 1]$,于是$\frac{a}{q} \in [- \frac{c}{q^p}, 1 + \frac{c}{q^p}]$,
乘以$q$可得$a \in [-cq^{1 - p}, q + cq^{1 - p}]$。
题设要求$a \in \mathbb{N}^+$,从而$a \in [1, q + cq^{1 - p}]$。

由于$p > 2$,于是我们有
\begin{align*}
  \lim\limits_{q \to \infty} cq^{1 - p} = 0
\end{align*}
因此在$q$足够大时,$a \in [1, q + 1)$,即$a \in [1, q]$。
(注意区间可以表示成$[1, q + cq^{1 - p}]_{q = 1}^\infty$,
区间是单调递增的,之前$q$不够大时的区间,最后会被覆盖。)

\end{document}