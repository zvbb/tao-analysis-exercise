\documentclass{article}
\usepackage{mathtools} 
\usepackage{fontspec}
\usepackage[UTF8]{ctex}
\usepackage{amsthm}
\usepackage{mdframed}
\usepackage{xcolor}
\usepackage{amssymb}
\usepackage{amsmath}


% 定义新的带灰色背景的说明环境 zremark
\newmdtheoremenv[
  backgroundcolor=gray!10,
  % 边框与背景一致,边框线会消失
  linecolor=gray!10
]{zremark}{说明}

% 通用矩阵命令: \flexmatrix{矩阵名}{元素符号}{行数}{列数}
\newcommand{\flexmatrix}[4]{
  \[
  #1 = \begin{pmatrix}
    #2_{11}     & #2_{12}     & \cdots & #2_{1#4}   \\
    #2_{21}     & #2_{22}     & \cdots & #2_{2#4}   \\
    \vdots      & \vdots      & \ddots & \vdots     \\
    #2_{#31}    & #2_{#32}    & \cdots & #2_{#3#4}
  \end{pmatrix}
  \]
}

% 简化版命令(默认矩阵名为A,元素符号为a): \quickmatrix{行数}{列数}
\newcommand{\quickmatrix}[2]{\flexmatrix{A}{a}{#1}{#2}}


\begin{document}
\title{19.3 习题}
\author{张志聪}
\maketitle

\section*{19.3.1}

我们有
\begin{align*}
  \left| \int_{\Omega} f \right|
  = \left|\int_{\Omega} f^+ - \int_{\Omega} f^-\right|
\end{align*}
因为$\int_{\Omega} f^+, \int_{\Omega} f^-$都是非负的有限实数,
运用实数的三角不等式, 我们有
\begin{align*}
  \left|\int_{\Omega} f^+ - \int_{\Omega} f^-\right| \leq \int_{\Omega} f^+ + \int_{\Omega} f^-
\end{align*}
综上可得
\begin{align*}
  \left| \int_{\Omega} f \right| \leq \int_{\Omega} f^+ + \int_{\Omega} f^-
\end{align*}

利用引理19.2.10,我们有
\begin{align*}
  \int_{\Omega} |f|
   & = \int_{\Omega} f^+ + f^-               \\
   & = \int_{\Omega} f^+ + \int_{\Omega} f^-
\end{align*}

所以,综上
\begin{align*}
  \left| \int_{\Omega} f \right| \leq \int_{\Omega} f^+ + \int_{\Omega} f^- = \int_{\Omega} |f|
\end{align*}

\section*{19.3.2}

\begin{itemize}
  \item (a)

        $f$是绝对可积函数,即$\int_{\Omega} |f| < \infty$是有限的,
        于是由命题19.2.6可知,
        \begin{align*}
          \int_{\Omega} |cf|
           & = \int_{\Omega} |c||f|          \\
           & = |c|\int_{\Omega} |f| < \infty
        \end{align*}
        所以,$cf$也是绝对可积函数。

        \begin{itemize}
          \item $c = 0$

                于是$cf = 0$,易得
                \begin{align*}
                  \int_{\Omega} cf = 0 = c \int_{\Omega} f
                \end{align*}

          \item $c > 0$

                于是由定义19.3.2和命题19.2.6(b)可知,
                \begin{align*}
                  \int_{\Omega} cf
                   & = \int_{\Omega} (cf)^+ - \int_{\Omega} (cf)^- \\
                   & = \int_{\Omega} cf^+ - \int_{\Omega} cf^-     \\
                   & = c \int_{\Omega} f^+ - c \int_{\Omega} f^-   \\
                   & = c \int_{\Omega} f
                \end{align*}

          \item $c < 0$

                此时,我们有
                \begin{align*}
                  (cf)^+ = |c|f^- \\
                  (cf)^- = |c|f^+
                \end{align*}

                于是由定义19.3.2和命题19.2.6(b)可知,
                \begin{align*}
                  \int_{\Omega} cf
                   & = \int_{\Omega} (cf)^+ - \int_{\Omega} (cf)^- \\
                   & = \int_{\Omega} |c|f^- - \int_{\Omega} |c|f^+ \\
                   & = |c|\int_{\Omega} f^- - |c|\int_{\Omega} f^+ \\
                   & = |c|(\int_{\Omega} f^- - \int_{\Omega} f^+)  \\
                   & = |c| (- \int_{\Omega} f)                     \\
                   & = c \int_{\Omega} f
                \end{align*}
        \end{itemize}

        综上可得
        \begin{align*}
          \int_{\Omega} c f = c \int_{\Omega} f
        \end{align*}

  \item (b)

        我们有
        \begin{align*}
          \int_{\Omega} |f| < \infty \\
          \int_{\Omega} |g| < \infty
        \end{align*}

        对任意$x \in \Omega$,我们有
        \begin{align*}
          |(f + g)(x)| = |f(x) + g(x)| \leq |f(x)| + |g(x)|
        \end{align*}
        即
        \begin{align*}
          |f + g| \leq |f| + |g|
        \end{align*}
        于是利用19.2.6(c)可得
        \begin{align*}
          \int_{\Omega} |f + g| \leq \int_{\Omega} |f| + \int_{\Omega} |g| < \infty
        \end{align*}
        所以,$f + g$是绝对可积函数。

        (2)

        我们有
        \begin{align}
          f + g & = (f + g)^+ - (f + g)^-     \\
          f + g & = (f^+ - f^-) + (g^+ - g^-)
        \end{align}
        由(1)(2)可得
        \begin{align*}
          (f + g)^+ + f^- + g^- = (f+g)^- + f^+ + g^+
        \end{align*}

        因为等式两边都是非负可测函数,利用引理19.2.10可得
        \begin{align*}
          \int_{\Omega} (f + g)^+ + f^- + g^-                             & = \int_{\Omega} (f+g)^- + f^+ + g^+                                             \\
          \int_{\Omega} (f + g)^+ + \int_{\Omega} f^- + \int_{\Omega} g^- & = \int_{\Omega} (f+g)^- + \int_{\Omega} f^+ + \int_{\Omega} g^+                 \\
          \int_{\Omega} (f + g)^+ - \int_{\Omega} (f+g)^-                 & = \int_{\Omega} f^+ - \int_{\Omega} f^- + \int_{\Omega} g^+ - \int_{\Omega} g^- \\
          \int_{\Omega} (f + g)                                           & = \int_{\Omega} f + \int_{\Omega} g
        \end{align*}

  \item (b-减法)

        扩展(b):函数$f - g$是绝对可积的,并且$\int_{\Omega} (f - g) = \int_{\Omega} f - \int_{\Omega} g$。
        (不假设$g,f$的大小关系)。

        对任意$x \in \Omega$,我们有
        \begin{align*}
          |f(x) - g(x)| \leq |f(x)| + |g(x)|
        \end{align*}
        因为$f, g$都是可测函数,由推论18.5.7可知
        $|f(x) - g(x)|$是可测函数。

        由命题19.2.6(c)和(b)可得
        \begin{align*}
          \int_{\Omega} |f(x) - g(x)|
          \leq \int_{\Omega} |f(x)| + |g(x)|
          = \int_{\Omega} |f(x)| + \int_{\Omega} |g(x)|
          < \infty
        \end{align*}
        所以,$f - g$是绝对可积函数。

        我们有
        \begin{align}
          f - g & = (f - g)^+ - (f - g)^-     \\
          f - g & = (f^+ - f^-) - (g^+ - g^-)
        \end{align}
        由(3)(4)可得
        \begin{align*}
          (f - g)^+ + f^- + g^+= (f - g)^- + f^+ + g^-
        \end{align*}

        因为等式两边都是非负可测函数,利用引理19.2.10可得
        \begin{align*}
          \int_{\Omega} (f - g)^+ + f^- + g^+ = \int_{\Omega} (f - g)^- + f^+ + g^- \\
          \int_{\Omega} (f - g)^+ + \int_{\Omega} f^- + \int_{\Omega} g^+
          = \int_{\Omega} (f - g)^- + \int_{\Omega} f^+ + \int_{\Omega} g^-         \\
          \implies                                                                  \\
          \int_{\Omega} (f - g) = \int_{\Omega} f - \int_{\Omega} g
        \end{align*}

  \item (c)

        因为$f(x) \leq g(x)$,于是可得
        \begin{align*}
          f^+(x) \geq g^+(x) \\
          f^-(x) \leq g^-(x)
        \end{align*}
        所以
        \begin{align*}
          \int_{\Omega} f^+ - \int_{\Omega} f^- \leq \int_{\Omega} g^+ - \int_{\Omega} g^- \\
          \implies                                                                         \\                                                                        \\
          \int_{\Omega} f \leq \int_{\Omega} g
        \end{align*}

  \item (d)

        由命题19.2.6(d)可知,
        \begin{align*}
          \int_{\Omega} f^+ = \int_{\Omega} g^+ \\
          \int_{\Omega} f^- = \int_{\Omega} g^-
        \end{align*}
        所以
        \begin{align*}
          \int_{\Omega} f^+  + \int_{\Omega} f^- = \int_{\Omega} g^+ + \int_{\Omega} g^- \\
          \implies                                                                       \\
          \int_{\Omega} f = \int_{\Omega} g
        \end{align*}

\end{itemize}

\section*{19.3.3}

注意:不能直接利用命题19.3.3(b)。

\begin{itemize}
  \item 方法一:使用习题19.3.2(b-减法)

        \begin{align*}
          \int_{\mathbb{R}} g - f = \int_{\mathbb{R}} g - \int_{\mathbb{R}} f = 0
        \end{align*}
        由命题19.2.6(a)可得,$g - f$是几乎处处为零,
        命题得证。

  \item 方法二:直接证明

        考虑集合
        \begin{align*}
          A : = \{ x \in \mathbb{R} : f(x) \neq g(x) \}
        \end{align*}
        我们需要证明$m(A) = 0$。

        反证法,假设$m(A) \neq 0$。

        令$h: \mathbb{R} \to \mathbb{R}$为
        \begin{align*}
          h(x) = g(x) - f(x)
        \end{align*}
        因为$g(x) \geq f(x)$且$g, f$都是绝对可积的可测函数,
        所以$h$是可测的(根据引理18.5.10)且是非负的。而且还是绝对可积的。

        因为$m(A) \neq 0$,所以$h$不是几乎处处为零,利用命题19.2.6(a)可得
        \begin{align*}
          \int_{\mathbb{R}} h > 0
        \end{align*}
        我们有
        \begin{align*}
          g(x) = f(x) + h(x)
        \end{align*}
        由命题19.3.3(b)可得
        \begin{align*}
          \int_{\mathbb{R}} g = \int_{\mathbb{R}} f + \int_{\mathbb{R}} h \\
          \implies                                                        \\
          \int_{\mathbb{R}} g > \int_{\mathbb{R}} f
        \end{align*}
        这与题设$\int_{\mathbb{R}} g = \int_{\mathbb{R}} f$矛盾。
        假设不成立,命题得证。

\end{itemize}


\end{document}