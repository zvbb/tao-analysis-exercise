\documentclass{article}
\usepackage{mathtools} 
\usepackage{fontspec}
\usepackage[UTF8]{ctex}
\usepackage{amsthm}
\usepackage{mdframed}
\usepackage{xcolor}
\usepackage{amssymb}
\usepackage{amsmath}


% 定义新的带灰色背景的说明环境 zremark
\newmdtheoremenv[
  backgroundcolor=gray!10,
  % 边框与背景一致,边框线会消失
  linecolor=gray!10
]{zremark}{说明}

% 通用矩阵命令: \flexmatrix{矩阵名}{元素符号}{行数}{列数}
\newcommand{\flexmatrix}[4]{
  \[
  #1 = \begin{pmatrix}
    #2_{11}     & #2_{12}     & \cdots & #2_{1#4}   \\
    #2_{21}     & #2_{22}     & \cdots & #2_{2#4}   \\
    \vdots      & \vdots      & \ddots & \vdots     \\
    #2_{#31}    & #2_{#32}    & \cdots & #2_{#3#4}
  \end{pmatrix}
  \]
}

% 简化版命令(默认矩阵名为A,元素符号为a): \quickmatrix{行数}{列数}
\newcommand{\quickmatrix}[2]{\flexmatrix{A}{a}{#1}{#2}}


\begin{document}
\title{19.2 注释}
\author{张志聪}
\maketitle

\begin{zremark}
  定理19.2.9的证明中:
  \textbf{“
    $\sup\limits_{n} m(F_j \cap E_n) = m(F_j)$可以利用习题18.2.3(a)得到。
    ”}
  的具体证明过程。
\end{zremark}

\textbf{证明:}

对每一个$n$都有
\begin{align*}
  F_j \cap E_1 \subseteq F_j
\end{align*}
而且我们有
\begin{align*}
  E_1 \subset E_2  \subset \cdots
\end{align*}
于是可得
\begin{align*}
  F_j \cap E_1 \subseteq F_j \cap E_2 \subseteq \cdots
\end{align*}
所以,$(m(F_j \cap E_n))_{n = 1}^\infty$是单调的递增序列,
于是我们有
\begin{align*}
  \sup\limits_{n} m(F_j \cap E_n) = \lim\limits_{n \to \infty} m(F_j \cap E_n)
\end{align*}
由习题18.2.3(a)可知
\begin{align*}
  \lim\limits_{n \to \infty} m(F_j \cap E_n) = m(\bigcup\limits_{n = 1}^\infty F_j \cap E_n)
\end{align*}
接下来证明:
\begin{align*}
  m(\bigcup\limits_{n = 1}^\infty F_j \cap E_n) = m(F_j)
\end{align*}
为了完成证明,我们只需证明
\begin{align*}
  \bigcup\limits_{n = 1}^\infty F_j \cap E_n = F_j
\end{align*}
对任意$n$都有
\begin{align*}
  F_j \cap E_n \subseteq F_j \\
  \implies                   \\
  \bigcup\limits_{n = 1}^\infty F_j \cap E_n \subseteq F_j
\end{align*}

对任意$x \in F_j$,因为$F_j \subseteq \Omega$,
又因为$\bigcup\limits_{n = 1}^\infty E_n = \Omega$,
所以存在$n$使得$x \in E_n$,于是$x \in F_j \cap E_n \subseteq \bigcup\limits_{n = 1}^\infty F_j \cap E_n$,所以
\begin{align*}
  F_j \subseteq \bigcup\limits_{n = 1}^\infty F_j \cap E_n
\end{align*}
综上可得
\begin{align*}
  F_j = \bigcup\limits_{n = 1}^\infty F_j \cap E_n
\end{align*}

\begin{zremark}
  引理19.2.10中:
  \textbf{“
    简单函数序列$0 \leq s_1 \leq s_2 \leq \cdots \leq f$使得$\sup\limits_{n} s_n = f$。
    ”}
  的证明。
\end{zremark}

\textbf{证明:}

文中的说明存在歧义,应该是:
简单函数序列$0 \leq s_1 \leq s_2 \leq \cdots $逐点收敛于$f$,使得$\sup\limits_{n} s_n = f$。

先解释下$\sup\limits_{n} s_n$的定义:
\begin{align*}
  (\sup\limits_{n} s_n)(x) := \sup\limits_{n \in \mathbb{N}} s_n(x) \,\, \text{对每个$x \in \Omega$}
\end{align*}
即
\begin{itemize}
  \item $\sup\limits_{n} s_n$是一个函数;
  \item 它在每个点$x$的取值是实数序列$(s_n(x))_{n = 1}^\infty$的上确界(注意不是极限点。
  因为实数序列只要有界,就有上确界,但序列本身不一定收敛)。
\end{itemize}

对任意$x \in \Omega$,
题设可知$(s_n(x))_{n = 1}^\infty$的单调递增的,
所以$(s_n(x))_{n = 1}^\infty$收敛于上确界$(\sup\limits_{n} s_n)(x)$。

如果$(\sup\limits_{n} s_n)(x) = +\infty$,由
$(s_n)_{n = 1}^\infty$逐点收敛于$f$可知,$f(x) = +\infty$,
我们有
\begin{align*}
  (\sup\limits_{n} s_n)(x) = f(x) = +\infty
\end{align*}

如果$(\sup\limits_{n} s_n)(x)$是实数,
那么对任意$\epsilon > 0$,存在$N_0$,使得只要$n \geq N_0$,就有 
\begin{align}
  |(\sup\limits_{n} s_n)(x) - s_n(x)| < \frac{1}{2}\epsilon
\end{align}

$(s_n)_{n = 1}^\infty$逐点收敛于$f$,那么存在$N_1$,使得只要$n \geq N_1$,就有 
\begin{align}
  |f(x) - s_n(x)| < \frac{1}{2}\epsilon
\end{align}

综上,$n \geq \max(N_0, N_1)$,式子(1)(2)同时成立。

由三角不等式可知
\begin{align*}
  |(\sup\limits_{n} s_n)(x) - f(x)| < \epsilon
\end{align*}
由$\epsilon$的任意性可知,$(\sup\limits_{n} s_n)(x) = f(x)$,
由$x$的任意性可知,$\sup\limits_{n} s_n = f$。

\end{document}