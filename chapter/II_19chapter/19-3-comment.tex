\documentclass{article}
\usepackage{mathtools} 
\usepackage{fontspec}
\usepackage[UTF8]{ctex}
\usepackage{amsthm}
\usepackage{mdframed}
\usepackage{xcolor}
\usepackage{amssymb}
\usepackage{amsmath}


% 定义新的带灰色背景的说明环境 zremark
\newmdtheoremenv[
  backgroundcolor=gray!10,
  % 边框与背景一致,边框线会消失
  linecolor=gray!10
]{zremark}{说明}

% 通用矩阵命令: \flexmatrix{矩阵名}{元素符号}{行数}{列数}
\newcommand{\flexmatrix}[4]{
  \[
  #1 = \begin{pmatrix}
    #2_{11}     & #2_{12}     & \cdots & #2_{1#4}   \\
    #2_{21}     & #2_{22}     & \cdots & #2_{2#4}   \\
    \vdots      & \vdots      & \ddots & \vdots     \\
    #2_{#31}    & #2_{#32}    & \cdots & #2_{#3#4}
  \end{pmatrix}
  \]
}

% 简化版命令(默认矩阵名为A,元素符号为a): \quickmatrix{行数}{列数}
\newcommand{\quickmatrix}[2]{\flexmatrix{A}{a}{#1}{#2}}


\begin{document}
\title{19.3 注释}
\author{张志聪}
\maketitle

\begin{zremark}
  定理19.3.4中:如何从
  \begin{align*}
    \int_{\Omega} F + f \leq \liminf\limits_{n \to \infty} \int_{\Omega} F + f_n
  \end{align*}
  从而有
  \begin{align*}
    \int_{\Omega} f \leq \liminf\limits_{n \to \infty} \int_{\Omega} f_n
  \end{align*}
\end{zremark}

\textbf{证明:}

左侧,由命题19.3.3(b)可知
\begin{align*}
  \int_{\Omega} F + f = \int_{\Omega} F + \int_{\Omega} f
\end{align*}

右侧,我们有
\begin{align*}
  \liminf\limits_{n \to \infty} \int_{\Omega} F + f_n
   & = \liminf\limits_{n \to \infty} \left(\int_{\Omega} F + \int_{\Omega} f_n\right)
\end{align*}
因为$\int_{\Omega} F$是定值,不妨设$c = \int_{\Omega} F$。

所以
\begin{align*}
  \liminf\limits_{n \to \infty} \left(\int_{\Omega} F + \int_{\Omega} f_n\right)
   & = \liminf\limits_{n \to \infty} \left(c + \int_{\Omega} f_n\right)                       \\
   & = \sup\limits_{n} \inf\limits_{m \geq n} \left(c + \int_{\Omega} f_m\right)              \\
   & = \sup\limits_{n} \left(c + \inf\limits_{m \geq n} \left(\int_{\Omega} f_m\right)\right) \\
   & = c + \sup\limits_{n} \left(\inf\limits_{m \geq n} \left(\int_{\Omega} f_m\right)\right) \\
   & = \int_{\Omega} F + \liminf\limits_{n \to \infty} \int_{\Omega} f_n
\end{align*}

综上可得
\begin{align*}
  \int_{\Omega} f \leq \liminf\limits_{n \to \infty} \int_{\Omega} f_n
\end{align*}

\begin{zremark}
  定理19.3.4中:
  为什么
  \begin{align*}
    \liminf\limits_{n \to \infty} \int_{\Omega} F - f_n
    =
    \int_{\Omega} F - \limsup\limits_{n \to \infty} \int_{\Omega} f_n
  \end{align*}
\end{zremark}

\textbf{证明:}

我们有
\begin{align*}
  \liminf\limits_{n \to \infty} \int_{\Omega} F - f_n
  = \liminf\limits_{n \to \infty} \int_{\Omega} F - \int_{\Omega} f_n
\end{align*}
因为$\int_{\Omega} F$是定值,不妨设$c = \int_{\Omega} F$。

所以
\begin{align*}
  \liminf\limits_{n \to \infty} \left(\int_{\Omega} F - \int_{\Omega} f_n\right)
   & = \liminf\limits_{n \to \infty} \left( c - \int_{\Omega} f_n \right)            \\
   & = \sup\limits_{n} \inf\limits_{m \geq  n} \left( c - \int_{\Omega} f_m \right)  \\
   & = c + \sup\limits_{n} \inf\limits_{m \geq  n} \left(- \int_{\Omega} f_m \right)
\end{align*}

接下来需要证明:
\begin{align*}
  \sup\limits_{n} \inf\limits_{m \geq  n} \left(- \int_{\Omega} f_m \right)
  = - \inf\limits_{n} \sup\limits_{m \geq  n} \left(\int_{\Omega} f_m \right)
\end{align*}
把$(\int_{\Omega} f_n)_{n = 1}^\infty$看做实数序列$(a_n)_{n = 1}^\infty$,
相应的,我们需要证明:
\begin{align*}
  \liminf\limits_{n \to \infty} - a_n = - \limsup\limits_{n \to \infty} a_n
\end{align*}

对任意集合$A$,设$\sup(A) = L^+$,$\inf(A) = L^-$,
于是对任意$x \in A$,我们有
\begin{align*}
  x \leq L^+ \\
  x \geq L^-
\end{align*}
于是
\begin{align*}
  -x \geq -L^+ \\
  -x \leq -L^-
\end{align*}
所以,
\begin{align*}
  \sup(-A) = -L^- = - \inf(A) \\
  \inf(-A) = -L^+ = - \sup(A)
\end{align*}

应用以上命题,我们有
\begin{align*}
  \liminf\limits_{n \to \infty} - a_n
   & = \sup\limits_{n}\inf\limits_{m \geq n} - a_m   \\
   & = \sup\limits_{n}(- \sup\limits_{m \geq n} a_m) \\
   & = - \inf\limits_{n}\sup\limits_{m \geq n} a_m   \\
   & = - \limsup\limits_{n \to \infty} a_n
\end{align*}
\end{document}