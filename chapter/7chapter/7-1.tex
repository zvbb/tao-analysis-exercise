\documentclass{article}
\usepackage{mathtools} 
\usepackage{fontspec}
\usepackage[UTF8]{ctex}
\usepackage{amsthm}
\usepackage{mdframed}
\usepackage{xcolor}
\usepackage{amssymb}
\usepackage{amsmath}

\newmdtheoremenv[
  backgroundcolor=gray!10,
  linewidth=0pt,
  innerleftmargin=10pt,
  innerrightmargin=10pt,
  innertopmargin=10pt,
  innerbottommargin=10pt
]{zgraytheorem}{}
% 定义说明环境样式
\newtheoremstyle{mystyle}% 说明环境样式的名称
  {1em}% 上方间距
  {1em}% 下方间距
  {\normalfont}% 说明内容的字体样式
  {}% 缩进量
  {\bfseries}% 说明标记的字体样式
  {.}% 说明标记和说明内容之间的标点
  {1em}% 说明标记后的水平空间
  {}% 说明标记后的垂直空间
% 使用新定义的样式创建说明环境
\theoremstyle{mystyle}
\newtheorem*{zremark}{说明}


\begin{document}
\title{7.1 习题}
\maketitle

\section*{7.1.1}

\textbf{【a】}

由定义7.1.1 可知,

\begin{align*}
   & \sum \limits_{i=m}^n a_i       \\
   & = a_m + a_{m+1} + \cdots + a_n
\end{align*}

\begin{align*}
   & \sum \limits_{i=n+1}^p a_i         \\
   & = a_{n+1} + a_{n+2} + \cdots + a_p
\end{align*}

\begin{align*}
   & \sum \limits_{i=m}^p a_i                                          \\
   & = a_m + a_{m+1} + \cdots + a_n + a_{n+1} + a_{n+2} + \cdots + a_p
\end{align*}

于是,
\begin{align*}
   & \sum \limits_{i=m}^n a_i + \sum \limits_{i=n+1}^p a_i \\
   & = \sum \limits_{i=m}^p a_i
\end{align*}

\textbf{【b】【c】【d】的证明与【a】类似,证明略}

\textbf{【e】}

归纳法证明。

归纳基始$m=n$,此时,
\begin{align*}
   & |\sum \limits_{i=m}^n a_i| = |a_m| \\
   & \sum \limits_{i=m}^n |a_i| = |a_m|
\end{align*}
满足$|\sum \limits_{i=m}^n a_i| \leq \sum \limits_{i=m}^n |a_i|$

归纳假设$m < n = j-1$时,命题成立。

$n=j++$时,由(a)可知,
\begin{align*}
   & |\sum \limits_{i=m}^j a_i|                  \\
   & = |\sum \limits_{i=m}^{j-1} a_i + a_j|      \\
   & \leq |\sum \limits_{i=m}^{j-1} a_i| + |a_j|
\end{align*}

\begin{align*}
   & \sum \limits_{i=m}^j |a_i|                                       \\
   & = \sum \limits_{i=m}^{j-1} |a_i| + |a_j|                         \\
   & \geq |\sum \limits_{i=m}^{j-1} a_i| + |a_j| & \textbf{【归纳假设保证的】}
\end{align*}
于是$|\sum \limits_{i=m}^j |a_i| \geq |\sum \limits_{i=m}^{j-1} a_i| + |a_j| \geq |\sum \limits_{i=m}^j a_i|$

归纳完毕。

\textbf{【f】} 与\textbf{【e】}类似,可通过归纳法证明。


\end{document}