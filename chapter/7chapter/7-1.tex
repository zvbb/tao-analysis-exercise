\documentclass{article}
\usepackage{mathtools} 
\usepackage{fontspec}
\usepackage[UTF8]{ctex}
\usepackage{amsthm}
\usepackage{mdframed}
\usepackage{xcolor}
\usepackage{amssymb}
\usepackage{amsmath}

\newmdtheoremenv[
  backgroundcolor=gray!10,
  linewidth=0pt,
  innerleftmargin=10pt,
  innerrightmargin=10pt,
  innertopmargin=10pt,
  innerbottommargin=10pt
]{zgraytheorem}{}
% 定义说明环境样式
\newtheoremstyle{mystyle}% 说明环境样式的名称
  {1em}% 上方间距
  {1em}% 下方间距
  {\normalfont}% 说明内容的字体样式
  {}% 缩进量
  {\bfseries}% 说明标记的字体样式
  {.}% 说明标记和说明内容之间的标点
  {1em}% 说明标记后的水平空间
  {}% 说明标记后的垂直空间
% 使用新定义的样式创建说明环境
\theoremstyle{mystyle}
\newtheorem*{zremark}{说明}


\begin{document}
\title{7.1 习题}
\maketitle

\section*{7.1.1}

\textbf{【a】}

由定义7.1.1 可知,

\begin{align*}
         & \sum \limits_{i=m}^n a_i       \\
         & = a_m + a_{m+1} + \cdots + a_n
\end{align*}

\begin{align*}
         & \sum \limits_{i=n+1}^p a_i         \\
         & = a_{n+1} + a_{n+2} + \cdots + a_p
\end{align*}

\begin{align*}
         & \sum \limits_{i=m}^p a_i                                          \\
         & = a_m + a_{m+1} + \cdots + a_n + a_{n+1} + a_{n+2} + \cdots + a_p
\end{align*}

于是,
\begin{align*}
         & \sum \limits_{i=m}^n a_i + \sum \limits_{i=n+1}^p a_i \\
         & = \sum \limits_{i=m}^p a_i
\end{align*}

\textbf{【b】【c】【d】的证明与【a】类似,证明略}

\textbf{【e】}

归纳法证明。

归纳基始$m=n$,此时,
\begin{align*}
         & |\sum \limits_{i=m}^n a_i| = |a_m| \\
         & \sum \limits_{i=m}^n |a_i| = |a_m|
\end{align*}
满足$|\sum \limits_{i=m}^n a_i| \leq \sum \limits_{i=m}^n |a_i|$

归纳假设$m < n = j-1$时,命题成立。

$n=j++$时,由(a)可知,
\begin{align*}
         & |\sum \limits_{i=m}^j a_i|                  \\
         & = |\sum \limits_{i=m}^{j-1} a_i + a_j|      \\
         & \leq |\sum \limits_{i=m}^{j-1} a_i| + |a_j|
\end{align*}

\begin{align*}
         & \sum \limits_{i=m}^j |a_i|                                       \\
         & = \sum \limits_{i=m}^{j-1} |a_i| + |a_j|                         \\
         & \geq |\sum \limits_{i=m}^{j-1} a_i| + |a_j| & \textbf{【归纳假设保证的】}
\end{align*}
于是$|\sum \limits_{i=m}^j |a_i| \geq |\sum \limits_{i=m}^{j-1} a_i| + |a_j| \geq |\sum \limits_{i=m}^j a_i|$

归纳完毕。

\textbf{【f】} 与\textbf{【e】}类似,可通过归纳法证明。

\section*{7.1.2}

\textbf{【a】}
由于$X$是空集,所以定义7.1.6 中的$n=0$,于是,
取一个从$\{i \in \mathbb{N}: 1 \leq i \leq 0\}$到$X$的双射$g$,所以,
\begin{align*}
         & \sum \limits_{x \in X}f(x)    \\
         & =\sum \limits_{i=1}^0 f(g(i)) \\
         & = 0
\end{align*}

\textbf{【b】}

定义双射函数$g:\{i \in \mathbb{N}: 1 \leq i \leq 1\} \rightarrow X$如下:当$i=1,g(x)=x_0$。
于是,
\begin{align*}
         & \sum \limits_{x \in X}f(x)    \\
         & =\sum \limits_{i=1}^1 f(g(i)) \\
         & = f(g(1))                     \\
         & = f(x_0)
\end{align*}

\textbf{【c】}
设$X$有$n$个元素,

取一个从$\{i \in \mathbb{N}: 1 \leq i \leq n\}$到$Y$的双射函数$h$,
于是函数$g \circ h$是从$\{i \in \mathbb{N}: 1 \leq i \leq n\}$到$X$的双射函数;

取一个从$\{i \in \mathbb{N}: 1 \leq i \leq n\}$到$X$的双射函数$h^\prime$。

由命题7.1.8可知,
\begin{align*}
        \sum \limits_{i=1}^n f(h^\prime(i)) = \sum \limits_{i=1}^n f(g \circ h(i))
\end{align*}
于是,
\begin{align*}
        \sum \limits_{x \in X}f(x) = \sum \limits_{y \in Y}f(g(x(y)))
\end{align*}

\textbf{【d】}

题设中,对每一个整数$i \in X$都指定了一个实数$a_i$,
其实是定义了一个函数$f:X \rightarrow \mathbb{R}$如下:$i \in X, f(i) = a_i$。

所以,
\begin{align*}
        \sum\limits_{i=n}^m a_i & =\sum\limits_{i=n}^m f(i)
\end{align*}
由引理 7.1.4 (b)可知,

\begin{align*}
        \sum\limits_{i=n}^m f(i) = \sum\limits_{j=1}^{m-(n-1)} f(j+(n-1))
\end{align*}
此时,定义一个从$Y := \{j \in \mathbb{N}: 1 \leq j \leq m-(n-1)\}$到$X$的双射函数$g$如下:
\begin{align*}
        g(j) = j+(n-1)
\end{align*}
于是,
\begin{align*}
         & \sum\limits_{j=1}^{m-(n-1)} f(j+(n-1))                     \\
         & =\sum\limits_{j=1}^{m-(n-1)} f(g(j))                       \\
         & =\sum\limits_{x \in X} f(x)            & \textbf{定义 7.1.6} \\
\end{align*}

\textbf{【e】}

设$X,Y$的元素个数分别为$n,m$,选取一个从$\{i \in \mathbb{N}: 1 \leq i \leq n+m\}$
到$X \cup Y$的双射函数$g$,并且限定$\{i \in \mathbb{N}: 1 \leq i \leq n\}$的值域是$X$,
$\{i \in \mathbb{N}: n+1 \leq i \leq n+m\}$的值域是$Y$。于是,
\begin{align*}
         & \sum \limits_{z \in X \cup Y} f(z)                                                           \\
         & = \sum \limits_{i=1}^{n+m} f(g(i))                                                           \\
         & = \sum \limits_{i=1}^{n} f(g(i)) + \sum \limits_{i=n+1}^{n+m} f(g(i)) & \textbf{引理7.1.4 (a)} \\
         & = \sum \limits_{x \in X}^{n} f(x) + \sum \limits_{y \in Y}^{n} f(y)   & \textbf{命题7.1.11(d)} \\
\end{align*}

\textbf{【f】}

设$X$有$n$个元素,
取一个从选取一个从$\{i \in \mathbb{N}: 1 \leq i \leq n\}$到$X$的双射函数$h$,
于是,
\begin{align*}
         & \sum \limits_{x \in X} (f(x) + g(x))                                                \\
         & = \sum \limits_{i=1}^n (f(h(i)) + g(h(i)))                                          \\
         & = \sum \limits_{i=1}^n f(h(i)) + \sum \limits_{i=1}^n g(h(i)) & \textbf{引理7.1.4(c)} \\
         & = \sum \limits_{x \in X} f(x) +\sum \limits_{x \in X} g(x)    & \textbf{定义7.1.6}    \\
\end{align*}

\textbf{【g】}

设$X$的元素个数为$n$。

把定义函数$g = cf$,并取一个从选取一个从$\{i \in \mathbb{N}: 1 \leq i \leq n\}$到$X$的双射函数$h$。
此时,
\begin{align*}
         & \sum \limits_{x \in X} cf(x)       \\
         & = \sum \limits_{x \in X} g(x)      \\
         & = \sum \limits_{i = 1}^n g(h(i))   \\
         & = \sum \limits_{i = 1}^n cf(h(i))  \\
         & = c \sum \limits_{i = 1}^n f(h(i)) \\
         & = c \sum \limits_{x \in X}^n f(x)  \\
\end{align*}

\textbf{【h】}

设$X$的元素个数为$n$,取一个从$\{i \in \mathbb{N}: 1 \leq i \leq n\}$到$X$的双射函数$h$。

对$n$进行归纳。

当$n = 0$时,
\begin{align*}
        \sum \limits_{x \in X} f(x) = \sum \limits_{x \in X} g(x) = 0
\end{align*}
此时,命题成立。

当$n=j-1$时,归纳假设命题成立。

当$n=j$时,
\begin{align*}
         & \sum \limits_{x \in X} f(x)                    \\
         & = \sum \limits_{i = 1}^j f(h(i))               \\
         & = \sum \limits_{i = 1}^{j-1} f(h(i)) + f(h(j)) \\
\end{align*}

\begin{align*}
         & \sum \limits_{x \in X} g(x)                    \\
         & = \sum \limits_{i = 1}^j g(h(i))               \\
         & = \sum \limits_{i = 1}^{j-1} g(h(i)) + g(h(j)) \\
\end{align*}

由归纳假设可知$\sum \limits_{i = 1}^{j-1} f(h(i)) \leq \sum \limits_{i = 1}^{j-1} g(h(i))$;
又因为$f(h(j)) \leq g(h(j))$,于是,
\begin{align*}
        \sum \limits_{i = 1}^{j-1} f(h(i)) + f(h(j)) \leq \sum \limits_{i = 1}^{j-1} g(h(i)) + g(h(j))
\end{align*}
即:
\begin{align*}
        \sum \limits_{x \in X} f(x) \leq \sum \limits_{x \in X} g(x)
\end{align*}

\textbf{【i】}

与(h)类似,使用归纳法证明。略

\section*{7.1.3}

嫌麻烦!!! 略

\section*{7.1.4}
$\sum \limits_{j=0}^n \frac{n!}{j!(n-j)!} x^j y^{n-j}$

对n进行归纳。

$n=0$时,$(x+y)^0=1$。

\noindent 又
\begin{align*}
         & \sum \limits_{j=0}^0 \frac{0!}{j!(0-j)!} x^j y^{0-j} \\
         & = \frac{0!}{0!(0-0)!} x^0 y^{0-0}                    \\
         & = 1
\end{align*}
故,$n=0$时,命题成立。

归纳假设$n$时,命题成立。

对$n+1$,
\begin{align*}
         & (x+y)^{n+1}                                                                                        \\
         & = (x+y)^n (x+y)                                                                                    \\
         & = (\sum \limits_{j=0}^n \frac{n!}{j!(n-j)!} x^j y^{n-j}) (x+y)                                     \\
         & =
        \sum \limits_{j=0}^n \frac{n!}{j!(n-j)!} x^{j+1} y^{n-j}
        +
        \sum \limits_{j=0}^n \frac{n!}{j!(n-j)!} x^j y^{n-j+1}                                                \\
         & =
        (
        \sum \limits_{j=0}^{n-1} \frac{n!}{j!(n-j)!} x^{j+1} y^{n-j}
        +
        \sum \limits_{j=n}^{n} \frac{n!}{j!(n-j)!} x^{j+1} y^{n-j}
        )                                                                                                     \\
         & +
        (
        \sum \limits_{j=1}^n \frac{n!}{j!(n-j)!} x^j y^{n-j+1}
        +
        \sum \limits_{j=0}^0 \frac{n!}{j!(n-j)!} x^j y^{n-j+1}
        )                                                                                                     \\
         & = \sum \limits_{j=0}^{n-1} \frac{n!}{j!(n-j)!} x^{j+1} y^{n-j} + x^{n+1}
        + \sum \limits_{j=1}^n \frac{n!}{j!(n-j)!} x^j y^{n-j+1} + y^{n+1}                                    \\
         & = \sum \limits_{j=1}^{n} \frac{n!}{(j-1)!(n+1-j)!} x^{j} y^{n-j+1}
        + \sum \limits_{j=1}^n \frac{n!}{j!(n-j)!} x^j y^{n-j+1}
        + x^{n+1} + y^{n+1}                                                                                   \\
         & = (\sum \limits_{j=0}^{n-1} (\frac{n!}{(j-1)!(n+1-j)!} + \frac{n!}{j!(n-j)!}) x^{j} y^{n+1-j})
        + x^{n+1} + y^{n+1}                                                                                   \\
         & = \sum \limits_{j=0}^{n-1} (\frac{n!j}{j!(n+1-j)!} + \frac{n!(n-j+1)}{j!(n-j+1)!}) x^{j} y^{n+1-j}
        + x^{n+1} + y^{n+1}                                                                                   \\
         & =  \sum \limits_{j=0}^{n-1} \frac{(n+1)!}{j!(n+1-j)!} x^{j} y^{n+1-j}
        + x^{n+1} + y^{n+1}                                                                                   \\
         & =  \sum \limits_{j=0}^{n-1} \frac{(n+1)!}{j!(n+1-j)!} x^{j} y^{n+1-j}
        +  \sum \limits_{j=n+1}^{n+1} \frac{(n+1)!}{j!(n+1-j)!} x^{j} y^{n+1-j}
        +  \sum \limits_{j=0}^{0} \frac{(n+1)!}{j!(n+1-j)!} x^{j} y^{n+1-j}                                   \\
         & = \sum \limits_{j=0}^{n+1} \frac{(n+1)!}{j!(n+1-j)!} x^{j} y^{n+1-j}
\end{align*}

归纳完成,命题得证。

\section*{7.1.5}

设$X$的基数为$K$,通过对$K$进行归纳,来证明该命题。

归纳基始$K=0$,有命题7.1.11 可知,
\begin{align*}
         & \lim\limits_{n \rightarrow \infty} \sum\limits_{x \in X} a_n(x) \\
         & = \lim\limits_{n \rightarrow \infty} 0                          \\
         & = 0
\end{align*}
又因为
\begin{align*}
         & \sum\limits_{x \in X} \lim\limits_{n \rightarrow \infty} a_n(x) \\
         & = 0
\end{align*}
所以,$K=0$时,命题成立。

归纳假设$K = k$时,命题成立。

$K = k + 1$,
取一个从$\{i \in \mathbb{N}: 1 \leq i \leq k+1\}$到$X$的双射$g$,所以,
\begin{align*}
          & \sum\limits_{x \in X} \lim\limits_{n \rightarrow \infty} a_n(x)                \\
          & = \sum\limits_{i = 1}^{k+1} \lim\limits_{n \rightarrow \infty} a_n(g(i))       \\
          & = \sum\limits_{i = 1}^{k} \lim\limits_{n \rightarrow \infty} a_n(g(i))
        +
        \sum\limits_{i = k+1}^{k+1} \lim\limits_{n \rightarrow \infty} a_n(g(i))           \\
          & = \lim\limits_{n \rightarrow \infty} \sum\limits_{x \in X-\{g(k+1)\}} a_n(x)
        +
        \sum\limits_{i = k+1}^{k+1} \lim\limits_{n \rightarrow \infty} a_n(g(i))           \\
          & = \lim\limits_{n \rightarrow \infty} \sum\limits_{x \in X-\{g(k+1)\}} a_n(x)
        +
        \lim\limits_{n \rightarrow \infty} a_n(g(k+1))                                     \\
          & = \lim\limits_{n \rightarrow \infty}  (\sum\limits_{x \in X-\{g(k+1)\}} a_n(x)
        +
        a_n(g(k+1))
        ) & \textbf{定理6.1.19(a)}                                                           \\
          & = \lim\limits_{n \rightarrow \infty}  (\sum\limits_{x \in X-\{g(k+1)\}} a_n(x)
        +
        \sum\limits_{i = k+1}^{k+1} a_n(g(i))
        )                                                                                  \\
          & = \lim\limits_{n \rightarrow \infty}  (\sum\limits_{x \in X-\{g(k+1)\}} a_n(x)
        +
        \sum\limits_{x \in \{g(k+1)\}} a_n(x)
        )                                                                                  \\
          & = \lim\limits_{n \rightarrow \infty} \sum\limits_{x \in X} a_n(x)
\end{align*}

归纳完成,命题得证。
\end{document}