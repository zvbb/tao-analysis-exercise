\documentclass{article}
\usepackage{mathtools} 
\usepackage{fontspec}
\usepackage[UTF8]{ctex}
\usepackage{amsthm}
\usepackage{mdframed}
\usepackage{xcolor}
\usepackage{amssymb}
\usepackage{amsmath}

\newmdtheoremenv[
  backgroundcolor=gray!10,
  linewidth=0pt,
  innerleftmargin=10pt,
  innerrightmargin=10pt,
  innertopmargin=10pt,
  innerbottommargin=10pt
]{zgraytheorem}{}
% 定义说明环境样式
\newtheoremstyle{mystyle}% 说明环境样式的名称
  {1em}% 上方间距
  {1em}% 下方间距
  {\normalfont}% 说明内容的字体样式
  {}% 缩进量
  {\bfseries}% 说明标记的字体样式
  {.}% 说明标记和说明内容之间的标点
  {1em}% 说明标记后的水平空间
  {}% 说明标记后的垂直空间
% 使用新定义的样式创建说明环境
\theoremstyle{mystyle}
\newtheorem*{zremark}{说明}


\begin{document}
\title{7.5 习题}
\maketitle

\section*{7.5.1}
记$L^\prime := \lim \inf\limits_{n \rightarrow \infty} \frac{c_{n+1}}{c_n}$,
因为$\frac{c_{n+1}}{c_n}$总是正的,所以$L^\prime \geq 0$。

设$\epsilon > 0$,由命题6.4.12(a)可知存在一个$N \geq m$
使得$\frac{c_{n+1}}{c_n} \geq L^\prime - \epsilon$对所有的$n \geq N$均成立。
所以$c_{n+1} \geq c_n(L^\prime - \epsilon)$对所有的$n \geq N$均成立。
根据归纳法(对$n$进行归纳),这表明
\begin{align*}
  c_n \geq c_N(L^\prime - \epsilon)^{n-N}
\end{align*}
\textbf{对所有的$n \geq N$均成立}。

如果我们记$A := c_N(L^\prime - \epsilon)^{-N}$,那么
\begin{align*}
  c_n \geq A(L^\prime - \epsilon)^n
\end{align*}
\textbf{对所有的$n \geq N$均成立}。

从而
\begin{align*}
  c_n^{1/n} \geq A^{1/n}(L^\prime - \epsilon)
\end{align*}
\textbf{对所有的$n \geq N$均成立}。

而根据极限定律(定理6.1.19)和引理6.5.3,我们有
\begin{align*}
  \lim \limits_{n \rightarrow \infty} A^{1/n}(L^\prime - \epsilon) = L^\prime - \epsilon
\end{align*}
于是由比较原理(引理6.4.13)可知,
\begin{align*}
  \lim \inf\limits_{n \rightarrow \infty} c_n^{1/n} \geq L^\prime - \epsilon
\end{align*}
而上式对任意的$\epsilon > 0$都成立,因此
\begin{align*}
  \lim \inf\limits_{n \rightarrow \infty} c_n^{1/n} \geq L^\prime
\end{align*}
(为什么?见下方的“说明”)
这就是要证明的结论。


\begin{zremark}
  反证法,假设$\lim \inf\limits_{n \rightarrow \infty} c_n^{1/n} = K < L^\prime$,
  那么取$\delta = (L^\prime - K) /2  > 0$,由命题6.4.12(b)可知,存在一个$N_1 \geq m$,使得
  $c_n < K + \delta$对所有的$n \geq N_1$均成立。

  取$\epsilon = \delta /2$,$L^\prime - \epsilon > K + \delta$是显然的。

  \begin{align*}
    \lim \inf\limits_{n \rightarrow \infty} c_n^{1/n} \geq L^\prime - \epsilon
  \end{align*}可知,存在一个$N_2 \geq m$,使得
  $c_n \geq L^\prime - \epsilon$对所有的$n \geq N_2$均成立。

  取$N = max(N_1, N_2)$,此时对所有的$n \geq N$有
  \begin{align}
    c_n \geq L^\prime - \epsilon \\
    c_n < K + \delta
  \end{align}
  与$L^\prime - \epsilon > K + \delta$矛盾。
\end{zremark}

\section*{7.5.2}

比值判别法。

\begin{align*}
  \lim\sup\limits_{n \rightarrow \infty}\frac{|a_{n+1}|}{|a_{n}|}
   & = \lim\sup\limits_{n \rightarrow \infty}\frac{|(n+1)^qx^{n+1}|}{|n^qx^{n}|} \\
   & = \lim\sup\limits_{n \rightarrow \infty}|\big(\frac{n+1}{n}\big)^qx|        \\
   & = 1^q |x|                                                                  \\
   & = |x|
\end{align*}
因为$|x| < 1$,有推论7.5.3(比值判别法)可知级数是绝对收敛的。
于是级数也是条件收敛的。
又由推论7.2.6(零判别法)可知$\lim\limits_{n \rightarrow \infty}n^qx^n = 0$


\textcolor{red}{注意}
上面的等式使用了以下命题:

\textbf{如果序列$(a_n)_{n=m}^\infty$的收敛于$x$,
  那么$(a_n^r)_{n=m}^\infty$收敛于$x^r$,其中$r$是实数。}

结论是显然的,但我想到的证明过程比较复杂,要使用公理8.1(选择公理),感兴趣的可以看看,
个人认为可能不是最优解。




\section*{7.5.3}

\begin{align*}
  \sum \limits_{n=1}^\infty 1/n^{2n}
\end{align*}

\end{document}