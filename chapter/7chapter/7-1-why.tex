\documentclass{article}
\usepackage{mathtools} 
\usepackage{fontspec}
\usepackage[UTF8]{ctex}
\usepackage{amsthm}
\usepackage{mdframed}
\usepackage{xcolor}
\usepackage{amssymb}
\usepackage{amsmath}

\newmdtheoremenv[
  backgroundcolor=gray!10,
  linewidth=0pt,
  innerleftmargin=10pt,
  innerrightmargin=10pt,
  innertopmargin=10pt,
  innerbottommargin=10pt
]{zgraytheorem}{}
% 定义说明环境样式
\newtheoremstyle{mystyle}% 说明环境样式的名称
  {1em}% 上方间距
  {1em}% 下方间距
  {\normalfont}% 说明内容的字体样式
  {}% 缩进量
  {\bfseries}% 说明标记的字体样式
  {.}% 说明标记和说明内容之间的标点
  {1em}% 说明标记后的水平空间
  {}% 说明标记后的垂直空间
% 使用新定义的样式创建说明环境
\theoremstyle{mystyle}
\newtheorem*{zremark}{说明}



\begin{document}
\title{7.1 为什么}
\maketitle

\section*{推论 7.1.14}

\begin{align*}
  \sum \limits_{(x,y) \in X \times Y} f(x, y) = \sum \limits_{(y,x) \in Y \times X} f(x, y)
\end{align*}
的证明。

\noindent 在证明过程中,有几点认知是比较重要:
\newline
\textcircled{1} $X \times Y$是只表示一个集合。
\newline
\textcircled{2} $(x, y)$是可以看做一个变量的,在书中的例 3.5.6中有说明。

\noindent 证明:

定义$h(x,y):=(y,x)$的双射$h: X \times Y \rightarrow Y \times X$。那么,
\begin{align*}
   & \sum \limits_{(x,y) \in X \times Y} f(x, y)      \\
   & = \sum \limits_{(y,x) \in Y \times X} f(h(y, x)) \\
   & = \sum \limits_{(y,x) \in Y \times X} f(x, y)    \\
\end{align*}

\end{document}