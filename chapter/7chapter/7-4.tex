\documentclass{article}
\usepackage{mathtools} 
\usepackage{fontspec}
\usepackage[UTF8]{ctex}
\usepackage{amsthm}
\usepackage{mdframed}
\usepackage{xcolor}
\usepackage{amssymb}
\usepackage{amsmath}

\newmdtheoremenv[
  backgroundcolor=gray!10,
  linewidth=0pt,
  innerleftmargin=10pt,
  innerrightmargin=10pt,
  innertopmargin=10pt,
  innerbottommargin=10pt
]{zgraytheorem}{}
% 定义说明环境样式
\newtheoremstyle{mystyle}% 说明环境样式的名称
  {1em}% 上方间距
  {1em}% 下方间距
  {\normalfont}% 说明内容的字体样式
  {}% 缩进量
  {\bfseries}% 说明标记的字体样式
  {.}% 说明标记和说明内容之间的标点
  {1em}% 说明标记后的水平空间
  {}% 说明标记后的垂直空间
% 使用新定义的样式创建说明环境
\theoremstyle{mystyle}
\newtheorem*{zremark}{说明}


\begin{document}
\title{7.4 习题}
\maketitle

\section*{7.4.1}
结论很明显,难度在于如何阐明清楚。

要证明$\sum \limits_{n=0}^\infty |a_{f(n)}|$有收敛,只要证明其部分和序列有上界。

证明过程可以参考命题7.4.3(级数的重排序),这里还是比较好处理的,因为这里只需考虑收敛性。

对任意正整数$N$,序列$(f^{-1}(n))_{n=0}^N$是有限的,从而是有界的,
于是存在一个$M$使得对所有的$0 \leq n \leq N$都有$f^{-1}(n) \leq M$。

特别地,对任意的$M^\prime \geq M$,集合$A = \{f(m): m \in \mathbb{N}; m \leq M^\prime \}$
是$B = \{n \in \mathbb{N}; n \leq M^\prime\}$的子集。

于是根据命题7.1.11,对任意的$M^\prime \geq M$都有
\begin{align*}
  \sum \limits_{m=0}^{M^\prime} |a_{f(m)}| & = \sum \limits_{n \in A} |a_{f(m)}| \\
                                           & \leq \sum \limits_{n \in B} |a_n|
\end{align*}
因为$\sum \limits_{n \in B} |a_n|$是有界的,所以
$\sum \limits_{m=0}^{M^\prime} |a_{f(m)}|$也是有界的。
由$M^\prime$的任意性可知其收敛。

\end{document}