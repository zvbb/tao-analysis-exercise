\documentclass{article}
\usepackage{mathtools} 
\usepackage{fontspec}
\usepackage[UTF8]{ctex}
\usepackage{amsthm}
\usepackage{mdframed}
\usepackage{xcolor}
\usepackage{amssymb}
\usepackage{amsmath}

\newmdtheoremenv[
  backgroundcolor=gray!10,
  linewidth=0pt,
  innerleftmargin=10pt,
  innerrightmargin=10pt,
  innertopmargin=10pt,
  innerbottommargin=10pt
]{zgraytheorem}{}
% 定义说明环境样式
\newtheoremstyle{mystyle}% 说明环境样式的名称
  {1em}% 上方间距
  {1em}% 下方间距
  {\normalfont}% 说明内容的字体样式
  {}% 缩进量
  {\bfseries}% 说明标记的字体样式
  {.}% 说明标记和说明内容之间的标点
  {1em}% 说明标记后的水平空间
  {}% 说明标记后的垂直空间
% 使用新定义的样式创建说明环境
\theoremstyle{mystyle}
\newtheorem*{zremark}{说明}


\begin{document}
\title{7.2 习题}
\maketitle

\section*{7.2.1}

发散的。

只需证明,级数的部分和序列$(S_N)_{N=1}^\infty$是发散的即可,证明如下:

当$N$是奇数$S_N = -1$;

当$N$是偶数$S_N = 0$;

所以,该序列不是柯西序列,也就不会收敛,即是发散的。

1.2.2的问题已解决,提到的答案都是不对的。

\section*{7.2.2}

$\bigstar \Rightarrow$

因为$\sum \limits_{n=m} ^{\infty} a_n$收敛,
那么这个级数的部分和序列$(S_N)_{N=m}^{\infty}$是收敛的,
由定理6.4.18(实数的完备性)可知$(S_N)_{N=m}^{\infty}$也是柯西序列,
于是,对任意的$\epsilon > 0$,存在$N \geq m$,$p,q \geq N$使得
\begin{align}
  |S_p - S_q|                & \leq \epsilon \\
  |\sum \limits_{n=p}^q a_n| & \leq \epsilon
\end{align}

$\bigstar \Leftarrow$

对任意$\epsilon > 0$都有$|\sum \limits_{n=p}^q a_n| \leq \epsilon$,
可知级数的部分和序列$(S_N)_{N=m}^{\infty}$是柯西序列,
由定理6.4.18(实数的完备性)可知其也是收敛的,
由部分和收敛可知级数收敛。

\section*{7.2.3}

由命题7.2.5可知,$\sum \limits_{n=m} ^{\infty} a_n$收敛,
对任意$\epsilon > 0$,都存在一个$N \geq m$使得$n \geq N$有
\begin{align*}
  |\sum \limits_{n=n}^n a_n| & \leq \epsilon \\
  |a_n|                      & \leq \epsilon
\end{align*}
由$\epsilon$的任意性可知,$\lim \limits_{n \to \infty} a_n$收敛且收敛于$0$


\section*{7.2.4}

$\bigstar \text{绝对收敛} \Rightarrow \text{条件收敛}$

$\sum \limits_{n=m} ^{\infty} a_n$是绝对收敛,
即$\sum \limits_{n=m} ^{\infty} |a_n|$是收敛的,
由命题7.2.5可知,对任意$\epsilon > 0$,都存在一个整数$N \geq m$,使得$q,p \geq N$,均有,
\begin{align*}
  |\sum \limits_{n=p}^q |a_n|| & \leq \epsilon
\end{align*}
由命题7.1.4(e)可知,
\begin{align*}
  |\sum \limits_{n=p}^q a_n| & \leq |\sum \limits_{n=p}^q |a_n|| \leq \epsilon \\
  |\sum \limits_{n=p}^q a_n| & \leq \epsilon
\end{align*}
再次利用命题7.2.5可知,$\sum \limits_{n=m} ^{\infty} a_n$收敛

$\bigstar \text{三角不等式}$

不妨设$\sum \limits_{n=m} ^{\infty} |a_n|$的部分和序列为$(S_N^\prime)_{N=m}^{\infty}$,
$|\sum \limits_{n=m} ^{\infty} a_n|$的部分和序列为$(S_N)_{N=m}^{\infty}$。
显然,对任意$N \geq m$都有,
\begin{align*}
  S_N^\prime \geq S_N
\end{align*}
又因为两个序列的极限都存在,于是(推论5.4.10 的变形),
\begin{align*}
  \lim \limits_{N \rightarrow \infty} S_N^\prime
  \geq \lim \limits_{N \rightarrow \infty} S_N
\end{align*}
即:
\begin{align*}
  \sum \limits_{n=m} ^{\infty} |a_n| \geq |\sum \limits_{n=m} ^{\infty} a_n|
\end{align*}

\section*{7.2.5}

$\bigstar \text{(a)}$

$\sum \limits_{n=m} ^{\infty} a_n$收敛于$x$,
于是其部分和序列$(A_N)_{N=m}^{\infty}$收敛于$x$。

同理,$\sum \limits_{n=m} ^{\infty} b_n$收敛于$y$,
于是其部分和序列$(B_N)_{N=m}^{\infty}$收敛于$y$。

由题设可知,$\sum \limits_{n=m} ^{\infty} a_n + b_n$的部分和$S_N = A_N + B_N$,
由定理6.1.19(极限定律)可知,序列$(S_N)_{n=m}^\infty$收敛于$x + y$。

$\bigstar \text{(b)}$

略,与(a)证明步骤类似。

$\bigstar \text{(c)}$

不妨设$\sum \limits_{n=m} ^{\infty} a_n$、$\sum \limits_{n=m+k} ^{\infty} a_n$
的部分和分别为$S_N, S_N^\prime$,
并设$M = \sum \limits_{n=m} ^{m+k-1} a_n$。

当$N \geq m+k-1$时,$S_N = M + S_N^\prime$。

$\textbf{(1)}$ 如果$\sum \limits_{n=m} ^{\infty} a_n$收敛。

设$\sum \limits_{n=m} ^{\infty} a_n$收敛于$x$。


由于$\sum \limits_{n=m} ^{\infty} a_n$收敛$x$,所以对任意$\epsilon > 0$,
存在$N_0 \geq m$使得$|S_N - x| \leq \epsilon$,对任意$N \geq N_0$均成立。
取$N_0^\prime = max(N_0, m+k-1)$,此时$|S_N - x| \leq \epsilon$,对任意$N \geq N_0^\prime$均成立。

反证法,假设$\sum \limits_{n=m+k} ^{\infty} a_n$是发散的,
则序列$(S_N^\prime)_{N=m+k}^\infty$是发散的,那么,也就不会收敛于$x-M$。

所以,对存在$\epsilon > 0, N \geq N_0^\prime$,使得,
\begin{align*}
  |S_N^\prime + M - x| & > \epsilon     \\
  S_N^\prime + M       & > x + \epsilon \\
                       & \textbf{或}     \\
  S_N^\prime + M       & < x - \epsilon
\end{align*}

因为$S_N = M + S_N^\prime$,所以$S_N > x + \epsilon$或$S_N < x - \epsilon$,
这与 $|S_N - x| \leq \epsilon$矛盾,所以$\sum \limits_{n=m+k} ^{\infty} a_n$是收敛的。

$\textbf{(2)}$ 如果$\sum \limits_{n=m+k} ^{\infty} a_n$收敛。

把$\sum \limits_{n=m} ^{m+k-1} a_n$,转换成一个新的序列$\sum \limits_{n=m} ^{\infty} b_n$,
当$m \leq n \leq m + k - 1$时,$a_n = b_n$;当$n>m + k - 1$时,$b_n = 0$。

此时新的序列$\sum \limits_{n=m} ^{\infty} b_n$收敛且收敛于$M$。

设序列$\sum \limits_{n=m} ^{\infty} b_n$的部分和为$S_B$。

那么,$S_N = S_B + S_N^\prime$,由极限定律(定理6.1.19)可知,
序列$(S_N)_m^\infty$收敛,且收敛于$M + \sum \limits_{n=m+k} ^{\infty} a_n$。

$\bigstar \text{(d)}$

不妨设$\sum \limits_{n=m} ^{\infty} a_n$、$\sum \limits_{n=m+k} ^{\infty} a_{n-k}$
的部分和分别为$S_N, S_{N^\prime}^\prime$。

如果,序列$(S_{N^\prime}^\prime)_{N^\prime=m+k}^\infty$与$(S_N)_{N=N^\prime-k}^\infty$是等价的,
则它们有相同的极限。

对$N^\prime$进行归纳。

$N^\prime = m+k$时,$S_N^\prime = a_m, S_N = S_{N^\prime - k} = S_m = a_m$,所以$S_N^\prime = S_N$。

归纳假设,$N^\prime = q$时,$S_q^\prime = S_{q-k}$。

当$N^\prime = q + 1$时,
$S_{q+1}^\prime = S_q^\prime + a_{q+1-k}$;
$S_{q+1-k} = S_{q-k} + a_{q+1-k}$,由归纳假设可知$S_q^\prime = S_{q-k}$,
于是$S_{q+1}^\prime = S_{q+1-k}$。

归纳完成。

又由习题6.1.3可知,$(S_N)_{N=N^\prime-k}^\infty$与$(S_N)_{N=m}^\infty$收敛于同一个实数。

于是序列$(S_{N^\prime}^\prime)_{N^\prime=m+k}^\infty$与$(S_N)_{N=m}^\infty$收敛于同一个实数。

\section*{7.2.6}

$\bigstar (a_n)_{n=0}^\infty \text{收敛于} 0$

级数的部分和$S_N = a_0 - a_{N+1}$,由极限定律(定理6.1.19),以及$(a_n)_{n=0}^\infty$收敛于0,可知,
序列$(S_N)_{N=0}^\infty$收敛于$a_0$。

$\bigstar (a_n)_{n=0}^\infty \text{收敛于} L(L \neq 0) $

同理可证,序列$(S_N)_{N=0}^\infty$收敛于$a_0 - L$。
\end{document}