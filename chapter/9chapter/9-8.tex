\documentclass{article}
\usepackage{mathtools} 
\usepackage{fontspec}
\usepackage[UTF8]{ctex}
\usepackage{amsthm}
\usepackage{mdframed}
\usepackage{xcolor}
\usepackage{amssymb}
\usepackage{amsmath}
\usepackage{tikz}


% 定义新的带灰色背景的说明环境 zremark
\newmdtheoremenv[
  backgroundcolor=gray!10,
  % 边框与背景一致,边框线会消失
  linecolor=gray!10
]{zremark}{说明}


\begin{document}
\title{9.8 习题}
\author{张志聪}
\maketitle

\section*{9.8.1}

以单调递增为例,其他情况类似。

设闭区间为$[a, b]$,$f$为$[a, b]$上的单调递增函数。

此时$f(b)$是最大值,$f(a)$为最小值,

因为任意$x_0 \in [a, b]$都有$x_0 \leq b$,按照定义9.8.1可知,
$f(x_0) \leq f(b)$,于是由定义9.6.5可知,此时的$f(b)$就是最大值。

类似地,可证$f(a)$是最小值。

\section*{9.8.2}

函数$f: [1, 3] \to \mathbb{R}$定义如下:

\begin{eqnarray*}
  f(x) =
  \begin{cases*}
    x, x \in [1, 2] \\
    x+1, x \in (2,3]
  \end{cases*}
\end{eqnarray*}

此时$f(1) = 1, f(3) = 4$,$1 \leq 2.5 \leq 4$,但不存在$c \in [1, 3]$使得$f(c) = 2.5$。

\section*{9.8.3}

因为$f: [a, b] \to \mathbb{R}$是即连续又一对一的函数,所以$f(a) \neq f(b)$。

于是
\begin{itemize}
  \item $f(a) < f(b)$

        显然,如果$f$是严格单调的,只能是严格递增的。反证法,假设不是严格递增的,
        即存在$x_0, x_1 \in [a,b]$且$x_0 < x_1$使得$f(x_0) > f(x_1)$。

        \begin{itemize}
          \item[$\circ$]如果$f(x_0) < f(b)$

                于是$\{y : f(a) < y < f(x_0) \text{且} f(x_1) < y < f(b) \}$是非空集合,
                从中任意一个值$y_0$,由介值定理,存在$c_0 \in [a, x_0]$使得$f(c_0) = y_0$,
                且存在$c_1 \in [x_1, b]$使得$f(c_1) = y_0$。

                此时$f(c_0) = y_0 = f(c_1)$,这与题设$f$是即连续又一对一的函数矛盾。

          \item[$\circ$]如果$f(x_0) > f(b)$
                于是$\{y : f(a) < y < f(x_0) \text{且} f(b) < y < f(x_0) \}$是非空集合,
                从中任意一个值$y_0$,由介值定理,存在$c_0 \in [a, x_0]$使得$f(c_0) = y_0$,
                且存在$c_1 \in [x_0, b]$使得$f(c_1) = y_0$。

                此时$f(c_0) = y_0 = f(c_1)$,这与题设$f$是即连续又一对一的函数矛盾。
        \end{itemize}

        于是有矛盾可知,$f$是严格递增的。

  \item $f(a) > f(b)$

        显然,如果$f$是严格单调的,只能是严格递减的。
        证明方法和上同理。
\end{itemize}

综上,$f$是严格单调的。

\section*{9.8.4}

分别证明$f^{-1}$连续性和严格单调递增性。

\begin{itemize}
  \item 连续性

        任意$y_0 \in [f(a), f(b)]$,由于$f$在$[a, b]$上连续又严格递增,
        那么,$f(a) \leq y_0 \leq f(b)$由介值定理可知存在$x_0 \in [a, b]$使得$f(x_0) = y_0$。

        对于任意$\epsilon > 0$,

        设$f([a, b] \cap [x_0 - \epsilon, x_0 + \epsilon]) = [y^\prime, y^{\prime\prime}]$,
        因为$f$是连续又严格递增,那么任意$y^\prime \leq y \leq y^{\prime\prime}$,
        只能在$[a, b] \cap [x_0 - \epsilon, x_0 + \epsilon]$找到唯一的$x$使得$y = f(x)$
        (介值定理保证是存在的,严格递增保证是唯一的)。

        于是取$\delta = min(y_0 - y^\prime, y^{\prime\prime} - y_0)$,使得
        \begin{align*}
           & |f^{-1}(y) - f^{-1}(y_0)| \\
           & = |x - x_0| \leq \epsilon
        \end{align*}
        对所有满足$|y - y_0| < \delta$的$y \in [f(a),f(b)]$均成立
        (因为满足条件的$x$只会在区间$[a, b] \cap [x_0 - \epsilon, x_0 + \epsilon]$)。

  \item 严格单调递增性

        任意$y_0, y_1 \in [f(a), f(b), y_0 < y_1]$,由于$f$在$[a, b]$上连续又严格递增,
        那么,$f(a) \leq y_0 < y_1 \leq f(b)$
        由介值定理可知存在$x_0,x_1 \in [a, b]$使得$f(x_0) = y_0, f(x_1) = y_1$。

        首先说明$x_0, x_1$的唯一性,然后说明$x_0 < x_1$。

        \begin{itemize}
          \item[$\circ$] 唯一性

                假设有多个$x$使得其函数值相同,这与$f$严格递增矛盾。
          \item[$\circ$] $x_0 < x_1$

              如果$x_1 < x_0$,那么$f(x_1) > f(x_0)$(因为$y_0 < y_1$),这与$f$严格递增矛盾。
        \end{itemize}

        综上,$f^{-1}(y_0) < f^{-1}(y_1)$。所以$f^{-1}$严格递增。
\end{itemize}

\section*{9.8.5}

不是命题,不想证,哈哈哈

\end{document}
