\documentclass{article}
\usepackage{mathtools} 
\usepackage{fontspec}
\usepackage[UTF8]{ctex}
\usepackage{amsthm}
\usepackage{mdframed}
\usepackage{xcolor}
\usepackage{amssymb}
\usepackage{amsmath}
\usepackage{tikz}


% 定义新的带灰色背景的说明环境 zremark
\newmdtheoremenv[
  backgroundcolor=gray!10,
  % 边框与背景一致,边框线会消失
  linecolor=gray!10
]{zremark}{说明}


\begin{document}
\title{9.8 习题}
\author{张志聪}
\maketitle

\section*{9.8.1}

以单调递增为例,其他情况类似。

设闭区间为$[a, b]$,$f$为$[a, b]$上的单调递增函数。

此时$f(b)$是最大值,$f(a)$为最小值,

因为任意$x_0 \in [a, b]$都有$x_0 \leq b$,按照定义9.8.1可知,
$f(x_0) \leq f(b)$,于是由定义9.6.5可知,此时的$f(b)$就是最大值。

类似地,可证$f(a)$是最小值。

\section*{9.8.2}

函数$f: [1, 3] \to \mathbb{R}$定义如下:

\begin{eqnarray*}
      f(x) =
      \begin{cases*}
            x, x \in [1, 2] \\
            x+1, x \in (2,3]
      \end{cases*}
\end{eqnarray*}

此时$f(1) = 1, f(3) = 4$,$1 \leq 2.5 \leq 4$,但不存在$c \in [1, 3]$使得$f(c) = 2.5$。

\section*{9.8.3}

因为$f: [a, b] \to \mathbb{R}$是即连续又一对一的函数,所以$f(a) \neq f(b)$。

于是
\begin{itemize}
      \item $f(a) < f(b)$

            显然,如果$f$是严格单调的,只能是严格递增的。反证法,假设不是严格递增的,
            即存在$x_0, x_1 \in [a,b]$且$x_0 < x_1$使得$f(x_0) > f(x_1)$。

            \begin{itemize}
                  \item[$\circ$]如果$f(x_0) < f(b)$

                        于是$\{y : f(a) < y < f(x_0) \text{且} f(x_1) < y < f(b) \}$是非空集合,
                        从中任意一个值$y_0$,由介值定理,存在$c_0 \in [a, x_0]$使得$f(c_0) = y_0$,
                        且存在$c_1 \in [x_1, b]$使得$f(c_1) = y_0$。

                        此时$f(c_0) = y_0 = f(c_1)$,这与题设$f$是即连续又一对一的函数矛盾。

                  \item[$\circ$]如果$f(x_0) > f(b)$
                        于是$\{y : f(a) < y < f(x_0) \text{且} f(b) < y < f(x_0) \}$是非空集合,
                        从中任意一个值$y_0$,由介值定理,存在$c_0 \in [a, x_0]$使得$f(c_0) = y_0$,
                        且存在$c_1 \in [x_0, b]$使得$f(c_1) = y_0$。

                        此时$f(c_0) = y_0 = f(c_1)$,这与题设$f$是即连续又一对一的函数矛盾。
            \end{itemize}

            于是有矛盾可知,$f$是严格递增的。

      \item $f(a) > f(b)$

            显然,如果$f$是严格单调的,只能是严格递减的。
            证明方法和上同理。
\end{itemize}

综上,$f$是严格单调的。

\section*{9.8.4}

分别证明$f^{-1}$连续性和严格单调递增性。

\begin{itemize}
      \item 连续性

            任意$y_0 \in [f(a), f(b)]$,由于$f$在$[a, b]$上连续又严格递增,
            那么,$f(a) \leq y_0 \leq f(b)$由介值定理可知存在$x_0 \in [a, b]$使得$f(x_0) = y_0$。

            对于任意$\epsilon > 0$,

            设$f([a, b] \cap [x_0 - \epsilon, x_0 + \epsilon]) = [y^\prime, y^{\prime\prime}]$,
            因为$f$是连续又严格递增,那么任意$y^\prime \leq y \leq y^{\prime\prime}$,
            只能在$[a, b] \cap [x_0 - \epsilon, x_0 + \epsilon]$找到唯一的$x$使得$y = f(x)$
            (介值定理保证是存在的,严格递增保证是唯一的)。

            于是取$\delta = min(y_0 - y^\prime, y^{\prime\prime} - y_0)$,使得
            \begin{align*}
                   & |f^{-1}(y) - f^{-1}(y_0)| \\
                   & = |x - x_0| \leq \epsilon
            \end{align*}
            对所有满足$|y - y_0| < \delta$的$y \in [f(a),f(b)]$均成立
            (因为满足条件的$x$只会在区间$[a, b] \cap [x_0 - \epsilon, x_0 + \epsilon]$)。

      \item 严格单调递增性

            任意$y_0, y_1 \in [f(a), f(b), y_0 < y_1]$,由于$f$在$[a, b]$上连续又严格递增,
            那么,$f(a) \leq y_0 < y_1 \leq f(b)$
            由介值定理可知存在$x_0,x_1 \in [a, b]$使得$f(x_0) = y_0, f(x_1) = y_1$。

            首先说明$x_0, x_1$的唯一性,然后说明$x_0 < x_1$。

            \begin{itemize}
                  \item[$\circ$] 唯一性

                        假设有多个$x$使得其函数值相同,这与$f$严格递增矛盾。
                  \item[$\circ$] $x_0 < x_1$

                        如果$x_1 < x_0$,那么$f(x_1) > f(x_0)$(因为$y_0 < y_1$),这与$f$严格递增矛盾。
            \end{itemize}

            综上,$f^{-1}(y_0) < f^{-1}(y_1)$。所以$f^{-1}$严格递增。
\end{itemize}

\section*{9.8.5}
(a)

$x, y \in \mathbb{R}$且$y > x$,令集合$J := \{r \in Q: r < x\}, K := \{r \in Q: r < y\}$,于是$J \subseteq K$。

我们有
\begin{align*}
      f(x) = \sum\limits_{r \in J} g(r) \\
      f(y) = \sum\limits_{r \in K} g(r)
\end{align*}
由命题5.4.14可知,存在有理数$q$使得$x < q < y$,于是$J \subset K$。\\
由命题7.1.11(e)可得
\begin{align*}
      f(x) + \sum\limits_{r \in K \setminus J} g(r) = f(y)
\end{align*}
又因为任意$r \in Q$都有$g(r) > 0$,于是
\begin{align*}
      \sum\limits_{r \in K \setminus J} g(r) > 0
\end{align*}
所以
\begin{align*}
      f(x) < f(y)
\end{align*}

(b)

与(a)同理,对任意$x > r$可得
\begin{align*}
      f(x) & = f(r) + \sum\limits_{k \in Q: r \leq k < x} g(k) \\
           & \geq f(r) + g(r)
\end{align*}
根据$r$是有理数可知,存在某个自然数$n$使得$r = q(n)$,于是代入上述不等式
\begin{align*}
      f(x) & \geq f(r) + 2^{-n}
\end{align*}
由$x$的任意性可知,无法找到满足连续性的定义(定义9.4.1)的$x$,所以$f$在$r$处是间断的。

(c)

\begin{zremark}
      书中定义的$f_n$很巧妙,而且需要把$n$看做常量,$x$为自变量,否则$f_n$会成为多元函数,
      到此时为止,书中还未提及多元函数。
\end{zremark}

对于$f_n(x)$至多由$n$个有理数$r_1,r_2,\dots,r_n$使得$g(r_i) \geq 2^{-n}$,
取
\begin{align*}
      \delta = min|r_i - x|>0, 1 \leq i \leq n
\end{align*}
于是对任意$|x - y| \leq \delta$,
如果$y \in (x-\delta, x+\delta)$,那么$f_n(y) = f_n(x)$(此时他们的有理数个数是相同的),所以$f_n$在$x$处是连续的。

我们有
\begin{align*}
      |f(x) - f_n(x)| & = \sum \limits_{r \in Q; r < x, g(r) < 2^{-n}} \\
                      & \leq \sum \limits_{k = n + 1}^\infty 2^{-k}    \\
                      & = 2^{-n}
\end{align*}
最后一个等式是通过等比数列求和$\sum \limits_{k = 0}^n 2^{-k} = 2 - 2^{-n}$且$\sum\limits_{n=0}^\infty 2^{-n} = 2$(几何级数)算出。

对任意$\epsilon > 0$,选出一个$n$使得$2^{-n} < \epsilon / 2$,并重复之前的步骤获取$x$对应的$\delta$,
于是任意$y \in (x - \delta, x + \delta)$,$f_n(y) = f_n(x)$,而且
\begin{align*}
      |f(y) - f(x)| & \leq |f(y) - f_n(y)| + |f_n(y) - f(x)| \\
                    & = |f(y) - f_n(y)| + |f_n(x) - f(x)|    \\
                    & \leq 2^{-n} + 2^{-n}                   \\
                    & = \epsilon
\end{align*}
命题得证。

\end{document}
