\documentclass{article}
\usepackage{mathtools} 
\usepackage{fontspec}
\usepackage[UTF8]{ctex}
\usepackage{amsthm}
\usepackage{mdframed}
\usepackage{xcolor}
\usepackage{amssymb}
\usepackage{amsmath}


% 定义新的带灰色背景的说明环境 zremark
\newmdtheoremenv[
  backgroundcolor=gray!10,
  % 边框与背景一致,边框线会消失
  linecolor=gray!10
]{zremark}{说明}


\begin{document}
\title{9.8 习题}
\author{张志聪}
\maketitle

\section*{9.8.1}

以单调递增为例,其他情况类似。

设闭区间为$[a, b]$,$f$为$[a, b]$上的单调递增函数。

此时$f(b)$是最大值,$f(a)$为最小值,

因为任意$x_0 \in [a, b]$都有$x_0 \leq b$,按照定义9.8.1可知,
$f(x_0) \leq f(b)$,于是由定义9.6.5可知,此时的$f(b)$就是最大值。

类似地,可证$f(a)$是最小值。

\section*{9.8.2}

函数$f: [1, 3] \to \mathbb{R}$定义如下:

\begin{eqnarray*}
  f(x) =
  \begin{cases*}
    x, x \in [1, 2] \\
    x+1, x \in (2,3]
  \end{cases*}
\end{eqnarray*}

此时$f(1) = 1, f(3) = 4$,$1 \leq 2.5 \leq 4$,但不存在$c \in [1, 3]$使得$f(c) = 2.5$。

\section*{9.8.3}

\begin{itemize}
  \item 
\end{itemize}

\end{document}
