\documentclass{article}
\usepackage{mathtools} 
\usepackage{fontspec}
\usepackage[UTF8]{ctex}
\usepackage{amsthm}
\usepackage{mdframed}
\usepackage{xcolor}
\usepackage{amssymb}
\usepackage{amsmath}


% 定义新的带灰色背景的说明环境 zremark
\newmdtheoremenv[
  backgroundcolor=gray!10,
  % 边框与背景一致,边框线会消失
  linecolor=gray!10
]{zremark}{说明}


\begin{document}
\title{9.7 习题}
\author{张志聪}
\maketitle

\section*{9.7.1}

由命题9.6.7(最大值原理)可知,存在$x_{max} \in [a, b]$使得$f(x_{max}) = M$,
类似地,存在$x_{min} \in [a, b]$使得$f(x_{min}) = m$。

由习题9.4.6可知$f$在$[x_{min}, x_{max}]$(这里假设$x_{min} \leq x_{max}$,其他证明类似)上也是连续的。
由定理9.7.1(介值定理)可知,任意$y \in [m, M]$(即:$m \leq y \leq M$)
都存在$c \in [x_{min}, x_{max}]$(此时$c \in [a, b]$也是成立的)使得$f(c) = y$。

\section*{9.7.2}

定义函数$g$如下:
\begin{align*}
  g(x) := f(x) - x
\end{align*}
其中函数$g$的定义域为$[0, 1]$。因为$f,x$都是$[0, 1]$上的连续函数,
由命题9.4.9可知,于是函数$g$也是$[0, 1]$上的连续函数。

思路是证明存在$x \in [0, 1]$使得$g(x) \leq 0$($g(x) \geq 0$),然后利用介值定理。

\begin{itemize}
  \item 存在$x \in [0, 1]$使得$g(x) \leq 0$。

        当$x = 1$时,$g(1) = f(1) - 1$,因为函数$f$的值域是$[0, 1]$,即任意$x \in [0, 1]$都满足$0 \leq f(x) \leq 1$,
        所以$g(1) \leq 0$。

  \item 存在$x \in [0, 1]$使得$g(x) \geq 0$。

        当$x = 0$时,$g(0) = f(0) - 0$,因为函数$f$的值域是$[0, 1]$,即任意$x \in [0, 1]$都满足 $ 0 \leq f(x) \leq 1$,
        所以$g(0) \geq 0$。
\end{itemize}

因为$g(0) \leq 0 \leq g(1)$,且$g$是$[0, 1]$上的连续函数,
由定理9.7.1(介值定理)可知,存在$c \in [0, 1]$使得$g(c) = 0$,此时$f(c) = c$。

\end{document}
