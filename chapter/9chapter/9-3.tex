\documentclass{article}
\usepackage{mathtools} 
\usepackage{fontspec}
\usepackage[UTF8]{ctex}
\usepackage{amsthm}
\usepackage{mdframed}
\usepackage{xcolor}
\usepackage{amssymb}
\usepackage{amsmath}


% 定义新的带灰色背景的说明环境 zremark
\newmdtheoremenv[
  backgroundcolor=gray!10,
  % 边框与背景一致,边框线会消失
  linecolor=gray!10
]{zremark}{说明}


\begin{document}
\title{9.3 习题}
\author{张志聪}
\maketitle

\section*{9.3.1}

\begin{itemize}
  \item (a)$\Rightarrow$ (b)

        对任意$\epsilon > 0$,由(a)f在$x_0$处沿着$E$收敛于$L$可知,
        都存在$\delta > 0$使得$f$被限制在集合$\{x \in E: |x - x_0| < \delta\}$上时,
        $f$是$\epsilon-$接近于$L$的,即$|f(x) - L| \leq \epsilon$。

        由于$(a_n)_{n=0}^\infty$收敛于$x_0$,那么存在正整数$N$,使得
        \begin{align*}
          |a_n - x_0| \leq \frac{1}{2} \delta
        \end{align*}
        对$n \geq N$均成立。又因为此时$a_n \in \{x \in E: |x - x_0| < \delta\}$,
        所以
        \begin{align*}
          |f(a_n) - L| \leq \epsilon
        \end{align*}
        由$\epsilon$的任意性,可得$f((a_n))_{n=0}^\infty$收敛于$L$。

  \item (b)$\Rightarrow$(a)

        反证法,假设(a)不成立,即对某一个$\epsilon _{0} > 0$不存在$\delta > 0$使得
        \begin{align*}
          |f(x) - L| \leq \epsilon_{0}
        \end{align*}
        对所有满足$|x - x_0| < \delta$对$x \in E$均成立。

        那么,对于任意的正整数$n$,设$X_n$表示集合
        \begin{align*}
          X_n := \{x: |f(x) - L| > \epsilon_{0}, |x - x_0| < 1/n\}
        \end{align*}
        是非空集合(其中$|x - x_0| < 1/n$由$x_0$是附着点保证,
        $|f(x) - L| > \epsilon_{0}$由假设(a)不成立保证)。

        利用选择公理,能够找到一个序列$(a_n)_{n=0}^\infty$使得$a_n \in X_n$对所有的$n \geq 1$均成立(特别的,
        $a_0$可以任选$E$中的一个元素)。
        于是这里构造的序列$(a_n)_{n=0}^\infty$收敛于$x_0$,由题设(b)可知,序列$f((a_n))_{n=0}^\infty$收敛于$L$,
        即存在正整数$N$,使得
        \begin{align*}
          |f(a_n) - L| \leq \epsilon_{0}
        \end{align*}
        对$n \geq N$均成立。因为$a_n \in X_n$所以$|f(a_n) -L| > \epsilon_{0}$,存在矛盾。

\end{itemize}

\end{document}
