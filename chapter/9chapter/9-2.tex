\documentclass{article}
\usepackage{mathtools} 
\usepackage{fontspec}
\usepackage[UTF8]{ctex}
\usepackage{amsthm}
\usepackage{mdframed}
\usepackage{xcolor}
\usepackage{amssymb}
\usepackage{amsmath}


% 定义新的带灰色背景的说明环境 zremark
\newmdtheoremenv[
  backgroundcolor=gray!10,
  % 边框与背景一致,边框线会消失
  linecolor=gray!10
]{zremark}{说明}


\begin{document}
\title{9.2 习题}
\author{张志聪}
\maketitle

\section*{9.2.1}

\begin{itemize}
  \item $(f + g) \circ h = (f \circ h) + (g \circ h)$

        成立;因为对任意$x \in R$都有(用到了定义3.3.10)
        \begin{align*}
          ((f + g) \circ h)(x) & = (f + g)(h(x))               \\
                               & = f(h(x)) + g(h(x))           \\
                               & = f \circ h(x) + g \circ h(x) \\
        \end{align*}

  \item $f \circ (g + h) = (f \circ h) + (f \circ h)$

        不成立;设$f(x) = x^2, g(x)=x, h(x) = -x$,则
        \begin{align*}
          f \circ (g + h)(1) & = f((g+h)(1))    \\
                             & = f(g(1) + h(1)) \\
                             & = f(1 - 1)       \\
                             & = f(0)           \\
                             & = 0^2            \\
                             & = 0
        \end{align*}
        \begin{align*}
          (f \circ h) + (f \circ h) & = f(h(1)) + f(h(1)) \\
                                    & = f(1) + f(-1)      \\
                                    & = 1^2 + (-1)^2      \\
                                    & = 1 + 1             \\
                                    & = 2
        \end{align*}
        $f \circ (g + h)(1) \neq (f \circ h) + (f \circ h)$
  \item $(f + g) \cdot h = (f \cdot h) + (g \cdot h)$

        成立;因为对任意$x \in R$都有(第3个等式用到了命题5.3.11),
        \begin{align*}
          ((f + g) \cdot h)(x) & = (f + g)(x) \cdot h(x)             \\
                               & = (f(x) + g(x))\cdot h(x)           \\
                               & = f(x) \cdot h(x) + g(x) \cdot h(x) \\
                               & = (f \cdot h)(x) + (g \cdot h)(x)   \\
        \end{align*}
  \item $f \cdot (g + h) = (f \cdot g) + (f \cdot h)$

        成立;因为对任意$x \in R$都有(第3个等式用到了命题5.3.11),
        \begin{align*}
          (f \cdot (g + h))(x) & = f(x) \cdot (g + h)(x)             \\
                               & = f(x) \cdot (g(x) + h(x))          \\
                               & = f(x) \cdot g(x) + f(x) \cdot h(x) \\
                               & = (f \cdot g)(x) + (f \cdot h)(x)   \\
        \end{align*}
\end{itemize}


\end{document}
