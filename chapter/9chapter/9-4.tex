\documentclass{article}
\usepackage{mathtools} 
\usepackage{fontspec}
\usepackage[UTF8]{ctex}
\usepackage{amsthm}
\usepackage{mdframed}
\usepackage{xcolor}
\usepackage{amssymb}
\usepackage{amsmath}


% 定义新的带灰色背景的说明环境 zremark
\newmdtheoremenv[
  backgroundcolor=gray!10,
  % 边框与背景一致,边框线会消失
  linecolor=gray!10
]{zremark}{说明}


\begin{document}
\title{9.4 习题}
\author{张志聪}
\maketitle

\section*{9.4.1}
按照定义9.4.1可知(a)等价于$f$在$x_0$处沿着$X$收敛于$f(x_0)$(定义9.3.6),
即:$(a) \Leftrightarrow$ $f$在$x_0$处沿着$X$收敛于$f(x_0)$

\begin{itemize}
  \item (b) $\Rightarrow$ $f$在$x_0$处沿着$X$收敛于$f(x_0)$

        (b)满足9.3.9(b),所以$f$在$x_0$处沿着$X$收敛于$f(x_0)$。
  \item (c) $\Rightarrow$ $f$在$x_0$处沿着$X$收敛于$f(x_0)$

        $|f(x) - f(x_0)| < \epsilon$成立,那么$|f(x) - f(x_0)| \leq \epsilon$成立,
        于是满足定义9.3.6,
        所以$f$在$x_0$处沿着$X$收敛于$f(x_0)$。
  \item (d) $\Rightarrow$ $f$在$x_0$处沿着$X$收敛于$f(x_0)$

        因为$(x_0 - \delta, x_0 + \delta) \subset [x_0 - \delta, x_0 + \delta]$,
        所以$|x - x_0| \leq \delta$命题成立,于是$|x - x_0| < \delta$时命题也成立。

        于是满足定义9.3.6,
        所以$f$在$x_0$处沿着$X$收敛于$f(x_0)$。
\end{itemize}

\section*{9.4.2}

例9.4.2、例9.4.3已经说明了证明过程,唯一的区别是定义域的不同的。

\section*{9.4.3}

任意$x_0 \in R$,设序列$(a_n)_{n=0}^\infty$是任意一个完全由$R$中元素构成并且收敛于$x_0$的序列。

对任意$\epsilon > 0$,我们希望

\begin{align*}
  |f(x) - f(x_0)| \leq \epsilon           \\
  |a^x - a^{x_0}| \leq \epsilon           \\
  a^{x_0} |a^{x-x_0} - 1| \leq \epsilon   \\
  |a^{x-x_0} - 1| \leq \epsilon / a^{x_0} \\
\end{align*}

\begin{itemize}
  \item  当$x - x_0 > 0$,
        由引理6.5.3可知,存在正整数$N^\prime$,使得
        \begin{align*}
          |a^{x-x_0} - 1| \leq \epsilon / a^{x_0}
        \end{align*}
        当$x-x_0 \leq 1/N^\prime$时成立。

        所以当$\delta^\prime = 1/N^\prime$时,$|f(x) - f(x_0)| \leq \epsilon$对
        所有满足$|x - x_0| < \delta^\prime$的$x \in R$均成立。
        % 于是$f$在$x_0$处沿着$R$收敛于$f(x_0)$。
  \item 当$x - x_0 < 0$,
        由引理6.5.3和极限定律可知,$\lim\limits_{n \rightarrow \infty} x^{-(1/n)} = 1$,
        类似地,存在正整数$N^{\prime\prime}$,使得
        \begin{align*}
          |a^{x-x_0} - 1| \leq \epsilon / a^{x_0}
        \end{align*}
        当$x-x_0 \geq -(1/N^{\prime\prime})$即$x_0 - x \leq 1/N^{\prime\prime}$时成立。

        所以当$\delta^{\prime\prime} = 1/N^{\prime\prime}$时,$|f(x) - f(x_0)| \leq \epsilon$对
        所有满足$|x - x_0| < \delta^{\prime\prime}$的$x \in R$均成立。
\end{itemize}
取$\delta = min(\delta^\prime, \delta^{\prime\prime})$时,
$|f(x) - f(x_0)| \leq \epsilon$对
所有满足$|x - x_0| < \delta$的$x \in R$均成立。
所以$f$在$x_0$处沿着$R$收敛于$f(x_0)$。
于是$f$在每一个点$x_0 \in R$处都连续。

\section*{9.4.4}

任意$x_0 \in (0, +\infty)$,对任意$\epsilon > 0$,

\begin{align*}
   & |f(x) - f(x_0)| \leq \epsilon                   \\
   & |x^p - x_0^p| \leq \epsilon                     \\
   & | (\frac{x}{x_0})^p - 1 | \leq \epsilon / x_0^p
\end{align*}

\begin{zremark}
  到这里,也就知道书中那样提示的原因了,接下来,我们先按照提示证明:
  \begin{align*}
    \lim\limits_{x \rightarrow 1} x^p = 1
  \end{align*}

  因为$\lim\limits_{x \rightarrow 1} x = 1$,所以利用极限定理(命题9.3.14)可知证明
  $\lim\limits_{x \rightarrow 1} x^n = 1$对所有的非负整数$n$均成立(对$n$进行归纳即可);

  于是$\lim\limits_{x \rightarrow 1} x^n = 1$对所有的负整数$n$均成立(因为$x^n = 1/x^{-n}$然后利用极限定理可证);

  由命题5.4.12和命题4.4.1可知,存在整数$n$使得$n \leq p < n+1$,这里以$n \geq 0, x > 1$为例(其他情况类似,不做赘述),
  因为$x^n \leq x^p < x^{n+1}$,于是由习题9.3.5(夹逼定理的连续形式)可得$\lim\limits_{x \rightarrow 1} x^p = 1$。
\end{zremark}

所以,$\lim\limits_{\frac{x}{x_0} \rightarrow 1} (\frac{x}{x_0})^p = 1$
(不妨把$x^\prime := \frac{x}{x_0}$整体看做自变量),
所以存在$\delta > 0$使得
\begin{align*}
  | (\frac{x}{x_0})^p - 1 | \leq \epsilon / x_0^p
\end{align*}
即
\begin{align*}
   & |f(x) - f(x_0)| \leq \epsilon \\
\end{align*}
对所有满足$|\frac{x}{x_0} - 1| < \delta$即$(|x - x_0| \leq \delta x_0)$的$x \in (0, +\infty)$均成立。

所以$f$在$x_0$处沿着$(0, +\infty)$收敛于$f(x_0)$,
于是$f$在每一个点$x_0 \in (0, +\infty)$处都连续。

\section*{9.4.5}
对任意$\epsilon > 0$,只要能找到$\delta > 0$使得
\begin{align*}
  |(g \circ f)(x) - (g \circ f)(x_0)| \leq \epsilon
\end{align*}
对所有满足$|x - x_0| < \delta$的$x \in X$均成立,即可证明复合函数$g \circ f: X \rightarrow Y$在$x_0$处是连续的。

为了表述方便,定义
\begin{align*}
  y_x & := f(x)                                  \\
  y_0 & := f(x_0)                                \\
  r_x & := (g \circ f)(x) = g(f(x))  = g(y_x)    \\
  r_0 & := (g \circ f)(x_0) = f(f(x_0)) = g(y_0) \\
\end{align*}

因为$g$在$f(x_0)$处是连续的,所以,存在$\delta_{g} > 0$使得
\begin{align*}
  |g(y) - g(y_0)| \leq \epsilon
\end{align*}
对所有满足$|y - y_0| \leq \delta_{g}$的$y \in Y$均成立。

又因为$f$在$x_0$处是连续的,所以,存在$\delta_{f} > 0$使得
\begin{align*}
  |y_x - y_0| \leq \delta_{g}
\end{align*}
对所有满足$|x - x_0| \leq \delta_{f}$的$x \in X$均成立。

综上,取$\delta = \delta_{f}$时,对满足$|x - x_0| \leq \delta$且$x \in X$的$x$来说,
\begin{align*}
  |f(x) - f(x_0)| \leq \delta_{g} \\
  \Rightarrow                     \\
  |y_x - y_0| \leq \delta_{g}     \\
\end{align*}
进而$|g(y_x) - g(y_0)| \leq \epsilon$。
于是可得,复合函数$g \circ f: X \rightarrow Y$在$x_0$处是连续的。

\section*{9.4.6}

任意$x_0 \in Y$。

对任意一个由$Y$中元素构成的且满足$\lim\limits_{x \rightarrow x_0} a_n = x_0$的序列$(a_n)_{n=0}^\infty$,
因为$Y \subseteq X$,所以序列$(a_n)_{n=0}^\infty$中的项也是$X$中元素,
因为$f: X \rightarrow Y$是连续函数,由命题9.4.7(b)可知,
$\lim\limits_{x \rightarrow x_0} f(a_n) = f(x_0)$,
再次利用命题9.4.7(b)可知$f|_{Y}$在$x_0$处是连续的。

于是$f$在$Y$上是连续的。

\begin{zremark}
  这个习题给了一个启发,就算$Y$是一个孤立的点,也是连续的,也是严格符合定义的。
\end{zremark}

\section*{9.4.7}

对$n$进行归纳。

归纳基始$n = 0$,$P(x) = c_0 x^0 = c_0$,由例9.4.2可知此时$P$是连续的。

归纳假设$n = k$时,$P$是连续的。

$n = k+1$时,
\begin{align*}
  P(x) & = \sum\limits_{i=0}^{k+1}c_i x^i                \\
       & = \sum\limits_{i=0}^{k}c_i x^i + c_{k+1}x^{k+1}
\end{align*}

由例9.4.3可知$f(x) = x$是连续函数,于是由命题9.4.9可知$c_{k+1}x^{k+1}$是连续的,
又由归纳假设可知$\sum\limits_{i=0}^{k}c_i x^i$是连续的,
再次利用命题9.4.9可知$\sum\limits_{i=0}^{k}c_i x^i + c_{k+1}x^{k+1}$是连续的,
即$P(x)$是连续的。

\end{document}
