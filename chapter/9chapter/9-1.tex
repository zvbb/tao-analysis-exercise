\documentclass{article}
\usepackage{mathtools} 
\usepackage{fontspec}
\usepackage[UTF8]{ctex}
\usepackage{amsthm}
\usepackage{mdframed}
\usepackage{xcolor}
\usepackage{amssymb}
\usepackage{amsmath}


% 定义新的带灰色背景的说明环境 zremark
\newmdtheoremenv[
  backgroundcolor=gray!10,
  % 边框与背景一致,边框线会消失
  linecolor=gray!10
]{zremark}{说明}


\begin{document}
\title{9.1 习题}
\author{张志聪}
\maketitle

\section*{9.1.1}

$\overline{X} = \overline{Y}$等价于$\overline{X} \subseteq \overline{Y}, \overline{Y} \subseteq \overline{X}$。

\begin{itemize}
      \item $\overline{X} \subseteq \overline{Y}$。

            任意$x \in \overline{X}$,因为$x$是附着点,所以对任意$\epsilon > 0$,都存在$y \in X$
            使得$|x - y| \leq \epsilon$。

            由题设$X \subseteq Y$可知,$y \in Y$,于是由定义9.1.8(附着点)可得,
            $x$也是$\overline{Y}$的附着点,即$x \in \overline{Y}$。

            由$x$的任意性可知$\overline{X} \subseteq \overline{Y}$。
      \item $\overline{Y} \subseteq \overline{X}$。

            任意$x \in \overline{Y}$,因为$x$是附着点,所以对任意$\frac{1}{2}\epsilon > 0$,
            都存在$y \in Y$使得$|x - y| \leq \frac{1}{2}\epsilon$。

            由题设$Y \subseteq \overline{X}$可知,$y \in \overline{X}$,所以$y$是$X$的附着点,
            于是存在$y_x \in X$使得$|y - y_x| \leq \frac{1}{2}\epsilon$。

            于是由命题4.3.7(c)可知$|x - y_x| \leq \epsilon$,所以$x$也是$X$的附着点。

            由$x$的任意性可知$\overline{Y} \subseteq \overline{X}$。
\end{itemize}

\section*{9.1.2}

\begin{itemize}
      \item $X \subseteq  \overline{X}$。

            任意$x \in X$,对于任意的$\epsilon > 0$,有$|x - x| \leq \epsilon$,所以$x$是$X$的附着点。

            由$x$的任意性可知$X \subseteq \overline{X}$。

      \item $\overline{X \cup Y} = \overline{X} \cup \overline{Y}$。
            \begin{itemize}
                  \item[$\circ$] $\overline{X \cup Y} \subseteq \overline{X} \cup \overline{Y}$。

                        任意$x \in \overline{X \cup Y}$,因为$x$是附着点,所以对任意$\epsilon > 0$,
                        都存在$y \in X \cup Y$使得$|x - y| \leq \epsilon$。

                        如果$y \in X$则由定义9.1.8(附着点)可得,
                        $x$也是$X$的附着点。

                        如果$y \in Y$则由定义9.1.8(附着点)可得,
                        $x$也是$Y$的附着点。

                        综上$x \in \overline{X} \cup \overline{Y}$。

                  \item[$\circ$] $\overline{X} \cup \overline{Y} \subseteq \overline{X \cup Y}$。

                        任意$x \in \overline{X} \cup \overline{Y}$,于是要么$x \in \overline{X}$,要么$x \in \overline{Y}$(或者两个皆成立)。

                        以$x \in \overline{X}$为例,因为$x$是$X$的附着点,所以对任意$\epsilon > 0$,
                        存在$y \in X$使得$|x - y| \leq \epsilon$。

                        因为$y \in X \cup Y$则由定义9.1.8(附着点)可得,
                        $x$也是$X \cup Y$的附着点。

                        同理,$x \in \overline{Y}$时也成立。

                        综上$x \in \overline{X \cup Y}$。

            \end{itemize}
      \item $\overline{X \cap Y} \subseteq \overline{X} \cap \overline{Y}$。

            任意$x \in \overline{X \cap Y}$,因为$x$是$X \cap Y$的附着点,
            所以对任意的$\epsilon > 0$,存在$y \in X \cap Y$,使得$|x - y| \leq \epsilon$。

            因为$y \in X \cap Y$,所以$y \in X$且$y \in Y$,则由定义9.1.8(附着点)可得
            $x$是$X$的附着点且是$Y$的附着点,即$x \in \overline{X} \cap \overline{Y}$

      \item 如果$X \subseteq Y$,那么$\overline{X} \subseteq \overline{Y}$。

            任意$x \in \overline{X}$,因为$x$是$X$的附着点,
            所以对任意的$\epsilon > 0$,存在$y \in X$,使得$|x - y| \leq \epsilon$。

            因为$X \subseteq Y$,所以$y \in Y$则由定义9.1.8(附着点)可得
            $x$也是$Y$的附着点,即$x \in \overline{Y}$。

\end{itemize}

\section*{9.1.3}

\begin{itemize}
      \item $\mathbb{N}$的闭包是$\mathbb{N}$。

            由引理9.1.11可得$\mathbb{N} \subseteq \overline{\mathbb{N}}$。

            现在证明附着于$\mathbb{N}$的点只能是$\mathbb{N}$的元素。

            假设实数$x$是$\mathbb{N}$的附着点且$x \notin \mathbb{N}$,
            由命题5.4.12(有理数对实数的界定)与命题4.4.1(由有理数确定的整数散布)可得,
            存在唯一的整数$n$使得$n < x < n + 1$(即:$x$在两个自然数之间)。

            设$\epsilon = \frac{1}{2}min(x - n, n + 1 - x )$,
            此时不存在$y \in \mathbb{N}$使得$|x - y| \leq \epsilon$,与$x$是附着点矛盾。

      \item $\mathbb{Z}$的闭包是$\mathbb{Z}$。

            由引理9.1.11可得$\mathbb{Z} \subseteq \overline{\mathbb{Z}}$。

            现在证明附着于$\mathbb{Z}$的点只能是$\mathbb{Z}$的元素。

            证明过程与$\mathbb{N}$一致,这里不做赘述。

      \item $\mathbb{Q}$的闭包是$\mathbb{R}$。

            即任意实数$x$都是$\mathbb{Q}$的附着点。
            对任意$\epsilon > 0$,取$y = x + \epsilon$,由命题5.4.14可知,存在有理数$q \in \mathbb{Q}$使得
            $x < q < y$,此时$|x - q| \leq \epsilon$。

      \item $\mathbb{R}$的闭包是$\mathbb{R}$。

            由引理9.1.11可得$\mathbb{R} \subseteq \overline{\mathbb{R}}$。

            而有定义9.1.8可知,不存在$\mathbb{R}$外的附着点,否则不满足定义了。

      \item $\varnothing$的闭包是$\varnothing$。

            因为$\varnothing$中没有元素,也就没有$x \in R$能够满足定义9.1.8(附着点)的定义。
\end{itemize}

\section*{9.1.4}

\begin{align*}
      X := [0, 1) \\
      Y := (1, 2] \\
\end{align*}
此时,
\begin{align*}
      \overline{X \cap Y}            & = \varnothing \\
      \overline{X} \cap \overline{Y} & = \{1\}
\end{align*}

\section*{9.1.5}

\begin{itemize}
      \item $\Rightarrow$

            任意$\alpha \in \overline{X}$,
            对任意的正自然数$n$,设$X_n$表示集合
            \begin{align*}
                  X_n := \{ x \in X, |x - \alpha| \leq 1/n \}
            \end{align*}
            由于$\alpha$是附着点,所以$X_n$是非空集合。

            利用选择公理,能够找到一个序列$(a_n)_{n=1}^\infty$使得$a_n \in X_n$对所有的$n \geq 1$均成立。

            以上构造的序列$(a_n)_{n=1}^\infty$是收敛于$x$且每一个元素都属于$X$。

      \item $\Leftarrow$

            对任意$\epsilon > 0$,由$(a_n)_{n=0}^\infty$收敛于$x$可知,
            存在$N$使得$n \geq N$时,
            \begin{align*}
                  |a_n - x| \leq \epsilon
            \end{align*}
            因为序列中的完全是由$X$中的元素构成的,于是可得$x$是附着点。
\end{itemize}

\section*{9.1.6}

\begin{zremark}
      这里所说的闭集,应该是和定义9.1.15对应的,所以应该是$\overline{\overline{X}} = \overline{X}$
\end{zremark}

\begin{itemize}
      \item $\overline{X}$是闭集(即$\overline{\overline{X}} = \overline{X}$)

            由引理9.1.11可知$\overline{X} \subseteq \overline{\overline{X}}$,
            现在需要证明$\overline{\overline{X}} \subseteq \overline{X}$。

            设任意$x^{\prime\prime} \in \overline{\overline{X}}$,
            对任意$\epsilon > 0$,都存在$y^\prime \in \overline{X}$,使得
            \begin{align*}
                  |x^{\prime\prime} - y^\prime| \leq \frac{1}{2} \epsilon
            \end{align*}

            因为$y^\prime$也是$X$的附着点,所以存在$y \in X$使得
            \begin{align*}
                  |y - y^\prime| \leq \frac{1}{2} \epsilon
            \end{align*}
            于是由命题4.3.7(c)可知,
            \begin{align*}
                  |x^{\prime\prime} - y| \leq \epsilon
            \end{align*}

            所以$x^{\prime\prime}$也是$X$的附着点,即$x^{\prime\prime} \in \overline{X}$。

      \item 换个表达方式:$X \subseteq Y, \overline{Y} = Y$,那么$\overline{X} \subseteq Y$(即:$\overline{X} \subseteq \overline{Y}$)。

            任意$x \in \overline{X}$,所以对于任意$\epsilon > 0$,都存在$y \in Y$使得
            \begin{align*}
                  |x - y| \leq \epsilon
            \end{align*}
            因为$X \subseteq Y$,
            于是$y \in Y$,所以$x$也是$Y$的附着点,即$x \in \overline{Y}$。
\end{itemize}

\section*{9.1.7}
设
\begin{align*}
      X := X_1 \cup X_2 \cup \cdots \cup X_n = \bigcup_{i \in \{1,2,...,n\}} X_i
\end{align*}
换句话说,要证明$\overline{X} = X$。

由引理9.1.11可知,$X \subseteq \overline{X}$,接下来我们需要证明$\overline{X} \subseteq X$。

任意$x \in \overline{X}$,对任意$\epsilon > 0$,都存在$y \in X$使得
\begin{align*}
      |x - y| \leq \epsilon
\end{align*}

因为$y \in X$,由公理3.11(并集)可知存在$X_i$使得$y \in X_i$,
于是$x \in \overline{X_i}$,由题设可知$X_i = \overline{X_i}$,
所以$x \in X_i$,于是$x \in X$。

\section*{9.1.8}

设
\begin{align*}
      X := \bigcap \limits_{\alpha \in I} X_{\alpha}
\end{align*}
换句话说,要证明$\overline{X} = X$。

由引理9.1.11可知,$X \subseteq \overline{X}$,接下来我们需要证明$\overline{X} \subseteq X$。

任意$x \in \overline{X}$,对任意$\epsilon > 0$,都存在$y \in X$使得
\begin{align*}
      |x - y| \leq \epsilon
\end{align*}

因为$y \in X$,由式(3.4)可知对任意$X_{\alpha}$都有$y \in X_{\alpha}$,
于是$x \in \overline{X_{\alpha}}$,由题设可知$X_{\alpha} = \overline{X_{\alpha}}$,
再次由式(3.4)可知$x \in X$。

\section*{9.1.9}

\begin{itemize}
      \item $\Rightarrow$

            任意$x \in \overline{X}$,对任意$\epsilon > 0$,都有存在$y \in X$使得
            \begin{align*}
                  |x - y| \leq \epsilon
            \end{align*}

            即$W_x :=\{y: y \in X, |x-y| \leq \epsilon \}$是非空集;

            \begin{itemize}
                  \item[$\circ$] $W_x \setminus \{x\} \neq \varnothing$,则$x$
                        也是$X \setminus \{x\}$的附着点,所以$x$是极限点。
                  \item[$\circ$] $W_x \setminus \{x\} = \varnothing$,可知$x \in X$,
                        且因为$W_x \setminus \{x\}$是空集,
                        所以任意$y \in X \setminus \{x\}$都满足$|x-y| > \epsilon$
                        (特别地$X \setminus \{x\} = \varnothing$空虚的成立),
                        所以$x$是$X$的孤立点。
            \end{itemize}
      \item $\Leftarrow$

            观察定义,如果$x$是$X$的极限点,无法说明$x \in X$,如果是孤立点却能保证$x \in X$
            (定义9.1.18孤立点是按蕴含关系定义的,应该是表述的不准确应是当且仅当的关系,
            否则无法推出$x \in X$),
            而根据引理9.1.11可知$X \subseteq \overline{X}$,所以孤立点肯定是附着点。

            接下来要对极限点进行说明。
            按照定义9.1.18可知,$X$的任意极限点$x$是$X \setminus \{x\}$的附着点,
            因为$X \setminus \{x\} \subseteq X$,所以$x$是$X$的附着点。
\end{itemize}


\begin{zremark}
      错误推论:$X$是实直线的一个子集,对于任意实数$x$,要么是$X$的极限点,要么是$X$的孤立点。

      按照定义9.1.8可知,一个实数$x$要么是$X$的附着点,要么不是。

      习题9.1.9中已经证明,当$x$是附着点,则$x$要么是$X$的极限点,要么是孤立点。

      当$x$不是附着点,则$x \notin X$(否则肯定是附着点),按照定义9.1.18可知$x$不会是孤立点;
      $x$也不会极限点,因为$X \setminus \{x\} = X$(因为$x \notin X$),
      所以如果是极限点,则是附着点(习题9.1.10的反推),与假设矛盾。至此可知,此时的$x$既不是孤立点也不是极限点。

\end{zremark}

\section*{9.1.10}

\begin{itemize}
      \item $\Rightarrow$ $X$是有界的,按照定义9.1.22可知,存在某个实数$M > 0$使得
            $X \subset [-M, M]$,即任意$x \in X$都满足$|x| \leq M$,那么由定义5.5.1(上界)可知
            $M$是$X$的一个上界,由定理5.5.9(最小上界的存在性)可知$sup(X)$是存在,且由定义5.5.5(最小上界)
            可知$sup(X) \leq M$;
            同理可得最大下界$inf(X)$且$-M \leq inf(X)$;

            又因为$inf(X) \leq sup(X)$可得(可以直接通过定义证明,但不能直接使用引理6.4.13,因为实数子集可能不是至多可数的)
            \begin{align*}
                  -M \leq inf(X) \leq sup(X) \leq M
            \end{align*}

      \item $\Leftarrow$ 设$M := max(|inf(X)|, |sup(X)|)$,有最小上界的定义可知,
            对任意$x \in X$都有
            \begin{align*}
                  -M \leq  inf(X) \leq x \leq sup(X) \leq M
            \end{align*}

            所以$X \subset [-M, M]$,由此可知$X$是有界的。
\end{itemize}

\section*{9.1.11}

反证法,假设$\overline{X}$不是有界的。

因为$X$是有界,那么存在$M > 0$使得$X \subset [-M, M]$,即任意$x \in X$都满足$-M \leq x \leq M$;

因为$\overline{X}$不是有界的,
那么对任意实数$M + \epsilon, \epsilon > 0$,
存在$x \in \overline{X}$,$x > M + \epsilon$或$x < -M - \epsilon$,这里以$x > M + \epsilon$为例;

因为$x$是$X$的附着点,都存在$y \in X$使得
\begin{align*}
      |x - y|           & \leq \epsilon       \\
                        & \Rightarrow         \\
      y - \epsilon \leq & x \leq y + \epsilon
\end{align*}
因为$y \in X$,所以$-M \leq y \leq M$,于是可得
\begin{align*}
      -M - \epsilon \leq x \leq M + \epsilon
\end{align*}
这与$x > M + \epsilon$矛盾。

\section*{9.1.12}

\begin{itemize}
      \item 有限个;

            不妨设集合个数为$n, n \in \mathbb{N}$,设每个有界子集的找到$M$分别为
            $M_1, M_2, \dots, M_n$,
            那么定义$M := max(M_1, M_2, \dots, M_n)$(对$n$进行归纳,就可以确定该$M$是可以得到的,
            参照引理5.1.14的证明),
            可证并集$X \subset [-M, M]$(证明略)

      \item 无限个;

            这里无法使用归纳原理进行证明,因为实数子集可能不是至多可数的。





\end{itemize}
\end{document}
