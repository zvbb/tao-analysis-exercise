\documentclass{article}
\usepackage{mathtools} 
\usepackage{fontspec}
\usepackage[UTF8]{ctex}
\usepackage{amsthm}
\usepackage{mdframed}
\usepackage{xcolor}
\usepackage{amssymb}
\usepackage{amsmath}


% 定义新的带灰色背景的说明环境 zremark
\newmdtheoremenv[
  backgroundcolor=gray!10,
  % 边框与背景一致,边框线会消失
  linecolor=gray!10
]{zremark}{说明}


\begin{document}
\title{9.1 习题}
\author{张志聪}
\maketitle

\section*{9.1.1}

$\overline{X} = \overline{Y}$等价于$\overline{X} \subseteq \overline{Y}, \overline{Y} \subseteq \overline{X}$。

\begin{itemize}
      \item $\overline{X} \subseteq \overline{Y}$。

            任意$x \in \overline{X}$,因为$x$是附着点,所以对任意$\epsilon > 0$,都存在$y \in X$
            使得$|x - y| \leq \epsilon$。

            由题设$X \subseteq Y$可知,$y \in Y$,于是由定义9.1.8(附着点)可得,
            $x$也是$\overline{Y}$的附着点,即$x \in \overline{Y}$。

            由$x$的任意性可知$\overline{X} \subseteq \overline{Y}$。
      \item $\overline{Y} \subseteq \overline{X}$。

            任意$x \in \overline{Y}$,因为$x$是附着点,所以对任意$\frac{1}{2}\epsilon > 0$,
            都存在$y \in Y$使得$|x - y| \leq \frac{1}{2}\epsilon$。

            由题设$Y \subseteq \overline{X}$可知,$y \in \overline{X}$,所以$y$是$X$的附着点,
            于是存在$y_x \in X$使得$|y - y_x| \leq \frac{1}{2}\epsilon$。

            于是由命题4.3.7(c)可知$|x - y_x| \leq \epsilon$,所以$x$也是$X$的附着点。

            由$x$的任意性可知$\overline{Y} \subseteq \overline{X}$。
\end{itemize}

\section*{9.1.2}

\begin{itemize}
      \item $X \subseteq  \overline{X}$。

            任意$x \in X$,对于任意的$\epsilon > 0$,有$|x - x| \leq \epsilon$,所以$x$是$X$的附着点。

            由$x$的任意性可知$X \subseteq \overline{X}$。

      \item $\overline{X \cup Y} = \overline{X} \cup \overline{Y}$。
            \begin{itemize}
                  \item[$\circ$] $\overline{X \cup Y} \subseteq \overline{X} \cup \overline{Y}$。

                        任意$x \in \overline{X \cup Y}$,因为$x$是附着点,所以对任意$\epsilon > 0$,
                        都存在$y \in X \cup Y$使得$|x - y| \leq \epsilon$。

                        如果$y \in X$则由定义9.1.8(附着点)可得,
                        $x$也是$X$的附着点。

                        如果$y \in Y$则由定义9.1.8(附着点)可得,
                        $x$也是$Y$的附着点。

                        综上$x \in \overline{X} \cup \overline{Y}$。

                  \item[$\circ$] $\overline{X} \cup \overline{Y} \subseteq \overline{X \cup Y}$。

                        任意$x \in \overline{X} \cup \overline{Y}$,于是要么$x \in \overline{X}$,要么$x \in \overline{Y}$(或者两个皆成立)。

                        以$x \in \overline{X}$为例,因为$x$是$X$的附着点,所以对任意$\epsilon > 0$,
                        存在$y \in X$使得$|x - y| \leq \epsilon$。

                        因为$y \in X \cup Y$则由定义9.1.8(附着点)可得,
                        $x$也是$X \cup Y$的附着点。

                        同理,$x \in \overline{Y}$时也成立。

                        综上$x \in \overline{X \cup Y}$。

            \end{itemize}
      \item $\overline{X \cap Y} \subseteq \overline{X} \cap \overline{Y}$。

            任意$x \in \overline{X \cap Y}$,因为$x$是$X \cap Y$的附着点,
            所以对任意的$\epsilon > 0$,存在$y \in X \cap Y$,使得$|x - y| \leq \epsilon$。

            因为$y \in X \cap Y$,所以$y \in X$且$y \in Y$,则由定义9.1.8(附着点)可得
            $x$是$X$的附着点且是$Y$的附着点,即$x \in \overline{X} \cap \overline{Y}$

      \item 如果$X \subseteq Y$,那么$\overline{X} \subseteq \overline{Y}$。

            任意$x \in \overline{X}$,因为$x$是$X$的附着点,
            所以对任意的$\epsilon > 0$,存在$y \in X$,使得$|x - y| \leq \epsilon$。

            因为$X \subseteq Y$,所以$y \in Y$则由定义9.1.8(附着点)可得
            $x$也是$Y$的附着点,即$x \in \overline{Y}$。

\end{itemize}

\section*{9.1.3}

\begin{itemize}
      \item $\mathbb{N}$的闭包是$\mathbb{N}$。

            由引理9.1.11可得$\mathbb{N} \subseteq \overline{\mathbb{N}}$。

            现在证明附着于$\mathbb{N}$的点只能是$\mathbb{N}$的元素。

            假设实数$x$是$\mathbb{N}$的附着点且$x \notin \mathbb{N}$,
            由命题5.4.12(有理数对实数的界定)与命题4.4.1(由有理数确定的整数散布)可得,
            存在唯一的整数$n$使得$n < x < n + 1$(即:$x$在两个自然数之间)。

            设$\epsilon = \frac{1}{2}min(x - n, n + 1 - x )$,
            此时不存在$y \in \mathbb{N}$使得$|x - y| \leq \epsilon$,与$x$是附着点矛盾。

      \item $\mathbb{Z}$的闭包是$\mathbb{Z}$。

            由引理9.1.11可得$\mathbb{Z} \subseteq \overline{\mathbb{Z}}$。

            现在证明附着于$\mathbb{Z}$的点只能是$\mathbb{Z}$的元素。

            证明过程与$\mathbb{N}$一致,这里不做赘述。

      \item $\mathbb{Q}$的闭包是$\mathbb{R}$。

            即任意实数$x$都是$\mathbb{Q}$的附着点。
            对任意$\epsilon > 0$,取$y = x + \epsilon$,由命题5.4.14可知,存在有理数$q \in \mathbb{Q}$使得
            $x < q < y$,此时$|x - q| \leq \epsilon$。

      \item $\mathbb{R}$的闭包是$\mathbb{R}$。

            由引理9.1.11可得$\mathbb{R} \subseteq \overline{\mathbb{R}}$。

            而有定义9.1.8可知,不存在$\mathbb{R}$外的附着点,否则不满足定义了。

      \item $\varnothing$的闭包是$\varnothing$。

            因为$\varnothing$中没有元素,也就没有$x \in R$能够满足定义9.1.8(附着点)的定义。
\end{itemize}

\section*{9.1.4}

\end{document}
