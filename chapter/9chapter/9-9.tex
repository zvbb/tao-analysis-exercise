\documentclass{article}
\usepackage{mathtools} 
\usepackage{fontspec}
\usepackage[UTF8]{ctex}
\usepackage{amsthm}
\usepackage{mdframed}
\usepackage{xcolor}
\usepackage{amssymb}
\usepackage{amsmath}
\usepackage{tikz}


% 定义新的带灰色背景的说明环境 zremark
\newmdtheoremenv[
  backgroundcolor=gray!10,
  % 边框与背景一致,边框线会消失
  linecolor=gray!10
]{zremark}{说明}


\begin{document}
\title{9.9 习题}
\author{张志聪}
\maketitle

\section*{9.9.1}

\begin{itemize}
  \item $\Rightarrow$

        对于任意$\epsilon > 0$,
        因为序列$(a_n)_{n=1}^\infty, (b_n)_{n=1}^\infty$是等价的,
        由定义9.9.5两者是最终$\epsilon -$接近的,即存在
        正整数$N \geq 1$使得$|a_n - b_n| \leq \epsilon$对任意$n \geq N$均成立,
        即序列$(a_n - b_n)_{n=1}^\infty$是最终$\epsilon -$接近于$0$,
        由定义6.1.5(序列的收敛)可知序列$(a_n - b_n)_{n=1}^\infty$收敛于$0$,
        即$\lim\limits_{n \rightarrow \infty}(a_n - b_n) = 0$。

  \item $\Leftarrow$

        $\lim\limits_{n \rightarrow \infty}(a_n - b_n) = 0$,那么,
        对于任意$\epsilon > 0$,都存在
        正整数$N \geq 1$使得$|a_n - b_n| \leq \epsilon$对任意$n \geq N$均成立,
        于是可得,序列$(a_n)_{n=1}^\infty, (b_n)_{n=1}^\infty$是最终$\epsilon-$接近的。
        由定义9.9.5可知,序列$(a_n)_{n=1}^\infty, (b_n)_{n=1}^\infty$是等价的。
\end{itemize}

\section*{9.9.2}

\begin{itemize}
  \item $(a) \implies (b)$

        $f$在$X$是一致连续的,则对任意$\epsilon > 0$都存在$\delta > 0$使得
        $|f(x) - f(y)| \leq \epsilon$对任意$x, y \in X, |x - y| \leq \delta$均成立。

        因为$(x_n)_{n=0}^\infty$和$(y_n)_{n=0}^\infty$是由$X$中元素构成的等价序列,
        那么,存在正整数$N$使得
        \begin{align*}
          |x_n - y_n| \leq \delta
        \end{align*}
        此时
        \begin{align*}
          | f(x_n) - f(y_n) | \leq \epsilon
        \end{align*}

        由定义9.9.5可知
        $(f(x_n))_{n=0}^\infty$和$(f(y_n))_{n=0}^\infty$是等价的。


  \item $(b) \implies (a)$

        反证法,假设$f$在$X$上不是一致连续的。那么,存在$\epsilon_0 > 0$,
        对任意$n \in \mathbb{N}$存在$x_n, y_n \in X$当$|x_n - y_n| < 1/n$都有$|f(x_n) - f(y_n)| > \epsilon_0$。

        由定义9.9.5可知,$(x_n)_{n=1}^\infty, (y_n)_{n=1}^\infty$是等价的,
        但因为对任意$n$都有$|f(x_n) - f(y_n)| > \epsilon_0$可知,
        $(f(x_n))_{n=1}^\infty,(f(y_n))_{n=0}^\infty$不是等价的。
        这与题设$(b)$矛盾。
\end{itemize}

\section*{9.9.3}

$f$在$X$是一致连续的,则对任意$\epsilon > 0$都存在$\delta > 0$使得
$|f(x) - f(y)| \leq \epsilon$对任意$x, y \in X, |x - y| \leq \delta$均成立。

因为$(x_n)_{n=0}^\infty$是柯西序列,即存在$N$使得对任意$n,m \geq N$都有
\begin{align*}
  |x_n - x_m| \leq \delta
\end{align*}
此时
\begin{align*}
  |f(x_n) - f(x_m)| \leq \epsilon
\end{align*}
于是可得$(f(x_n))_{n=0}^\infty$是柯西序列。

\section*{9.9.4}

$x_0$是$X$的附着点,由引理9.1.14可知,存在一个完全由$X$中元素构成的序列$(a_n)_{n=0}^\infty$收敛于$x_0$。
由定理6.4.18可得收敛序列是柯西序列,则由命题9.9.12可知,$(f(x_n))_{n=0}^\infty$是柯西序列,
再次利用定理6.4.18可得柯西序列收敛,于是
$\lim\limits_{x \to x_0; x \in X} f(x)$存在。

例9.9.10另一种证明:

序列$(1/n)_{n=1}^\infty$是$(0,2)$中的柯西序列,但是序列$f(1/n)_{n=1}^\infty$发散(不是柯西序列),
所以根据命题9.9.12可知,$f$不是一致连续的。

\section*{9.9.5}

反证法,假设$f(E)$是无界的,那么对任意的实数$M$都存在一个元素$x \in E$使得$f(x) \geq M$.

特别地,对于每一个自然数$n$,集合$\{x \in E: |f(x)| \geq n\}$都是非空的。所以我们可以使用选择公理
选取$E$中的一个序列$(x_n)_{n=0}^\infty$使得$|f(x_n)| \geq n$对所有的$n$均成立。
由于这个序列属于有界子集$E$,由定理6.6.8可知,$(x_n)_{n=0}^\infty$有一个收敛的子序列$(x_{n_j})_{j=0}^\infty$,
其中$n_0 < n_1 < n_2 < ...$是一个递增的自然数序列。特别地,对于所有的$j \in N$均有$n_j \geq j$。

由定理6.4.18可得收敛序列是柯西序列,则由命题9.9.12可知,$(f(x_{n_j}))_{n=0}^\infty$是柯西序列,再次利用定理6.4.18可得
$(f(x_{n_j}))_{n=0}^\infty$是收敛序列。

另外,我们从序列的构造过程中看出$|f(x_{n_j})| \geq n_j \geq j$对所有的$j$均成立,从而序列$(f(x_{n_j}))_{n=0}^\infty$是无界的,
这是一个矛盾。


\section*{9.9.6}


\end{document}
