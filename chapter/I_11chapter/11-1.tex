\documentclass{article}
\usepackage{mathtools} 
\usepackage{fontspec}
\usepackage[UTF8]{ctex}
\usepackage{amsthm}
\usepackage{mdframed}
\usepackage{xcolor}
\usepackage{amssymb}
\usepackage{amsmath}


% 定义新的带灰色背景的说明环境 zremark
\newmdtheoremenv[
  backgroundcolor=gray!10,
  % 边框与背景一致,边框线会消失
  linecolor=gray!10
]{zremark}{说明}


\begin{document}
\title{11.1 习题}
\author{张志聪}
\maketitle

\section*{11.1.1}

$X$空集或单点集,命题显然是真的。

$X$非空且不是单点集,证明如下:
\begin{itemize}
  \item $(a) \implies (b)$

        因为$X$是有界的,由命题5.5.9(最小上界的存在性)可知,$X$存在最小上界$M$与最大下界$m$,
        此时$X$只会是以下情况中的一个:$[m, M]$、$(m, M)$、$[M, m)$、$(m, M]$,否则不满足连通性。

        反证法,假设$x,y \in X, x < y$,存在介于$x,y$之间的元素$c \in X$,
        但不属于$[m, M]$、$(m, M)$、$[M, m)$、$(m, M]$,
        于是可得$c < m$或$c > M$,这与$M,m$是$X$的最小上界,最大下界矛盾。

        $[m, M]$、$(m, M)$、$[M, m)$、$(m, M]$都是有界区间,于是(b)成立。


  \item $(b) \implies (a)$

        由定义9.1.1(区间)和例9.1.3中有界区间的定义,可知有界区间,
        一定是有界的并且是连通的。

\end{itemize}

\section*{11.1.2}

反证法,假设$X = I \cap J$,$X$不是有界区间。

由引理11.1.4可知,$X$不会是有界的并且是连通的。因为任意$x \in X$都有$x \in I \cap J$,
而$I$与$J$都是有界的,所以$X$也是有界的。

如果假设成立,那么$X$不是连通的,即存在$x, y \in X, x < y$的元素,有界区间$[x, y]$不是$X$的子集。
即存在$c \in [x, y]$但$c \notin X$。

因为$x, y \in X$,所以$x, y \in I$,由$I$是连通的可知$c \in I$。
类似地,$c \in J$,综上$c \in I \cap J$,即$c \in X$,这与$c \notin X$矛盾。

\section*{11.1.3}

\begin{zremark}
  没采用书中提示的证明方式,主要是没搞懂$\sup I_j$的含义,可能是定义8.5.12中严格上界的意思。
  不用在这概念,也能证明。
\end{zremark}

反证法,不存在形如$I_{j} = (c, b)$或$I_{j} = [c, b)$的区间$I_j$,其中$a \leq c \leq b$。

现在证明如果没有形如$I_{j}$的区间,那么$I$会存在一个洞。

$I_i$是$I$划分中的任意区间元素,$I_j$的左右端点$L, R$满足,
\begin{equation*}
  \begin{cases*}
    a \leq L \leq b \\
    a \leq R \leq b \\
    L \leq R
  \end{cases*}
\end{equation*}
因为$I_i$都不是形如$[c, b)$和$(c, b)$的区间,由此可知,
\begin{align*}
  R < b \\
\end{align*}
由于$I_i$的任意性和划分的基数是有限的,可取所有$I_i$的右端点的最大值为$M$,满足
\begin{align*}
  M < b
\end{align*}

取$x \in (M, b)$,此时$x$不在任何一个划分元素中,与定义11.1.10(划分)矛盾。

\section*{11.1.4}

先证明$P \# P^\prime$也是$I$的一个划分。

反证法,假设$P \# P^\prime$不是$I$的一个划分。
由定义11.1.16(公共加细)可知,$P \# P^\prime$中的元素都会是$I$的子集,于是如果假设成立,

\begin{itemize}
  \item 存在$x \in I$,$x \notin P \# P^\prime$。

        由定义11.1.10(划分)可知,$P$中存在元素$P_i$,使得$x \in P_i$。
        类似地,$P^\prime$中存在元素$P^\prime_i$,使得$x \in P^\prime_i$。综上可得$x \in P_i \cap P^\prime_i$。

        由定义11.1.16(公共加细)可知,$P_i \cap P^\prime_i \in P \# P^\prime$,
        此时可得$x \in P \# P^\prime$,存在矛盾。

  \item 存在$x \in I$,$x$属于$P \# P^\prime$中的多个元素。

        由定义11.1.16(公共加细)可知,有多个属于$P$的区间包含$x$,
        这与定义11.1.10(划分)矛盾。
\end{itemize}


接下来证明,$P \# P^\prime$比$P$更细,也比$P^\prime$更细。

$P \# P^\prime$中的任意元素$X$,由定义11.1.16(公共加细)可知,
存在$K \in P$,$J \in P^\prime$使得$X \subseteq K, X \subseteq J$,
由$X$的任意性,结合定义11.1.14可知,$P \# P^\prime$比$P$更细,也比$P^\prime$更细 。


\end{document}