\documentclass{article}
\usepackage{mathtools} 
\usepackage{fontspec}
\usepackage[UTF8]{ctex}
\usepackage{amsthm}
\usepackage{mdframed}
\usepackage{xcolor}
\usepackage{amssymb}
\usepackage{amsmath}


% 定义新的带灰色背景的说明环境 zremark
\newmdtheoremenv[
  backgroundcolor=gray!10,
  % 边框与背景一致,边框线会消失
  linecolor=gray!10
]{zremark}{说明}


\begin{document}
\title{11.1 习题}
\author{张志聪}
\maketitle

\section*{11.1.1}

$X$空集或单点集,命题显然是真的。

$X$非空且不是单点集,证明如下:
\begin{itemize}
  \item $(a) \implies (b)$

        因为$X$是有界的,由命题5.5.9(最小上界的存在性)可知,$X$存在最小上界$M$与最大下界$m$,
        此时$X$只会是以下情况中的一个:$[m, M]$、$(m, M)$、$[M, m)$、$(m, M]$,否则不满足连通性。

        反证法,假设$x,y \in X, x < y$,存在介于$x,y$之间的元素$c \in X$,
        但不属于$[m, M]$、$(m, M)$、$[M, m)$、$(m, M]$,
        于是可得$c < m$或$c > M$,这与$M,m$是$X$的最小上界,最大下界矛盾。

        $[m, M]$、$(m, M)$、$[M, m)$、$(m, M]$都是有界区间,于是(b)成立。


  \item $(b) \implies (a)$

  由定义9.1.1(区间)和例9.1.3中有界区间的定义,可知


\end{itemize}

\end{document}