\documentclass{article}
\usepackage{mathtools} 
\usepackage{fontspec}
\usepackage[UTF8]{ctex}
\usepackage{amsthm}
\usepackage{mdframed}
\usepackage{xcolor}
\usepackage{amssymb}
\usepackage{amsmath}


% 定义新的带灰色背景的说明环境 zremark
\newmdtheoremenv[
  backgroundcolor=gray!10,
  % 边框与背景一致,边框线会消失
  linecolor=gray!10
]{zremark}{说明}


\begin{document}
\title{11.2 习题}
\author{张志聪}
\maketitle

\section*{11.2.1}

划分$P^\prime$比$P$更细,由命题11.1.14可知,
任意$J \subseteq P^\prime$,都存在$K \subseteq P$使得$J \subseteq K$。
此时$J \subseteq K \subseteq P$,由于$f$是关于$P$上的分段常数函数,
那么$f$在$K$上是常值的,于是$f$在$J$上也是常值的。

由$J$的任意性可知,$f$也是关于$P^\prime$上的分段常数函数。

\section*{11.2.2}

由于$f,g$都是$I$上的分段常数函数,于是存在$I$上的划分$P_f, P_g$分别使得
$f$是关于$P_f$上的分段常数函数,$g$是关于$P_g$上的分段常数函数。

由引理11.1.18可知,$P := P_f \# P_g$是$I$的划分,且比$P_f$和$P_g$更细。
由引理11.2.7可知,$f,g$都是关于$P$的分段常数函数,
那么,由定义11.2.3,对任意$J \subseteq P$,$f|_J$是常数函数,$g|_J$是常数函数,
于是$(f-g)|J$也是常量函数,由$J$的任意性可知,$f-g$是关于$P$的分段常数函数,
由定义11.2.5可知,$f-g$是$I$上的分段函数。

其他情况类似。

\section*{11.2.3}

令$Q := P \# P^\prime$,由引理11.2.7可知,$f$是关于$Q$上的分段常数函数。

接下来证明:
\begin{align}
  p.c.\int_{[P]} f = p.c.\int_{[Q]}f \\
  p.c.\int_{[P^\prime]} f = p.c.\int_{[Q]}f
\end{align}

对任意$K \in P$,定义
\begin{align*}
  Q_K := \{X \in Q : X \subseteq K\}
\end{align*}
证明$K = Q_K$。反证法,假设$K \neq Q_K$。由$Q_K$的构造方式,易知$Q_K \subseteq K$,
如果假设成立,那么,存在$x \in K$,$x \notin Q_K$。

$Q$中一定存在$J$使得$x \in J$,由$Q$比$P$更细,可知存在$W \in P$使得$J \subseteq W$,
由划分的定义可知$W = K$,因为如果$W \neq K$,则与定义11.1.10(划分)中每个元素恰好
属于$P$中的一个有界区间矛盾,存在了两个区间都包含$x$。于是可知$J \in Q_K$,与假设矛盾。

由$K = Q_K$可知,$p.c.\int_{[K]} f = p.c.\int_{[Q_K]}f$,由$K$的任意性,可知(1)式成立。

类似地,可证(2)式成立。

\section*{11.2.4}

设$P_f,P_g$是满足条件的划分,即$f,g$分别是关于$P_f,P_g$的分段常数函数。
令$P := P_f \# P_g$,由引理11.1.18和命题11.2.13可知,
\begin{align*}
  p.c.\int_{I} f = p.c.\int_{[P]}f \\
  p.c.\int_{I} g = p.c.\int_{[P]}g
\end{align*}

\begin{itemize}
  \item (a)

        由定义11.2.9可知,
        \begin{align*}
          p.c.\int_{I} (f + g) = \sum\limits_{J \in P} (F_J + G_J) |J|
        \end{align*}
        其中$F_J, G_J$表示$f,g$分别在$J$上的常数值。
        由命题7.1.11可知,

        \begin{align*}
          p.c.\int_{I} (f + g) & = \sum\limits_{J \in P} (F_J + G_J) |J|                         \\
                               & = \sum\limits_{J \in P} F_J |J| + \sum\limits_{J \in P} G_J |J| \\
                               & = p.c.\int_{[P]} f + p.c.\int_{[P]} g                           \\
                               & = p.c.\int_{I} f + p.c.\int_{I} g
        \end{align*}

  \item (b)

        证明与(a)类似,略
  \item (c)

        利用(a),(b)可证,略

  \item (d)

        由定义11.2.9可知,
        \begin{align*}
          p.c.\int_{I} f = \sum\limits_{J \in P} F_J |J|
        \end{align*}
        其中$F_J$表示$f$在$J$上的常数值。

        由题设可知,$F_J \geq 0$,又由定义11.1.8可知$|J| \geq 0$,
        于是,任意$J \in P$都有$F_J |J| \geq 0$,所以,
        \begin{align*}
          \sum\limits_{J \in P} F_J |J| \geq 0
        \end{align*}
        即:
        \begin{align*}
          p.c.\int_{I} f \geq 0
        \end{align*}

  \item (e)

        证明框架:对$P$的基数$n$进行归纳。略

  \item (f)

        由定义11.2.9可知,
        \begin{align*}
          p.c.\int_{I} f = \sum\limits_{J \in P} F_J |J|
        \end{align*}
        其中$F_J$表示$f$在$J$上的常数值,又$f$是常量函数$f(x) = c$,所以总是$F_J = c$,于是,
        \begin{align*}
          p.c.\int_{I} f & = \sum\limits_{J \in P} F_J |J| \\
                         & = \sum\limits_{J \in P} c |J|   \\
                         & = c\sum\limits_{J \in P} |J|    \\
                         & = c|I|
        \end{align*}
        最有一个等式利用了定理11.1.13


  \item (g)

        \begin{align*}
          J_L := \{x \in J: x \leq min(P) ; x \notin P \} \\
          J_R := \{x \in J: x \geq max(P) ; x \notin P \}
        \end{align*}
        $\{J_L, P, J_R\}$划分是符合定义11.1.10(划分)的。

        于是,
        \begin{align*}
          p.c.\int_{J} F & = \sum\limits_{K \in J} C_K |K|                                                                     \\
                         & = \sum\limits_{K \in J_L} C_K |K| + \sum\limits_{K \in P} C_K |K| + \sum\limits_{K \in J_R} C_K |K| \\
                         & = 0 + \sum\limits_{K \in P} C_K |K| + 0                                                             \\
                         & = \sum\limits_{K \in P} C_K |K|                                                                     \\
                         & = p.c.\int_{I} f
        \end{align*}
        其中,$K \in J, C_K$表示$F$在$K$上的常数值。
        而$x \in J_L$时,$C_K = 0$,即此时的$C_J = 0$;
        $x \in J_R$时,$C_K = 0$,即此时的$C_J = 0$。

  \item (h)

        \begin{itemize}
          \item[$\circ$]

                \begin{align*}
                  J_P := \{J \cap X: X \in P\} \\
                  K_P := \{K \cap X: X \in P\} \\
                \end{align*}
                可见$J_P, K_P$分别是$J, K$的划分。

                对任意$X \in J_P$,按照$J_P$的构造方式存在$Y \in P$使得$X \subseteq Y$,
                因为$f$是关于$P$的分段常数函数,所以$f$在$Y$上都是常值的,那么在$X$上也是常值的,由定义11.2.3
                可得$f|_J$是关于$J_P$的分段常数函数。于是$f|_J$是$J$上的分段常数函数。

                类似地,$f|_K$是$K_P$上的分段常数函数。

          \item[$\circ$]

                定义
                \begin{equation*}
                  F(x) :=
                  \begin{cases*}
                    f_J(x) & $x \in J$             \\
                    0      & $x \in I \setminus J$
                  \end{cases*}
                \end{equation*}

                \begin{equation*}
                  G(x) :=
                  \begin{cases*}
                    f|_K(x) & $x \in K$             \\
                    0       & $x \in I \setminus K$
                  \end{cases*}
                \end{equation*}

                于是$f = F + G$,利用(a)(g)可得,
                \begin{align*}
                  p.c.\int_{I} f & = p.c.\int_{I} (F + G)                  \\
                                 & = p.c.\int_{I} F + p.c.\int_{I} G       \\
                                 & = p.c.\int_{J} f|_J + p.c.\int_{K} f|_K \\
                \end{align*}
        \end{itemize}
\end{itemize}

\end{document}