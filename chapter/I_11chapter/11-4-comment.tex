\documentclass{article}
\usepackage{mathtools} 
\usepackage{fontspec}
\usepackage[UTF8]{ctex}
\usepackage{amsthm}
\usepackage{mdframed}
\usepackage{xcolor}
\usepackage{amssymb}
\usepackage{amsmath}


% 定义新的带灰色背景的说明环境 zremark
\newmdtheoremenv[
  backgroundcolor=gray!10,
  % 边框与背景一致,边框线会消失
  linecolor=gray!10
]{zremark}{说明}


\begin{document}
\title{11.4 注释}
\author{张志聪}
\maketitle

\begin{zremark}
  设$\epsilon > 0$,由$\int_I f = \underline{\int}_I f$可知,存在一个分段常数函数函数
  $\underline{f} : I \to \mathbb{R}$在$I$上从下方控制$f$,并且有
  \begin{align*}
    \int_I \underline{f} \geq \int_I f - \epsilon
  \end{align*}
\end{zremark}

反证法,不存在满足条件的函数$\overline{f}$。
因为$f$是有界函数,那么,在$I$上从下方控制$f$的分段常数函数一定是存在的,
由假设可知,对任意$g$是在$I$上从下方控制$f$的分段常数函数,都有
\begin{align*}
  p.c.\int_I g \leq \int_I f - \epsilon 
\end{align*}
而$f$是黎曼可积的,于是
\begin{align*}
  \int_I f = \underline{\int}_I f = sup\{p.c.\int_I g: g \text{在} I \text{上从下方控制}f \text{的分段常数函数}\}
\end{align*}
因为$\int_I f - \epsilon $是任意$p.c.\int_I g$的上界,且
\begin{align*}
  \int_I f - \epsilon < \int_I f 
\end{align*}
这与$\int_I f$是最小上界矛盾。

\end{document}