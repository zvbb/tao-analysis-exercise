\documentclass{article}
\usepackage{mathtools} 
\usepackage{fontspec}
\usepackage[UTF8]{ctex}
\usepackage{amsthm}
\usepackage{mdframed}
\usepackage{xcolor}
\usepackage{amssymb}
\usepackage{amsmath}


% 定义新的带灰色背景的说明环境 zremark
\newmdtheoremenv[
  backgroundcolor=gray!10,
  % 边框与背景一致,边框线会消失
  linecolor=gray!10
]{zremark}{说明}


\begin{document}
\title{11.8 注释}
\author{张志聪}
\maketitle

\begin{zremark}
  命题11.5.3和命题11.5.6不成立的原因。 
\end{zremark}

黎曼积分与黎曼-斯蒂尔杰斯积分的区别在于后者存在$\alpha$长度的概念,
这里命题的不成立都是因为这一点不同导致的。

命题11.5.3和命题11.5.6的前置条件都是$I$为有界区间,如何假设$I := (0, 1),\alpha := \frac{1}{x}$,
那么$\alpha[(0, 1)]$是无法计算的。如果是闭区间就可以避免这个问题。



\end{document}