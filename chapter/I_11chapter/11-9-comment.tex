\documentclass{article}
\usepackage{mathtools} 
\usepackage{fontspec}
\usepackage[UTF8]{ctex}
\usepackage{amsthm}
\usepackage{mdframed}
\usepackage{xcolor}
\usepackage{amssymb}
\usepackage{amsmath}


% 定义新的带灰色背景的说明环境 zremark
\newmdtheoremenv[
  backgroundcolor=gray!10,
  % 边框与背景一致,边框线会消失
  linecolor=gray!10
]{zremark}{说明}


\begin{document}
\title{11.9 注释}
\author{张志聪}
\maketitle

\begin{zremark}
  定理11.9.1前置条件: $f: [a, b] \to \mathbb{R}$是黎曼可积的函数,是否能说明当$a \leq x \leq b$时
  $f: [a, x] \to \mathbb{R}$也是黎曼可积的函数,即
  \begin{align*}
    \int_{[a, x]} f
  \end{align*}
\end{zremark}
我们可以定义$[a,b]$的一个划分$P := \{[a,x], (x, b]\}$,于是由定理11.4.1(h)可以得到结论。

\begin{zremark}
  定理11.9.1中
  \begin{align*}
    F(y) - F(x) = \int_{[a,y]} f - \int_{[a,x]} f = \int_{[x,y]} f
  \end{align*}
  如果严格使用定理11.4.1(h)应该是
  \begin{align*}
    F(y) - F(x) = \int_{[a,y]} f - \int_{[a,x]} f = \int_{(x,y]} f
  \end{align*}
  那么,是否可以推论出
  \begin{align*}
    \int_{[x,y]} f = \int_{(x,y]} f
  \end{align*}
\end{zremark}
这里可以使用定理11.4.1(h),定义$[x,y]$的一个划分$ P:= \{\{x\}, (x,y]\}$,于是
\begin{align*}
  \int_{[x,y]} f & = \int_{\{x\}} f + \int_{(x,y]} f \\
                 & = 0 + \int_{(x,y]} f \\
                 & = \int_{(x,y]} f
\end{align*}

\end{document}