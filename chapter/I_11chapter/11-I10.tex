\documentclass{article}
\usepackage{mathtools} 
\usepackage{fontspec}
\usepackage[UTF8]{ctex}
\usepackage{amsthm}
\usepackage{mdframed}
\usepackage{xcolor}
\usepackage{amssymb}
\usepackage{amsmath}


% 定义新的带灰色背景的说明环境 zremark
\newmdtheoremenv[
  backgroundcolor=gray!10,
  % 边框与背景一致,边框线会消失
  linecolor=gray!10
]{zremark}{说明}


\begin{document}
\title{11.10 习题}
\author{张志聪}
\maketitle

\section*{11.10.1}

因为$F, G$在闭区间$[a, b]$上可微,则$F, G$都是连续函数,于是推论11.5.2可知,$F, G$都是$[a, b]$上的黎曼可积的函数。

由定理11.4.5可知,$F G^\prime, F^\prime G$都是$[a, b]$上的黎曼可积的函数。

由定理10.1.13(d)可知
\begin{align*}
  (FG)^\prime = F^\prime G + F G^\prime
\end{align*}
所以
\begin{align*}
  \int_{[a, b]} (FG)^\prime & = \int_{[a, b]} F^\prime G + \int_{[a, b]} F G^\prime \\
                            & = (FG)(b) - (FG)(a)                                   \\
\end{align*}
第一个等式使用了定理11.4.1(a),第二个等式使用了定理11.9.4(微积分第二基本定理)。
于是经过变换可得
\begin{align*}
  \int_{[a, b]} FG^\prime & = F(b)G(b) - F(a)G(a) - \int_{[a, b]} F^\prime G
\end{align*}

\section*{11.10.2}

\begin{itemize}
  \item $\phi^{-1}(J)$是连通的。

        反证法,假设$\phi^{-1}(J)$不是连通的,那么存在$x, y \in \phi^{-1}(J),x \neq y$且$x < c < y$满足
        $c \notin \phi^{-1}(J)$。

        由$\phi$在闭区间$[a, b]$上单调递增的连续函数可知
        \begin{align*}
          \phi(x) \leq \phi(c) \leq \phi(y)
        \end{align*}
        而由假设可知$c \notin \phi^{-1}(J)$,所以$c$应该小于$J$的左端点$J_l$(大于$J$的右端点,同理),于是
        \begin{align*}
          \phi(c) \leq \phi(J_l)
        \end{align*}
        满足上述两个不等式只能是
        \begin{align*}
          \phi(c) = \phi(x) = \phi(y)
        \end{align*}
        因为$\phi(x) \in J$,于是$\phi(c) \in J$进而$c \in \phi^{-1}(J)$,存在矛盾。

  \item $c_J$还是$f \circ \phi$在$\phi^{-1}(J)$上的常数值。

        任意$x \in \phi^{-1}(J)$,由集合$\phi^{-1}(J)$的定义可知,
        \begin{align*}
          \phi(x) \in J
        \end{align*}
        于是
        \begin{align*}
          (f \circ \phi)(x) = f(\phi(x)) = c_J
        \end{align*}

  \item $\textbf{Q}$是$[a,b]$的一个划分。

        任意$x \in [a, b]$,都有$\phi(x) \in [\phi(a), \phi(b)]$,
        因为$\textbf{P}$是$[\phi(a), \phi(b)]$的一个划分,所以存在一个$J \in \textbf{P}$使得
        \begin{align*}
          \phi(x) \in J
        \end{align*}
        于是
        \begin{align*}
          x \in \phi^{-1}(J)
        \end{align*}
        是否还存在另一个区间$K \in \textbf{P}$使得
        \begin{align*}
          x \in \phi^{-1}(K)
        \end{align*}
        不会存在,因为如果存在,则$\phi(x) \in K, \phi(x) \in J$这与$\textbf{P}$是划分矛盾。

  \item $\phi[\phi^{-1}(J)] = |J|$

        不妨设$J$的左右端点为$l, r$,现在需要证明
        \begin{equation*}
          \begin{cases*}
            \sup \phi^{-1}(J) = \phi^{-1}(r) \\
            \inf \phi^{-1}(J) = \phi^{-1}(l)
          \end{cases*}
        \end{equation*}
        反证法,假设存在$x \in \phi^{-1}(J)$且$x > \phi^{-1}(r)$。因为$x \in \phi^{-1}(J)$,所以
        \begin{align*}
          \phi(x) \in J
        \end{align*}
        又因为$x > \phi^{-1}(r)$,且$\phi$在闭区间$[a, b]$上单调递增的连续函数,所以
        \begin{align*}
          \phi(x) > \phi(\phi^{-1}(r)) = r
        \end{align*}
        这与$\phi(x) \in J$矛盾,所以不存在这样的$x$。
        于是
        \begin{align*}
          \sup \phi^{-1}(J) = \phi^{-1}(r)
        \end{align*}
        类似地,可得
        \begin{align*}
          \inf \phi^{-1}(J) = \phi^{-1}(l)
        \end{align*}
        于是
        \begin{align*}
          \phi[\phi^{-1}(J)] & = \phi(\sup \phi^{-1}(J)) - \phi(\inf \phi^{-1}(J)) \\
                             & = r - l
        \end{align*}
        又因为
        \begin{align*}
          |J| = r - l
        \end{align*}
        综上,命题得证。
\end{itemize}

\section*{11.10.3}

设$\epsilon > 0$,那么我们能够找到一个在$[a,b]$上从上方控制$f$的分段常数函数$\overline{f}$
和一个在$[a,b]$上从下方控制$f$的分段常数函数$\underline{f}$,它们使得
\begin{align*}
  \int_{[a,b]} f - \epsilon \leq \int_{[a,b]} \underline{f} \leq \int_{[a,b]} \overline{f} \leq \int_{[a,b]} f + \epsilon
\end{align*}
令$\overline{g}(x) = \overline{f}(-x), \underline{g}(x) = \underline{f}(-x)$是$[-b, -a]$上的函数。
对任意$x \in [-b, -a]$,有
\begin{align*}
  \overline{g}(x) = \overline{f}(-x) \geq f(-x) = g(x)
\end{align*}
所以$\overline{g}$在$[-b, -a]$上从上方控制$g$。类似地,$\underline{g}$在$[-b, -a]$上从下方控制$g$。

对任意$J \in \textbf{P}$我们定义$G_J := \{x \in [-b, -a] : -x \in J\}$,
于是$\textbf{P}^\prime := \{G_J : J \in \textbf{P}\}$是$[-b, -a]$的一个划分。

此外,
任意$G_J \in \textbf{P}^\prime,\overline{g}$在$G_J$上的常数值也是
$\overline{f}$在$J$上的常数值。类似地,任意$G_J \in \textbf{P}^\prime,\underline{g}$在$G_J$上的常数值也是
$\underline{f}$在$J$上的常数值。

于是
\begin{align*}
  \int_{[-b, -a]} \overline{g} & = p.c.\int_{[-b, -a]} \overline{g}                   \\
                               & = \sum \limits_{G_J \in \textbf{P}^\prime} c_J |G_J| \\
                               & = \sum \limits_{J \in \textbf{P}} c_J |J|            \\
                               & = \int_{[a, b]} \overline{f}
\end{align*}
类似地,
\begin{align*}
  \int_{[-b, -a]} \underline{g} = \int_{[a, b]} \underline{f}
\end{align*}
所以
\begin{align*}
  \int_{[a,b]} f - \epsilon
  \leq \int_{[-b, -a]} \underline{g}
  \leq \underline{\int}_{[-b, -a]} g \leq \overline{\int}_{[-b, -a]} g
  \leq \int_{[-b, -a]} \overline{g}
  \leq \int_{[a,b]} f + \epsilon
\end{align*}
由$\epsilon$的任意性可知,据此可得出结论。

\section*{11.10.4}
(1)命题

设$[a, b]$是一个闭区间,$\phi : [a, b] \to [\phi(b), \phi(a)]$是一个单调递减的可微函数,
并且使得$\phi ^\prime$是黎曼可积的。设$f: [\phi(b), \phi(a)] \to \mathbb{R}$是$[\phi(b), \phi(a)]$上的
黎曼可积的函数,那么$(f \circ \phi)\phi^\prime : [a, b] \to \mathbb{R}$是$[a, b]$上是黎曼可积的,并且
\begin{align*}
  \int_{[a, b]} (f \circ \phi) \phi^\prime = - \int_{[\phi(b), \phi(a)]} f
\end{align*}

(2)证明

令$h : [-b, -a] \to [\phi(b), \phi(a)], h(x) := \phi(-x)$,
那么$h$是单调递增的可微函数;
因为
\begin{align*}
  h ^\prime(x) = -\phi^\prime(-x)
\end{align*}
由习题11.10.3可知$\phi^\prime(-x)$是黎曼可积的,利用定理11.4.1(b)可知$-\phi^\prime(-x)$是黎曼可积的,
即$h^\prime$是黎曼可积的。

利用命题11.10.7可得
\begin{align*}
  \int_{[-b, -a]} (f \circ h) h^\prime                   & = \int_{[\phi(b), \phi(a)]} f \\
  \int_{[-b, -a]} (f \circ \phi(-x)) -\phi^\prime(-x) dx & = \int_{[\phi(b), \phi(a)]} f \\
  -\int_{[-b, -a]} (f \circ \phi(-x)) \phi^\prime(-x) dx & = \int_{[\phi(b), \phi(a)]} f
\end{align*}
再由习题11.10.3可知
\begin{align*}
  -\int_{[-b, -a]} (f \circ \phi(-x)) \phi^\prime(-x) dx & = \int_{[\phi(b), \phi(a)]} f \\
  -\int_{[a, b]} (f \circ \phi(x)) \phi^\prime(x) dx     & = \int_{[\phi(b), \phi(a)]} f \\
  -\int_{[a, b]} (f \circ \phi) \phi^\prime              & = \int_{[\phi(b), \phi(a)]} f \\
\end{align*}
注意:什么时候会加$dx$书中P239有说明。
据此可得出结论。



\end{document}