\documentclass{article}
\usepackage{mathtools} 
\usepackage{fontspec}
\usepackage[UTF8]{ctex}
\usepackage{amsthm}
\usepackage{mdframed}
\usepackage{xcolor}
\usepackage{amssymb}
\usepackage{amsmath}


% 定义新的带灰色背景的说明环境 zremark
\newmdtheoremenv[
  backgroundcolor=gray!10,
  % 边框与背景一致,边框线会消失
  linecolor=gray!10
]{zremark}{说明}


\begin{document}
\title{11.4 习题}
\author{张志聪}
\maketitle

\section*{11.4.1}

仿照定理11.4.3的证明,做以下说明:

对任意$\epsilon > 0$,由$\int_I f = \underline{\int}_I f$可知,存在一个分段常数函数函数
$\underline{f} : I \to \mathbb{R}$在$I$上从下方控制$f$,并且有
\begin{align*}
  \int_I \underline{f} \geq \int_I f - \epsilon
\end{align*}
类似地,我们能够找到一个分段常数函数$\underline{g}: I \to \mathbb{R}$在$I$上从下方控制$g$,并且有
\begin{align*}
  \int_I \underline{g} \geq \int_I g - \epsilon
\end{align*}
而且我们还能找到分段常数函数$\overline{f}$和$\overline(g)$分别在$I$上从上方控制$f$和$g$,并且有
\begin{align*}
  \int_I \overline{f} \leq \int_I f + \epsilon
\end{align*}
和
\begin{align*}
  \int_I \overline{g} \leq \int_I g + \epsilon
\end{align*}
特别地,如果$h : I \to \mathbb{R}$表示函数
\begin{align*}
  h := (\overline{f} - \underline{f}) + (\overline{g} - \underline{g})
\end{align*}
那么
\begin{align*}
  \int_I h \leq 4\epsilon
\end{align*}

\begin{itemize}
  \item (a)

        由以上说明,可得$\underline{f} + \underline{g}$在$I$上从下方控制$f + g$的分段常数函数,
        而$\overline{f} + \overline{g}$在$I$上从上方控制$f + g$的分段常数函数,所以有
        \begin{align*}
          \int_{I} (\underline{f} + \underline{g}) \leq \underline{\int}_I (f + g) \leq \overline{\int}_I (f + g) \leq \int_I (\overline{f} + \overline{g})
        \end{align*}

        从而
        \begin{align*}
          0 \leq \overline{\int}_I (f + g) - \underline{\int}_I (f + g) \leq \int_I (\overline{f} + \overline{g}) -  (\underline{f} - \underline{g})
        \end{align*}
        于是
        \begin{align*}
          0 \leq \overline{\int}_I (f + g) - \underline{\int}_I (f + g) \leq \int_I h(x)
        \end{align*}

        综上所述,对任意的$\epsilon > 0$,都有
        \begin{align*}
          0 \leq \overline{\int}_I (f + g) - \underline{\int}_I (f + g) \leq 4\epsilon
        \end{align*}

        由于$\overline{\int}_I (f + g) - \underline{\int}_I (f + g)$与$\epsilon$无关(这里表达的是$\epsilon$取任何值,等式都要成立),
        所以
        \begin{align*}
          \overline{\int}_I (f + g) = \underline{\int}_I (f + g)
        \end{align*}

        因此,$f + g$是黎曼可积的。

        因为
        \begin{align*}
          \int_{I} (\underline{f} + \underline{g}) \leq \int_I (f + g) \leq \int_I (\overline{f} + \overline{g})
        \end{align*}
        从而
        \begin{align*}
          \int_{I} \underline{f} + \int_{I}\underline{g} \leq \int_I (f + g) \leq \int_I \overline{f} + \int_I \overline{g}
        \end{align*}
        对左右两端分别取上确界和下确界(其实这也可作为$f+g$黎曼可积的证明),可得
        \begin{align*}
          \int_{I} f + \int_{I} g \leq \int_I (f + g) \leq \int_I f + \int_I g
        \end{align*}
        所以
        \begin{align*}
          \int_I (f + g) = \int_I f + \int_I g
        \end{align*}


  \item (b)

        $c = 0$,任意$x \in I$都有$cf(x) = 0$,于是$cf$是常数函数,
        于是$\int_I cf = p.c.\int_I cf = 0$;

        $c > 0,c < 0$的证明类似,这里以$c > 0$为例。

        因为
        \begin{align*}
          c \int_I \underline{f} \geq c \int_I f - c\epsilon \\
          c \int_I \overline{f} \leq c \int_I f + c\epsilon  \\
        \end{align*}

        特别地,如果$h : I \to \mathbb{R}$表示函数
        \begin{align*}
          h := c\overline{f} - c\underline{f}
        \end{align*}
        那么
        \begin{align*}
          \int_I h \leq 2c\epsilon
        \end{align*}
        因为$c\underline{f}$是在$I$上从下方控制$cf$的分段常数函数,$c\overline{f}$是从$I$上从上方控制$cf$的分段常数函数,
        于是
        \begin{align*}
          \int_I c\overline{f} \leq \underline{\int}_I cf \leq \overline{\int}_I cf \leq \int_I c\underline{f}
        \end{align*}
        从而
        \begin{align*}
          0 \leq \overline{\int}_I cf - \underline{\int}_I cf \leq \int_I (c\overline{f} - c\underline{f})
        \end{align*}
        于是
        \begin{align*}
          0 \leq \overline{\int}_I cf - \underline{\int}_I cf \leq \int_I h
        \end{align*}
        综上所述,对任意的$\epsilon > 0$,都有
        \begin{align*}
          0 \leq \overline{\int}_I cf - \underline{\int}_I cf \leq 2c\epsilon
        \end{align*}
        由于$\overline{\int}_I cf, \underline{\int}_I cf$与$\epsilon$无关,所以
        \begin{align*}
          \overline{\int}_I cf = \underline{\int}_I cf
        \end{align*}
        因此,$cf$是黎曼可积的。


        \begin{align*}
          \int_I cf = c\int_I f
        \end{align*}
        的证明与(a)中的类似,这里不再赘述。

  \item (c)

        利用(a)(b)可得
        \begin{align*}
          \int_I (f - g) & = \int_I (f + (- g))      \\
                         & = \int_I f + \int_I (- g) \\
                         & = \int_I f - \int_I g
        \end{align*}

  \item (d)

        定义$\underline{f}: I \to \mathbb{R}$为$\underline{f}(x) = 0$,于是$\underline{f}$在$I$上从下方控制$f$的分段常数函数。
        从而
        \begin{align*}
          \int_I \underline{f} \leq \int_I f
        \end{align*}
        因为
        \begin{align*}
          \int_I \underline{f} = 0
        \end{align*}
        于是
        \begin{align*}
          0 \leq \int_I f
        \end{align*}

  \item (e)

        因为$f(x) \geq g(x)$,所以$f(x) - g(x) \geq 0$,
        利用(d)可得
        \begin{align*}
          \int_I (f - g) \geq 0
        \end{align*}
        利用(c)可得
        \begin{align*}
          \int_I f - \int_I g \geq 0
        \end{align*}
        从而
        \begin{align*}
          \int_I f \geq \int_I g
        \end{align*}

  \item (f)

        因为$f$是常数函数,由引理11.3.7可知
        \begin{align*}
          \int_I f & = p.c.\int_I f
        \end{align*}
        由定理11.2.16(f)可知
        \begin{align*}
          \int_I f & = p.c.\int_I f \\
                   & = c|I|
        \end{align*}

  \item (g)

        定义$\underline{F}, \overline{F}$分别是在$J$上从下方控制和从上方控制的分段函数,如下
        \begin{align*}
          \underline{F} (x) =
          \begin{cases*}
            \underline{f}(x),  x \in I    \\
            0,                 x \notin I \\
          \end{cases*}
          ,\overline{F}(x) =
          \begin{cases*}
            \overline{f}(x),  x \in I     \\
            0,                 x \notin I \\
          \end{cases*}
        \end{align*}
        从而
        \begin{align*}
          \int_I \underline{F} \leq \underline{\int}_I F \leq \overline{\int}_I F \leq \int_I \overline{F}
        \end{align*}
        利用11.2.16(g)可得
        \begin{align*}
          \int_I \underline{f} \leq \underline{\int}_I F \leq \overline{\int}_I F \leq \int_I \overline{f}
        \end{align*}
        对左右两端分别取上确界和下确界,可得
        \begin{align*}
          \underline{\int}_I f \leq \underline{\int}_I F \leq \overline{\int}_I F \leq \overline{\int}_I f
        \end{align*}
        因为$f$是黎曼可积的,所以$\int_I f = \underline{\int}_I f = \overline{\int}_I f$,
        所以,
        \begin{align*}
          \int_I f \leq \underline{\int}_I F \leq \overline{\int}_I F \leq \int_I f
        \end{align*}

        综上所述,
        $F$是黎曼可积的,且
        \begin{align*}
          \int_I F = \int_I f
        \end{align*}

  \item (h)

        由定理11.2.16可得
        \begin{align*}
          \int_I \overline{f} = \int_J \overline{f}|_J + \int_K \overline{f}|_K    \\
          \int_I \underline{f} = \int_J \underline{f}|_J + \int_K \underline{f}|_K \\
        \end{align*}

        两者相减
        \begin{align*}
          \int_I \overline{f} - \int_I \underline{f} & = \int_J \overline{f}|_J - \int_J \underline{f}|_J + \int_K \overline{f}|_K - \int_K \underline{f}|_K \\
                                                     & \leq 4\epsilon
        \end{align*}
        因为任意$x \in I$都有$\overline{f}(x) \geq \underline{f}(x)$,所以
        \begin{equation*}
          \begin{cases*}
            \int_J \overline{f}|_J \geq \int_J \underline{f}|_J \\
            \int_K \overline{f}|_K \geq \int_K \underline{f}|_K
          \end{cases*}
        \end{equation*}
        从而
        \begin{align*}
          \int_J \overline{f}|_J - \int_J \underline{f}|_J \leq 4\epsilon \\
          \int_K \overline{f}|_K - \int_K \underline{f}|_K \leq 4\epsilon
        \end{align*}

        而我们有
        \begin{align*}
          \int_J \underline{f} \leq \underline{\int}_J f|_J \leq \overline{\int}_J f|_J \leq \int_J \overline{f}|_J \\
          \int_K \underline{f} \leq \underline{\int}_K f|_K \leq \overline{\int}_J f|_K \leq \int_K \overline{f}|_K
        \end{align*}
        于是可得
        \begin{align*}
          0 \leq \overline{\int}_J f|_J - \underline{\int}_J f|_J \leq \int_J \overline{f}|_J - \int_J \underline{f}|_J \leq 4\epsilon \\
          0 \leq \overline{\int}_K f|_J - \underline{\int}_K f|_K \leq \int_K \overline{f}|_K - \int_K \underline{f}|_K \leq 4\epsilon \\
        \end{align*}
        综上所述,对任意的$\epsilon > 0$,都有
        \begin{align*}
          0 \leq \overline{\int}_J f|_J - \underline{\int}_J f|_J \leq 4\epsilon \\
          0 \leq \overline{\int}_K f|_J - \underline{\int}_K f|_K \leq 4\epsilon \\
        \end{align*}
        由于$\overline{\int}_J f|_J, \underline{\int}_J f|_J,\overline{\int}_K f|_K, \underline{\int}_K f|_K$与$\epsilon$无关,所以
        \begin{align*}
          \overline{\int}_J f|_J = \underline{\int}_J f|_J \\
          \overline{\int}_K f|_K = \underline{\int}_K f|_K \\
        \end{align*}

        于是$f|_J, f|_K$分别在$J$和$K$上黎曼可积。

        定义
        \begin{equation*}
          F(x) :=
          \begin{cases*}
            f|_J(x) & $x \in J$             \\
            0       & $x \in I \setminus J$
          \end{cases*}
          ,G(x) :=
          \begin{cases*}
            f|_K(x) & $x \in K$             \\
            0       & $x \in I \setminus K$
          \end{cases*}
        \end{equation*}

        于是$f = F + G$,利用(a)(g)可得,
        \begin{align*}
          \int_I f = \int_J f|_J + \int_K f|_K
        \end{align*}
\end{itemize}

\section*{11.4.2}

反证法,假设存在$x_0 \in [a, b]$使得$f(x_0) > 0$。

因为$f: [a, b] \to \mathbb{R}$是一个连续函数,由定义9.4.1可知
\begin{align*}
  \lim\limits_{x \to x_0; x \in [a, b]} f(x)  = f(x_0)
\end{align*}
从而由定义9.3.6可知,特别地$\epsilon = \frac{1}{2}f(x_0) > 0 $存在一个$\delta > 0$使得
\begin{align*}
   & |f(x) - f(x_0)| \leq \frac{1}{2}f(x_0)             \\
   & \implies                                           \\
   & \frac{1}{2}f(x_0) \leq f(x) \leq \frac{3}{2}f(x_0) \\
\end{align*}
对所有满足$|x - x_0| \leq \delta$的$x \in [a, b]$均成立。

我们定义函数$g: [a, b] \to \mathbb{R}$如下
\begin{equation*}
  g(x) =
  \begin{cases*}
    \frac{1}{2}f(x_0), x \in [a,b] \cap (x_0 - \delta, x_0 + \delta) \\
    0, x \notin [a,b] \cap (x_0 - \delta, x_0 + \delta)
  \end{cases*}
\end{equation*}
上面构造的$g$是在$[a, b]$上从下方控制$f$的函数,因为
\begin{align*}
  \int_{[a, b]} g & = p.c.\int_{[a, b]} f         \\
                  & \geq \frac{1}{2}f(x_0) \delta \\
                  & = \frac{1}{2}f(x_0) \delta    \\
                  & > 0                           \\
\end{align*}
注意,这里没有乘以$2\delta$是因为$x_0$有可能在左右端点上。

因为
\begin{align*}
  0 < \int_{[a, b]} g \leq \int_{[a, b]} f
\end{align*}
于是
\begin{align*}
  0 < \int_{[a, b]} f
\end{align*}
与题设矛盾。

\section*{11.4.3}
对$P$的基数$n$进行归纳。

归纳基始$n = 1$,即$P = \{I\}$,于是
\begin{align*}
  \int_I f & = \sum\limits_{J \in P} \int_J f \\
           & = \int_I f
\end{align*}

归纳假设$n = k$时,命题成立。

$n = k+1$,取$P$中取出一个区间$K$,保证$I - K$是有界区间。

\begin{zremark}
  这样取的原因是要保证定义11.1.10的前置条件成立,习题11.1.3保证$(a,b),[a,b)$类型的可以取出符合条件的区间;
  如果是$(a, b], [a, b]$类型的,直接取包含$b$的区间即可。
\end{zremark}

因为$\{I - K, K\}$是$I$的一个划分,于是利用定理11.4.1(h)可知
\begin{align*}
  \int_I f = \int_{I - K} f|_{I - K} + \int_K f|_K
\end{align*}
因为$P^\prime :=P - K$是$I - K$的划分,由归纳假设可知
\begin{align*}
  \int_{I - K} f|_{I - K} & = \sum\limits_{J \in P^\prime} \int_J f
\end{align*}
从而
\begin{align*}
  \int_I f & = \int_{I - K} f|_{I - K} + \int_K f|_K \\
           & = \sum\limits_{J \in P^\prime} \int_J f + \int_K f|_K \\
           & = \sum\limits_{J \in P} \int_J f
\end{align*}

\section*{11.4.4}

对定理11.4.3
\begin{align*}
  min(f, g)(x) = -max(-f, -g)(x)
\end{align*}
然后利用定理11.4.1(b)可证。

对定义11.4.5

\begin{align*}
  f_{+}g_{-} = -f_{+}(-g)_{-} \\
  f_{-}g_{+} = -(-f)_{-}g_{+} \\
  f_{-}g_{-} = (-f)_{-}(-g)_{-}
\end{align*}
然后利用定理11.4.1可证。


\end{document}