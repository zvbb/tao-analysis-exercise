\documentclass{article}
\usepackage{mathtools} 
\usepackage{fontspec}
\usepackage[UTF8]{ctex}
\usepackage{amsthm}
\usepackage{mdframed}
\usepackage{xcolor}
\usepackage{amssymb}
\usepackage{amsmath}


% 定义新的带灰色背景的说明环境 zremark
\newmdtheoremenv[
  backgroundcolor=gray!10,
  % 边框与背景一致,边框线会消失
  linecolor=gray!10
]{zremark}{说明}


\begin{document}
\title{11.4 习题}
\author{张志聪}
\maketitle

\section*{11.4.1}

仿照定理11.4.3的证明,做以下说明:

对任意$\epsilon > 0$,由$\int_I f = \underline{\int}_I f$可知,存在一个分段常数函数函数
$\underline{f} : I \to \mathbb{R}$在$I$上从下方控制$f$,并且有
\begin{align*}
  \int_I \underline{f} \geq \int_I f - \epsilon
\end{align*}
类似地,我们能够找到一个分段常数函数$\underline{g}: I \to \mathbb{R}$在$I$上从下方控制$g$,并且有
\begin{align*}
  \int_I \underline{g} \geq \int_I g - \epsilon
\end{align*}
而且我们还能找到分段常数函数$\overline{f}$和$\overline(g)$分别在$I$上从上方控制$f$和$g$,并且有
\begin{align*}
  \int_I \overline{f} \leq \int_I f + \epsilon
\end{align*}
和
\begin{align*}
  \int_I \overline{g} \leq \int_I g + \epsilon
\end{align*}

\begin{itemize}
  \item (a)

        由以上说明,可得$\underline{f} + \underline{g}$在$I$上从下方控制$f + g$,
        而$\overline{f} + \overline{g}$在$I$上从上方控制$f + g$,所以有
        \begin{align*}
          \int_{I} \underline{f} + \underline{g} \leq \underline{\int}_I f + g \leq \overline{\int}_I f + g \leq \int_I \overline{f} + \overline{g} 
        \end{align*}

        从而
        \begin{align*}
          0 \leq \overline{\int}_I f + g - \underline{\int}_I f - g \leq \int_I \overline{f} + \overline{g} - \int_I \underline{f} - \underline{g}
        \end{align*}

  \item (b)
  \item (c)
\end{itemize}

\end{document}