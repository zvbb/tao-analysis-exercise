\documentclass{article}
\usepackage{mathtools} 
\usepackage{fontspec}
\usepackage[UTF8]{ctex}
\usepackage{amsthm}
\usepackage{mdframed}
\usepackage{xcolor}
\usepackage{amssymb}
\usepackage{amsmath}


% 定义新的带灰色背景的说明环境 zremark
\newmdtheoremenv[
  backgroundcolor=gray!10,
  % 边框与背景一致,边框线会消失
  linecolor=gray!10
]{zremark}{说明}


\begin{document}
\title{11.10 注释}
\author{张志聪}
\maketitle

\section*{1}

\begin{zremark}
  通过命题11.10.6,推导出以下命题(同济大学高等数学-定积分的换元法,即黎曼积分的换元法):

  假设函数$f(x)$在区间$[a, b]$上连续,函数$x = \varphi(t)$满足条件:
  \begin{itemize}
    \item (1)$\varphi(\alpha) = a, \varphi(\beta) = b$;
    \item (2)$\varphi(t)$在$[\alpha, \beta]$(或$[\beta, \alpha]$)上具有连续导数,
          且其值域$R_{\varphi} = [a, b]$,则有
          \begin{align*}
            \int_{[a, b]} f(x) dx = \int_{[\alpha, \beta]} f[\varphi(t)]\varphi^\prime(t) dt
          \end{align*}
  \end{itemize}
\end{zremark}

这里对$\varphi$的前置条件是不全的,以下条件是必须的:
\begin{itemize}
  \item (1)$\varphi$是可导的;
  \item (2)$\varphi$是单调的;
\end{itemize}

\textbf{证明:}

以$\varphi$单调递增为例,
$\varphi = [\alpha, \beta] \to [a, b]$,
又因为$f$在$[a, b]$上是连续的,所以$f$是$[a, b]$上黎曼可积的函数,
利用命题11.10.6可知,
\begin{align*}
  \int_{[a, b]} f
   & = \int_{[\alpha, \beta]} f \circ \varphi d\varphi \\
   & = \int_{[\alpha, \beta]} f[\varphi(t)]\varphi^\prime(t) dt
\end{align*}



\end{document}