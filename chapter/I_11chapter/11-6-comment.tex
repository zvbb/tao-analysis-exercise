\documentclass{article}
\usepackage{mathtools} 
\usepackage{fontspec}
\usepackage[UTF8]{ctex}
\usepackage{amsthm}
\usepackage{mdframed}
\usepackage{xcolor}
\usepackage{amssymb}
\usepackage{amsmath}


% 定义新的带灰色背景的说明环境 zremark
\newmdtheoremenv[
  backgroundcolor=gray!10,
  % 边框与背景一致,边框线会消失
  linecolor=gray!10
]{zremark}{说明}


\begin{document}
\title{11.6 注释}
\author{张志聪}
\maketitle

\begin{zremark}
  命题11.6.1 的证明存在问题,文中“根据嵌套级数(引理7.2.15)可得”是错误的。
\end{zremark}

错误的原因在于引理7.2.15说的是无限级数的性质,由划分的定义(定义11.1.10)
可知$N$是自然数,所以是有限级数。

但文中的结论是正确的,不用引理7.2.15可以直接得到结论。仔细级数的部分和就能得到结论。


\end{document}