\documentclass{article}
\usepackage{mathtools} 
\usepackage{fontspec}
\usepackage[UTF8]{ctex}
\usepackage{amsthm}
\usepackage{mdframed}
\usepackage{xcolor}
\usepackage{amssymb}
\usepackage{amsmath}


% 定义新的带灰色背景的说明环境 zremark
\newmdtheoremenv[
  backgroundcolor=gray!10,
  % 边框与背景一致,边框线会消失
  linecolor=gray!10
]{zremark}{说明}


\begin{document}
\title{11.8 习题}
\author{张志聪}
\maketitle

\section*{11.8.1}

我们通过对$n$进行归纳来证明。更准确地说,设$P(n)$具有如下性质:
如果$I$是有界区间且$P$是$I$的一个基数为$n$的划分。那么
\begin{align*}
  \alpha[I] = \sum\limits_{J \in P} \alpha[J]
\end{align*}

最基本的情况$P(0)$是平凡的,$I$能够被分割成一个空划分的唯一可能是$I$本身就是空集,
在这种情况下,容易得到结论。情形$P(1)$也非常容易,$I$能够被划分成一个单元素集合$\{J\}$的唯一可能
就是$I = J$,此时同样可以容易地得到结论。

现在归纳地假设存在某个$n \geq 1$使得$P(n)$为真,接下来我们证明$P(n + 1)$为真。
设$I$是一个有界区间,并且设$P$是$I$的一个基数为$n + 1$的划分。

如果$I$是空集或者单点集,那么$P$中的所有区间也一定是空集或者单点集,
从而每个区间的$\alpha$长度都为零,此时结论是平凡的。因此,我们假设$I$是形如
$(a, b), (a, b], [a, b)$或$[a, b]$的区间。

        首先假设$b \in I$,即$I$是$(a, b]$或$[a, b]$。由$b \in I$可知,
$P$中存在一个区间$K$包含$b$。由于$K$是包含在$I$中的,所以$K$一定是形如
$(c,b], [c, b]$或$\{b\}$的区间,其中$c$是满足$a \leq c \leq b$的实数(当$K = \{b\}$时,令$c := b$)。
特别地,这意味着当$c > a$时,集合$I - K$是形如$[a, c]$、$(a, c)$、$(a, c]$或$[a, c)$的区间;
而当$c = a$时,$I - K$就是单点集或者空集。无论是哪种情形,容易得出
\begin{equation*}
  \begin{cases*}
    \alpha[I] = \alpha[b] - \alpha[a] \\
    \alpha[K] = \alpha[b] - \alpha[c] \\
    \alpha[I - K] = \alpha[c] - \alpha[a]
  \end{cases*}
\end{equation*}
于是
\begin{align*}
  \alpha[I] = \alpha[K] + \alpha[I - K]
\end{align*}
另外,因为$P$构成了$I$的一个划分,所以$P - \{K\}$构成了$I - K$的一个划分。根据归纳假设可知,
\begin{align*}
  \alpha[I - K] = \sum\limits_{J \in P - \{K\}} \alpha[J]
\end{align*}
结合这两个等式(利用有限集合的加法定律,参见命题7.1.11)可得,
\begin{align*}
  \alpha[I] = \sum\limits_{J \in P} \alpha[J]
\end{align*}
结论得证。

现在假设$b \notin I$,即$I$是$(a, b)$或$[a, b)$,同样存在一个形如$(c, b)$或$[c, b)$的区间$K$(参见习题11.1.3)。
        特别地,这意味着当$c > a$时,集合$I - K$是形如$[a, c]$、$(a, c)$、$(a, c]$或$[a, c)$的区间;
        而当$a = c$时,$I - K$就是单点集或者空集。接下来的论证与上文一样。

        \section*{11.8.2}
        (1)叙述

        分段常值黎曼-斯蒂尔杰斯积分是独立于划分的:设$I$是一个有界区间,并且设$f: I \to \mathbb{R}$是一个函数。
        如果$P$和$P^\prime$都是$I$的划分,并且$f$关于$P$和$P^\prime$都是分段常数函数,
      $\alpha: X \to \mathbb{R}$是定义在某个包含$I$的区域$X$上的函数,那么
      $p.c.\int_{[P]} f d\alpha = p.c.\int_{[P^\prime]} f d\alpha$

        (2)证明

        参考习题11.2.3的证明。

        令$Q := P \# P^\prime$,由引理11.2.7可知,$f$是关于$Q$上的分段常数函数。

        接下来证明:
        \begin{align}
          p.c.\int_{[P]} f d\alpha = p.c.\int_{[Q]}f d\alpha \\
          p.c.\int_{[P^\prime]} f d\alpha = p.c.\int_{[Q]}f d\alpha
        \end{align}

        对任意$K \in P$,定义
        \begin{align*}
          Q_K := \{X \in Q : X \subseteq K\}
        \end{align*}
        证明$K = Q_K$。反证法,假设$K \neq Q_K$。由$Q_K$的构造方式,易知$Q_K \subseteq K$,
        如果假设成立,那么,存在$x \in K$,$x \notin Q_K$。
        因为$Q$中一定存在$J$使得$x \in J$,由$Q$比$P$更细,可知存在$W \in P$使得$J \subseteq W$,
        由划分的定义可知$W = K$,因为如果$W \neq K$,则与定义11.1.10(划分)中每个元素恰好
        属于$P$中的一个有界区间矛盾,存在了两个区间都包含$x$。于是可知$J \in Q_K$,与假设矛盾。

        由$K = Q_K$可知,$p.c.\int_{[K]} f d\alpha= p.c.\int_{[Q_K]}f d\alpha$,由$K$的任意性,可知(1)式成立。

        类似地,可证(2)式成立。命题得证。

        \section*{11.8.3}

        可以参考11.2.16的证明;P98中关于\textbf{元证明}的说明可能帮助理解。证明略

        \section*{11.8.4}
        (1)叙述

        设$I$是一个有界区间,并且设$f$是定义在$I$上的一致连续函数,
      $\alpha: X \to \mathbb{R}$是定义在某个包含$I$的区域$X$上的单调递增函数,那么$f$在$I$上关于$\alpha$是黎曼-斯蒂尔杰斯可积的。

        (2)证明

        参考定理11.5.1的证明。

        根据命题9.9.15可知,$f$是有界的。那么我们要证明$\underline{\int}_{I} f d\alpha = \overline{\int}_{I} f d\alpha$。

        如果$I$是一个单点集或者空集,那么命题是平凡的。设$I$是四个区间$[a,b]$、$(a,b)$、$(a, b]$和$[a, b)$中的任意一个,其中$a < b$是实数。

设$\epsilon > 0$是任意的,由一致连续性可知,存在一个$\delta > 0$使得只要$x, y \in I$满足$|x - y| < \delta$,就有$|f(x) - f(y)| < \epsilon$。
根据阿基米德性质可知,存在一个整数$N > 0$使得$(b - a)/N < \delta$。

注意,我们可以把$I$划分成$N$个区间$J_1,...,J_N$,设该划分为$P$,其中每个区间长度都是$(b-a)/N$。于是可得,
\begin{align*}
  \overline{\int}_{I} f d\alpha \leq \sum\limits_{k=1}^{N}\left(\sup\limits_{x \in J_k}f(x) \alpha[J_k] \right)
\end{align*}
注意,以上无需类似11.3.9(黎曼和)的定义,可以看做函数$g: I \to \mathbb{R}$对任意$J_k \in P, x \in J_k$都有
$g(x) = \sup\limits_{x \in J_k}f(x)$,于是$g$是定义在$I$上的分段常数函数,并且它从上方控制$f$。

又
\begin{align*}
  \underline{\int}_{I} f d\alpha \geq \sum\limits_{k=1}^{N}\left(\inf\limits_{x \in J_k}f(x) \alpha[J_k] \right)
\end{align*}

特别地,
\begin{align*}
  \overline{\int}_{I} f d\alpha - \underline{\int}_{I} f d\alpha \leq \sum\limits_{k=1}^{N}\left(\sup\limits_{x \in J_k}f(x) - \inf\limits_{x \in J_k}f(x) \right) \alpha[J_k]
\end{align*}
但是,因为$|J_k| = (b - a) / N < \delta$,所以对所有的$x, y \in J_k$均有$|f(x) - f(y)| < \epsilon$。\\
特别地,我们有
\begin{align*}
  f(x) < f(y) + \epsilon \text{对所有的} x,y \in J_k \text{均成立}
\end{align*}
对$f(x)$取上确界可得(应该是对$f(x)$取上确界,而不是书中的对$x$取上确界),
\begin{align*}
  \sup\limits_{x \in J_k}f(x) \leq f(y) + \epsilon \text{对所有的} y \in J_k \text{均成立}
\end{align*}
然后对$f(y)$取下确界可得,
\begin{align*}
  \sup\limits_{x \in J_k}f(x) \leq \inf\limits_{x \in J_k} f(y) + \epsilon
\end{align*}
把这个式子代入到前面的不等式可得,
\begin{align*}
  \overline{\int}_{I} f d\alpha - \underline{\int}_{I} f d\alpha & \leq \sum\limits_{k=1}^{N} \epsilon \alpha[J_k] \\
                                                                 & = \epsilon \sum\limits_{k=1}^{N} \alpha[J_k]    \\
                                                                 & = \epsilon \alpha[I]
\end{align*}

又因为$\epsilon > 0$是任意的,而$\alpha[I]$是定值。所以$\overline{\int}_{I} f d\alpha - \underline{\int}_{I} f d\alpha$
不可能是正的。命题得证。

\section*{11.8.5}

由定理9.9.16可知,$f$在$[-1, 1]$上是一致连续的。由习题11.8.4可知,
\begin{align*}
  \int_{[-1, 1]} f dsgn
\end{align*}
是黎曼-斯蒂尔杰斯可积的。

由引理9.6.3可知,$f$是有界的,那么,存在$M > 0$使得
\begin{align*}
  |f(x)| \leq M, \text{对任意}x \in [-1, 1]
\end{align*}

对任意$\epsilon > 0$,因为$f$在$[-1, 1]$上是连续函数,所以存在$\delta > 0$使得
\begin{align*}
  |f(x) - f(0)| < \epsilon, \text{对任意}x \in (-\delta, \delta)
\end{align*}

接下来对$[-1,1]$进行如下划分$P := \{[-1, -\delta], (-\delta, \delta), [\delta, 1]\}$,
并定义函数$g: [-1, 1] \to \mathbb{R}$如下
\begin{equation*}
  g(x) =
  \begin{cases*}
    f(0) + \epsilon, x \in (-\delta, \delta) \\
    M, x \in [-1, -\delta] \cup [\delta, 1]
  \end{cases*}
\end{equation*}
因为$g$是从上方控制$f$的,于是
\begin{align*}
  \overline{\int}_{[-1, 1]} f dsgn & \leq p.c.\int_{[P]} g dsgn                                                                  \\
                                   & = \int_{[-1, -\delta]} g dsgn + \int_{[\delta, 1]} g dsgn + \int_{(-\delta, \delta)} g dsgn \\
                                   & = M sgn[[-1, -\delta]] + M sgn[[\delta, 1]] + (f(0) + \epsilon)sgn[(-\delta, \delta)]       \\
                                   & = M (-1 - (-1)) + M (1 - 1) + (f(0) + \epsilon) (1 - (-1))                                  \\
                                   & = 2f(0) + 2\epsilon
\end{align*}
类似地,可得
\begin{align*}
  \underline{\int}_{[-1, 1]} f dsgn \geq 2f(0) - 2\epsilon
\end{align*}
于是我们有
\begin{align*}
  2f(0) - 2\epsilon \leq \underline{\int}_{[-1, 1]} f dsgn \leq \overline{\int}_{[-1, 1]} f dsgn \leq 2f(0) + 2\epsilon
\end{align*}
因为$\epsilon$是任意的,所以$\int_{[-1, 1]} f dsgn = 2f(0)$

\end{document}