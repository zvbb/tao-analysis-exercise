\documentclass{article}
\usepackage{mathtools} 
\usepackage{fontspec}
\usepackage[UTF8]{ctex}
\usepackage{amsthm}
\usepackage{mdframed}
\usepackage{xcolor}
\usepackage{amssymb}
\usepackage{amsmath}


% 定义新的带灰色背景的说明环境 zremark
\newmdtheoremenv[
  backgroundcolor=gray!10,
  % 边框与背景一致,边框线会消失
  linecolor=gray!10
]{zremark}{说明}


\begin{document}
\title{11.5 习题}
\author{张志聪}
\maketitle

\section*{11.5.1}

因为$f: I \to \mathbb{R}$既是分段连续的,由定义11.5.4可知,
存在一个$I$的划分$P$,使得对所有的$J \in P$,$f|_J$都是$J$上的连续函数。
又因为$f$在$I$上是有界的,由命题11.5.3可知,
任意$J \in P$,$f|_J$在$J$上是黎曼可积的。

设$P$的基数为$n$,对任意$\epsilon/n > 0$,对每一个$J \in P$,我们能找到一个分段常数函数$h_J: J \to \mathbb{R}$在
$J$上从上方控制$f$,并且有
\begin{align*}
  \int_J h_J \leq \int_J f + \epsilon/n.
\end{align*}
定义函数$h: I \to \mathbb{R}$,对$x \in J, J \in P$为$h(x) = h_J(x)$。于是$h$在$I$上从上方控制$f$的分段常数函数。\\
从而
\begin{align*}
  \overline{\int}_I f & \leq \int_I h
\end{align*}
由习题11.4.3可知(分段常值积分是特例)
\begin{align*}
  \int_I h = \sum\limits_{J \in P} \int_J h_J
\end{align*}
于是
\begin{align*}
  \overline{\int}_I f & \leq \int_I h                                  \\
                      & = \sum\limits_{J \in P} \int_J h_J             \\
                      & \leq \sum\limits_{J \in P} \int_J f + \epsilon
\end{align*}
同理可得
\begin{align*}
  \underline{\int}_I f \geq \sum\limits_{J \in P} \int_J f - \epsilon.
\end{align*}
于是可得
\begin{align*}
  0 \leq \overline{\int}_I - \underline{\int}_I \leq 2\epsilon.
\end{align*}
但$\epsilon$是任意的,所以$f$是黎曼可积的。

\end{document}