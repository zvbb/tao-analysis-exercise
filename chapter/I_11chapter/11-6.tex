\documentclass{article}
\usepackage{mathtools} 
\usepackage{fontspec}
\usepackage[UTF8]{ctex}
\usepackage{amsthm}
\usepackage{mdframed}
\usepackage{xcolor}
\usepackage{amssymb}
\usepackage{amsmath}


% 定义新的带灰色背景的说明环境 zremark
\newmdtheoremenv[
  backgroundcolor=gray!10,
  % 边框与背景一致,边框线会消失
  linecolor=gray!10
]{zremark}{说明}


\begin{document}
\title{11.6 习题}
\author{张志聪}
\maketitle

\section*{11.6.1}

证明框架参考了命题11.5.3的证明。

如果$I$是一个单点集或者空集,那么结论是平凡的。如果$I$是一个闭区间,那么根据命题11.6.1可以得到结论。
于是我们假设$I$是形如$(a, b], (a, b)$或$[a, b)$的区间,其中$a < b$。

设$M$是$f$的界,所以对所有的$x \in I$均有$-M \leq f(x) \leq M$。现在设$0 < \epsilon < (b - a)/2$是一个很小的数。
当$f$被限制在区间$[a + \epsilon, b - \epsilon]$上时,它就是单调有界的,从而再次利用11.6.1可知,它是黎曼可积的。
特别地,我们能够找到一个分段常数函数$h: [a + \epsilon, b - \epsilon]$上从上方控制$f$,并且有
\begin{align*}
  \int_{[a + \epsilon, b - \epsilon]} h \leq \int_{[a + \epsilon, b - \epsilon]} f + \epsilon
\end{align*}
定义$\widetilde{h} : I \to \mathbb{R}$为
\begin{equation*}
  \widetilde{h}(x) =
  \begin{cases*}
    h(x), x \in [a + \epsilon, b - \epsilon] \\
    M, x \in I \setminus [a + \epsilon, b - \epsilon]
  \end{cases*}
\end{equation*}
$\widetilde{h}$显然是$I$上从上方控制$f$的分段常数函数。根据定理11.2.16可知,
\begin{align*}
  \int_{I}\widetilde{h}  = \epsilon M + \int_{[a + \epsilon, b - \epsilon]} h + \epsilon M
                         \leq \int_{[a + \epsilon, b - \epsilon]} f + (2M + 1)\epsilon
\end{align*}

特别地
\begin{align*}
  \overline{\int}_I f \leq \int_{[a + \epsilon, b - \epsilon]} f + (2M + 1)\epsilon
\end{align*}

类似地,有
\begin{align*}
  \underline{\int}_I f \geq \int_{[a + \epsilon, b - \epsilon]} f - (2M + 1)\epsilon
\end{align*}
从而

\begin{align*}
  \overline{\int}_I f - \underline{\int}_I f\leq (4M + 2)\epsilon
\end{align*}
综上由$\epsilon$的任意性且$\overline{\int}_I f - \underline{\int}_I f$与$\epsilon$无关可得,$f$是黎曼可积的。



\end{document}