\documentclass{article}
\usepackage{mathtools} 
\usepackage{fontspec}
\usepackage[UTF8]{ctex}
\usepackage{amsthm}
\usepackage{mdframed}
\usepackage{xcolor}
\usepackage{amssymb}
\usepackage{amsmath}


% 定义新的带灰色背景的说明环境 zremark
\newmdtheoremenv[
  backgroundcolor=gray!10,
  % 边框与背景一致,边框线会消失
  linecolor=gray!10
]{zremark}{说明}


\begin{document}
\title{11.6 习题}
\author{张志聪}
\maketitle

\section*{11.6.1}

证明框架参考了命题11.5.3的证明。

如果$I$是一个单点集或者空集,那么结论是平凡的。如果$I$是一个闭区间,那么根据命题11.6.1可以得到结论。
于是我们假设$I$是形如$(a, b], (a, b)$或$[a, b)$的区间,其中$a < b$。

设$M$是$f$的界,所以对所有的$x \in I$均有$-M \leq f(x) \leq M$。现在设$0 < \epsilon < (b - a)/2$是一个很小的数。
当$f$被限制在区间$[a + \epsilon, b - \epsilon]$上时,它就是单调有界的,从而再次利用11.6.1可知,它是黎曼可积的。
特别地,我们能够找到一个分段常数函数$h: [a + \epsilon, b - \epsilon]$上从上方控制$f$,并且有
\begin{align*}
  \int_{[a + \epsilon, b - \epsilon]} h \leq \int_{[a + \epsilon, b - \epsilon]} f + \epsilon
\end{align*}
定义$\widetilde{h} : I \to \mathbb{R}$为
\begin{equation*}
  \widetilde{h}(x) =
  \begin{cases*}
    h(x), x \in [a + \epsilon, b - \epsilon] \\
    M, x \in I \setminus [a + \epsilon, b - \epsilon]
  \end{cases*}
\end{equation*}
$\widetilde{h}$显然是$I$上从上方控制$f$的分段常数函数。根据定理11.2.16可知,
\begin{align*}
  \int_{I}\widetilde{h}  = \epsilon M + \int_{[a + \epsilon, b - \epsilon]} h + \epsilon M
  \leq \int_{[a + \epsilon, b - \epsilon]} f + (2M + 1)\epsilon
\end{align*}

特别地
\begin{align*}
  \overline{\int}_I f \leq \int_{[a + \epsilon, b - \epsilon]} f + (2M + 1)\epsilon
\end{align*}

类似地,有
\begin{align*}
  \underline{\int}_I f \geq \int_{[a + \epsilon, b - \epsilon]} f - (2M + 1)\epsilon
\end{align*}
从而

\begin{align*}
  \overline{\int}_I f - \underline{\int}_I f\leq (4M + 2)\epsilon
\end{align*}
综上由$\epsilon$的任意性且$\overline{\int}_I f - \underline{\int}_I f$与$\epsilon$无关可得,$f$是黎曼可积的。


\section*{11.6.2}
(1)分段单调函数的定义参考定义11.5.4:

设$I$是有一个有界区间,并设$f: I \to \mathbb{R}$。我们称$f$在$I$上是有界分段单调函数,当且仅当存在一个$I$的划分$P$,
使得对于所有的$J \in P$,$f|_J$都是$J$上的单调有界函数。

(2)由(1)可知存在一个$I$的划分$P$,使得对于所有的$J \in P$,$f|_J$都是$J$上的单调有界函数。
于是对任意$J \in P$,由推论11.6.3可知$f|_J$在$J$上是黎曼可积的。剩余部分的证明与习题11.5.1类似,这里不做赘述。

\section*{11.6.3}

注意这里无法假设$N$是正整数。

\begin{itemize}
  \item $\Rightarrow$

        因为$x \geq 0, f(x) \geq 0$,可知$\int_{[0, N]} f$是关于实数$N$的单调递增函数,
        由定理5.5.9(最小上界的存在性)可知只要证明其有上界,则最小上界存在且有限。

        由推论11.6.3可知$\int_{[0, N]} f$在$[0, N]$上是黎曼可积的。

        由推论5.4.12和命题4.4.1可知,对实数$N$存在一个自然数$n$使得$n \leq N < n + 1$,
        现在把$[0, N]$划分成$n+1$个半开区间
        \begin{align*}
          \{[0, 1), [1, 2),...,[n-1, n), [n, N]\}
        \end{align*}

        由命题11.3.12可知
        \begin{align*}
          \overline{\int}_{[0, N]} f & \leq \sum\limits_{j=0}^{n} \left( \sup\limits_{x \in [j, j+1)} f(x)  \right)
          + \sup\limits_{x \in [n, N]} f(x)                                                                         \\
                                     & \leq \sum\limits_{j=0}^{n + 1} f(j)
        \end{align*}
        以上最后一个等式由$f: [0, +\infty) \to \mathbb{R}$是一个单调递增的函数保证的。

        因为对任意的$x \in [0, +\infty), f(x) \geq 0$,由定理11.4.1(d)可知
        \begin{align*}
          \int_{[0, N]} f  \geq 0
        \end{align*}


        综上,对任意$N > 0$都有
        \begin{align*}
          0 \leq \underline{\int}_{[0, N]} f \leq \overline{\int}_{[0, N]} f \leq \sum\limits_{j=0}^{n + 1} f(j)
        \end{align*}
        即
        \begin{align*}
          0 \leq \int_{[0, N]} f \leq \sum\limits_{j=0}^{n + 1} f(j)
        \end{align*}

        不妨设$\sum\limits_{n = 0}^\infty f(n)$收敛于$L$,于是对任意$N$都有
        \begin{align*}
          0 \leq \int_{[0, N]} f \leq  L
        \end{align*}
        $\int_{[0, N]} f$是有界的,于是命题成立。

  \item $\Leftarrow$

        反证法,假设$\sum \limits_{n=0}^\infty$是发散的。
        那么,对任意实数$M$都存在正整数$N$使得
        \begin{align*}
          \sum \limits_{n=0}^{N} > M
        \end{align*}
        现在把$[0, N]$划分成$N+1$个半开区间
        \begin{align*}
          \{[0, 1), [1, 2),...,[n-1, n), [n, N), \{N\}\}
        \end{align*}

        由命题11.3.12可知
        \begin{align*}
          \underline{\int}_{[0, N]} f & \geq \sum\limits_{j=0}^{N} \left( \inf\limits_{x \in [j, j+1)} f(x)  \right)            \\
                                      & \geq \sum\limits_{j=1}^{N} f(j) = M - f(0)
        \end{align*}
        因为$M$是任取的,而$f(0)$是固定值,于是可得$\underline{\int}_{[0, N]} f$是无限的,
        这与题设矛盾。
\end{itemize}

\section*{11.6.4}

\section*{11.6.5}

由引理5.6.9(d)可知,$x > 0$时,$\frac{1}{x^q}$是非负且递减的,


\end{document}