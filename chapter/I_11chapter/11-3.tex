\documentclass{article}
\usepackage{mathtools} 
\usepackage{fontspec}
\usepackage[UTF8]{ctex}
\usepackage{amsthm}
\usepackage{mdframed}
\usepackage{xcolor}
\usepackage{amssymb}
\usepackage{amsmath}


% 定义新的带灰色背景的说明环境 zremark
\newmdtheoremenv[
  backgroundcolor=gray!10,
  % 边框与背景一致,边框线会消失
  linecolor=gray!10
]{zremark}{说明}


\begin{document}
\title{11.3 习题}
\author{张志聪}
\maketitle

\section*{11.3.1}

\begin{itemize}
  \item[$\circ$]

        $f$从上方控制$g$,于是由定义11.3.1可知,对任意$x \in I$都有$f(x) \geq g(x)$,
        $g$从上方控制$h$,类似的,对任意$x \in I$都有$g(x) \geq h(x)$,
        于是对任意$x \in I$都有$f(x) \geq h(x)$,再次利用定义11.3.1,
        $f$从上方控制$h$

  \item[$\circ$]

        $f$从上方控制$g$,于是由定义11.3.1可知,对任意$x \in I$都有$f(x) \geq g(x)$;
        $g$从上方控制$f$,于是由定义11.3.1可知,对任意$x \in I$都有$g(x) \geq f(x)$;
        综上,对任意$x \in I$,都有,
        \begin{equation*}
          \begin{cases*}
            f(x) \geq g(x) \\
            f(x) \leq g(x) \\
          \end{cases*}
        \end{equation*}
        于是可得$f(x) = g(x)$,由函数相等的定义可知$f=g$。

\end{itemize}

\section*{11.3.2}

\begin{itemize}
  \item $f + h$是否从上方控制$g + h$?

        是;证明略

  \item $f \cdot h$是否从上方控制$g \cdot h$?

        否;反例$f(x) = 1, g(x) = -1, h(x) = -1$,此时,
        \begin{align*}
          f \cdot h & = -1 \\
          g \cdot h & = 1
        \end{align*}
        于是任意$x \in I$都有$(f \cdot h)(x) < (g \cdot h)(x)$,
        $f \cdot h$从上方控制$g \cdot h$不成立。

  \item $cf$是否从上方控制$cg$?

        否;把上面的反例中的$h(x)=-1$看做$c=-1$。

\end{itemize}

\section*{11.3.3}

由定义11.3.1,$f$在$I$上从上方控制$f$,于是
\begin{align*}
  \overline{\int}_{I} f \leq p.c.\int_{I}f
\end{align*}
类似的,$f$在$I$上从下方控制$f$,可得
\begin{align*}
  p.c.\int_{I}f \leq \underline{\int}_I  f
\end{align*}

又由引理11.3.3可知
\begin{align*}
  \underline{\int}_I  f \leq \overline{\int}_{I} f
\end{align*}

于是可得,
\begin{align*}
  p.c.\int_{I}f = \underline{\int}_I  f = \overline{\int}_{I} f
\end{align*}
即:
\begin{align*}
  \int_I  f = p.c.\int_{I}f
\end{align*}

\section*{11.3.4}

由定义11.2.14和定义11.2.9可知
\begin{align*}
  p.c.\int_{I}g & = p.c.\int_{[P]}g               \\
                & = \sum\limits_{J \in P} C_J |J| \\
\end{align*}
其中,对任意的$J \in P$,我们令$C_J$表示$g$在$J$上的常数值。

由定义11.3.9可知
\begin{align*}
  U(f, P) = \sum\limits_{J \in P; J \neq \emptyset} (\sup\limits_{x \in J}f(x)) |J|
\end{align*}

对任意$J \in P$,$g|_J$是常数函数,不妨设为$c_j$,此时任意$x \in J$都有$c_j \geq \sup\limits_{x \in J}f(x)$,
否则与题设$g$是从上方控制$f$的函数矛盾。
所以任意$J \in P, |J| \geq 0$都有
\begin{align*}
  C_J |J| \geq (\sup\limits_{x \in J}f(x)) |J|
\end{align*}

由命题7.1.11(h)可知
\begin{align*}
  \sum\limits_{J \in P} C_J |J| \geq \sum\limits_{J \in P; J \neq \emptyset} (\sup\limits_{x \in J}f(x)) |J|
\end{align*}
即:
\begin{align*}
  p.c.\int_{I}g \geq U(f, P)
\end{align*}

类似地,
\begin{align*}
  p.c.\int_{I}h \leq L(f, P)
\end{align*}

\begin{zremark}
  $I$是空集的话,空虚的成立,无需讨论,定义11.3.9未定义空集的情况。

  划分$P$中可能存在空集的情况,可以去掉空集元素,得到划分$P^\prime$,命题11.2.13保证结论任然成立。
\end{zremark}

\section*{11.3.5}

由引理11.3.11可知,任意$g$是从上方控制$f$的函数,并且$g$是关于$I$的某个划分$P$的分段常量函数,
于是
\begin{align*}
  p.c.\int_{I}g \geq U(f, P)
\end{align*}
于是可得
\begin{align*}
  p.c.\int_{I}g \geq \inf\{U(f,P): P \text{是} I \text{的划分}\}
\end{align*}
又由定义11.3.2可知
\begin{align*}
  p.c.\int_{I}g \geq \overline{\int}_{I} f
\end{align*}

下面证明$\overline{\int}_{I} f  = \inf\{U(f,P): P \text{是} I \text{的划分}\}$。

反证法,假设$ \overline{\int}_{I} f  > \inf\{U(f,P): P \text{是} I \text{的划分}\}$,
于是存在$I$的划分$P_0$使得
\begin{equation}
  \begin{cases}
    \overline{\int}_{I} f  > U(f,P_0)                   \\
    U(f, P_0) > \inf\{U(f,P): P \text{是} I \text{的划分}\} \\
  \end{cases}
\end{equation}
我们可以在划分$P_0$的基础上,定义函数$g_0$是关于$P_0$的分段常量函数如下:
\begin{align*}
  g_0(x) = \sup\limits_{x \in J}f(x)
\end{align*}
其中,$J \in P_0$,于是
\begin{align*}
  p.c.\int_{I}g_0 = p.c.\int_{[P_0]}g_0 = U(f, P_0)
\end{align*}
由(1)式可得
\begin{align*}
  \overline{\int}_{I} f  > p.c.\int_{I}g_0
\end{align*}
因为$g_0$也是满足引理11.3.11前置条件的函数,所以
\begin{align*}
  \overline{\int}_{I} f \leq p.c.\int_{I}g_0
\end{align*}
存在矛盾,假设不成立。

同理可证$\overline{\int}_{I} f  < \inf\{U(f,P): P \text{是} I \text{的划分}\}$不成立。

综上,$\overline{\int}_{I} f  = \inf\{U(f,P): P \text{是} I \text{的划分}\}$

类似的,可证$\underline{\int}_I  f = \sup\{L(f,P): P \text{是} I \text{的划分}\}$

\end{document}