\documentclass{article}
\usepackage{mathtools} 
\usepackage{fontspec}
\usepackage[UTF8]{ctex}
\usepackage{amsthm}
\usepackage{mdframed}
\usepackage{xcolor}
\usepackage{amssymb}
\usepackage{amsmath}


% 定义新的带灰色背景的说明环境 zremark
\newmdtheoremenv[
  backgroundcolor=gray!10,
  % 边框与背景一致,边框线会消失
  linecolor=gray!10
]{zremark}{说明}


\begin{document}
\title{11.3 习题}
\author{张志聪}
\maketitle

\section*{11.3.1}

\begin{itemize}
  \item[$\circ$]

        $f$从上方控制$g$,于是由定义11.3.1可知,对任意$x \in I$都有$f(x) \geq g(x)$,
        $g$从上方控制$h$,类似的,对任意$x \in I$都有$g(x) \geq h(x)$,
        于是对任意$x \in I$都有$f(x) \geq h(x)$,再次利用定义11.3.1,
        $f$从上方控制$h$

  \item[$\circ$]

        $f$从上方控制$g$,于是由定义11.3.1可知,对任意$x \in I$都有$f(x) \geq g(x)$;
        $g$从上方控制$f$,于是由定义11.3.1可知,对任意$x \in I$都有$g(x) \geq f(x)$;
        综上,对任意$x \in I$,都有,
        \begin{equation*}
          \begin{cases*}
            f(x) \geq g(x) \\
            f(x) \leq g(x) \\
          \end{cases*}
        \end{equation*}
        于是可得$f(x) = g(x)$,由函数相等的定义可知$f=g$。

\end{itemize}

\section*{11.3.2}

\begin{itemize}
  \item $f + h$是否从上方控制$g + h$?

        是;证明略

  \item $f \cdot h$是否从上方控制$g \cdot h$?

        否;反例$f(x) = 1, g(x) = -1, h(x) = -1$,此时,
        \begin{align*}
          f \cdot h & = -1 \\
          g \cdot h & = 1
        \end{align*}
        于是任意$x \in I$都有$(f \cdot h)(x) < (g \cdot h)(x)$,
        $f \cdot h$从上方控制$g \cdot h$不成立。

  \item $cf$是否从上方控制$cg$?

        否;把上面的反例中的$h(x)=-1$看做$c=-1$。

\end{itemize}

\section*{11.3.3}


\end{document}