\documentclass{article}
\usepackage{mathtools} 
\usepackage{fontspec}
\usepackage[UTF8]{ctex}
\usepackage{amsthm}
\usepackage{mdframed}
\usepackage{xcolor}
\usepackage{amssymb}
\usepackage{amsmath}


% 定义新的带灰色背景的说明环境 zremark
\newmdtheoremenv[
  backgroundcolor=gray!10,
  % 边框与背景一致,边框线会消失
  linecolor=gray!10
]{zremark}{说明}


\begin{document}
\title{11.9 习题}
\author{张志聪}
\maketitle

\section*{11.9.1}

由命题11.6.1可知,单调函数$f$在$[0, 1]$上是黎曼可积的。于是由定理11.9.1可知,
$F$函数在$[0, 1]$上是连续的。有一点需要注意,虽然$f$在有理数点不连续,但不能以此来推断$F$函数在有理数点处不可微。

\begin{itemize}
  \item $q \in (0, 1)$

        $q$同时是$[0, 1] \cap (q, +\infty)$和$[0, 1] \cap (-\infty, q)$的附着点,$\frac{F(x) - F(q)}{x - q}$在$q$处可微分(即存在极限)
        ,当且仅当左右极限存在且相等。我们按照这个框架来证明。(书中P189处有说明)

        由习题9.8.5可知,存在一个某个自然数$n$使得$q = q(n)$。

        当$x > q$时,由习题9.8.5(b)可知
        \begin{align*}
          \frac{F(x) - F(q)}{x - q} & = \frac{\int_{[q, x]} f}{x - q}                  \\
                                    & \geq \frac{\left(f(q) + 2^{-n}\right)(x-q)}{x-q} \\
                                    & = f(q) + 2^{-n}
        \end{align*}
        当$x < q$时,类似地
        \begin{align*}
          \frac{F(x) - F(q)}{x - q} \leq f(q)
        \end{align*}
        于是
        \begin{align*}
          \lim\limits_{x \to q; x \in (q, 1]} & \geq f(q) + 2^{-n} \\
          \lim\limits_{x \to q; x \in [0, q)} & \leq f(q)
        \end{align*}
        综上,$F$在$q$处不可微。
  \item $q = 0$ or $q = 1$
        通过左右极限的方式无法证明,当前还未找到解决方法。
\end{itemize}

\section*{11.9.2}

(1)方法1,利用推论10.2.9
如果$I$是空集或者单点集,那么结论是平凡的。

设$x_0 \in I$,任意$x \in I, x \neq x_0$。因为$F, G$都是$f$的原函数,
由定义11.9.3可知,$F, G$都在$I$上可微,于是由推论10.2.9可知,
存在$y \in [x, x_0]$(这里假设$x > x_0$,$x < x_0$同理)使得
\begin{align*}
  (F - G)^\prime(y)         & = \frac{(F - G)(x) - (F - G)(x_0)}{x - x_0} \\
  F^\prime(y) - G^\prime(y) & = \frac{(F - G)(x) - (F - G)(x_0)}{x - x_0} \\
  f(y) - f(y)               & = \frac{(F - G)(x) - (F - G)(x_0)}{x - x_0} \\
  0                         & = \frac{(F - G)(x) - (F - G)(x_0)}{x - x_0} \\
  0                         & = (F - G)(x) - (F - G)(x_0)                 \\
  F(x) - G(x)               & = F(x_0) - G(x_0)
\end{align*}
令$C = F(x_0) - G(x_0)$,于是任意$x \in I$且$x \neq x_0$都有
\begin{align*}
  F(x) = G(x) + C
\end{align*}
当$x = x_0$
\begin{align*}
   & G(x_0) + C                 \\
   & = G(x_0) + F(x_0) - G(x_0) \\
   & = F(x_0)
\end{align*}
综上,命题得证。

(2)方法2

任意$x \in I$,我们有
\begin{align*}
  (F - G)^\prime(x) & = F^\prime(x) - G^\prime(x) \\
                    & = f(x) - f(x)               \\
                    & = 0
\end{align*}
令$h : I \to \mathbb{R}, h = (F - G)^\prime(x)$,于是$h$是$I$上的常数函数,常数值为0,所以$h$是黎曼可积的。
设$x_0 \in I$,任意$x \in I, x \neq x_0$,我们有(这里假设$x > x_0$,$x < x_0$同理)
\begin{align*}
  \int_{[x_0, x]} h = (F - G)(x) - (F - G)(x_0) & = 0               \\
  F(x) - G(x)                                   & = F(x_0) - G(x_0) \\
\end{align*}
令$C = F(x_0) - G(x_0)$,于是任意$x \in I$且$x \neq x_0$都有
\begin{align*}
  F(x) = G(x) + C
\end{align*}
当$x = x_0$
\begin{align*}
   & G(x_0) + C                 \\
   & = G(x_0) + F(x_0) - G(x_0) \\
   & = F(x_0)
\end{align*}
综上,命题得证。

\section*{11.9.3}

\begin{itemize}
  \item $\Leftarrow$

        $f$是$[a, b]$上的单调递增函数,由命题11.6.1可知,$f$在$[a, b]$上是黎曼可积的。

        于是由定理11.9.1(微积分第一基本定理)可知,$f$在$x_0$处连续,则$F$在$x_0$处可微。

  \item $\Rightarrow$

        反证法,假设$f$在$x_0$处不连续,那么有定义9.3.6和$f$是单调递增函数可知,
        存在$\epsilon_0 > 0$使得时
        \begin{align*}
          f(x) > f(x_0) + \epsilon_0, if \; x > x_0 \\
          f(x) < f(x_0) - \epsilon_0, if \; x < x_0
        \end{align*}
        我们有
        \begin{equation*}
          \begin{cases*}
            F^\prime(x_0^{-}) = \lim\limits_{x \to x_0^{-}} \frac{F(x) - F(x_0)}{x - x_0} \\
            F^\prime(x_0^{+}) = \lim\limits_{x \to x_0^{+}} \frac{F(x) - F(x_0)}{x - x_0}   \\
          \end{cases*}
        \end{equation*}
        结合两组式子,我们有
        \begin{align*}
            F^\prime(x_0^{-}) = \lim\limits_{x \to x_0^{-}} \frac{F(x) - F(x_0)}{x - x_0} \leq f(x_0) - \epsilon \\
            F^\prime(x_0^{+}) = \lim\limits_{x \to x_0^{+}} \frac{F(x) - F(x_0)}{x - x_0} \geq f(x_0) + \epsilon 
        \end{align*}
        于是
        \begin{align*}
          F^\prime(x_0^{-}) \neq F^\prime(x_0^{+})
        \end{align*}
        与题设$F$在$x_0$处可微矛盾。

        特别地,$x = a \; or \; x = b$还未找到证明方法。
\end{itemize}

\end{document}