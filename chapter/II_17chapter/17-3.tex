\documentclass{article}
\usepackage{mathtools} 
\usepackage{fontspec}
\usepackage[UTF8]{ctex}
\usepackage{amsthm}
\usepackage{mdframed}
\usepackage{xcolor}
\usepackage{amssymb}
\usepackage{amsmath}


% 定义新的带灰色背景的说明环境 zremark
\newmdtheoremenv[
  backgroundcolor=gray!10,
  % 边框与背景一致,边框线会消失
  linecolor=gray!10
]{zremark}{说明}

% 通用矩阵命令: \flexmatrix{矩阵名}{元素符号}{行数}{列数}
\newcommand{\flexmatrix}[4]{
  \[
  #1 = \begin{pmatrix}
    #2_{11}     & #2_{12}     & \cdots & #2_{1#4}   \\
    #2_{21}     & #2_{22}     & \cdots & #2_{2#4}   \\
    \vdots      & \vdots      & \ddots & \vdots     \\
    #2_{#31}    & #2_{#32}    & \cdots & #2_{#3#4}
  \end{pmatrix}
  \]
}

% 简化版命令(默认矩阵名为A,元素符号为a): \quickmatrix{行数}{列数}
\newcommand{\quickmatrix}[2]{\flexmatrix{A}{a}{#1}{#2}}


\begin{document}
\title{17.3 习题}
\author{张志聪}
\maketitle

\section*{17.3.1}

$v$是零向量,等式显然成立,接下来我们讨论$v$不是零向量的情况。

$f$在$x_0$处可微,所以由定义17.2.2(可微性),我们有
\begin{align*}
  \lim\limits_{x \to x_0; x \in E - \{x_0\}} \frac{\|f(x) - (f(x_0) + f^\prime(x_0)(x- x_0)) \|}{\|x-x_0\|} = 0
\end{align*}

令$x = x_0 + tv$,则当$x \to x_0$时,$t \to 0$(只关注$t > 0$)。代入后:
\begin{align*}
  \lim\limits_{t \to 0; t > 0} \frac{\|f(x_0 + tv) - (f(x_0) + f^\prime(x_0)(tv)) \|}{\|tv\|} = 0 \\
  \lim\limits_{t \to 0; t > 0} \frac{\|f(x_0 + tv) - f(x_0) - f^\prime(x_0)(tv)\|}{\|tv\|} = 0    
\end{align*}

对任意$\epsilon > 0$,存在$\delta > 0$,使得对所有的$x \in B(x_0, \delta) \ \{x_0\}$(即:$(x_0 + tv) \in B(x_0, \delta) \ \{x_0\}$)。
都有
\begin{align*}
  \frac{\|f(x_0 + tv) - f(x_0) - f^\prime(x_0)(tv)\|}{\|tv\|} < \epsilon         \\
  \frac{\|f(x_0 + tv) - f(x_0) - tf^\prime(x_0)(v)\|}{t\|v\|} < \epsilon         \\
  \frac{\|\frac{f(x_0 + tv) - f(x_0)}{t} - f^\prime(x_0)(v)\|}{\|v\|} < \epsilon \\
  \|\frac{f(x_0 + tv) - f(x_0)}{t} - f^\prime(x_0)(v)\| < \epsilon \|v\|
\end{align*}
(变换过程中,$f^\prime(x_0)$的线性性来自可微性的定义)

由$\epsilon > 0$是任意值,$\|v\|$是定值,可知
\begin{align*}
  \frac{f(x_0 + tv) - f(x_0)}{t} = f^\prime(x_0)(v) \\
  \implies                                          \\
  D_vf(x_0) = f^\prime(x_0)(v)
\end{align*}

\section*{17.3.2}

\end{document}


