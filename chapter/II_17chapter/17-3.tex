\documentclass{article}
\usepackage{mathtools} 
\usepackage{fontspec}
\usepackage[UTF8]{ctex}
\usepackage{amsthm}
\usepackage{mdframed}
\usepackage{xcolor}
\usepackage{amssymb}
\usepackage{amsmath}


% 定义新的带灰色背景的说明环境 zremark
\newmdtheoremenv[
  backgroundcolor=gray!10,
  % 边框与背景一致,边框线会消失
  linecolor=gray!10
]{zremark}{说明}

% 通用矩阵命令: \flexmatrix{矩阵名}{元素符号}{行数}{列数}
\newcommand{\flexmatrix}[4]{
  \[
  #1 = \begin{pmatrix}
    #2_{11}     & #2_{12}     & \cdots & #2_{1#4}   \\
    #2_{21}     & #2_{22}     & \cdots & #2_{2#4}   \\
    \vdots      & \vdots      & \ddots & \vdots     \\
    #2_{#31}    & #2_{#32}    & \cdots & #2_{#3#4}
  \end{pmatrix}
  \]
}

% 简化版命令(默认矩阵名为A,元素符号为a): \quickmatrix{行数}{列数}
\newcommand{\quickmatrix}[2]{\flexmatrix{A}{a}{#1}{#2}}


\begin{document}
\title{17.3 习题}
\author{张志聪}
\maketitle

\section*{17.3.1}

$v$是零向量,等式显然成立,接下来我们讨论$v$不是零向量的情况。

$f$在$x_0$处可微,所以由定义17.2.2(可微性),我们有
\begin{align*}
  \lim\limits_{x \to x_0; x \in E - \{x_0\}} \frac{\|f(x) - (f(x_0) + f^\prime(x_0)(x- x_0)) \|}{\|x-x_0\|} = 0
\end{align*}

令$x = x_0 + tv$,则当$x \to x_0$时,$t \to 0$(只关注$t > 0$)。代入后:
\begin{align*}
  \lim\limits_{t \to 0; t > 0} \frac{\|f(x_0 + tv) - (f(x_0) + f^\prime(x_0)(tv)) \|}{\|tv\|} = 0 \\
  \lim\limits_{t \to 0; t > 0} \frac{\|f(x_0 + tv) - f(x_0) - f^\prime(x_0)(tv)\|}{\|tv\|} = 0
\end{align*}

对任意$\epsilon > 0$,存在$\delta > 0$,使得对所有的$x \in B(x_0, \delta) \ \{x_0\}$(即:$(x_0 + tv) \in B(x_0, \delta) \ \{x_0\}$)。
都有
\begin{align*}
  \frac{\|f(x_0 + tv) - f(x_0) - f^\prime(x_0)(tv)\|}{\|tv\|} < \epsilon         \\
  \frac{\|f(x_0 + tv) - f(x_0) - tf^\prime(x_0)(v)\|}{t\|v\|} < \epsilon         \\
  \frac{\|\frac{f(x_0 + tv) - f(x_0)}{t} - f^\prime(x_0)(v)\|}{\|v\|} < \epsilon \\
  \|\frac{f(x_0 + tv) - f(x_0)}{t} - f^\prime(x_0)(v)\| < \epsilon \|v\|
\end{align*}
(变换过程中,$f^\prime(x_0)$的线性性来自可微性的定义)

由$\epsilon > 0$是任意值,$\|v\|$是定值,可知
\begin{align*}
  \frac{f(x_0 + tv) - f(x_0)}{t} = f^\prime(x_0)(v) \\
  \implies                                          \\
  D_vf(x_0) = f^\prime(x_0)(v)
\end{align*}

\section*{17.3.2}

\begin{itemize}
  \item (a)偏导数$\implies$方向导数

        $\frac{\partial f}{\partial x_j}(x_0)$存在,由偏导数的定义可知,
        \begin{align*}
          \frac{\partial f}{\partial x_j}(x_0)
           & = \lim\limits_{t \to 0; t \neq 0, x_0 + te_j \in E} \frac{f(x_0 + te_j) - f(x_0)}{t}
        \end{align*}

        所以(右极限等于$x_0$处极限),
        \begin{align*}
          \lim\limits_{t \to 0; t \neq 0, x_0 + te_j \in E} \frac{f(x_0 + te_j) - f(x_0)}{t}
           & = \lim\limits_{t \to 0; t > 0, x_0 + te_j \in E} \frac{f(x_0 + te_j) - f(x_0)}{t} \\
           & = D_{e_j}f(x_0)
        \end{align*}

        又因为(左极限等于$x_0$处极限)
        \begin{align*}
          \lim\limits_{t \to 0; t \neq 0, x_0 + te_j \in E} \frac{f(x_0 + te_j) - f(x_0)}{t}
           & = \lim\limits_{t \to 0; t < 0, x_0 + te_j \in E} \frac{f(x_0 + te_j) - f(x_0)}{t}
        \end{align*}
        令$t = - t^\prime$,则当$t \to 0, t < 0$时,$t^\prime \to 0, t^\prime > 0$。代入后
        \begin{align*}
          \lim\limits_{t \to 0; t < 0, x_0 + te_j \in E} \frac{f(x_0 + te_j) - f(x_0)}{t}
           & = \lim\limits_{t^\prime \to 0; t^\prime > 0, x_0 + (-1)t^\prime e_j \in E} \frac{f(x_0 + (-t^\prime)e_j) - f(x_0)}{(-1)t^\prime} \\
           & = \lim\limits_{t^\prime \to 0; t^\prime > 0, x_0 + t^\prime (-e_j) \in E} (-1) \frac{f(x_0 + t^\prime(-e_j)) - f(x_0)}{t^\prime} \\
           & = - D_{(-e_j)} f(x_0)
        \end{align*}

  \item (b) 方向导数$\implies$偏导数

        $D_{e_j}f(x_0)$存在,由方向导数的定义,我们有
        \begin{align*}
          D_{e_j}f(x_0) & = \lim\limits_{t \to 0; t > 0, x_0 + t e_j \in E} \frac{f(x_0 + t e_j) - f(x_0)}{t} \\
                        & = \frac{\partial f}{\partial x_j^+}(x_0)
        \end{align*}
        即$\frac{\partial f}{\partial x_j}(x_0)$的右极限存在,且等于$D_{e_j}f(x_0)$。

        同理可得,
        \begin{align*}
          - D_{(-e_j)} f(x_0) & = \lim\limits_{t^\prime \to 0; t^\prime > 0, x_0 + t^\prime (-e_j) \in E} (-1) \frac{f(x_0 + t^\prime(-e_j)) - f(x_0)}{t^\prime} \\
        \end{align*}
        令$t^\prime = -t$,则当$t^\prime \to 0, t^\prime < 0$时,$t \to 0, t < 0$。代入后
        \begin{align*}
          - D_{(-e_j)} f(x_0) & = (-1) \lim\limits_{t^\prime \to 0; t^\prime > 0, x_0 + t^\prime (-e_j) \in E} \frac{f(x_0 + t^\prime(-e_j)) - f(x_0)}{t^\prime} \\
                              & = \lim\limits_{t \to 0; t < 0, x_0 + te_j \in E} \frac{f(x_0 + te_j) - f(x_0)}{t}                                                \\
                              & = \frac{\partial f}{\partial x_j^-}(x_0)
        \end{align*}
        即$\frac{\partial f}{\partial x_j}(x_0)$的左极限存在,且等于$- D_{(-e_j)} f(x_0)$。

        题设知$D_{e_j}f(x_0)$和$D_{(-e_j)} f(x_0)$互为相反数,
        所以,$D_{e_j}f(x_0) = - D_{(-e_j)} f(x_0)$。
        于是,$\frac{\partial f}{\partial x_j}(x_0)$左右极限相等,
        所以$\frac{\partial f}{\partial x_j}(x_0)$存在,且
        \begin{align*}
          \frac{\partial f}{\partial x_j}(x_0) = D_{e_j}f(x_0)
        \end{align*}

\end{itemize}

\section*{17.3.3}
\begin{itemize}
  \item (1)在$(0, 0)$处不可微。

        反证法,假设$f$在$(0, 0)$处可微,导数为$L = f^\prime(0, 0)$。

        我们有(P362有说明),
        \begin{align*}
          f^\prime(0, 0)(x, y)
           & = xf^\prime(0, 0)e_1 + yf^\prime(0, 0)e_2
        \end{align*}
        其中
        \begin{align*}
          f^\prime(0, 0)e_1 = \frac{\partial f}{\partial x_1}(0, 0) \\
          f^\prime(0, 0)e_2 = \frac{\partial f}{\partial x_2}(0, 0)
        \end{align*}
        我们先计算这两个偏导数
        \begin{align*}
          \frac{\partial f}{\partial x_1}(0, 0)
           & = \lim\limits_{t \to 0; t \neq 0, (0, 0) + te_1 \in \mathbb{R}^2} \frac{f((0, 0) + te_1) - f(0, 0)}{t} \\
           & = \lim\limits_{t \to 0; t \neq 0, (0, 0) + te_1 \in \mathbb{R}^2} \frac{f(t, 0) - 0}{t}                \\
           & = \lim\limits_{t \to 0; t \neq 0, (0, 0) + te_1 \in \mathbb{R}^2} \frac{\frac{t^3}{t^2}}{t}            \\
           & = 1
        \end{align*}
        \begin{align*}
          \frac{\partial f}{\partial x_2}(0, 0)
           & = \lim\limits_{t \to 0; t \neq 0, (0, 0) + te_2 \in \mathbb{R}^2} \frac{f((0, 0) + te_2) - f(0, 0)}{t} \\
           & = \lim\limits_{t \to 0; t \neq 0, (0, 0) + te_2 \in \mathbb{R}^2} \frac{f(0, t) - 0}{t}                \\
           & = \lim\limits_{t \to 0; t \neq 0, (0, 0) + te_2 \in \mathbb{R}^2} \frac{\frac{0^3}{0^2 + t^2} - 0}{t}  \\
           & = 0
        \end{align*}

        综上可得
        \begin{align*}
          f^\prime(0, 0)(x, y)
           & = xf^\prime(0, 0)e_1 + yf^\prime(0, 0)e_2 \\
           & = x
        \end{align*}

        于是由可微性(定义17.2.2),我们有
        \begin{align*}
          \lim\limits_{(x, y) \to (0, 0); (x, y) \in \mathbb{R}^2 - \{(0, 0)\}} \frac{\|f(x, y) - f(0, 0) - f^\prime(0, 0)(x, y)\|}{\|(x, y) - (0, 0)\|} = 0 \\
          \lim\limits_{(x, y) \to (0, 0); (x, y) \in \mathbb{R}^2 - \{(0, 0)\}} \frac{\|f(x, y) - 0 - x\|}{\|(x, y)\|} = 0                                   \\
          \lim\limits_{(x, y) \to (0, 0); (x, y) \in \mathbb{R}^2 - \{(0, 0)\}} \frac{\|\frac{x^3}{x^2 + y^2} - x\|}{\|(x, y)\|} = 0
        \end{align*}
        $(x, y)$以任何方式趋近于$(0, 0)$时,都等于0。
        不妨假设$y = x$且$x, y \to 0$,于是
        \begin{align*}
          \frac{\|\frac{x^3}{x^2 + x^2} - x\|}{\sqrt{x^2 + x^2}}
           & = \frac{\|\frac{x}{2} - x\|}{\sqrt{2}|x|} \\
           & = \frac{|-\frac{x}{2}|}{\sqrt{2}|x|}      \\
           & = \frac{\frac{1}{2}|x|}{\sqrt{2}|x|}      \\
           & = \frac{\sqrt{2}}{4}
        \end{align*}
        存在矛盾,假设不成立。

  \item (2) 不与定理17.3.8矛盾。

        任意$(x_0, y_0) \neq (0, 0)$求偏导数(这里选择对$y$求偏导数,避免处理3次方)
        \begin{align*}
          \frac{\partial f}{\partial x_2}
           & = \lim\limits_{t \to 0; t \neq 0, (x_0, y_0) + te_2 \in \mathbb{R}^2} \frac{f((x_0, y_0) + te_2) - f(x_0, y_0)}{t}                                                             \\
           & = \lim\limits_{t \to 0; t \neq 0, (x_0, y_0) + te_2 \in \mathbb{R}^2} \frac{f(x_0, y_0 + t) - f(x_0, y_0)}{t}                                                                  \\
           & = \lim\limits_{t \to 0; t \neq 0, (x_0, y_0) + te_2 \in \mathbb{R}^2} \frac{\frac{x_0^3}{x_0^2 + (y_0 + t)^2} - \frac{x_0^3}{x_0^2 + y_0^2}}{t}                                \\
           & = \lim\limits_{t \to 0; t \neq 0, (x_0, y_0) + te_2 \in \mathbb{R}^2} \frac{\frac{x_0^3(x_0^2 + y_0^2) - x_0^3[x_0^2 + (y_0 + t)^2]}{[x_0^2 + (y_0 + t)^2](x_0^2 + y_0^2)}}{t} \\
           & = \lim\limits_{t \to 0; t \neq 0, (x_0, y_0) + te_2 \in \mathbb{R}^2} x_0^3 \frac{-t^2 - 2ty_0}{t[x_0^2 + (y_0 + t)^2](x_0^2 + y_0^2)}                                         \\
           & = \lim\limits_{t \to 0; t \neq 0, (x_0, y_0) + te_2 \in \mathbb{R}^2} x_0^3 \frac{-t - 2y_0}{[x_0^2 + (y_0 + t)^2](x_0^2 + y_0^2)}                                             \\
           & = \frac{- 2x_0^3 y_0}{(x_0^2 + y_0^2)^2}
        \end{align*}

        (x, y)以任何方式趋近于$(0, 0)$,偏导数公式相同。
        不妨设$y_0 = x_0$且$(x_0, y_0) \to (0, 0)$,那么
        \begin{align*}
          \frac{- 2x_0^3 y_0}{(x_0^2 + y_0^2)^2}
           & = \frac{- 2x_0^4}{4x_0^4} \\
           & = \frac{- 1}{2}
        \end{align*}
        可见$\frac{\partial f}{\partial x_2}$在$(0, 0)$处不连续。不满足定理17.3.8的前置条件。

\end{itemize}

\section*{17.3.4}

因为对于所有的$x \in \mathbb{R}^n, f^\prime(x) = 0$(表示是零线性变换,任意向量$v$,都有$f^\prime(x)v = 0$),
那么,任意偏导数都有
\begin{align*}
  \frac{\partial f}{\partial x_j}(x) = f^\prime(x) e_j = 0
\end{align*}
于是$f$在$\mathbb{R}^n$上的全体偏导数都是零向量,所以,在任意$x \in \mathbb{R}^n$处
偏导数都是连续的(总是零向量)。
把$f$写成$(f_1, f_2, \cdots, f_m)$,于是
\begin{align*}
  \frac{\partial f}{\partial x_j}(x)
   & = (\frac{\partial f_1}{\partial x_j}(x), \cdots, \frac{\partial f_m}{\partial x_j}(x)) \\
   & = (0, \cdots, 0)
\end{align*}
从而,每个分量的偏导数也是连续的。

设$y_0 \neq x_0, y_0 - x_0 = v = (v_1, \cdots, v_n) = \sum \limits_{i=1}^n v_i e_i$。
对变量$x_1$使用平均值定理可得,存在一个介于$0$和$v_1$之间的$t_i$,使得
\begin{align*}
  f_i(x_0 + v_1e_1) - f_i(x_0) = \frac{\partial f_i}{\partial x_1}(x_0 + t_ie_1)v_1 = 0,\ 1 \leq i \leq m
\end{align*}
所以,$f(x_0 + v_1e_1) - f(x_0) = 0$。

同理可得
\begin{align*}
  f(x_0 + v_1e_1 + v_2e_2) - f(x_0 + v_1e_1) = 0
\end{align*}
那么以此类推有
\begin{align*}
  f(x_0 + v_1e_1 + \cdots + v_ne_n) - f(x_0 + v_1e_1 + \cdots + v_{n-1}e_{n - 1}) = 0
\end{align*}
把这$n$个等式相加,得到
\begin{align*}
  f(x_0 + v_1e_1 + \cdots + v_ne_n) - f(x_0) = 0 \\
  f(y_0) - f(x_0) = 0
\end{align*}
所以$f(y_0) = f(x_0)$,
由$y_0, x_0$的任意性可得,$f(x)$是常值函数。


\end{document}


