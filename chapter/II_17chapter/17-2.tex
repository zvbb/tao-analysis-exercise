\documentclass{article}
\usepackage{mathtools} 
\usepackage{fontspec}
\usepackage[UTF8]{ctex}
\usepackage{amsthm}
\usepackage{mdframed}
\usepackage{xcolor}
\usepackage{amssymb}
\usepackage{amsmath}


% 定义新的带灰色背景的说明环境 zremark
\newmdtheoremenv[
  backgroundcolor=gray!10,
  % 边框与背景一致,边框线会消失
  linecolor=gray!10
]{zremark}{说明}

% 通用矩阵命令: \flexmatrix{矩阵名}{元素符号}{行数}{列数}
\newcommand{\flexmatrix}[4]{
  \[
  #1 = \begin{pmatrix}
    #2_{11}     & #2_{12}     & \cdots & #2_{1#4}   \\
    #2_{21}     & #2_{22}     & \cdots & #2_{2#4}   \\
    \vdots      & \vdots      & \ddots & \vdots     \\
    #2_{#31}    & #2_{#32}    & \cdots & #2_{#3#4}
  \end{pmatrix}
  \]
}

% 简化版命令(默认矩阵名为A,元素符号为a): \quickmatrix{行数}{列数}
\newcommand{\quickmatrix}[2]{\flexmatrix{A}{a}{#1}{#2}}


\begin{document}
\title{17.2 习题}
\author{张志聪}
\maketitle

\section*{17.2.1}

\begin{itemize}
  \item (a) $\implies$ (b)

        $f$在$x_0$处可微,所以由牛顿逼近法(命题10.1.7)可得,
        对任意的$\epsilon > 0$都存在一个$\delta > 0$使得,只要$x \in E$且
        $|x - x_0| \leq \delta$,就有
        \begin{align*}
          |f(x) - (f(x_0) + L(x - x_0))| \leq \epsilon|x - x_0|
          % \frac{|f(x) - (f(x_0) + L(x - x_0))|}{|x - x_0|} \leq \epsilon
        \end{align*}
        令函数$h: E - \{x_0\} \to \mathbb{R}$为$h(x) = \frac{|f(x) - (f(x_0) + L(x - x_0))|}{|x - x_0|}$,
        所以(因为符合定义9.3.6)
        \begin{align*}
          \lim\limits_{x \to x_0; x \in E - \{x_0\}} h(x) = 0 \\
          \lim\limits_{x \to x_0; x \in E - \{x_0\}} \frac{|f(x) - (f(x_0) + L(x - x_0))|}{|x - x_0|} = 0
        \end{align*}

  \item (b) $\implies$ (a)

        \begin{align*}
          \lim\limits_{x \to x_0; x \in E - \{x_0\}} \frac{|f(x) - (f(x_0) + L(x - x_0))|}{|x - x_0|} = 0
        \end{align*}

        那么,对任意$\epsilon > 0$都存在一个$\delta > 0$使得,只要$x \in E - \{x_0\}$且
        $|x - x_0| \leq \delta$,就有
        \begin{align*}
          \left|\frac{|f(x) - (f(x_0) + L(x - x_0))|}{|x - x_0|} - 0\right| \leq \epsilon \\
          |f(x) - (f(x_0) + L(x - x_0))| \leq \epsilon|x - x_0|
        \end{align*}
        特别地,$x = x_0$时,$|f(x) - (f(x_0) + L(x - x_0))| \leq \epsilon|x - x_0|$也成立。

        由命题10.1.7(牛顿逼近法)可知(a)成立。
\end{itemize}

\section*{17.2.2 $\circledast$} 

反证法,如果$L_1 \neq L2$,那么存在一个向量$v$使得
$L_1v \neq L_2v$。这个向量$v$一定不是零向量(因为由线性变换的可加性,对任意的线性变换$T$,都有$T(0) = 0$)。

由$f$在$x_0$处可微,那么对任意$\epsilon = \frac{\|L_1(v) - L_2(v)\|}{2\|v\|} > 0$,
存在$\delta > 0$使得只要$|x - x_0| \leq \delta$,就有
\begin{align*}
  \frac{\|f(x) - f(x_0 - L_1(x - x_0))\|}{\|x - x_0\|} < \epsilon \\
  \frac{\|f(x) - f(x_0) - L_2(x - x_0)\|}{\|x - x_0\|} < \epsilon
\end{align*}

令$x = x_0 + tv$($t$是标量),使得$\|x - x_0\| < \delta$。
\begin{align*}
  \frac{\|f(x) - f(x_0) - L_1(tv)\|}{\|tv\|} < \epsilon \\
  \frac{\|f(x) - f(x_0) - L_2(tv)\|}{\|tv\|} < \epsilon
\end{align*}

因为
\begin{align*}
  \|L_1(tv) - L_2(tv)\| & = \|L_1(tv) - (f(x) + f(x_0)) + (f(x) + f(x_0)) - L_2(tv)\|        \\
                        & \leq \|L_1(tv) - (f(x) + f(x_0))\| + \|(f(x) + f(x_0)) - L_2(tv)\|
\end{align*}

所以
\begin{align*}
   & \frac{\|L_1(tv) - L_2(tv)\|}{\|tv\|} = \frac{\|L_1(v) - L_2(v)\|}{\|v\|}                                                              \\
   & \leq \frac{\|f(x) - (f(x_0) + L_1(tv))\|}{\|tv\|} + \frac{\|f(x) - (f(x_0) + L_2(tv))\|}{\|tv\|} \\
   & \leq 2\epsilon = \frac{\|L_1(v) - L_2(v)\|}{\|v\|}
\end{align*}
存在矛盾。

\end{document}


