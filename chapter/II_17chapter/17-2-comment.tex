\documentclass{article}
\usepackage{mathtools} 
\usepackage{fontspec}
\usepackage[UTF8]{ctex}
\usepackage{amsthm}
\usepackage{mdframed}
\usepackage{xcolor}
\usepackage{amssymb}
\usepackage{amsmath}


% 定义新的带灰色背景的说明环境 zremark
\newmdtheoremenv[
  backgroundcolor=gray!10,
  % 边框与背景一致,边框线会消失
  linecolor=gray!10
]{zremark}{说明}

% 通用矩阵命令: \flexmatrix{矩阵名}{元素符号}{行数}{列数}
\newcommand{\flexmatrix}[4]{
  \[
  #1 = \begin{pmatrix}
    #2_{11}     & #2_{12}     & \cdots & #2_{1#4}   \\
    #2_{21}     & #2_{22}     & \cdots & #2_{2#4}   \\
    \vdots      & \vdots      & \ddots & \vdots     \\
    #2_{#31}    & #2_{#32}    & \cdots & #2_{#3#4}
  \end{pmatrix}
  \]
}

% 简化版命令(默认矩阵名为A,元素符号为a): \quickmatrix{行数}{列数}
\newcommand{\quickmatrix}[2]{\flexmatrix{A}{a}{#1}{#2}}


\begin{document}
\title{17.2 注释}
\author{张志聪}
\maketitle

\begin{zremark}
  $f: \mathbb{R}^n \to \mathbb{R}^m$在点$x_0$处可微,那么,在$x_0$处$f$的分量函数也是可微的。
\end{zremark}

\textbf{证明:}

由可微性定义可知,存在线性变换$L: \mathbb{R}^n \to \mathbb{R}^m$,
使得以下极限存在:
\begin{align*}
  \lim\limits_{x \to x_0;} \frac{\|f(x) - f(x_0) - L(x - x_0)\|}{\|x - x_0\|} = 0
\end{align*}
把$f,L$写成$(f_1, f_2, \cdots, f_m)$和$(L_1, L_2, \cdots, L_m)$,
其中$f_i: \mathbb{R}^n \to \mathbb{R}$和$L_i: \mathbb{R}^n \to \mathbb{R}$。
至于如何拆线性变换,书中P355,已经说明:
\begin{align*}
  L_A(x_j)_{1 \leq j \leq n} = \left(\sum \limits_{j = 1}^n a_{ij}x_j\right)_{1 \leq i \leq m}
\end{align*}
那么
\begin{align*}
   & \lim\limits_{x \to x_0;} \frac{\|f(x) - f(x_0) - L(x - x_0)\|}{\|x - x_0\|}                                                                        \\
   & = \lim\limits_{x \to x_0;} \frac{\|(f_1(x), \cdots ,f_m(x)) - (f_1(x_0), \cdots ,f_m(x_0)) - (L_1(x - x_0)+ \cdots + L_m(x - x_0))\|}{\|x - x_0\|} \\
   & = \lim\limits_{x \to x_0;} \frac{\|(f_1(x) - f_1(x_0) - L_1(x - x_0), \cdots , f_m(x) - f_m(x_0) - L_m(x - x_0))\|}{\|x - x_0\|}                   \\
   & = 0
\end{align*}
对任意$\epsilon > 0$,存在$\delta > 0$,使得只要$\|x - x_0\| \leq \delta$,就有
\begin{align*}
  \frac{\|(f_1(x) - f_1(x_0) - L_1(x - x_0), \cdots , f_m(x) - f_m(x_0) - L_m(x - x_0))\|}{\|x - x_0\|} < \epsilon
\end{align*}
于是可得,任意分量($1 \leq j \leq m$)都有
\begin{align*}
  \frac{\|f_j(x) - f_j(x_0) - L_j(x - x_0)\|}{\|x - x_0\|} \leq \epsilon
\end{align*}

综上可得,在$x_0$处$f$的分量函数也是可微的,且$f_j^\prime(x_0) = L_j$。


\begin{zremark}
  如果$l: \mathbb{R}^n \to \mathbb{R}^m$的一个线性变换,
  那么$l$在点$x_0 \in \mathbb{R}^n$处的导数如何确定。
\end{zremark}

\textbf{证明:}

由定义17.2.2(可微性),问题就是找到一个线性变换$L$,使得以下极限存在:
\begin{align*}
  \lim\limits_{x \to x_0; x \in \mathbb{R}^n - \{x_0\}} \frac{\|l(x) - l(x_0) - L(x - x_0)\|}{\|x - x_0\|}
   & = \lim\limits_{x \to x_0; x \in \mathbb{R}^n - \{x_0\}} \frac{\|l(x - x_0) - L(x - x_0)\|}{\|x - x_0\|} \\
   & = 0
\end{align*}
令$L = l$以上极限就可以存在,又由17.2.4(导数的唯一性)可知,
$l$在$x_0$处的导数是其本身$l$。

而且比较有趣的是:$l$的在任何点的导数都是$l$本身,与点无关。
和实数函数$f(x) = c x$类似,导数都是$c$,与$x$无关。

\end{document}


