\documentclass{article}
\usepackage{mathtools} 
\usepackage{fontspec}
\usepackage[UTF8]{ctex}
\usepackage{amsthm}
\usepackage{mdframed}
\usepackage{xcolor}
\usepackage{amssymb}
\usepackage{amsmath}


% 定义新的带灰色背景的说明环境 zremark
\newmdtheoremenv[
  backgroundcolor=gray!10,
  % 边框与背景一致,边框线会消失
  linecolor=gray!10
]{zremark}{说明}

% 通用矩阵命令: \flexmatrix{矩阵名}{元素符号}{行数}{列数}
\newcommand{\flexmatrix}[4]{
  \[
  #1 = \begin{pmatrix}
    #2_{11}     & #2_{12}     & \cdots & #2_{1#4}   \\
    #2_{21}     & #2_{22}     & \cdots & #2_{2#4}   \\
    \vdots      & \vdots      & \ddots & \vdots     \\
    #2_{#31}    & #2_{#32}    & \cdots & #2_{#3#4}
  \end{pmatrix}
  \]
}

% 简化版命令(默认矩阵名为A,元素符号为a): \quickmatrix{行数}{列数}
\newcommand{\quickmatrix}[2]{\flexmatrix{A}{a}{#1}{#2}}


\begin{document}
\title{17.2 注释}
\author{张志聪}
\maketitle

\begin{zremark}
  如果$l: \mathbb{R}^n \to \mathbb{R}^m$的一个线性变换,
  那么$l$在点$x_0 \in \mathbb{R}^n$处的导数如何确定。
\end{zremark}

\textbf{证明:}

由定义17.2.2(可微性),问题就是找到一个线性变换$L$,使得以下极限存在:
\begin{align*}
  \lim\limits_{x \to x_0; x \in \mathbb{R}^n - \{x_0\}} \frac{\|l(x) - l(x_0) - L(x - x_0)\|}{\|x - x_0\|}
   & = \lim\limits_{x \to x_0; x \in \mathbb{R}^n - \{x_0\}} \frac{\|l(x - x_0) - L(x - x_0)\|}{\|x - x_0\|} \\
   & = 0
\end{align*}
令$L = l$以上极限就可以存在,又由17.2.4(导数的唯一性)可知,
$l$在$x_0$处的导数是其本身$l$。

而且比较有趣的是:$l$的在任何点的导数都是$l$本身,与点无关。
和实数函数$f(x) = c x$类似,导数都是$c$,与$x$无关。

\end{document}


