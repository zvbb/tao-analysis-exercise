\documentclass{article}
\usepackage{mathtools} 
\usepackage{fontspec}
\usepackage[UTF8]{ctex}
\usepackage{amsthm}
\usepackage{mdframed}
\usepackage{xcolor}
\usepackage{amssymb}
\usepackage{amsmath}


% 定义新的带灰色背景的说明环境 zremark
\newmdtheoremenv[
  backgroundcolor=gray!10,
  % 边框与背景一致,边框线会消失
  linecolor=gray!10
]{zremark}{说明}

% 通用矩阵命令: \flexmatrix{矩阵名}{元素符号}{行数}{列数}
\newcommand{\flexmatrix}[4]{
  \[
  #1 = \begin{pmatrix}
    #2_{11}     & #2_{12}     & \cdots & #2_{1#4}   \\
    #2_{21}     & #2_{22}     & \cdots & #2_{2#4}   \\
    \vdots      & \vdots      & \ddots & \vdots     \\
    #2_{#31}    & #2_{#32}    & \cdots & #2_{#3#4}
  \end{pmatrix}
  \]
}

% 简化版命令(默认矩阵名为A,元素符号为a): \quickmatrix{行数}{列数}
\newcommand{\quickmatrix}[2]{\flexmatrix{A}{a}{#1}{#2}}


\begin{document}
\title{17.6 注释}
\author{张志聪}
\maketitle

\begin{zremark}
  $\overline{B(0, r)}$是$\mathbb{R}^n$中以原点为中心的球,
  $\overline{B(0, r)}$是完备度量空间。
\end{zremark}

设$(x_n)_{n = 1}^\infty$是$\overline{B(0, r)}$中的柯西序列,
即对任意$\epsilon > 0$,存在$N \geq 1$,使得只要$p, q \geq N$,都有
\begin{align*}
  d(x_p, x_q) < \epsilon
\end{align*}
于是,对$x_p, x_q$的任意分量$1 \leq j \leq n$,都有
\begin{align*}
  x_{p}^{(j)} - x_{q}^{(j)} < \epsilon
\end{align*}
因此,每个分量序列$(x_n^{(j)})_{n = 1}^\infty$都是柯西序列,
由定理6.4.18可知,
$(x_n^{(j)})_{n = 1}^\infty$收敛于某个实数$L_j$,
因此,由命题12.1.18可知$(x_n)_{n = 1}^\infty$收敛,
因为$\overline{B(0, r)}$是闭集,于是由命题12.2.5(b)可知,
$(L_1, L_2, \cdots, L_n) \in \overline{B(0, r)}$。

综上,命题成立。



\end{document}