\documentclass{article}
\usepackage{mathtools} 
\usepackage{fontspec}
\usepackage[UTF8]{ctex}
\usepackage{amsthm}
\usepackage{mdframed}
\usepackage{xcolor}
\usepackage{amssymb}
\usepackage{amsmath}


% 定义新的带灰色背景的说明环境 zremark
\newmdtheoremenv[
  backgroundcolor=gray!10,
  % 边框与背景一致,边框线会消失
  linecolor=gray!10
]{zremark}{说明}

% 通用矩阵命令: \flexmatrix{矩阵名}{元素符号}{行数}{列数}
\newcommand{\flexmatrix}[4]{
  \[
  #1 = \begin{pmatrix}
    #2_{11}     & #2_{12}     & \cdots & #2_{1#4}   \\
    #2_{21}     & #2_{22}     & \cdots & #2_{2#4}   \\
    \vdots      & \vdots      & \ddots & \vdots     \\
    #2_{#31}    & #2_{#32}    & \cdots & #2_{#3#4}
  \end{pmatrix}
  \]
}

% 简化版命令(默认矩阵名为A,元素符号为a): \quickmatrix{行数}{列数}
\newcommand{\quickmatrix}[2]{\flexmatrix{A}{a}{#1}{#2}}


\begin{document}
\title{17.5 习题}
\author{张志聪}
\maketitle

\section*{17.5.1}

% \begin{align*}
%   \frac{\partial f}{\partial x} = \frac{y^3(x^2 + y^2) - xy^3(2x)}{(x^2 + y^2)^2} = \frac{y^5 - x^2 y^3}{(x^2 + y^2)^2} \\
%   \frac{\partial f}{\partial y} = \frac{3xy^2(x^2 + y^2) - xy^3(2y)}{(x^2 + y^2)^2} = \frac{3x^3 y^2 + xy^4}{(x^2 + y^2)^2}
% \end{align*}
% 接下来,需要判断偏导数的连续性。
% 由引理13.2.2和推论13.2.3可知,$(x, y) \neq 0$时,
% $\frac{\partial f}{\partial x}$和$\frac{\partial f}{\partial y}$都是连续的,
% 我们还需判断$(x, y) = (0, 0)$时的情况:
% \begin{align*}
%   \lim\limits_{(x, y) \to (0, 0)} \frac{y^5 - x^2 y^3}{(x^2 + y^2)^2} = 0 \\
%   \lim\limits_{(x, y) \to (0, 0)} \frac{3x^3 y^2 + xy^4}{(x^2 + y^2)^2} = 0 
% \end{align*}
% (书中好像也没啥可靠的办法,也有可能是我学艺不精。)

略。
大家可以查考:https://christangdt.home.blog/2019/07/05/%e9%99%b6%e5%93%b2%e8%bd%a9%e5%ae%9e%e5%88%86%e6%9e%90%ef%bc%88%e4%b8%8b%ef%bc%896-5%e5%8f%8a%e4%b9%a0%e9%a2%98-analysis-ii-6-5/


\end{document}