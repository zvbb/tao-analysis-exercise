\documentclass{article}
\usepackage{mathtools} 
\usepackage{fontspec}
\usepackage[UTF8]{ctex}
\usepackage{amsthm}
\usepackage{mdframed}
\usepackage{xcolor}
\usepackage{amssymb}
\usepackage{amsmath}


% 定义新的带灰色背景的说明环境 zremark
\newmdtheoremenv[
  backgroundcolor=gray!10,
  % 边框与背景一致,边框线会消失
  linecolor=gray!10
]{zremark}{说明}

% 通用矩阵命令: \flexmatrix{矩阵名}{元素符号}{行数}{列数}
\newcommand{\flexmatrix}[4]{
  \[
  #1 = \begin{pmatrix}
    #2_{11}     & #2_{12}     & \cdots & #2_{1#4}   \\
    #2_{21}     & #2_{22}     & \cdots & #2_{2#4}   \\
    \vdots      & \vdots      & \ddots & \vdots     \\
    #2_{#31}    & #2_{#32}    & \cdots & #2_{#3#4}
  \end{pmatrix}
  \]
}

% 简化版命令(默认矩阵名为A,元素符号为a): \quickmatrix{行数}{列数}
\newcommand{\quickmatrix}[2]{\flexmatrix{A}{a}{#1}{#2}}


\begin{document}
\title{17.7 习题}
\author{张志聪}
\maketitle

\section*{17.7.1}

(1)

$x \neq 0$时
\begin{align*}
  f^\prime(x) & = (x + x^2sin(1/x^4))^\prime               \\
              & = 1 + 2xsin(1/x^4) + x^2cos(1/x^4)(-4/x^5) \\
              & = 1 + 2xsin(1/x^4) + (-4/x^3)cos(1/x^4)
\end{align*}

$x = 0$时
\begin{align*}
  \lim\limits_{x \to 0} \frac{f(x) - f(0)}{x - 0}
   & = \lim\limits_{x \to 0} \frac{x + x^2sin(1/x^4)}{x} \\
   & = \lim\limits_{x \to 0} 1 + xsin(1/x^4)
\end{align*}
因为$-1 \leq sin(1/x^4) \leq 1$,于是
\begin{align*}
  -x \leq xsin(1/x^4) \leq 1
\end{align*}
所以
\begin{align*}
  \lim\limits_{x \to 0} xsin(1/x^4) = 0
\end{align*}
综上可得
\begin{align*}
  f^\prime(0)
   & = \lim\limits_{x \to 0} 1 + xsin(1/x^4)                       \\
   & = \lim\limits_{x \to 0} 1 + \lim\limits_{x \to 0} xsin(1/x^4) \\
   & = 1 + 0 = 1
\end{align*}

(2)

序列$((2\pi k)^{-1})_{k = 1}^\infty$收敛于$0$。
令$x = (2\pi k)^{-1}$,我们有
\begin{align*}
  cos(1/x^4) =  cos(2 \pi k) = 1 \\
  sin(1/x^4) =  sin(2 \pi k) = 0
\end{align*}

于是
\begin{align*}
  f^\prime((2 \pi k)^{-1/4}) = 1 - \frac{4}{(2 \pi k)^{-1/4}} = 1 - \frac{(2 \pi k)^{1/4}}{4}
\end{align*}
所以
\begin{align*}
  \lim\limits_{k \to \infty} f^\prime((2 \pi k)^{-1/4}) = -\infty
\end{align*}

综上可得,$x$以序列$((2\pi k)^{-1})_{k = 1}^\infty$的方式趋近$0$,总有$f^\prime(x) < 0$。

\section*{17.7.2}

因为$T$是可逆的线性变换,那么,$T$既是单射也是满射。
于是,对任意$y_0 \in \mathbb{R}^n$
都存在唯一的$x_0 \in \mathbb{R}^n$使得$T(x_0) = y_0$,即$T^{-1}(y_0) = x_0$。

接下来,证明$T^{-1}$满足定义17.1.6(线性变换)的两个公理。
\begin{itemize}
  \item (a)(可加性)对任意$y, y^\prime \in \mathbb{R}^n$,都有$T^{-1}(y + y^\prime) = T^{-1}(y) + T^{-1}(y^\prime)$。

        存在$x, x^\prime \in \mathbb{R}^n$使得$T(x) = y$和$T(x^\prime) = y^\prime$,
        于是
        \begin{align*}
          T^{-1}(y) + T^{-1}(y^\prime) = T^{-1}(T(x)) + T^{-1}(T(x^\prime)) = x + x^\prime
        \end{align*}
        又因为
        \begin{align*}
          T(x) + T(x^\prime) = T(x + x^\prime)
        \end{align*}
        综上
        \begin{align*}
          T^{-1}(y + y^\prime)
           & = T^{-1}(T(x) + T(x^\prime)) \\
           & = T^{-1}(T(x + x^\prime))    \\
           & = x + x^\prime
        \end{align*}
        所以,$T^{-1}(y + y^\prime) = T^{-1}(y) + T^{-1}(y^\prime)$。

  \item (b)(齐次性)对任意$y \in \mathbb{R}^n$和任意的$c \in \mathbb{R}$,都有$T^{-1}(cy) = cT^{-1}(y)$。

        存在$x \in \mathbb{R}^n$使得$T(x) = y$,
        于是
        \begin{align*}
          cT^{-1}(y) = cx
        \end{align*}
        我们有
        \begin{align*}
          T(cx) = cT(x) = cy
        \end{align*}
        于是
        \begin{align*}
          T^{-1}(cy) & = T^{-1}(cT(x)) \\
                     & = T^{-1}(T(cx)) \\
                     & = cx
        \end{align*}
        所以,$T^{-1}(cy) = cT^{-1}(y)$。

\end{itemize}

\section*{17.7.3}

对任意$y_0 \in f(V)$,存在$x_0 \in V$使得$y_0 = f(x_0)$。
由题设可知$f^\prime(x_0)$是一个可逆的线性变换,
于是由反函数定理可知,
$V$中存在一个包含$x_0$的开集$V_{x_0}$,
并且在$\mathbb{R}^n$中存在一个包含$f(x_0)$的开集$U_{x_0}$,使得
$f$是从$V_{x_0}$到$U_{x_0}$的双射。

因为$U_{x_0}$是开集,且$f(x_0) \in U_{x_0}$,那么存在$r > 0$,使得
$B(f(x_0), r) \subseteq U_{x_0}$,即$B(y_0, r) \subseteq U_{x_0}$。

因为$V_{x_0} \subseteq V$,所以$U_{x_0} \subseteq f(V)$。

综上可得,$B(y_0, r) \subseteq f(V)$,
由命题12.2.15(a)可知$f(V)$是开集。

\end{document}