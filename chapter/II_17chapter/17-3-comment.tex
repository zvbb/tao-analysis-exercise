\documentclass{article}
\usepackage{mathtools} 
\usepackage{fontspec}
\usepackage[UTF8]{ctex}
\usepackage{amsthm}
\usepackage{mdframed}
\usepackage{xcolor}
\usepackage{amssymb}
\usepackage{amsmath}


% 定义新的带灰色背景的说明环境 zremark
\newmdtheoremenv[
  backgroundcolor=gray!10,
  % 边框与背景一致,边框线会消失
  linecolor=gray!10
]{zremark}{}


% 通用矩阵命令: \flexmatrix{矩阵名}{元素符号}{行数}{列数}
\newcommand{\flexmatrix}[4]{
  \[
  #1 = \begin{pmatrix}
    #2_{11}     & #2_{12}     & \cdots & #2_{1#4}   \\
    #2_{21}     & #2_{22}     & \cdots & #2_{2#4}   \\
    \vdots      & \vdots      & \ddots & \vdots     \\
    #2_{#31}    & #2_{#32}    & \cdots & #2_{#3#4}
  \end{pmatrix}
  \]
}

% 简化版命令(默认矩阵名为A,元素符号为a): \quickmatrix{行数}{列数}
\newcommand{\quickmatrix}[2]{\flexmatrix{A}{a}{#1}{#2}}


\begin{document}
\title{17.3 注释}
\author{张志聪}
\maketitle

\begin{zremark}
  把$f$写成$f = (f_1, f_2, \cdots, f_m)$,那么容易得出
  \begin{align*}
    \frac{\partial f}{\partial x_j}(x_0) = (\frac{\partial f_1}{\partial x_j}(x_0),..., \frac{\partial f_m}{\partial x_j}(x_0))
  \end{align*}
\end{zremark}

\textbf{证明:}

\begin{align*}
  \frac{\partial f}{\partial x_j} (x_0)
   & = \lim\limits_{t \to 0; t \neq 0, x_0 + te_j \in E} \frac{f(x_0 + te_j) - f(x_0)}{t}                                              \\
   & = \lim\limits_{t \to 0; t \neq 0, x_0 + te_j \in E} \frac{(f_1(x_0 + te_j), ..., f_m(x_0 + te_j)) - (f_1(x_0),..., f_m(x_0))}{t}  \\
   & = \lim\limits_{t \to 0; t \neq 0, x_0 + te_j \in E} \frac{(f_1(x_0 + te_j)-f_1(x_0), ..., f_m(x_0 + te_j) - f_m(x_0))}{t}         \\
   & = \lim\limits_{t \to 0; t \neq 0, x_0 + te_j \in E} (\frac{f_1(x_0 + te_j)-f_1(x_0)}{t}, ..., \frac{f_m(x_0 + te_j)-f_m(x_0)}{t}) \\
   & = (\frac{\partial f_1}{\partial x_j}(x_0),..., \frac{\partial f_m}{\partial x_j}(x_0))
\end{align*}

\begin{zremark}
  设$E$是$\mathbb{R}^n$的子集,$f : E \to \mathbb{R}^m$是一个函数,$F$是$E$的子集,
  并设$x_0$是$F$的内点。如果$f$在$F$上的全体偏导数$\frac{\partial f}{\partial x_j}$都存在,
  并且它们在$x_0$处都是连续的。
  那么,把$f$写成$f_1, f_2, \cdots, f_m$,(每一个$f_i$都是从$E$到$\mathbb{R}$的函数)。
  对变量$x_1$使用平均值定理(推论10.2.9中值定理),存在一个介于$0$和$v_1$之间的$t_i$,使得
  \begin{align*}
    f_i(x_0 + v_1e_1) - f_i(x_0) = \frac{\partial f_i}{\partial x_1}(x_0 + t_ie_1)v_1
  \end{align*}

  我的问题是,按照平均值定理,等式应该是
  \begin{align*}
    f_i(x_0 + v_1e_1) - f_i(x_0) = \frac{\partial f_i}{\partial x_1}(x_0 + t_ie_1)v_1e_1
  \end{align*}
  才对吧!
  因为$(x_0 + v_1e_1) - x_0 = v_1 e_1$。
\end{zremark}

\textbf{证明:}

我的理解是有问题的。

一方面,如果改为
\begin{align*}
  f_i(x_0 + v_1e_1) - f_i(x_0) = \frac{\partial f_i}{\partial x_1}(x_0 + t_ie_1)v_1e_1
\end{align*}
此时,等式左侧是标量,右侧是向量,存在矛盾。

另一方面,$f_i : \mathbb{R}^n \to \mathbb{R}$,对变量$x_1$求偏导数,即固定了$x_2, x_3, \cdots, x_n$,
仅让$x_1$变化,此时$f_i$可以看做单变量函数:
\begin{align*}
  g(t) = f_i(x_0 + te_1) \ t \in [0, v_1]
\end{align*}
于是
\begin{align*}
  g(v_1) - g(0) = g^\prime(t_i) (v_1 - 0)
\end{align*}

因为
\begin{align*}
  g^\prime(t_0)
   & = \lim\limits_{t \to t_0} \frac{g(t) - g(t_0)}{t - t_0}                       \\
   & = \lim\limits_{t \to t_0} \frac{f_i(x_0 + te_1) - f_i(x_0 + t_0e_1)}{t - t_0}
\end{align*}
令$h = t - t_0$,则当$t \to t_0$时,$h \to 0$。代入后:
\begin{align*}
  g^\prime(t_0) & = \lim\limits_{h \to 0} \frac{f_i(x_0 + (t_0 + h)e_1) - f_i(x_0 + t_0e_1)}{h}
\end{align*}

又因为
\begin{align*}
  \frac{\partial f_i}{\partial x_1}(x_0 + t_0e_1)
   & = \lim\limits_{t \to 0; t \neq 0, x_0 + te_1 \in F } \frac{f_i(x_0 + t_0e_1 + te_1) - f_i(x_0 + t_0e_1)}{t} \\
   & = \lim\limits_{t \to 0; t \neq 0, x_0 + te_1 \in F } \frac{f_i(x_0 + (t_0 + t)e_1) - f_i(x_0 + t_0e_1)}{t}  \\
\end{align*}

由$t_0$的任意性可得
\begin{align*}
  g^\prime(t) = \frac{\partial f_i}{\partial x_1}(x_0 + te_1)
\end{align*}

综上可得,
\begin{align*}
  g(v_1) - g(0) & = f_i(x_0 + v_1e_1) - f_i(x_0)                       \\
                & = g^\prime(t_i)v_1                                   \\
                & = \frac{\partial f_i}{\partial x_1}(x_0 + t_ie_1)v_1
\end{align*}

\begin{zremark}
  特别地,如果$E \subseteq \mathbb{R}^m$,$f: E \to \mathbb{R}$是一个实值函数,并且$f$在$x_0$处的梯度$\nabla f(x_0)$
  被定义为$n$维行向量
  \begin{align*}
    \nabla f(x_0) := \left(\frac{\partial f}{\partial x_1}(x_0),\cdots,\frac{\partial f}{\partial x_n}(x_0)\right)
  \end{align*}
  ,那么只要$x_0$是某个梯度存在且连续的区域的内点,我们就有熟知的公式
  \begin{align*}
    D_vf(x_0) = v \cdot \nabla f(x_0)
  \end{align*}
\end{zremark}

\textbf{说明:}

这里应该是有问题,$v$和$\nabla f(x_0)$都是行向量是不能乘的,
所以,要么把$\nabla f(x_0)$改为列向量,要么把$v \cdot \nabla f(x_0)$改为$\nabla f(x_0) \cdot v$。

\begin{zremark}
  为什么导数矩阵是书中定义的格式(P365)
  $D_f(x_0) = (\frac{\partial f_i}{\partial x_j})_{1 \leq i \leq m; 1 \leq j \leq n}$。
\end{zremark}

按照可微性(定义17.2.2)可知,$f^\prime(x_0)$是一个线性变换,
且$f^\prime(x_0) : \mathbb{R}^n \to \mathbb{R}^m$。
于是按照引理17.1.13可知,存在一个$m \times n$矩阵$A$使得$f^\prime(x_0) = L_A$。
所以,$D_f(x_0) = A$应该是一个$m \times n$矩阵,符合书中的定义。

不妨设
\quickmatrix{m}{n}

因为
\begin{align*}
  \frac{\partial f}{\partial x_j}(x_0)
   & = f^\prime(x_0) e_j                                               \\
   & = L_A e_j                                                         \\
   & = \left(\sum \limits_{k = 1}^n a_{ik}x_k\right)_{1 \leq i \leq m} \\
   & = \left(a_{ij}\right)_{1 \leq i \leq m}                           \\
\end{align*}
这表示的是矩阵$A$的$j$列。

又因为
\begin{align*}
  \frac{\partial f}{\partial x_j}(x_0) = (\frac{\partial f_1}{\partial x_j}(x_0),..., \frac{\partial f_m}{\partial x_j}(x_0))
\end{align*}

综上可得,
\begin{align*}
  a_ij = \frac{\partial f_i}{\partial x_j}(x_0)
\end{align*}

与书中定义一致。



\end{document}


