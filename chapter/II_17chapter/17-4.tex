\documentclass{article}
\usepackage{mathtools} 
\usepackage{fontspec}
\usepackage[UTF8]{ctex}
\usepackage{amsthm}
\usepackage{mdframed}
\usepackage{xcolor}
\usepackage{amssymb}
\usepackage{amsmath}


% 定义新的带灰色背景的说明环境 zremark
\newmdtheoremenv[
  backgroundcolor=gray!10,
  % 边框与背景一致,边框线会消失
  linecolor=gray!10
]{zremark}{说明}

% 通用矩阵命令: \flexmatrix{矩阵名}{元素符号}{行数}{列数}
\newcommand{\flexmatrix}[4]{
  \[
  #1 = \begin{pmatrix}
    #2_{11}     & #2_{12}     & \cdots & #2_{1#4}   \\
    #2_{21}     & #2_{22}     & \cdots & #2_{2#4}   \\
    \vdots      & \vdots      & \ddots & \vdots     \\
    #2_{#31}    & #2_{#32}    & \cdots & #2_{#3#4}
  \end{pmatrix}
  \]
}

% 简化版命令(默认矩阵名为A,元素符号为a): \quickmatrix{行数}{列数}
\newcommand{\quickmatrix}[2]{\flexmatrix{A}{a}{#1}{#2}}


\begin{document}
\title{17.4 习题}
\author{张志聪}
\maketitle

\section*{17.4.1}

先证明可微性,再证明导数作为关于$\mathbb{R}^n$的函数是连续的。

设$L : \mathbb{R}^n \to \mathbb{R}^m$的线性变换,令$L = T$,
于是,对任意$x_0 \in \mathbb{R}^n$,我们有
\begin{align*}
  \lim\limits_{x \to x_0; x \in \mathbb{R}^n - \{x_0\} } \frac{\|T(x) - T(x_0) - L(x - x_0) \|}{\|x - x_0\|}
   & = \lim\limits_{x \to x_0; x \in \mathbb{R}^n - \{x_0\} } \frac{\|T(x - x_0) - L(x - x_0) \|}{\|x - x_0\|} \\
   & = \lim\limits_{x \to x_0; x \in \mathbb{R}^n - \{x_0\} } \frac{\|T(x - x_0) - T(x - x_0) \|}{\|x - x_0\|} \\
   & = \lim\limits_{x \to x_0; x \in \mathbb{R}^n - \{x_0\} } \frac{0}{\|x - x_0\|}                            \\
   & = 0
\end{align*}

所以$T$在$x_0$处是可微的,并且导数为$T$。而且引理17.2.4也保证了导数的唯一性。

又由$x_0$的任意性可知,$T$在任意一点$x$处的导数都是$T$,即$T^\prime(x) = x$,
可见导数函数是定值,所以其是连续的。

\section*{17.4.2}

要证明$f$在$x_0$连续,我们需要证明
\begin{align*}
  \lim\limits_{x \to x_0; x \in E - \{x_0\} } f(x) = f(x_0)
\end{align*}

对任意$\epsilon > 0$,由于$f$在$x_0$处的可微性可知,
存在$\delta > 0$,使得只要$\|x - x_0\| \leq \delta, x \in E - \{x_0\}$,
就有
\begin{align*}
  \frac{\|f(x) - f(x_0) - f^\prime(x_0)(x - x_0)\|}{\|x - x_0\|} < \epsilon                                             \\
  \|f(x) - f(x_0) - f^\prime(x_0)(x - x_0)\| < \epsilon \|x - x_0\|                                                     \\
  \|f(x) - f(x_0)\| - \|f^\prime(x_0)(x - x_0)\| \leq \|f(x) - f(x_0) - f^\prime(x_0)(x - x_0)\| < \epsilon \|x - x_0\| \\
  \|f(x) - f(x_0)\| - \|f^\prime(x_0)(x - x_0)\| < \epsilon \|x - x_0\|                                                 \\
  \|f(x) - f(x_0)\| < \epsilon \|x - x_0\| + \|f^\prime(x_0)(x - x_0)\|
\end{align*}
因为$x_0$是$E$的内点,于是$B(x_0, min(\delta, \epsilon)) \subseteq E$,
令$x \in B(x_0, min(\delta, \epsilon))$。
又由习题17.1.4可知,存在$M > 0$,使得$\|f^\prime(x_0)(x - x_0)\| \leq M \|x - x_0\|$,
于是,我们有
\begin{align*}
  \|f(x) - f(x_0)\| & < \epsilon \|x - x_0\| + \|f^\prime(x_0)(x - x_0)\| \\
                    & \leq \epsilon^2 + M \epsilon
\end{align*}
$M$是定值和$\epsilon$的任意性可知,$f$在$x_0$处连续。

\section*{17.4.3}
不妨设$g^\prime(f(x_0)) = L_1, f^\prime(x_0) = L_2$。

按照可微性的定义(定义17.2.2),我们需要证明
\begin{align*}
  \lim_{x \to x_0; x \in E - \{x_0\}} \frac{\|g \circ f(x) - g \circ f(x_0) - L_1L_2(x - x_0)\|}{\|x - x_0\|} = 0
\end{align*}

因为$g$在$f(x_0)$处可微,
于是对任意的$\epsilon > 0$,存在$\delta > 0$,使得只要$\|y - f(x_0)\| < \delta$,
就有
\begin{align*}
  \frac{\|g(y) - g(f(x_0)) - L_1(y - f(x_0))\|}{\|y - f(x_0)\|} < \epsilon
\end{align*}
又$f$在$x_0$处可微,由习题17.4.2可知,$f$在$x_0$处连续,
所以,存在$\delta_f > 0$,使得只要$\|x - x_0\| < \delta_f$,
就有
\begin{align*}
  \|f(x) - f(x_0)\| < \delta
\end{align*}
综上,当$\|x - x_0\| < \delta_f$,我们有
\begin{align}
  \frac{\|g(f(x)) - g(f(x_0)) - L_1(y - f(x_0))\|}{\|f(x) - f(x_0)\|} < \epsilon
\end{align}

$f$在$x_0$处可微,存在$\delta^\prime > 0$,使得只要$|x - x_0| < min(\delta_f,\delta^\prime)|$,就有
\begin{align}
  \frac{\|f(x) - f(x_0) - L_2(x - x_0)\|}{\|x - x_0\|} < \epsilon
\end{align}
取$\delta = min(\delta_f,\delta^\prime), \|x - x_0 \| < \delta$,式(1)(2)同时成立。

对任意$v \in R^m, w \in R^n$,由习题17.1.4可知,存在$M_1, M_2 > 0$使得
\begin{align*}
  \|L_1 v\| \leq M_1 \|v\| \\
  \|L_2 w\| \leq M_2 \|w\|
\end{align*}

当$\| x - x_0 \| < \delta$时,我们有
\begin{align*}
   & \frac{\|g \circ f(x) - g \circ f(x_0) - L_1L_2(x - x_0)\|}{\|x - x_0\|}                                                              \\
   & = \frac{\|g(f(x)) - g(f(x_0)) - L_1L_2(x - x_0)\|}{\|x - x_0\|}                                                                      \\
   & = \frac{\|g(f(x)) - g(f(x_0)) - L_1(f(x) - f(x_0)) + L_1(f(x) - f(x_0)) - L_1L_2(x - x_0)\|}{\|x - x_0\|}                            \\
   & \leq \frac{\|g(f(x)) - g(f(x_0)) - L_1(f(x) - f(x_0))\|}{\|x - x_0\|} + \frac{\|L_1(f(x) - f(x_0)) - L_1L_2(x - x_0)\|}{\|x - x_0\|} \\
   & = \frac{\|g(f(x)) - g(f(x_0)) - L_1(f(x) - f(x_0))\|}{\|x - x_0\|} + \frac{\|L_1[f(x) - f(x_0) - L_2(x - x_0)]\|}{\|x - x_0\|}       \\
   & \leq \frac{\epsilon \|f(x) - f(x_0)\|}{\|x - x_0\|} + \frac{M_2\|f(x) - f(x_0) - L_2(x - x_0)\|}{\|x - x_0\|}                        \\
   & \leq \frac{\epsilon \|f(x) - f(x_0)\|}{\|x - x_0\|} + \frac{M_2\epsilon \|x - x_0\|}{\|x - x_0\|}                                    \\
   & = \frac{\epsilon \|f(x) - f(x_0) - L_2(x - x_0) + L_2(x - x_0)\|}{\|x - x_0\|} + M_2\epsilon                                         \\
   & \leq \frac{\epsilon \|f(x) - f(x_0) - L_2(x - x_0)\| + \epsilon\|L_2(x - x_0)\|}{\|x - x_0\|} + M_2\epsilon                          \\
   & \leq \frac{\epsilon^2\|x - x_0\| + \epsilon M_1\|x - x_0\|}{\|x - x_0\|} + M_2\epsilon                                               \\
   & = \epsilon(\epsilon + M_1 + M_2)
\end{align*}

由$\epsilon$是任意的,$M_1, M_2$是定值,所以
\begin{align*}
  \lim_{x \to x_0; x \in E - \{x_0\}} \frac{\|g \circ f(x) - g \circ f(x_0) - L_1L_2(x - x_0)\|}{\|x - x_0\|} = 0
\end{align*}
命题得证。

\section*{17.4.4}

\textbf{命题:}

设$f: \mathbb{R}^n \to \mathbb{R}$和$g: \mathbb{R}^n \to \mathbb{R}$都是可微函数,
且$g$在$\mathbb{R}^n$上不为零,那么$f/g$也是可微的,且
\begin{align*}
  \nabla (\frac{f}{g}) = \frac{(\nabla f) g  - f (\nabla g)  }{g^2}
\end{align*}

\textbf{证明:}

仿照例17.4.2进行证明。

把$h: \mathbb{R}^n \to \mathbb{R}^2$定义为$h(x) = (f(x), g(x))$。
现在令$k: \mathbb{R}^2 \to \mathbb{R}$表示除法函数$K(a, b) = \frac{a}{b}$。
注意,
\[
  D_h(x_0) =
  \begin{pmatrix}
    \nabla f(x_0) \\
    \nabla g(x_0)
  \end{pmatrix}
\]
和
\begin{align*}
  D_k(a,b) = (\frac{1}{b}, -\frac{a}{b^2})
\end{align*}
根据链式法则可得,
\begin{align*}
  D(k \circ h)
   & = (\frac{1}{g(x_0)}, -\frac{f(x_0)}{g(x_0)^2})
  \begin{pmatrix}
    \nabla f(x_0) \\
    \nabla g(x_0)
  \end{pmatrix}                                                                      \\
   & = \frac{\nabla f(x_0)}{g(x_0)} - \frac{f(x_0)\nabla g(x_0)}{g(x_0)^2}            \\
   & = \frac{(\nabla f(x_0)) g(x_0)}{g(x_0)^2} - \frac{f(x_0)\nabla g(x_0)}{g(x_0)^2} \\
   & = \frac{(\nabla f(x_0)) g(x_0) - f(x_0)\nabla g(x_0)}{g(x_0)^2}
\end{align*}
命题得证。

\section*{17.4.5}

注意,注意,注意!!! 书中的是点积(或点乘),正文中也没有提及过,坑人!!!

令$k: \mathbb{R}^3 \to \mathbb{R}$表示$l^2$度量函数
$k(a, b, c) = \|(a, b, c)\| = \sqrt{a^2 + b^2 + c^2}$。
我们有
\begin{align*}
  D_{\overrightarrow{x}}(t_0) =
  \begin{pmatrix}
    \frac{\partial \overrightarrow{x}_1}{\partial x_1} (t_0) \\
    \frac{\partial \overrightarrow{x}_2}{\partial x_1} (t_0) \\
    \frac{\partial \overrightarrow{x}_3}{\partial x_1} (t_0)
  \end{pmatrix}
\end{align*}
和
\begin{align*}
  D_k(a, b, c) = (\frac{a}{\sqrt{a^2 + b^2 + c^2}}, \frac{b}{\sqrt{a^2 + b^2 + c^2}}, \frac{c}{\sqrt{a^2 + b^2 + c^2}})
\end{align*}
令$\overrightarrow{x}(t_0) = (c_1, c_2, c_3)$
根据链式法则可得,
\begin{align*}
  D(k \circ \overrightarrow{x})(t_0)
   & = (\frac{c_1}{\|\overrightarrow{x}(t_0)\|}, \frac{c_2}{\|\overrightarrow{x}(t_0)\|}, \frac{c_3}{\|\overrightarrow{x}(t_0)\|})
  \begin{pmatrix}
    \frac{\partial \overrightarrow{x}_1}{\partial x_1} (t_0) \\
    \frac{\partial \overrightarrow{x}_2}{\partial x_1} (t_0) \\
    \frac{\partial \overrightarrow{x}_3}{\partial x_1} (t_0)
  \end{pmatrix}                                                                       \\
   & = \frac{\frac{\partial \overrightarrow{x}_1}{\partial x_1} (t_0) c_1 }{\|\overrightarrow{x}(t_0)\|}
  + \frac{\frac{\partial \overrightarrow{x}_2}{\partial x_1} (t_0) c_2}{\|\overrightarrow{x}(t_0)\|}
  + \frac{\frac{\partial \overrightarrow{x}_3}{\partial x_1} (t_0) c_3}{\|\overrightarrow{x}(t_0)\|}                                 \\
   & = \frac{\overrightarrow{x}^\prime(t_0) \cdot \overrightarrow{x}(t_0)}{r(t_0)}
\end{align*}

因为
\begin{align*}
  \overrightarrow{x}^\prime(t_0) \cdot \overrightarrow{x}(t_0)
  = \frac{\partial \overrightarrow{x}_1}{\partial x_1} (t_0) c_1 
  + \frac{\partial \overrightarrow{x}_2}{\partial x_1} (t_0) c_2
  + \frac{\partial \overrightarrow{x}_3}{\partial x_1} (t_0) c_3
\end{align*}

\end{document}


