\documentclass{article}
\usepackage{mathtools} 
\usepackage{fontspec}
\usepackage[UTF8]{ctex}
\usepackage{amsthm}
\usepackage{mdframed}
\usepackage{xcolor}
\usepackage{amssymb}
\usepackage{amsmath}


% 定义新的带灰色背景的说明环境 zremark
\newmdtheoremenv[
  backgroundcolor=gray!10,
  % 边框与背景一致,边框线会消失
  linecolor=gray!10
]{zremark}{说明}

% 通用矩阵命令: \flexmatrix{矩阵名}{元素符号}{行数}{列数}
\newcommand{\flexmatrix}[4]{
  \[
  #1 = \begin{pmatrix}
    #2_{11}     & #2_{12}     & \cdots & #2_{1#4}   \\
    #2_{21}     & #2_{22}     & \cdots & #2_{2#4}   \\
    \vdots      & \vdots      & \ddots & \vdots     \\
    #2_{#31}    & #2_{#32}    & \cdots & #2_{#3#4}
  \end{pmatrix}
  \]
}

% 简化版命令(默认矩阵名为A,元素符号为a): \quickmatrix{行数}{列数}
\newcommand{\quickmatrix}[2]{\flexmatrix{A}{a}{#1}{#2}}


\begin{document}
\title{17.4 习题}
\author{张志聪}
\maketitle

\section*{17.4.1}

先证明可微性,再证明导数作为关于$\mathbb{R}^n$的函数是连续的。

设$L : \mathbb{R}^n \to \mathbb{R}^m$的线性变换,令$L = T$,
于是,对任意$x_0 \in \mathbb{R}^n$,我们有
\begin{align*}
  \lim\limits_{x \to x_0; x \in \mathbb{R}^n - \{x_0\} } \frac{\|T(x) - T(x_0) - L(x - x_0) \|}{\|x - x_0\|}
   & = \lim\limits_{x \to x_0; x \in \mathbb{R}^n - \{x_0\} } \frac{\|T(x - x_0) - L(x - x_0) \|}{\|x - x_0\|} \\
   & = \lim\limits_{x \to x_0; x \in \mathbb{R}^n - \{x_0\} } \frac{\|T(x - x_0) - T(x - x_0) \|}{\|x - x_0\|} \\
   & = \lim\limits_{x \to x_0; x \in \mathbb{R}^n - \{x_0\} } \frac{0}{\|x - x_0\|}                            \\
   & = 0
\end{align*}

所以$T$在$x_0$处是可微的,并且导数为$T$。而且引理17.2.4也保证了导数的唯一性。

又由$x_0$的任意性可知,$T$在任意一点$x$处的导数都是$T$,即$T^\prime(x) = x$,
可见导数函数是定值,所以其是连续的。

\section*{17.4.2}

要证明$f$在$x_0$连续,我们需要证明
\begin{align*}
  \lim\limits_{x \to x_0; x \in E - \{x_0\} } f(x) = f(x_0)
\end{align*}

\end{document}


