\documentclass{article}
\usepackage{mathtools} 
\usepackage{fontspec}
\usepackage[UTF8]{ctex}
\usepackage{amsthm}
\usepackage{mdframed}
\usepackage{xcolor}
\usepackage{amssymb}
\usepackage{amsmath}


% 定义新的带灰色背景的说明环境 zremark
\newmdtheoremenv[
  backgroundcolor=gray!10,
  % 边框与背景一致,边框线会消失
  linecolor=gray!10
]{zremark}{说明}

% 通用矩阵命令: \flexmatrix{矩阵名}{元素符号}{行数}{列数}
\newcommand{\flexmatrix}[4]{
  \[
  #1 = \begin{pmatrix}
    #2_{11}     & #2_{12}     & \cdots & #2_{1#4}   \\
    #2_{21}     & #2_{22}     & \cdots & #2_{2#4}   \\
    \vdots      & \vdots      & \ddots & \vdots     \\
    #2_{#31}    & #2_{#32}    & \cdots & #2_{#3#4}
  \end{pmatrix}
  \]
}

% 简化版命令(默认矩阵名为A,元素符号为a): \quickmatrix{行数}{列数}
\newcommand{\quickmatrix}[2]{\flexmatrix{A}{a}{#1}{#2}}


\begin{document}
\title{17.8 注释}
\author{张志聪}
\maketitle

\begin{zremark}
  $F(x_1, x_2, \cdots, x_n) : = (x_1, x_2, \cdots, f(x_1, x_2, \cdots, x_n))$,
  其中$f$是连续可微的。
  $F^{-1}(y) = (h_1(y), h_2(y), \cdots, h_n(y))$,
  因为$F^{-1}$在$(y_1,\cdots,y_{n - 1}, 0)$处可微,
  所以$h_n$也在$(y_1,\cdots,y_{n - 1}, 0)$处可微。
\end{zremark}

\textbf{证明:}

17-2-comment.txt中有说明。

\end{document}