\documentclass{article}
\usepackage{mathtools} 
\usepackage{fontspec}
\usepackage[UTF8]{ctex}
\usepackage{amsthm}
\usepackage{mdframed}
\usepackage{xcolor}
\usepackage{amssymb}
\usepackage{amsmath}


% 定义新的带灰色背景的说明环境 zremark
\newmdtheoremenv[
  backgroundcolor=gray!10,
  % 边框与背景一致,边框线会消失
  linecolor=gray!10
]{zremark}{说明}

% 通用矩阵命令: \flexmatrix{矩阵名}{元素符号}{行数}{列数}
\newcommand{\flexmatrix}[4]{
  \[
  #1 = \begin{pmatrix}
    #2_{11}     & #2_{12}     & \cdots & #2_{1#4}   \\
    #2_{21}     & #2_{22}     & \cdots & #2_{2#4}   \\
    \vdots      & \vdots      & \ddots & \vdots     \\
    #2_{#31}    & #2_{#32}    & \cdots & #2_{#3#4}
  \end{pmatrix}
  \]
}

% 简化版命令(默认矩阵名为A,元素符号为a): \quickmatrix{行数}{列数}
\newcommand{\quickmatrix}[2]{\flexmatrix{A}{a}{#1}{#2}}


\begin{document}
\title{17.5 注释}
\author{张志聪}
\maketitle

\begin{zremark}
  定理17.5.4(克莱罗定理)证明过程中的,有
  \begin{align*}
    f(\delta e_i + \delta e_j) - f(\delta e_j)
    = \int_{0}^{\delta} \frac{\partial f}{\partial x_i} (x_ie_i + \delta e_j) d_{x_i}
  \end{align*}
\end{zremark}

\textbf{证明:}

要利用定理11.9.4(微积分第二基本定理),我们先要确定
$\frac{\partial f}{\partial x_i} (x_ie_i + \delta e_j)$的原函数。

先解释下右侧的表达式,方便理解。首先
\begin{align*}
  \frac{\partial f}{\partial x_i}
\end{align*}
是一个$\mathbb{R}^n \to \mathbb{R}^m$的函数。
于是右侧就是沿着路径$\gamma (t) = te_i + \delta e_j \ (t \in [0, \delta])$的积分,
即$\int_{0}^{\delta} \frac{\partial f}{\partial x_i} \circ \gamma (t) dt$。

将$f$的第$j$个变量固定为$\delta$,其余变量(除了$x_i$)固定为$0$,
为了找到原函数,我们定义一个辅助函数
\begin{align*}
  g(x_i) : = f(x_ie_i + \delta e_j), \ x_i \in [0, \delta]
\end{align*}
$g$是一元函数,而且其导数为:
\begin{align*}
  g^\prime(x_i)
   & = \lim \limits_{x \to x_i} \frac{g(x) - g(x_i)}{x - x_i}                                           \\
   & = \lim\limits_{t \to 0;t \neq 0} \frac{g(x_i + t) - g(x_i)}{t}                                     \\
   & = \lim\limits_{t \to 0;t \neq 0} \frac{f((x_i + t)e_i + \delta e_j) - f(x_ie_i + \delta e_j)}{t}   \\
   & = \lim\limits_{t \to 0;t \neq 0} \frac{f(x_i e_i + \delta e_j + te_i) - f(x_ie_i + \delta e_j)}{t} \\
   & = \frac{\partial f}{\partial x_i} (x_i e_i + \delta e_j)
\end{align*}
$g$就是原函数。

然后,利用利用定理11.9.4(微积分第二基本定理)

\begin{align*}
   & \int_{0}^{\delta} \frac{\partial f}{\partial x_i} (x_i e_i + \delta e_j) \\
   & = \int_{0}^{\delta} g^\prime(x_i)                                        \\
   & = g(\delta) - g(0)                                                       \\
   & = f(\delta e_i + \delta e_j) - f(\delta e_j)
\end{align*}

\begin{zremark}
  定理17.5.4(克莱罗定理)证明过程中的,
  $f \in C^2, f: \mathbb{R}^n \to \mathbb{R}^m$,
  我们有,由平均值定理可知,对于每一个$x_i$,都存在一个$0 \leq r \leq \delta$,使得
  \begin{align*}
    \frac{\partial f}{\partial x_i}(x_i e_i + \delta e_j) - \frac{\partial f}{\partial x_i}(x_i e_i)
    = \delta \frac{\partial}{\partial x_j} \frac{\partial f}{\partial x_i} (x_ie_i + r e_j)
  \end{align*}
\end{zremark}

\textbf{证明:}

令$g : = \frac{\partial f}{\partial x_i}(x_i e_i + t e_j), t \in [0, \delta]$,
令$w = \frac{\partial f}{\partial x_i}; h : = x_i e_i + te_j$,
于是$g = w \circ h$,
使用链式法则,并令$e_j = (v_1, v_2, \cdots, v_n)$。
\begin{align*}
  g^\prime(t) & = w^\prime (x_ie_i + t e_j) h^\prime(t)                                          \\
              & = w^\prime (x_ie_i + t e_j) e_j                                                  \\
              & = \frac{\partial w}{\partial x_j}(x_ie_i + t e_j)                                \\
              & = \frac{\partial}{\partial x_j} \frac{\partial f}{\partial x_i} (x_ie_i + t e_j)
\end{align*}
接下来,对平均值定理的使用就无需赘述了。

\end{document}