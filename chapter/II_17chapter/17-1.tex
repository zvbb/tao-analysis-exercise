\documentclass{article}
\usepackage{mathtools} 
\usepackage{fontspec}
\usepackage[UTF8]{ctex}
\usepackage{amsthm}
\usepackage{mdframed}
\usepackage{xcolor}
\usepackage{amssymb}
\usepackage{amsmath}


% 定义新的带灰色背景的说明环境 zremark
\newmdtheoremenv[
  backgroundcolor=gray!10,
  % 边框与背景一致,边框线会消失
  linecolor=gray!10
]{zremark}{说明}

% 通用矩阵命令: \flexmatrix{矩阵名}{元素符号}{行数}{列数}
\newcommand{\flexmatrix}[4]{
  \[
  #1 = \begin{pmatrix}
    #2_{11}     & #2_{12}     & \cdots & #2_{1#4}   \\
    #2_{21}     & #2_{22}     & \cdots & #2_{2#4}   \\
    \vdots      & \vdots      & \ddots & \vdots     \\
    #2_{#31}    & #2_{#32}    & \cdots & #2_{#3#4}
  \end{pmatrix}
  \]
}

% 简化版命令(默认矩阵名为A,元素符号为a): \quickmatrix{行数}{列数}
\newcommand{\quickmatrix}[2]{\flexmatrix{A}{a}{#1}{#2}}


\begin{document}
\title{17.1 习题}
\author{张志聪}
\maketitle

\section*{17.1.1}

略

\section*{17.1.2}

证明$TS$是否满足线性变换的定义(定义17.1.6)。

\begin{itemize}
  \item (可加性)对任意的$x, x^\prime \in \mathbb{R}^p$,都有$TS(x + x^\prime) = TS(x) + TS(x^\prime)$。

        因为$T,S$都是线性变换,我们有
        \begin{align*}
          TS(x + x^\prime) & = T(S(x + x^\prime))       \\
                           & = T(S(x) + S(x^\prime))    \\
                           & = T(S(x)) + T(S(x^\prime)) \\
                           & = TS(x) + TS(x^\prime)
        \end{align*}

  \item (齐次性)对于任意的$x \in \mathbb{R}^p$和任意的$c \in \mathbb{R}$,都有$TS(cx) = cTS(x)$。

        因为$T,S$都是线性变换,我们有
        \begin{align*}
          TS(cx) & = T(S(cx)) \\
                 & = T(cS(x)) \\
                 & = cT(S(x))
        \end{align*}
\end{itemize}

\section*{17.1.3}

由定义17.1.11可知,$AB$是一个$m \times p$的矩阵。
我们需证明对任意$x = (x_r)_{1 \leq r \leq p} \in \mathbb{R}^p$,都有
\begin{align*}
  L_AL_B(x) = L_{AB}(x)
\end{align*}

\quickmatrix{m}{n}
\flexmatrix{B}{b}{n}{p}

由矩阵的乘积的定义(定义17.1.11),我们有
\flexmatrix{AB}{c}{m}{p}

\text{其中 } $c_{ik} = \sum\limits_{j=1}^n a_{ij}b_{jk}$。

\begin{align*}
  L_{AB} (x) & = L_{AB}(x_r)_{1 \leq r \leq p}                                                         \\
             & = \left( \sum \limits_{r = 1}^p c_{ir}x_r\right)_{1 \leq i \leq m}                      \\
             & = \left( \sum \limits_{r = 1}^p (\sum_{j=1}^n a_{ij}b_{jr})x_r\right)_{1 \leq i \leq m} \\
\end{align*}

\begin{align*}
  L_AL_B(x) & = L_A(L_B(x))                                                                                              \\
            & = L_A\left(\left( \sum\limits_{r=1}^p b_{jr}x_r \right)_{1 \leq j \leq n}\right)                           \\
            & = \left( \sum\limits_{j = 1}^n a_{ij}\left( \sum\limits_{r=1}^p b_{jr}x_r \right)\right)_{1 \leq i \leq m} \\
            & = \left( \sum \limits_{r = 1}^p (\sum_{j=1}^n a_{ij}b_{jr})x_r\right)_{1 \leq i \leq m}                    \\
            & = L_{AB}(x)
\end{align*}

\section*{17.1.4}

符号$\|\|$和$\|\|_2$都是表示范数的。但它们的含义具体取决于上下文。因为$Tx$属于欧几里得空间里的元素,所以这里是欧几里得范数。
即:
\begin{align*}
  \|x\| = \sqrt{\sum\limits_{i=1}^n x_i^2}
\end{align*}

(1) $\|Tx\| \leq M \|x\|$。

由引理17.1.13可知,恰好存在一个$m \times n$矩阵$A$使得$T = L_A$。
不妨设
\quickmatrix{m}{n}

于是
\begin{align*}
  \|Tx\| & = \|L_A x\|                                                                                             \\
         & = \left\| \left( \sum\limits_{j = 1}^n a_{ij} x_j \right)_{1 \leq i \leq m} \right\|                    \\
         & = \sqrt{\sum\limits_{i=1}^m \left( \sum\limits_{j = 1}^n a_{ij} x_j \right)^2}                          \\
         & \leq \sqrt{\sum\limits_{i=1}^m \sum\limits_{j = 1}^n a_{ij}^2 x_j^2}                                    \\
         & = \sqrt{\sum\limits_{j = 1}^n \sum\limits_{i=1}^m a_{ij}^2 x_j^2}                                       \\
         & = \sqrt{\sum\limits_{j = 1}^n x_j^2 \sum\limits_{i=1}^m a_{ij}^2}                                       \\
         & \leq \sqrt{\sum\limits_{j = 1}^n \left(\sum\limits_{k = 1}^n x_k^2\right) \sum\limits_{i=1}^m a_{ij}^2} \\
         & = \sqrt{\sum\limits_{k = 1}^n x_k^2 \sum\limits_{j = 1}^n \sum\limits_{i=1}^m a_{ij}^2}                 \\
         & = \sqrt{\sum\limits_{k = 1}^n x_k^2} \sqrt{ \sum\limits_{j = 1}^n \sum\limits_{i=1}^m a_{ij}^2}         \\
         & = \|x\| \sqrt{\sum\limits_{j = 1}^n \sum\limits_{i=1}^m a_{ij}^2}                                       \\
         & = M \|x\|
\end{align*}
其中$M = \sqrt{\sum\limits_{j = 1}^n \sum\limits_{i=1}^m a_{ij}^2}$。

(2) 从$\mathbb{R}^n$到$\mathbb{R}^m$的每一个线性变换都是连续的。

由P253可知,本书在默认情况下,只要提到度量空间$\mathbb{R}^n$,指的都是欧几里得度量(也称$l^2$度量)。

P252的定义可知,任意$x, y \in \mathbb{R}$有$d_l^2(x, y) = \|x - y\|$。

对任意$x_0 \in \mathbb{R}^n$,设$(x^{(n)})_{n = 1}^\infty$是度量空间$(\mathbb{R}^n, d_{l^2})$
中收敛于$x_0$的序列。$T$是任意$\mathbb{R}^n$到$\mathbb{R}^m$的线性变换。

对任意$\epsilon > 0$,存在$N \geq 1$,使得只要$n \geq N$,就有
\begin{align*}
  d_{l^2} (x^{(n)}, x_0) = \|x^{(n)} - x_0\| \leq \epsilon
\end{align*}

利用(1),我们有
\begin{align*}
  d_{l^2} (T(x^{(n)}), T(x_0))
   & = \|T(x^{(n)} - x_0)\|   \\
   & \leq M \|x^{(n)} - x_0\| \\
   & \leq M \epsilon
\end{align*}

由于$M$是定值且$\epsilon$是任意的,因此$(T(x^{(n)}))_{n = 1}^\infty$收敛于$T(x_0)$。
又因为$x_0$是任意的,因此$T$是连续的。



\end{document}


