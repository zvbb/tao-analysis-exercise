\documentclass{article}
\usepackage{mathtools} 
\usepackage{fontspec}
\usepackage[UTF8]{ctex}
\usepackage{amsthm}
\usepackage{mdframed}
\usepackage{xcolor}
\usepackage{amssymb}
\usepackage{amsmath}


% 定义新的带灰色背景的说明环境 zremark
\newmdtheoremenv[
  backgroundcolor=gray!10,
  % 边框与背景一致,边框线会消失
  linecolor=gray!10
]{zremark}{说明}

% 通用矩阵命令: \flexmatrix{矩阵名}{元素符号}{行数}{列数}
\newcommand{\flexmatrix}[4]{
  \[
  #1 = \begin{pmatrix}
    #2_{11}     & #2_{12}     & \cdots & #2_{1#4}   \\
    #2_{21}     & #2_{22}     & \cdots & #2_{2#4}   \\
    \vdots      & \vdots      & \ddots & \vdots     \\
    #2_{#31}    & #2_{#32}    & \cdots & #2_{#3#4}
  \end{pmatrix}
  \]
}

% 简化版命令(默认矩阵名为A,元素符号为a): \quickmatrix{行数}{列数}
\newcommand{\quickmatrix}[2]{\flexmatrix{A}{a}{#1}{#2}}


\begin{document}
\title{17.6 习题}
\author{张志聪}
\maketitle

\section*{17.6.1}

\begin{itemize}
  \item (1) $f$是压缩映射。

        令$x_0, x_1 \in [a, b]$
        $x_0 = x_1$显然满足压缩映射的条件,
        我们接下来关注$x_0 < x_1$的情况。

        在区间$[x_0, x_1]$,由推论10.2.9(中值定理)可知,
        存在$x \in [x_0, x_1]$,使得
        \begin{align*}
          \frac{f(x_1) - f(x_0)}{x_1 - x_0} = f^\prime(x) \\
          f(x_1) - f(x_0) = f^\prime(x)(x_1 - x_0)        \\
          |f(x_1) - f(x_0)| = |f^\prime(x)||x_1 - x_0|
        \end{align*}
        由题设可知$|f^\prime(x)| \leq 1$,
        所以
        \begin{align*}
          |f(x_1) - f(x_0)| \leq |x_1 - x_0|
        \end{align*}
        所以,$f$是压缩映射。

  \item (2) $|f^\prime(x)| < 1, f^\prime$连续的,那么$f$是一个严格压缩映射。

        如果$f$是常值函数,那么对任意$x \in [a, b]$,都有$f^\prime(x) = 0$,显然命题成立。

        我们接下来关注$f$不是常值函数的情况。

        由(1)的证明过程可知,对任意$x_0, x_1 \in [a, b]$,
        存在$x \in (x_0, x_1)$有
        \begin{align*}
          |f(x_1) - f(x_0)| = |f^\prime(x)||x_1 - x_0|
        \end{align*}
        因为$f^\prime$连续,由命题9.6.7可知(最大值原理),
        存在$x_{max}, x_{min} \in [a, b]$使得$f^\prime$达到
        最大值$f^\prime(x_{max})$和最小值$f^\prime(x_{min})$。
        令$c = max(|f^\prime(x_{max})|, |f^\prime(x_{min})|)$,
        于是对任意$x$都有$|f^\prime(x)| \leq c < 1$,所以
        \begin{align*}
          |f(x_1) - f(x_0)| = |f^\prime(x)||x_1 - x_0| \leq c |x_1 - x_0|
        \end{align*}
        所以$f$是一个严格压缩映射。

\end{itemize}

\section*{17.6.2}

反证法,假设存在$x_0$使得$|f^\prime(x_0)| > 1$。

以$f^\prime(x_0) > 1$为例,
对$\epsilon = \frac{1}{2}(f^\prime(x_0) - 1)$,存在$\delta > 0$使得只要
$|x - x_0| < \delta$,就有
\begin{align*}
  |f(x) - f(x_0) - f^\prime(x_0)(x - x_0)| \leq \epsilon |x - x_0|                                                      \\
  |f^\prime(x_0)||x - x_0| - \epsilon |x - x_0| \leq |f(x) - f(x_0)| \leq |f^\prime(x_0)||x - x_0| + \epsilon |x - x_0| \\
  (|f^\prime(x_0)| - \epsilon)|x - x_0| \leq |f(x) - f(x_0)|                                                            \\
  (\frac{1}{2}f^\prime(x_0) + 1)|x - x_0| \leq |f(x) - f(x_0)|
\end{align*}
因为$(\frac{1}{2}f^\prime(x_0) + 1) > 1$,
这与$f$是压缩映射矛盾。

\section*{17.6.3}

$f: [0, \frac{\pi}{2}] \to \mathbb{R}, f(x) = \sin(x)$。

$f$在$[0, \frac{\pi}{2}]$上可微。

任意$x, y \in [0, \frac{\pi}{2}]$都有
\begin{align*}
  |f(x) - f(y)|
   & = |sin(x) - sin(y)|                                                                        \\
   & = |sin(\frac{x + y}{2} + \frac{x - y}{2}) - sin(\frac{x + y}{2} - \frac{x - y}{2})|        \\
   & = |sin(\frac{x + y}{2})cos(\frac{x - y}{2}) + cos(\frac{x + y}{2})sin(\frac{x - y}{2})     \\
   & - (sin(\frac{x + y}{2})cos(-\frac{x - y}{2}) + cos(\frac{x + y}{2})sin(-\frac{x - y}{2}))| \\
   & = |2 cos(\frac{x + y}{2})sin(\frac{x - y}{2})|                                             \\
   & \leq |2 sin(\frac{x - y}{2})|                                                              \\
   & < 2 |\frac{x - y}{2}| = |x - y|
\end{align*}

因为$f^\prime(x) = sin^\prime(x) = cos(x)$,
于是当$x = 0$时$f^\prime(0) = cos(0) = 1$。

\section*{17.6.4}

$f: [-1, 1] \to \mathbb{R}, f(x) = \frac{|x|}{2}$。

$f$在$x = 0$处是不可微的,
因为左右导数不相等:

左导数
\begin{align*}
  \lim\limits_{x \to 0^-} \frac{f(x) - f(0)}{x - 0}
   & = \lim\limits_{x \to 0^-} \frac{\frac{-x}{2} - 0}{x} \\
   & = -\frac{1}{2}
\end{align*}
右导数
\begin{align*}
  \lim\limits_{x \to 0^+} \frac{f(x) - f(0)}{x - 0}
   & = \lim\limits_{x \to 0^+} \frac{\frac{x}{2} - 0}{x} \\
   & = \frac{1}{2}
\end{align*}

\section*{17.6.5}

\begin{align*}
  f^\prime(x) & = (x - x^2)^\prime \\
              & = 1 - 2x
\end{align*}
于是任意$x \in [0, 1]$,我们有
\begin{align*}
  |f^\prime(x)| \leq 1
\end{align*}
由习题17.6.1可知$f$是一个压缩映射。

注意,现在我们还不能说$f$不是严格压缩映射。

反证法,假设存在$0 < c < 1$使得$|f(x) - f(y)| \geq c|x - y|$。

\begin{align*}
  |f(x) - f(y)|
   & = |x - x^2 - y + y^2|        \\
   & = |x - y + y^2 - x^2|        \\
   & = |(x - y) + (y + x)(y - x)| \\
   & = |(x - y)[1 - (y + x)]|     \\
   & = |x - y| |1 - (y + x)|      \\
\end{align*}
因为$x, y \in [0, 1]$,所以$|1 - (y + x)| \in [0, 1]$,
因为$0 < c < 1$是一个定值,那么,存在
\begin{align*}
  |1 - (y + x)| > c
\end{align*}
于是
\begin{align*}
  |f(x) - f(y)| = |x - y| |1 - (y + x)| > c |x - y|
\end{align*}
存在矛盾。

\section*{17.6.6}

设$f$是$X$上的压缩映射任意压缩映射。

任意$x_0 \in X$,设$(x^{(n)})_{n = 1}^\infty$是$X$中收敛于
$x_0$的序列。

对任意$\epsilon > 0$,存在$N \geq 1$使得只要$n \geq N$,都有
\begin{align*}
  |x^{(n)} - x_0| < \epsilon
\end{align*}
又因为$f$是$X$上的压缩映射,于是
\begin{align*}
  |f(x^{(n)}) - f(x_0)| \leq |x^{(n)} - x_0| < \epsilon
\end{align*}
所以,序列$(f(x^{(n)}))_{n = 1}^\infty$收敛于$f(x_0)$。

由定理13.1.4(b)可知,$f$在$x_0$处连续。
因为$x_0$的任意性可知$f$在是连续的。

\section*{17.6.7}

\begin{itemize}
  \item (1) $f$最多有一个不动点。

        反证法,假设$f$不止一个不动点。
        设$x_0, x_1$是$f$的不动点,即
        \begin{align*}
          x_0 = f(x_0) \\
          x_1 = f(x_1)
        \end{align*}

        于是
        \begin{align*}
          d(f(x_1), f(x_0)) & = d(x_1, x_0)
        \end{align*}
        这与$f$是严格压缩映射矛盾。

  \item (2) $X$是一个非空的完备空间,那么$f$恰好有一个不动点。

        因为$X$是非空的,可以任取$x_0 \in X$,递归地定义
        \begin{align*}
           & x_1 = f(x_0) \\
           & x_2 = f(x_1) \\
           & x_3 = f(x_2) \\
           & \vdots       \\
        \end{align*}
        于是
        \begin{align*}
          d(x_{n + 1}, x_n) \leq c^n d(x1, x_0)
        \end{align*}
        (通过归纳法证明)

        由引理7.3.3可知,$\sum\limits_{n = 0}^\infty c^n d(x_1, x_0) = \frac{d(x_1, x_0)}{1 - c}$。

        对任意$\epsilon > 0$,存在$N \geq 1$使得只要$p-1, q \geq N$其中$q > p$,都有
        \begin{align*}
          \sum\limits_{n = 0}^q c^n d(x_1, x_0) - \sum\limits_{n = 0}^{p - 1} c^n d(x_1, x_0)
          = d(x_{p+1}, x_p) + d(x_{p+2}, x_{p+1}) + \cdots + d(x_{q}, x_{q-1}) \leq \epsilon
        \end{align*}
        因为
        \begin{align*}
          d(x_q, x_p) \leq d(x_{p+1}, x_p) + d(x_{p+2}, x_{p+1}) + \cdots + d(x_{q}, x_{q-1}) \leq \epsilon
        \end{align*}
        又$p = q$时,$d(x_q, x_p) = 0 \leq \epsilon$。

        综上可得$(x_n)_{n = 0}^\infty$是柯西序列。
        因为$X$是完备的度量空间,所以$(x_n)_{n = 0}^\infty$收敛于某个值$x^\prime \in X$。

        接下来证明$x^\prime$是$f$的不动点。

        因为$(x_n)_{n = 0}^\infty$收敛于$x^\prime$,
        那么,对任意$\epsilon > 0$,存在$N_0 \geq 0$,
        使得只要$n \geq N_0$都有$d(x_n, x^\prime) < \frac{1}{2}\epsilon$,
        又因为$f$是一个严格压缩映射,所以
        \begin{align*}
          d(f(x^\prime), f(x_n)) \leq c d(x^\prime, x_n) \leq  c \frac{1}{2} \epsilon < \frac{1}{2} \epsilon
        \end{align*}

        又由$f(x_n)$的构造方式可得$(f(x_n))_{n = 0}^\infty$也收敛于$x^\prime$。
        那么,对任意$\epsilon > 0$,存在$N_1 \geq 0$,
        使得只要$n \geq N_1$都有$d(f(x_n), x^\prime) < \frac{1}{2}\epsilon$。

        综上,取$N = \max(N_0, N_1)$,$n \geq N$,我们有
        \begin{align*}
          d(f(x^\prime), x^\prime)
           & \leq d(f(x^\prime), f(x_n)) + d(f(x_n), x^\prime)      \\
           & < \frac{1}{2}\epsilon + \frac{1}{2}\epsilon = \epsilon
        \end{align*}
        由$\epsilon$的任意性可知,$x^\prime = f(x^\prime)$。
\end{itemize}

\section*{17.6.8 $\circledast$}
\begin{align*}
  d(x_0, y_0) = d(f(x_0), g(y_0)) = d(f(x_0), f(y_0)) + d(f(y_0), g(y_0)) \leq cd(x_0, y_0) + \epsilon \\
  d(x_0, y_0) = d(f(x_0), g(y_0)) = d(f(x_0), g(x_0)) + d(g(x_0), g(y_0)) \leq \epsilon + c^\prime d(x_0, y_0)
\end{align*}
可得
\begin{align*}
  d(x_0, y_0) \leq \frac{\epsilon}{1 - c} \\
  d(x_0, y_0) \leq \frac{\epsilon}{1 - c^\prime}
\end{align*}
所以
\begin{align*}
  d(x_0, y_0) \leq \frac{\epsilon}{1 - \min(c, c^\prime)}
\end{align*}

\end{document}