\documentclass{article}
\usepackage{mathtools} 
\usepackage{fontspec}
\usepackage[UTF8]{ctex}
\usepackage{amsthm}
\usepackage{mdframed}
\usepackage{xcolor}
\usepackage{amssymb}
\usepackage{amsmath}


% 定义新的带灰色背景的说明环境 zremark
\newmdtheoremenv[
  backgroundcolor=gray!10,
  % 边框与背景一致,边框线会消失
  linecolor=gray!10
]{zremark}{说明}

% 通用矩阵命令: \flexmatrix{矩阵名}{元素符号}{行数}{列数}
\newcommand{\flexmatrix}[4]{
  \[
  #1 = \begin{pmatrix}
    #2_{11}     & #2_{12}     & \cdots & #2_{1#4}   \\
    #2_{21}     & #2_{22}     & \cdots & #2_{2#4}   \\
    \vdots      & \vdots      & \ddots & \vdots     \\
    #2_{#31}    & #2_{#32}    & \cdots & #2_{#3#4}
  \end{pmatrix}
  \]
}

% 简化版命令(默认矩阵名为A,元素符号为a): \quickmatrix{行数}{列数}
\newcommand{\quickmatrix}[2]{\flexmatrix{A}{a}{#1}{#2}}


\begin{document}
\title{17.4 注释}
\author{张志聪}
\maketitle

\begin{zremark}
  作为链式法则和引理17.1.16(以及引理17.1.13)的一个推论,我们得到
  \begin{align*}
    D(g \circ f)(x_0) = Dg(f(x_0)) Df(x_0)
  \end{align*}
\end{zremark}

\textbf{证明:}

由链式法则,我们有
\begin{align*}
  (g \circ f)^\prime(x_0) = g^\prime(f(x_0)) f^\prime(x_0)
\end{align*}
$(g \circ f)^\prime(x_0)$是一个线性变换,
由P364中对导数矩阵的构造方式可知
\begin{align*}
  (g \circ f)^\prime(x_0) = L_{D(g \circ f)(x_0)}
\end{align*}
同理可得
\begin{align*}
  g^\prime(f(x_0)) & = L_{Dg(f(x_0))} \\
  f^\prime(x_0)    & = L_{Df(x_0)}
\end{align*}
综上,
\begin{align*}
  L_{D(g \circ f)(x_0)} = L_{Dg(f(x_0))} L_{Df(x_0)}
\end{align*}
利用引理17.1.16,我们得到
\begin{align*}
  L_{D(g \circ f)(x_0)} = L_{Dg(f(x_0)) Df(x_0)} 
\end{align*}
由引理17.1.13可知线性变换存在唯一的对应矩阵,于是我们有
\begin{align*}
  D(g \circ f)(x_0) = Dg(f(x_0)) Df(x_0)
\end{align*}

\begin{zremark}
  $D(fg) = \nabla (fg)$
\end{zremark}

\textbf{证明:}

$k \circ h = fg: \mathbb{R}^n \to \mathbb{R}$,
按照书中关于梯度的定义(P364),我们有
\begin{align*}
  \nabla (fg) = \left(\frac{\partial fg}{\partial x_1}, \cdots, \frac{\partial fg}{\partial x_n}\right)
\end{align*}
而$D(fg)$也将是$1 \times n$的矩阵。而且
\begin{align*}
  D(fg) & = \left(\frac{\partial fg}{\partial x_1}, \cdots, \frac{\partial fg}{\partial x_n}\right) \\
        & = \nabla (fg)
\end{align*}

\end{document}


