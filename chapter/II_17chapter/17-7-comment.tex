\documentclass{article}
\usepackage{mathtools} 
\usepackage{fontspec}
\usepackage[UTF8]{ctex}
\usepackage{amsthm}
\usepackage{mdframed}
\usepackage{xcolor}
\usepackage{amssymb}
\usepackage{amsmath}


% 定义新的带灰色背景的说明环境 zremark
\newmdtheoremenv[
  backgroundcolor=gray!10,
  % 边框与背景一致,边框线会消失
  linecolor=gray!10
]{zremark}{说明}

% 通用矩阵命令: \flexmatrix{矩阵名}{元素符号}{行数}{列数}
\newcommand{\flexmatrix}[4]{
  \[
  #1 = \begin{pmatrix}
    #2_{11}     & #2_{12}     & \cdots & #2_{1#4}   \\
    #2_{21}     & #2_{22}     & \cdots & #2_{2#4}   \\
    \vdots      & \vdots      & \ddots & \vdots     \\
    #2_{#31}    & #2_{#32}    & \cdots & #2_{#3#4}
  \end{pmatrix}
  \]
}

% 简化版命令(默认矩阵名为A,元素符号为a): \quickmatrix{行数}{列数}
\newcommand{\quickmatrix}[2]{\flexmatrix{A}{a}{#1}{#2}}


\begin{document}
\title{17.7 注释}
\author{张志聪}
\maketitle

\begin{zremark}
  “化归”技巧,用于将一般问题转化为已解决的特殊问题:
  \begin{itemize}
    \item (1) 只证明$f(x_0) = 0$,如何解决一般性问题?
    \item (2) 只证明$x_0 = 0$,如何解决一般性问题?
  \end{itemize}
\end{zremark}

\textbf{证明:}

\begin{itemize}
  \item (1)

        对于一般的$f$,即$x_0$处$f(x_0) \neq 0$的情况,
        通过把$f$替换成
        \begin{align*}
          \tilde{f} := f(x) - f(x_0)
        \end{align*}
        此时,$\tilde{f}(x_0) = f(x_0) - f(x_0) = 0$,
        其他$x \in E, x \neq x_0$的情况下
        \begin{align*}
          \tilde{f}(x) = f(x) - f(x_0)
        \end{align*}
        这里要知道,$f(x_0)$是定值,
        于是可知,相比于我们要研究的$f$本身,
        $\tilde{f}(x)$的值域发生了偏移,
        相对于$f$,偏移了$f(x_0)$(注意:不一定向左,因为$f(x_0)$的正负不确定)。

        至于反函数,设$f(x) = y$,于是$f^{-1}(y) = x$,
        我们有
        \begin{align*}
          f^{-1}(y) & = x                             \\
                    & = \tilde{f}^{-1}(f(x) - f(x_0)) \\
                    & = \tilde{f}^{-1}(y - f(x_0))
        \end{align*}
        综上,我们可以通过研究$\tilde{f}$来了解$f$。

  \item (2)

        对于一般的$f$,我们讨论的$x_0$可能不等于$0$。
        通过把$f$替换成
        \begin{align*}
          \tilde{f}(x) : = f(x + x_0)
        \end{align*}
        此时,$\tilde{f}(0) = f(0 + x_0) = f(x_0)$,
        于是对$f$在$x_0$处的研究,就转变成了对$\tilde{f}$在$0$点处的研究。

        这样的替换是有副作用的,因为$x \in E$,那么$x + x_0$则相对于$E$平移了$x_0$。

        值域反函数,设$f(x + x_0) = y$,于是$f^{-1}(y) = x + x_0$,
        我们有
        \begin{align*}
          f^{-1}(y) - x_0 & = x                            \\
                          & = \tilde{f}^{-1}(\tilde{f}(x)) \\
                          & = \tilde{f}^{-1}(f(x + x_0))   \\
                          & = \tilde{f}^{-1}(y)
        \end{align*}
        于是$f^{-1}(y) = \tilde{f}^{-1}(y) + x_0$。

        综上,我们可以通过研究$\tilde{f}$来了解$f$。
\end{itemize}

\begin{zremark}
  书中$I^\prime(x_0) = I$什么意思?
\end{zremark}

\textbf{说明:}

首先,$I^\prime(x_0)$,$I$在$x_0$处的$n \times n$导数矩阵,
所以,我们可以确定$I$是$n \times n$矩阵。

对任意$1 \leq j \leq n$,我们有,
\begin{align*}
  \frac{\partial I}{\partial x_j}(x_0)
   & = \lim\limits_{t \to 0; t \neq 0} \frac{I(x_0 + t e_j) - I(x_0)}{t} \\
   & = \lim\limits_{t \to 0; t \neq 0} \frac{x_0 + t e_j - x_0}{t}       \\
   & = e_j
\end{align*}

于是
\begin{align*}
  I = \begin{pmatrix}
        (e_1)^T & (e_2)^T & \cdots & (e_n)^T
      \end{pmatrix}
\end{align*}
所以,$I$是单位矩阵(对角线元素都是1,其余元素都是0)。

在线性代数中,如果矩阵$A, B$满足:
\begin{align*}
  A B = I
\end{align*}
那么,$A$是$B$的左逆矩阵,$B$是$A$的右逆矩阵,记为:
\begin{align*}
  A = B^{-1}, B = A^{-1}
\end{align*}
所以,书中得到结论:
\begin{align*}
  (f^{-1})^\prime(f(x_0)) = (f^\prime(x_0))^{-1}
\end{align*}

\begin{zremark}
  书中:
  最后,假设$f^\prime(0) = I$,其中$I: \mathbb{R}^n \to \mathbb{R}^n$是恒等变换
  $I(x) = x$。此时,我们把$f$替换成定义为$\tilde{f}(x) := f^\prime(0)^{-1}f(x)$
  的新函数$\tilde{f} : E \to \mathbb{R}^n$,并对它应用$f^\prime(0) = I$时的结果,
  这样就得到了一般情形下的结论。
\end{zremark}

\textbf{证明:}

问题的核心是:为什么通过定义$\tilde{f}(x) := f^\prime(0)^{-1}f(x)$,
并利用$f^\prime(0) = I$时的结果,可以推广到一般情形?

在特殊情况下才会有$f^\prime(0) = I$,一般情况下是不具备这个特性的。
但由题设可知$f^\prime(0)$是可逆的,
即:设$A = f^\prime(0)$,则$A^{-1}$存在。
$\tilde{f}$被定义为$A^{-1}f(x)$,这样做的目的是让$\tilde{f}$在
$x = 0$处的导数是单位矩阵:
\begin{align*}
  \tilde{f}^\prime(0) = A^{-1}f^\prime(0) = A^{-1}A = I
\end{align*}
以上利用了链式法则,且线性变换$A^{-1}$的导数是其本身$A^{-1}$(这部分在17.2.2-comment中有说明)。

\begin{zremark}
  \begin{align*}
    f^{-1}(y) = \tilde{f}^{-1}(f^\prime(0)^{-1}y)
  \end{align*}
\end{zremark}

\textbf{证明:}

因为$f$是双射,对任意$y$存在唯一的$x$使得$y = f(x)$,即
\begin{align*}
  y = f(x) = f^\prime(0)\tilde{f}(x)
\end{align*}
于是
\begin{align*}
  f^{-1}(y) = x
\end{align*}
接下来,我们需要证明$\tilde{f}^{-1}(f^\prime(0)^{-1}y) = x$。
\begin{align*}
  \tilde{f}^{-1}(f^\prime(0)^{-1}y)
   & = \tilde{f}^{-1}(f^\prime(0)^{-1}f(x))                    \\
   & = \tilde{f}^{-1}(f^\prime(0)^{-1}f^\prime(0)\tilde{f}(x)) \\
   & = \tilde{f}^{-1}(I \tilde{f}(x))                          \\
   & = \tilde{f}^{-1}(\tilde{f}(x))                            \\
   & = x
\end{align*}
综上,等式成立。

\begin{zremark}
  $V$是一个开集,$f$是连续的且可逆的,那么$U = f^{-1}(V)$也是开集。
\end{zremark}

\textbf{证明:}

首先,由题设可知$f$是单射和满射。

对任意的$x_0 \in U$,
$f(x_0) \in V$,
因为$V$是开集,存在$\epsilon > 0$,使得$B(f(x_0), \epsilon) \subseteq V$。

因为$f$连续,所以存在$\delta > 0$,使得只要
$|x - x_0| < \delta$就有
\begin{align*}
  |f(x) - f(x_0)| < \epsilon
\end{align*}

综上可得,任意$x \in B(x_0, \delta)$,都有
\begin{align*}
  f(x) \in V \\
  \implies   \\
  x \in U    \\
  \implies   \\
  B(x_0, \delta) \subseteq U
\end{align*}

由开集的定义(命题12.2.15(a))可知,$U$是开集。

\end{document}