\documentclass{article}
\usepackage{mathtools} 
\usepackage{fontspec}
\usepackage[UTF8]{ctex}
\usepackage{amsthm}
\usepackage{mdframed}
\usepackage{xcolor}
\usepackage{amssymb}
\usepackage{amsmath}


% 定义新的带灰色背景的说明环境 zremark
\newmdtheoremenv[
  backgroundcolor=gray!10,
  % 边框与背景一致,边框线会消失
  linecolor=gray!10
]{zremark}{说明}


\begin{document}
\title{16.5 习题}
\author{张志聪}
\maketitle

\section*{16.5.1}

\begin{itemize}
  \item (a)

        设$\epsilon > 0$,由傅里叶定理可知,当$N$足够大时,有
        \begin{align*}
          \left\|f - \sum\limits_{n = -N}^N \hat{f}(n)e_n \right\|_2 \leq \epsilon
        \end{align*}

        我们有
        \begin{align*}
           & \sum\limits_{n = -N}^N \hat{f}(n)e_n                                               \\
           & = \hat{f}(0)e_0 + \sum\limits_{n = 1}^N \hat{f}(n)e_n + \hat{f}(-n)e_{-n}          \\
           & = \int_{[0, 1]} f(x) dx
          + \sum\limits_{n = 1}^N (\int_{[0, 1]}f(x)e^{-2\pi n x} dx) e_n
          + (\int_{[0, 1]}f(x)e^{2\pi n x} dx)e_{-n}                                            \\
           & = \frac{1}{2}a_0
          + \sum\limits_{n = 1}^N e_n\int_{[0, 1]}f(x) (cos(2\pi n x) -i sin(2\pi n x)) dx
          + e_{-n}\int_{[0, 1]}f(x) (cos(2\pi n x) +i sin(2\pi n x)) dx                         \\
           & = \frac{1}{2}a_0                                                                   \\
           & + \sum\limits_{n = 1}^N e_n\int_{[0, 1]}f(x) cos(2\pi n x) -i f(x)sin(2\pi n x) dx
          + e_{-n}\int_{[0, 1]}f(x) cos(2\pi n x) +i f(x)sin(2\pi n x) dx                       \\
           & = \frac{1}{2}a_0
          + \sum\limits_{n = 1}^N (e_n + e_{-n})\int_{[0, 1]}f(x) cos(2\pi n x) dx
          + (-e_n + e_{-n})\int_{[0, 1]}i f(x)sin(2\pi n x) dx                                  \\
           & = \frac{1}{2}a_0
          + \sum\limits_{n = 1}^N (e_n + e_{-n})\int_{[0, 1]}f(x) cos(2\pi n x) dx
          + (-e_n + e_{-n})\int_{[0, 1]}i f(x)sin(2\pi n x) dx                                  \\
           & = \frac{1}{2}a_0
          + \sum\limits_{n = 1}^N 2cos(2\pi nx)\int_{[0, 1]}f(x) cos(2\pi n x) dx
          -2isin(2\pi nx)\int_{[0, 1]}i f(x)sin(2\pi n x) dx                                    \\
           & = \frac{1}{2}a_0
          + \sum\limits_{n = 1}^N 2cos(2\pi nx)\int_{[0, 1]}f(x) cos(2\pi n x) dx
          +2sin(2\pi nx)\int_{[0, 1]}f(x)sin(2\pi n x) dx                                       \\
           & = \frac{1}{2}a_0 + \sum\limits_{n = 1}^N a_n cos(2\pi nx) +b_nsin(2\pi nx)
        \end{align*}

        综上可得,$N$足够大时,
        \begin{align*}
          \left\|f - (\frac{1}{2}a_0 + \sum\limits_{n = 1}^N a_n cos(2\pi nx) +b_nsin(2\pi nx)) \right\|_2 \leq \epsilon
        \end{align*}
        所以,$\frac{1}{2}a_0 + \sum\limits_{n = 1}^N a_n cos(2\pi nx) +b_nsin(2\pi nx)$依$L^2$度量收敛于$f$。

  \item (b)

        部分和
        \begin{align*}
           & \sum \limits_{n = -N}^N |\hat{f}(n)|                                                                    \\
           & = \sum \limits_{n = -N}^N |\int_{[0, 1]} f(x)e^{-2\pi n x} dx|                                          \\
           & = \sum \limits_{n = -N}^N |\int_{[0, 1]} f(x)(cos(2\pi nx) - isin(2\pi nx)) dx|                         \\
           & = \sum \limits_{n = -N}^N |\frac{1}{2} (a_n - ib_n)|                                                    \\
           & = |\frac{1}{2} (a_0 - ib_0)| + \frac{1}{2} \sum \limits_{n = 1}^N |(a_n - ib_n)| + |(a_{-n} - ib_{-n})| \\
           & = |\frac{1}{2} a_0| + \frac{1}{2} \sum \limits_{n = 1}^N (|(a_n - ib_n)| + |(a_{-n} - ib_{-n})|)        \\
           & = |\frac{1}{2} a_0| + \frac{1}{2} \sum \limits_{n = 1}^N (|(a_n - ib_n)| + |(a_{n} + ib_{n})|)          \\
           & \leq |\frac{1}{2} a_0| + \sum \limits_{n = 1}^N |a_n| + |ib_n|
        \end{align*}
        于是,由题设$\sum \limits_{n = 1}^\infty a_n, \sum \limits_{n = 1}^\infty b_n$都是绝对收敛,可知
        $\sum \limits_{n = -\infty}^\infty |\hat{f}(n)|$绝对收敛。
        由定理16.5.3可知,
        $\sum \limits_{n = -\infty}^\infty \hat{f}(n)e_n$绝对收敛于$f$。

        于是对任意$\epsilon > 0$,存在$N_0$,使得只要$N \geq N_0$,有
        \begin{align*}
          \left\|f - \sum \limits_{n = -N}^N \hat{f}(n)e_n \right\|_{\infty} \leq \epsilon
        \end{align*}
        由$(a)$可知,
        \begin{align*}
          \left\|f - (\frac{1}{2}a_0 + \sum\limits_{n = 1}^N a_n cos(2\pi nx) +b_nsin(2\pi nx)) \right\|_{\infty} \leq \epsilon
        \end{align*}

        综上,级数$\frac{1}{2}a_0 + \sum\limits_{n = 1}^\infty a_n cos(2\pi nx) +b_nsin(2\pi nx)$一致收敛于$f$。

\end{itemize}

\section*{16.5.2}

\begin{itemize}
  \item (a)

        令$y = 2x - 1$,于是$x = \frac{y + 1}{2}$,函数$y: [-1, 1] \to [0, 1]$,
        于是由命题11.10.7可知,
        \begin{align*}
          a_n & = 2\int_{[0, 1]} f(x) cos(2\pi nx) dx         \\
              & = 2\int_{[0, 1]} (1 - 2x)^2 cos(2\pi nx) dx   \\
              & = \int_{[-1, 1]} y^2 cos(\pi(y + 1)n) dy      \\
              & = \int_{[-1, 1]} y^2 cos(\pi ny)cos(\pi n) dy \\
              & = (-1)^n\int_{[-1, 1]} y^2 cos(\pi ny) dy
        \end{align*}

        $n = 0$时,$a_0 = (-1)^0\int_{[-1, 1]} y^2 dy = \frac{2}{3}$。

        $n \geq 1$时,
        我们设$u = y^2, du = 2ydy; v = \frac{1}{\pi n}sin(\pi ny), dv = cos(\pi ny)$,
        利用命题11.10.1(分部积分法)可知,
        \begin{align*}
          \int_{[-1, 1]} y^2 cos(\pi ny) dy
           & = \int_{[-1, 1]} u dv                                                                  \\
           & = uv|_{-1}^1 - \int_{[-1, 1]} v du                                                     \\
           & = \frac{y^2}{\pi n}sin(\pi ny)|_{-1}^1 - \int_{[-1, 1]} \frac{1}{\pi n}sin(\pi ny)2ydy \\
           & = \frac{2sin(\pi n)}{\pi n} - \frac{2}{\pi n}\int_{[-1, 1]} ysin(\pi ny) dy            \\
           & = - \frac{2}{\pi n}\int_{[-1, 1]} ysin(\pi ny) dy
        \end{align*}
        (以上的处理,可以使$y^2$降幂)

        接下来,我们计算$\int_{[-1, 1]} ysin(\pi ny) dy$。

        令$u = y, du = dy; v = -\frac{cos(\pi ny)}{n\pi}, dv = sin(\pi ny)dy$,再次利用分部积分法,
        \begin{align*}
          \int_{[-1, 1]} ysin(\pi ny) dy
           & = \int_{[-1, 1]} u dv                                                            \\
           & = uv|_{-1}^1 - \int_{[-1, 1]} v du                                               \\
           & = -\frac{ycos(\pi ny)}{n\pi}|_{-1}^1 + \int_{[-1, 1]} \frac{cos(\pi ny)}{n\pi}dy \\
           & = -\frac{2cos(\pi n)}{\pi n} + \frac{1}{n\pi} \frac{sin(\pi ny)}{n \pi}|_{-1}^1  \\
           & = -\frac{2cos(\pi n)}{\pi n}
        \end{align*}

        综上可得,
        \begin{align*}
          a_n & =(-1)^n\int_{[-1, 1]} y^2 cos(\pi ny) dy            \\
              & = (-1)^n (\frac{2}{\pi n}\frac{2cos(\pi n)}{\pi n}) \\
              & = (-1)^n (\frac{4cos(\pi n)}{\pi^2 n^2})            \\
              & = (-1)^n(-1)^n (\frac{4}{\pi^2 n^2})                \\
              & = \frac{4}{\pi^2 n^2}
        \end{align*}

        类似地,我们有
        \begin{align*}
          b_n & = 2\int_{[0, 1]} f(x)sin(2\pi nx) dx        \\
              & = 2\int_{[0, 1]} (1 - 2x)^2 sin(2\pi nx) dx \\
              & = 0
        \end{align*}

        所以,
        \begin{align*}
           & \frac{1}{2}a_0 + \sum \limits_{n = 1}^\infty a_n cos(2\pi nx) + b_nsin(2\pi nx) \\
           & =\frac{1}{3} + \sum \limits_{n = 1}^\infty \frac{4}{\pi^2 n^2} cos(2\pi nx)
        \end{align*}

        因为
        \begin{align*}
          \sum \limits_{n = 1}^\infty a_n
           & = \sum \limits_{n = 1}^\infty \frac{4}{\pi^2 n^2}          \\
           & = \frac{4}{\pi^2}\sum \limits_{n = 1}^\infty \frac{1}{n^2}
        \end{align*}
        由推论7.3.7可知,$\sum \limits_{n = 1}^\infty \frac{1}{n^2}$收敛,
        进一步可知,$\sum \limits_{n = 1}^\infty a_n$绝对收敛。
        又因为$\sum \limits_{n = 1}^\infty b_n$是绝对收敛的。

        综上,由习题16.5.1可知,
        级数$\frac{1}{3} + \sum \limits_{n = 1}^\infty \frac{4}{\pi^2 n^2} cos(2\pi nx)$一致收敛于$f$。

  \item (b)

        因为级数$\frac{1}{3} + \sum \limits_{n = 1}^\infty \frac{4}{\pi^2 n^2} cos(2\pi nx)$一致收敛于$f$,

        对任意$\epsilon > 0$,存在$N_0 > 1$使得只要$N \geq N_0$和$x = 0$,都有
        \begin{align*}
          |(\frac{1}{3} + \sum \limits_{n = 1}^N \frac{4}{\pi^2 n^2} cos(0)) - f(0)| \leq \epsilon \\
          |-\frac{2}{3} + \sum \limits_{n = 1}^N \frac{4}{\pi^2 n^2}| \leq \epsilon                \\
          \frac{2}{3} - \epsilon \leq |\sum \limits_{n = 1}^N \frac{4}{\pi^2 n^2}| \leq \frac{2}{3} + \epsilon
        \end{align*}

        由$\epsilon$的任意性可知,$\sum \limits_{n = 1}^\infty \frac{4}{\pi^2 n^2} = \frac{2}{3}$,
        进一步可得,$\sum \limits_{n = 1}^\infty \frac{1}{n^2} = \frac{\pi^2}{4} \times \frac{2}{3} = \frac{\pi^2}{6}$。

  \item (c)

        由题设可知,
        \begin{align*}
          \hat{f}(n) & = \int_{[0, 1]} f(x) e_{-n} dx                         \\
                     & = \int_{[0, 1]} f(x) (cos(2\pi nx) - isin(2\pi nx)) dx \\
                     & = \frac{a_n -ib_n}{2}                                  \\
                     & = \frac{2}{\pi^2 n^2}
        \end{align*}
        (省略了积分的计算过程)

        同理可得,
        \begin{align*}
          \hat{f}(-n) & = \frac{a_n + ib_n}{2} \\
                      & = \frac{2}{\pi^2 n^2}
        \end{align*}
        \begin{align*}
          \hat{f}(0) & = \frac{a_0}{2} \\
                     & = \frac{1}{3}
        \end{align*}

        由命题16.5.4可知,
        \begin{align*}
          \|f\|_2^2 = \sum \limits_{n = -\infty}^\infty |\hat{f}(n)|^2
        \end{align*}

        综上可得,
        \begin{align*}
           & \sum \limits_{n = 0}^\infty |\hat{f}(n)|^2                                                                     \\
           & = |\hat{f}(0)|^2 + \sum \limits_{n = -\infty}^{-1} |\hat{f}(n)|^2 + \sum \limits_{n = 1}^\infty |\hat{f}(n)|^2 \\
           & = \frac{1}{9} + 2\sum \limits_{n = 1}^\infty |\frac{2}{\pi^2 n^2}|^2                                           \\
           & = \frac{1}{9} + 2\sum \limits_{n = 1}^\infty \frac{4}{\pi^4 n^4}                                               \\
           & = \frac{1}{9} + \frac{8}{\pi^4} \sum \limits_{n = 1}^\infty \frac{1}{n^4}
        \end{align*}

        又因为
        \begin{align*}
          \|f\|_2^2 & = \int{[0, 1]} |f(x)|^2 dx   \\
                    & = \int{[0, 1]} (1 - 2x)^2 dx \\
                    & = \frac{1}{5}
        \end{align*}
        综上可得,
        \begin{align*}
          \frac{1}{5} = \frac{1}{3} + \frac{8}{\pi^4} \sum \limits_{n = 1}^\infty \frac{1}{n^4} \\
          \frac{4}{45} = \frac{8}{\pi^4} \sum \limits_{n = 1}^\infty \frac{1}{n^4}              \\
          \sum \limits_{n = 1}^\infty \frac{1}{n^4} = \frac{\pi^4}{90}
        \end{align*}
\end{itemize}

\section*{16.5.3}

这道题不对吧!!!都没定义这个特殊符号是怎么运算的。

$P$是一个三角多项式,所以$P \in C(\mathbb{R}/\mathbb{Z}; \mathbb{C})$,
且存在一个整数$N \geq 0$和一个复数序列$(c_n)_{n = -N}^N$使得$P = \sum \limits_{n = -N}^N c_ne_n$。

\begin{itemize}
  \item (a)
        \begin{align*}
          f \ast P & = f \ast \sum\limits_{n = -N}^N c_n e_n       \\
                   & = \sum\limits_{n = -N}^N f \ast (c_n e_n)     \\
                   & = \sum\limits_{n = -N}^N c_n (f \ast e_n)     \\
                   & = \sum\limits_{n = -N}^N c_n (\hat{f}(n) e_n) \\
                   & = \sum\limits_{n = -N}^N \hat{f}(n) c_n e_n
        \end{align*}
        (书中P345,有类似说明)

        利用引理16.2.5和推论16.3.6,我们有
        \begin{align*}
          \widehat{f \ast P}(n) & = \langle \sum\limits_{k = -N}^N \hat{f}(k) c_k e_k, e_n \rangle \\
                                & = \langle \hat{f}(n) c_n e_n, e_n \rangle                        \\
                                & = \hat{f}(n) c_n \langle e_n, e_n \rangle                        \\
                                & = \hat{f}(n) c_n
        \end{align*}

        由推论16.3.6和定义16.3.7,我们有
        \begin{align*}
          \hat{f}(n) c_n = \langle f, e_n \rangle \langle P, e_n \rangle \\
          \hat{f}(n) \hat{P}(n) = \langle f, e_n \rangle \langle P, e_n \rangle
        \end{align*}
        综上可得,
        \begin{align*}
          \widehat{f \ast P}(n) = \hat{f}(n) c_n = \hat{f}(n) \hat{P}(n)
        \end{align*}

  \item (b) $\circledast$

        由定理16.5.4(Plancherel定理)可知,$|\hat{f}(n)|$对任意$n$有界,
        函数$f(x)$有界。所以存在$M > 0$使得$|\hat{f}(n)| < M$,$|f(x)| < M$。

        由定理16.4.1(三角多项式的威尔斯特拉斯逼近定理)可知,存在三角多项式$P$使得
        \begin{align*}
          \left\|g - P\right\|_{\infty} \leq \epsilon
        \end{align*}

        由引理16.2.7(b),我们有
        \begin{align*}
          \left|\widehat{f \ast g}(n) - \widehat{f \ast P}(n)\right|
           & = \left|\langle f \ast g, e_n \rangle - \langle f \ast P, e_n \rangle\right| \\
           & = \left|\langle f \ast (g - P), e_n \rangle \right|                          \\
           & \leq \|f \ast (g - P)\|_2 \|e_n\|_2                                          \\
           & = \|f \ast (g - P)\|_2
        \end{align*}
        又
        \begin{align*}
          \left| f \ast (g - P)\right|
           & = \left| \int_{[0, 1]}f(y) (g - P)(x - y) dy \right| \\
           & \leq \left| \epsilon \int_{[0, 1]}f(y) dy \right|    \\
           & \leq M \epsilon
        \end{align*}

        于是
        \begin{align*}
          \|f \ast (g - P)\|_2
           & = \left(\int_{[0, 1]} |f \ast (g - P)|^2 dx\right)^{1/2} \\
           & \leq M\epsilon
        \end{align*}

        对于
        \begin{align*}
          \left|\widehat{f \ast P}(n) - \hat{f}(n) \hat{g}(n)\right|
           & = \left|\hat{f}(n)\hat{P}(n) - \hat{f}(n) \hat{g}(n)\right|                \\
           & = \left|\hat{f}(n)(\hat{P}(n) - \hat{g}(n))\right|                         \\
           & = \left|\hat{f}(n)(\langle P, e_n \rangle - \langle f, e_n \rangle)\right| \\
           & = \left|\hat{f}(n)\langle P - f, e_n \rangle\right|                        \\
           & \leq |\hat{f}(n)|\left\|P - f, e_n \right\|_2                              \\
           & = M\epsilon
        \end{align*}

        综上所述,
        \begin{align*}
          \left|\widehat{f \ast g}(n) - \hat{f}(n) \hat{g}(n)\right|
           & = \left|\widehat{f \ast g}(n) - \widehat{f \ast P}(n) + \widehat{f \ast P}(n) - \hat{f}(n) \hat{g}(n)\right|                 \\
           & \leq \left|\widehat{f \ast g}(n) - \widehat{f \ast P}(n)\right| + \left|\widehat{f \ast P}(n) - \hat{f}(n) \hat{g}(n)\right| \\
           & \leq M \epsilon + M \epsilon
        \end{align*}

        由$\epsilon$的任意性可知,$\widehat{f \ast g}(n) = \hat{f}(n) \hat{g}(n)$。

\end{itemize}

\section*{16.5.4}
 (1)

由题设,
我们只需在说明$f^\prime$是1周期函数,即可说明$f^\prime \in C(\mathbb{R}/\mathbb{Z}; \mathbb{C})$。

对任意$x_0 \in \mathbb{R}$,我们有
\begin{align*}
  \lim\limits_{x \to x_0} \frac{f(x) - f(x_0)}{x - x_0} = f^\prime(x_0)
\end{align*}
于是,对任意$\epsilon > 0$,存在$\delta > 0$,使得只要$|x - x_0| < \delta$,都有
\begin{align*}
  |\frac{f(x) - f(x_0)}{x - x_0} - f^\prime(x_0)| \leq \epsilon
\end{align*}

令$y = x + 1$,
那么对$|y - (x_0 + 1)| < \delta$,即$|x + 1 - (x_0 + 1)| = |x - x_0| < \delta$,
由$f$是1周期函数,我们有
\begin{align*}
  |\frac{f(y) - f(x_0)}{y - (x_0 + 1)} - f^\prime(x_0)|
   & = |\frac{f(x + 1) - f(x_0)}{x + 1 - (x_0 + 1)} - f^\prime(x_0)| \\
   & = |\frac{f(x) - f(x_0)}{x - x_0} - f^\prime(x_0)|               \\
   & \leq \epsilon
\end{align*}
于是可得
\begin{align*}
  f^\prime(x_0 + 1)
   & = \lim\limits_{y \to (x_0 + 1)} \frac{f(y) - f(x_0 + 1)}{y - (x_0 + 1)} \\
   & = f^\prime(x_0)
\end{align*}
所以,$f^\prime$是1周期函数。

(2)

\begin{itemize}
  \item 方法一
        \begin{align*}
          \hat{f^\prime}(n)
           & = \int_{[0, 1]} f^\prime(x) e^{-2\pi i nx} dx \\
        \end{align*}
        令$u = f(x), v = e^{-2\pi i nx}$于是$du = f^\prime (x)dx, dv = -2\pi i n e^{-2\pi i nx}dx$。

        \begin{align*}
          \hat{f^\prime}(n)
           & = \int_{[0, 1]} f^\prime(x) e^{-2\pi i nx} dx                                  \\
           & = uv|_{0}^1 - \int_{[0, 1]} u dv                                               \\
           & = f(x) e^{-2\pi i nx} |_{0}^1 - \int_{[0, 1]} -2 f(x) \pi i n e^{-2\pi i nx}dx \\
           & = f(1) - f(0) + 2\pi i n \int_{[0, 1]} f(x) e^{-2\pi i nx}dx                   \\
           & = 2\pi i n \hat{f}(n)
        \end{align*}

        方法一的问题在于,本书中没有复数函数的导数的定义。

  \item 方法二

        利用分部积分法,
        \begin{align*}
          \hat{f^\prime}(n)
           & = \int_{[0, 1]} f^\prime(x) e^{-2\pi i nx} dx                                                          \\
           & = \int_{[0, 1]} f^\prime(x) (cos(2\pi nx) - isin(2\pi nx)) dx                                          \\
           & = \int_{[0, 1]} f^\prime(x) cos(2\pi nx)dx - i \int_{[0, 1]} f^\prime(x) sin(2\pi nx)dx                \\
           & = \left[f(x)cos(2\pi n x)|_0^1 - \int_{[0, 1]} f(x) \cdot (2\pi n)sin(2\pi n x)dx\right]               \\
           & - i\left[f(x)sin(2\pi n x)|_0^1 - \int_{[0, 1]} f(x) \cdot (2\pi n)cos(2\pi n x)dx\right]              \\
           & = \int_{[0, 1]} f(x) \cdot (2\pi n)sin(2\pi n x)dx + i\int_{[0, 1]} f(x) \cdot (2\pi n)cos(2\pi n x)dx \\
           & = 2\pi n \int_{[0, 1]} f(x) sin(2\pi n x) + if(x)cos(2\pi n x) dx                                      \\
           & = 2\pi n \int_{[0, 1]} f(x)(sin(2\pi n x) + icos(2\pi n x))                                            \\
           & = 2\pi n \int_{[0, 1]} f(x)ie^{-2\pi i nx} dx                                                          \\
           & = 2\pi n i \int_{[0, 1]} f(x)e^{-2\pi i nx} dx                                                         \\
           & = 2\pi n i \hat{f}(n)
        \end{align*}

\end{itemize}

\section*{16.5.5}



\end{document}


