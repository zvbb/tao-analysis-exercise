\documentclass{article}
\usepackage{mathtools} 
\usepackage{fontspec}
\usepackage[UTF8]{ctex}
\usepackage{amsthm}
\usepackage{mdframed}
\usepackage{xcolor}
\usepackage{amssymb}
\usepackage{amsmath}


% 定义新的带灰色背景的说明环境 zremark
\newmdtheoremenv[
  backgroundcolor=gray!10,
  % 边框与背景一致,边框线会消失
  linecolor=gray!10
]{zremark}{说明}


\begin{document}
\title{16.1 习题}
\author{张志聪}
\maketitle

\section*{16.1.1}

(1)$k$的存在性。

对任意实数$x$,
令$A : = \{l \in \mathbb{Z}: l \leq x\}, k: = \sup(A)$。

由命题5.4.12(负实数有类似的命题)可知,存在有理数$q$和整数$N$使得
\begin{align*}
  q \leq x \leq N
\end{align*}

由命题4.4.1可知,存在一个整数$M$使得$M \leq q$,于是
\begin{align*}
  M \leq x \leq N
\end{align*}

于是$A$非空,且有上界。

接下来,证明上确界是存在的,且上确界属于$A$。
换言之,任意一个元素为整数的非空有界集合都有一个最大元素。

由命题5.5.9(最小上界的存在性)可知,$A$存在上确界,设$k$是$A$的上确界。

反证法,假设$k \notin A$。

任取$a_0 \in A$(因为$A$是非空,所以$a_0$是存在的。)
因为$k \notin A, a_0 \in A$,所以存在$a_1 \in A$,且$a_1 > a_0$(否则$a_0 = k$就是上确界了,与假设矛盾)。
因为$a_1 > a_0$,所以,$a_1 \geq a_0 + 1$($A$中的元素都是整数)。
递归地构造出$a_2, a_3, \dots, a_n, \dots$。

所以,对任意$n \geq 1$,都有$a_n \geq a_0 + n$。
于是,只要$n$足够大,就可以取到$a_n > k$,这与$k$是$A$的上确界矛盾。

综上,整数$k$是存在的。

(2)$y \in [0, 1)$。

接下来,证明:令$y = x - k$,$y \in [0, 1)$。

如果,$y \geq 1$,那么,$x = k + y \geq k + 1$,
这与$k$是$A$的上确界矛盾(因为$k + 1 \in A$)。

同理,$y < 0$,$x = k + y < k$,
同样与$k$是$A$的上确界矛盾。(因为$k \notin A$。)

综上,$y \in y \in [0, 1)$。

\end{document}


