\documentclass{article}
\usepackage{mathtools} 
\usepackage{fontspec}
\usepackage[UTF8]{ctex}
\usepackage{amsthm}
\usepackage{mdframed}
\usepackage{xcolor}
\usepackage{amssymb}
\usepackage{amsmath}


% 定义新的带灰色背景的说明环境 zremark
\newmdtheoremenv[
  backgroundcolor=gray!10,
  % 边框与背景一致,边框线会消失
  linecolor=gray!10
]{zremark}{说明}


\begin{document}
\title{16.1 习题}
\author{张志聪}
\maketitle

\section*{16.1.1}

(1)$k$的存在性。

对任意实数$x$,
令$A : = \{l \in \mathbb{Z}: l \leq x\}, k: = \sup(A)$。

由命题5.4.12(负实数有类似的命题)可知,存在有理数$q$和整数$N$使得
\begin{align*}
  q \leq x \leq N
\end{align*}

由命题4.4.1可知,存在一个整数$M$使得$M \leq q$,于是
\begin{align*}
  M \leq x \leq N
\end{align*}

于是$A$非空,且有上界。

接下来,证明上确界是存在的,且上确界属于$A$。
换言之,任意一个元素为整数的非空有界集合都有一个最大元素。

由命题5.5.9(最小上界的存在性)可知,$A$存在上确界,设$k$是$A$的上确界。

反证法,假设$k \notin A$。

任取$a_0 \in A$(因为$A$是非空,所以$a_0$是存在的。)
因为$k \notin A, a_0 \in A$,所以存在$a_1 \in A$,且$a_1 > a_0$(否则$a_0 = k$就是上确界了,与假设矛盾)。
因为$a_1 > a_0$,所以,$a_1 \geq a_0 + 1$($A$中的元素都是整数)。
递归地构造出$a_2, a_3, \dots, a_n, \dots$。

所以,对任意$n \geq 1$,都有$a_n \geq a_0 + n$。
于是,只要$n$足够大,就可以取到$a_n > k$,这与$k$是$A$的上确界矛盾。

综上,整数$k$是存在的。

(2)$y \in [0, 1)$。

接下来,证明:令$y = x - k$,$y \in [0, 1)$。

如果,$y \geq 1$,那么,$x = k + y \geq k + 1$,
这与$k$是$A$的上确界矛盾(因为$k + 1 \in A$)。

同理,$y < 0$,$x = k + y < k$,
同样与$k$是$A$的上确界矛盾。(因为$k \notin A$。)

综上,$y \in y \in [0, 1)$。

\section*{16.1.2}

$f$是$Z$周期的,所以,我们只需要了解它在$[0, 1)$上的取值就行了。

\begin{itemize}
  \item (a)


        因为$f$在$\mathbb{R}$上连续的,那么,$f$在$[0, 1]$上也是连续的。

        证明与引理9.6.3相同(不能直接使用9.6.3,函数的值域不同)。

        反证法,假设$f$在$[0, 1]$是无界的。那么,对任意实数$M > 0$,
        都存在$x \in [0, 1)$,使得$|f(x)| \geq M$。

        特别地,对于每一个自然数$n$,集合$\{x \in [0, 1]: |f(x)| \geq n\}$
        (这里$f(x)$是复数,但$|f(x)|$是实数)都是非空的。
        所以我们可以选取$[0, 1]$中的一个序列$(x_n)_{n=1}^\infty$使得$|f(x_n)| \geq n$
        对所有的$n$均成立。由于这个序列属于$[0, 1]$,从而根据9.1.24可知,
        存在一个收敛于某个极限$L \in [0, 1]$的子序列$(x_{n_j})_{j = 0}^\infty$,
        其中$n_0 < n_1 < n_2 < \dots$是一个递增的自然数序列。
        特别地,对于所有的$j \in \mathbb{N}$,均有$n_j \geq j$。

        因为$f$在$[0, 1]$上连续,所以它在$L$处是连续的,并且我们有
        \begin{align}
          \lim_{j \to \infty} f(x_{n_j}) = f(L)
        \end{align}

        所以序列$(f(x_{n_j}))_{j = 0}^\infty$是收敛的,从而是有界的。
        另外,我们从序列的构造过程中看出$|f(x_{n_j})| \geq n_j \geq j$对所有的$j$均成立,
        从而序列$(f(x_{n_j}))_{j = 0}^\infty$是无界的,这是一个矛盾。

        $f$在$[0, 1]$上有界,所以在$[0, 1)$上也有界。

  \item (b)

        以$f + g$为例, 只需证明$f + g$,
        满足连续的$\mathbb{Z}$周期复值函数的空间(即:$C(\mathbb{R}/\mathbb{Z}; \mathbb{C})$)性质即可。

        $f + g$的连续性,证明略。

        现在证明$f + g$是$\mathbb{Z}$周期性的。

        \begin{align*}
          (f + g)(x) & = f(x) + g(x)         \\
                     & = f(x + k) + g(x + k) \\
                     & = (f + g)(x + k)
        \end{align*}
        命题得证。

  \item (c)

        连续性由推论14.3.2保证。

        接下来,需要证明,$f$是$\mathbb{Z}$周期性的。

        反证法,假设$f$不是$\mathbb{Z}$周期性的,那么,存在$x_0 \in [0, 1)$,
        使得$f(x_0) \neq f(x_0 + k)$(其中,$k$是任意整数)。

        因为,$(f_n)_{n = 1}^\infty$一致收敛于$f$,
        那么,令$\epsilon = |f(x_0) - f(x_0 + k)|$,
        存在$N \geq 1$使得只要$n \geq N$,就有
        \begin{align*}
          |f(x_0) - f_n(x_0)| < \frac{1}{4} \epsilon \\
          |f(x_0 + k) - f_n(x_0 + k)| = |f(x_0 + k) - f_n(x_0)| < \frac{1}{4} \epsilon
        \end{align*}

        我们有,
        \begin{align*}
          |f(x_0) - f(x_0 + k)| & = |f(x_0) - f_n(x_0) + f_n(x_0) - f(x_0 + k)|      \\
                                & \leq |f(x_0) - f_n(x_0)| + |f_n(x_0) - f(x_0 + k)| \\
                                & < \frac{1}{4} \epsilon + \frac{1}{4} \epsilon      \\
                                & =  \frac{1}{2}\epsilon
        \end{align*}

        存在一个矛盾。
\end{itemize}

\section*{16.1.3}

\begin{itemize}
  \item (a) $(C(\mathbb{R}/\mathbb{Z}; \mathbb{C}),d_{\infty})$符合定义12.1.2(度量空间)。

        符合下面四个公理:
        \begin{itemize}
          \item (1) 对任意的$f \in C(\mathbb{R}/\mathbb{Z}; \mathbb{C})$,我们有$d_{\infty}(f, f) = 0$。

                因为,$\sup\limits_{x \in \mathbb{R}} |f(x) - f(x)| = \sup\limits_{x \in \mathbb{R}} 0 = 0$,
                即:$d_{\infty}(f, f) = 0$。
          \item (2) (正性)对任意两个不同的$f, g \in C(\mathbb{R}/\mathbb{Z}; \mathbb{C})$,我们有$d_{\infty}(f, g) > 0$。

                因为$f \neq g$,所以,存在$x_0 \in \mathbb{R}$使得$f(x_0) \neq g(x_0)$,
                于是$d_{\infty}(f, g) \geq |f(x_0) - g(x_0)| > 0$。

          \item (3) (对称性)对任意的$f, g \in C(\mathbb{R}/\mathbb{Z}; \mathbb{C})$,我们有$d_{\infty}(f, g) = d_{\infty}(g, f)$。

                \begin{align*}
                  d_{\infty}(f, g) & = \sup\limits_{x \in \mathbb{R}} |f(x) - g(x)| \\
                                   & = \sup\limits_{x \in \mathbb{R}} |g(x) - f(x)| \\
                                   & = d_{\infty}(g, f)
                \end{align*}

          \item (4)(三角不等式)对任意的$f,g,h \in C(\mathbb{R}/\mathbb{Z}; \mathbb{C})$,我们有$d_{\infty}(f, h) \leq d_{\infty}(f, g) + d_{\infty}(g, h)$。

                反证法,假设$d_{\infty}(f, h) > d_{\infty}(f, g) + d_{\infty}(g, h)$。

                由上确界的定义和命题6.3.6可知,存在$x_0 \in \mathbb{R}$使得
                \begin{align*}
                  d_{\infty}(f, h) & > |f(x_0) - h(x_0)|                        \\
                                   & > d_{\infty}(f, g) + d_{\infty}(g, h)      \\
                                   & \geq |f(x_0) - g(x_0)| + |g(x_0) - h(x_0)|
                \end{align*}
                综上,我们有
                \begin{align*}
                  |f(x_0) - h(x_0)| > |f(x_0) - g(x_0)| + |g(x_0) - h(x_0)|
                \end{align*}

                因为对任意$x \in \mathbb{R}$,我们有
                \begin{align*}
                  |f(x) - h(x)| & = |f(x) - g(x) + g(x) - h(x)|      \\
                                & \leq |f(x) - g(x)| + |g(x) - h(x)|
                \end{align*}

                存在矛盾。

        \end{itemize}


  \item (d) $(C(\mathbb{R}/\mathbb{Z}; \mathbb{C}),d_{\infty})$是完备的。

        利用定理14.4.5可证。



\end{itemize}

\end{document}


