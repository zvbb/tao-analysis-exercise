\documentclass{article}
\usepackage{mathtools} 
\usepackage{fontspec}
\usepackage[UTF8]{ctex}
\usepackage{amsthm}
\usepackage{mdframed}
\usepackage{xcolor}
\usepackage{amssymb}
\usepackage{amsmath}


% 定义新的带灰色背景的说明环境 zremark
\newmdtheoremenv[
  backgroundcolor=gray!10,
  % 边框与背景一致,边框线会消失
  linecolor=gray!10
]{zremark}{说明}


\begin{document}
\title{16.2 习题}
\author{张志聪}
\maketitle

\section*{16.2.1}

设$f(x) = f_1(x) + if_2(x), g(x) = g_1(x) + ig_2(x), h(x) = h_1(x) + ih_2(x), c = c_1 + ic_2$。

\begin{itemize}
  \item (a)

        \begin{align*}
          \langle g, f \rangle & = \int_{[0, 1]} g(x) \overline{f(x)} dx                                                         \\
                               & = \int_{[0, 1]} \left(g_1(x) + ig_2(x)\right) \left(f_1(x) - if_2(x)\right) dx                  \\
                               & = \int_{[0, 1]} g_1(x) f_1(x) + g_2(x)f_2(x) + i(-g_1(x)f_2(x) + g_2(x)f_1(x)) dx               \\
                               & = \int_{[0, 1]} g_1(x) f_1(x) + g_2(x)f_2(x) dx + i\int_{[0, 1]} g_2(x)f_1(x) - g_1(x)f_2(x) dx
        \end{align*}

        \begin{align*}
          \langle f, g \rangle & = \int_{[0, 1]} f(x) \overline{g(x)} dx                                                        \\
                               & = \int_{[0, 1]} (f_1(x) + if_2(x)) (g_1(x) - ig_2(x)) dx                                       \\
                               & = \int_{[0, 1]} f_1(x)g_1(x) + f_2(x)g_2(x) + i(-f_1(x)g_2(x) + f_2(x)g_1(x)) dx               \\
                               & = \int_{[0, 1]} f_1(x)g_1(x) + f_2(x)g_2(x) dx + i\int_{[0, 1]}-f_1(x)g_2(x) + f_2(x)g_1(x) dx \\
        \end{align*}
        于是,
        \begin{align*}
          \overline{\langle f, g \rangle} & = \int_{[0, 1]} f_1(x)g_1(x) + f_2(x)g_2(x) dx - i\int_{[0, 1]}-f_1(x)g_2(x) + f_2(x)g_1(x) dx \\
                                          & = \int_{[0, 1]} f_1(x)g_1(x) + f_2(x)g_2(x) dx + i\int_{[0, 1]}f_1(x)g_2(x) - f_2(x)g_1(x) dx  \\
        \end{align*}

        所以,
        \begin{align*}
          \langle g, f \rangle = \overline{\langle f, g \rangle}
        \end{align*}

  \item (b)

        \begin{itemize}
          \item (1) $\langle f, f \rangle \geq 0$。

                \begin{align*}
                  \langle f, f \rangle & = \int_{[0, 1]} f(x) \overline{f(x)} dx                                         \\
                                       & = \int_{[0, 1]} (f_1(x) + if_2(x))(f_1(x) - if_2(x))                            \\
                                       & = \int_{[0, 1]} f_1(x)f_1(x) + f_2(x)f_2(x) + i(-f_1(x)f_2(x) + f_2(x)f_1(x))dx \\
                                       & = \int_{[0, 1]} f_1(x)f_1(x) + f_2(x)f_2(x)dx
                \end{align*}

                因为
                \begin{align*}
                  f_1(x)f_1(x) + f_2(x)f_2(x) \geq 0
                \end{align*}

                由定理11.4.1(e)可知,
                \begin{align*}
                  \int_{[0, 1]} f_1(x)f_1(x) + f_2(x)f_2(x)dx \geq \int_{[0, 1]} 0 dx = 0
                \end{align*}
                即
                \begin{align*}
                  \langle f, f \rangle \geq 0
                \end{align*}

          \item $\langle f, f \rangle = 0$当且仅当$f = 0$。

                $\Leftarrow$ 是易证的,略。

                $\Rightarrow$

                反证法,假设$f$不是零函数,那么存在某个$x_0 \in [0, 1], f(x_0) \neq 0$,
                又因为$f$是连续的,那么,对$\epsilon = \frac{1}{2}|f(x_0)|$,
                存在$\delta > 0$,使得只要,$|x - x_0| < \delta$,就有
                \begin{align*}
                  |f(x) - f(x_0)| \leq \epsilon \\
                  |f(x_0)| - \epsilon \leq |f(x)| \leq |f(x_0)| + \epsilon
                \end{align*}
                (以上是复数的性质,与实数是一致的,具体证明在15-6-comment.tex文件中)
                所以,
                \begin{align*}
                  \int_{[x_0 - \delta, x_0 + \delta]} |f(x)|^2 dx > 0
                \end{align*}

                于是,我们有
                \begin{align*}
                  \langle f, f \rangle & = \int_{[0, 1]} f_1(x)f_1(x) + f_2(x)f_2(x)dx \\
                                       & = \int_{[0, 1]} |f(x)|^2 dx                   \\
                                       & = \int_{[0, x_0 - \delta]} |f(x)|^2 dx
                  + \int_{(x_0 - \delta, x_0 + \delta)} |f(x)|^2 dx
                  + \int_{[x_0 + \delta, 1]} |f(x)|^2 dx                               \\
                                       & > 0
                \end{align*}
                存在矛盾。

        \end{itemize}

  \item (c)

        证明略

  \item (d)

        证明略
\end{itemize}

\section*{16.2.2}

证明$(C(\mathbb{R}/\mathbb{Z};\mathbb{C}), d_{L^2})$。




\end{document}


