\documentclass{article}
\usepackage{mathtools} 
\usepackage{fontspec}
\usepackage[UTF8]{ctex}
\usepackage{amsthm}
\usepackage{mdframed}
\usepackage{xcolor}
\usepackage{amssymb}
\usepackage{amsmath}


% 定义新的带灰色背景的说明环境 zremark
\newmdtheoremenv[
  backgroundcolor=gray!10,
  % 边框与背景一致,边框线会消失
  linecolor=gray!10
]{zremark}{说明}


\begin{document}
\title{16.2 习题}
\author{张志聪}
\maketitle

\section*{16.2.1}

设$f(x) = f_1(x) + if_2(x), g(x) = g_1(x) + ig_2(x), h(x) = h_1(x) + ih_2(x), c = c_1 + ic_2$。

\begin{itemize}
  \item (a)

        \begin{align*}
          \langle g, f \rangle & = \int_{[0, 1]} g(x) \overline{f(x)} dx                                                         \\
                               & = \int_{[0, 1]} \left(g_1(x) + ig_2(x)\right) \left(f_1(x) - if_2(x)\right) dx                  \\
                               & = \int_{[0, 1]} g_1(x) f_1(x) + g_2(x)f_2(x) + i(-g_1(x)f_2(x) + g_2(x)f_1(x)) dx               \\
                               & = \int_{[0, 1]} g_1(x) f_1(x) + g_2(x)f_2(x) dx + i\int_{[0, 1]} g_2(x)f_1(x) - g_1(x)f_2(x) dx
        \end{align*}

        \begin{align*}
          \langle f, g \rangle & = \int_{[0, 1]} f(x) \overline{g(x)} dx                                                        \\
                               & = \int_{[0, 1]} (f_1(x) + if_2(x)) (g_1(x) - ig_2(x)) dx                                       \\
                               & = \int_{[0, 1]} f_1(x)g_1(x) + f_2(x)g_2(x) + i(-f_1(x)g_2(x) + f_2(x)g_1(x)) dx               \\
                               & = \int_{[0, 1]} f_1(x)g_1(x) + f_2(x)g_2(x) dx + i\int_{[0, 1]}-f_1(x)g_2(x) + f_2(x)g_1(x) dx \\
        \end{align*}
        于是,
        \begin{align*}
          \overline{\langle f, g \rangle} & = \int_{[0, 1]} f_1(x)g_1(x) + f_2(x)g_2(x) dx - i\int_{[0, 1]}-f_1(x)g_2(x) + f_2(x)g_1(x) dx \\
                                          & = \int_{[0, 1]} f_1(x)g_1(x) + f_2(x)g_2(x) dx + i\int_{[0, 1]}f_1(x)g_2(x) - f_2(x)g_1(x) dx  \\
        \end{align*}

        所以,
        \begin{align*}
          \langle g, f \rangle = \overline{\langle f, g \rangle}
        \end{align*}

  \item (b)

        \begin{itemize}
          \item (1) $\langle f, f \rangle \geq 0$。

                \begin{align*}
                  \langle f, f \rangle & = \int_{[0, 1]} f(x) \overline{f(x)} dx                                         \\
                                       & = \int_{[0, 1]} (f_1(x) + if_2(x))(f_1(x) - if_2(x))                            \\
                                       & = \int_{[0, 1]} f_1(x)f_1(x) + f_2(x)f_2(x) + i(-f_1(x)f_2(x) + f_2(x)f_1(x))dx \\
                                       & = \int_{[0, 1]} f_1(x)f_1(x) + f_2(x)f_2(x)dx
                \end{align*}

                因为
                \begin{align*}
                  f_1(x)f_1(x) + f_2(x)f_2(x) \geq 0
                \end{align*}

                由定理11.4.1(e)可知,
                \begin{align*}
                  \int_{[0, 1]} f_1(x)f_1(x) + f_2(x)f_2(x)dx \geq \int_{[0, 1]} 0 dx = 0
                \end{align*}
                即
                \begin{align*}
                  \langle f, f \rangle \geq 0
                \end{align*}

          \item $\langle f, f \rangle = 0$当且仅当$f = 0$。

                $\Leftarrow$ 是易证的,略。

                $\Rightarrow$

                反证法,假设$f$不是零函数,那么存在某个$x_0 \in [0, 1], f(x_0) \neq 0$,
                又因为$f$是连续的,那么,对$\epsilon = \frac{1}{2}|f(x_0)|$,
                存在$\delta > 0$,使得只要,$|x - x_0| < \delta$,就有
                \begin{align*}
                  |f(x) - f(x_0)| \leq \epsilon \\
                  |f(x_0)| - \epsilon \leq |f(x)| \leq |f(x_0)| + \epsilon
                \end{align*}
                (以上是复数的性质,与实数是一致的,具体证明在15-6-comment.tex文件中)
                所以,
                \begin{align*}
                  \int_{[x_0 - \delta, x_0 + \delta]} |f(x)|^2 dx > 0
                \end{align*}

                于是,我们有
                \begin{align*}
                  \langle f, f \rangle & = \int_{[0, 1]} f_1(x)f_1(x) + f_2(x)f_2(x)dx \\
                                       & = \int_{[0, 1]} |f(x)|^2 dx                   \\
                                       & = \int_{[0, x_0 - \delta]} |f(x)|^2 dx
                  + \int_{(x_0 - \delta, x_0 + \delta)} |f(x)|^2 dx
                  + \int_{[x_0 + \delta, 1]} |f(x)|^2 dx                               \\
                                       & > 0
                \end{align*}
                存在矛盾。

        \end{itemize}

  \item (c)

        证明略

  \item (d)

        证明略
\end{itemize}

\section*{16.2.2}

证明$(C(\mathbb{R}/\mathbb{Z};\mathbb{C}), d_{L^2})$是一个度量空间,
按照定义12.1.2,我们需要证明满足下面四个公理:
\begin{itemize}
  \item (a) 对任意的$f \in C(\mathbb{R}/\mathbb{Z};\mathbb{C})$,
        我们有$d_{L^2}(f, f) = 0$。

        \begin{align*}
          d_{L^2}(f, f) & = \| f - f\|_2                                       \\
                        & = \left(\int_{[0, 1]} |f(x) - f(x)|^2dx\right)^{1/2} \\
                        & = \left(\int_{[0, 1]} 0 dx\right)^{1/2}              \\
                        & = 0
        \end{align*}

  \item (b) (正性)对任意两个不同的$f, g \in C(\mathbb{R}/\mathbb{Z};\mathbb{C})$,
        我们有$d_{L^2}(f, g) > 0$。

        \begin{align*}
          d_{L^2}(f, g) & = \| f - g\|_2                                                            \\
                        & = \left(\int_{[0, 1]} |f(x) - g(x)|^2dx\right)^{1/2}                      \\
                        & = \left(\int_{[0, 1]} (f(x) - g(x))\overline{f(x) - g(x)} dx\right)^{1/2} \\
                        & \geq 0
        \end{align*}
        最后一个不等式,使用了引理16.2.5(b)。

  \item (c)(对称性)对任意的$f, g \in C(\mathbb{R}/\mathbb{Z};\mathbb{C})$,
        我们有$d_{L^2}(f, g) = d_{L^2}(g, f)$。

        \begin{align*}
          d_{L^2}(f, g) & = \| f - g\|_2                                       \\
                        & = \left(\int_{[0, 1]} |f(x) - g(x)|^2dx\right)^{1/2}
        \end{align*}

        \begin{align*}
          d_{L^2}(g, f) & = \| g - f\|_2                                       \\
                        & = \left(\int_{[0, 1]} |g(x) - f(x)|^2dx\right)^{1/2}
        \end{align*}

        由于,对任意$x \in [0, 1]$,都有
        \begin{align*}
          |f(x) - g(x)| & = |(-1)(g(x) - f(x))| \\
                        & = |-1||g(x) - f(x)|   \\
                        & = |g(x) - f(x)|
        \end{align*}

        所以,
        \begin{align*}
          \left(\int_{[0, 1]} |f(x) - g(x)|^2dx\right)^{1/2} = \left(\int_{[0, 1]} |g(x) - f(x)|^2dx\right)^{1/2}
        \end{align*}

        综上,$d_{L^2}(f, g) = d_{L^2}(g, f)$。

  \item (d) (三角不等式)对任意的$f,g,h \in C(\mathbb{R}/\mathbb{Z};\mathbb{C})$,
        我们有$d_{L^2}(f, h) \leq d_{L^2}(f, g) + d_{L^2}(g, h)$。

        这里有个开方,不好处理,需要使用一个小命题:
        \begin{zremark}
          \textbf{命题:实数$a, b, c \geq 0$,如果$a^2 + b^2 \geq c^2$,那么$a + b \geq c$。}

          \textbf{证明:}

          反证法,假设$a + b < c$,那么
          \begin{align*}
            (a + b)^2 < c^2 \\
            a^2 + b^2 + 2ab < c^2
          \end{align*}
          因为,$2ab \geq 0$,所以,
          \begin{align*}
            a^2 + b^2 + 2ab \geq a^2 + b^2
          \end{align*}

          $a^2 + b^2 + 2ab < c^2$与$a^2 + b^2 \geq c^2$矛盾。
        \end{zremark}

        \begin{align*}
          (d_{L^2}(f, g))^2 + (d_{L^2}(g, h))^2
           & = \int_{[0, 1]} |f(x) - g(x)|^2dx + \int_{[0, 1]} |g(x) - h(x)|^2dx \\
           & = \int_{[0, 1]} |f(x) - g(x)|^2 + |g(x) - h(x)|^2dx
        \end{align*}

        又因为
        \begin{align*}
          (d_{L^2}(f, h))^2
           & = \int_{[0, 1]} |f(x) - h(x)|^2dx
        \end{align*}

        由引理15.6.11的三角不等式可知,对任意$x \in [0, 1]$,
        \begin{align*}
          |f(x) - h(x)| & = |f(x) - g(x) + g(x) - h(x)|      \\
                        & \leq |f(x) - g(x)| + |g(x) - h(x)|
        \end{align*}

        综上可得,
        \begin{align*}
          |f(x) - h(x)|^2 \leq |f(x) - g(x)|^2 + |g(x) - h(x)|^2
        \end{align*}

        于是由定理11.4.1(e)可知,
        \begin{align*}
          \int_{[0, 1]} |f(x) - h(x)|^2dx \leq \int_{[0, 1]} |f(x) - g(x)|^2 + |g(x) - h(x)|^2dx
        \end{align*}
        即,
        \begin{align*}
          (d_{L^2}(f, h))^2 \leq (d_{L^2}(f, g))^2 + (d_{L^2}(g, h))^2
        \end{align*}
        由刚才的小命题可得,
        \begin{align*}
          d_{L^2}(f, h) \leq d_{L^2}(f, g) + d_{L^2}(g, h)
        \end{align*}
\end{itemize}

\section*{16.2.3}

(1)

由定义可知,$\|f\|_2, \|f\|_{L^\infty}$都是正值,
于是可以通过$(\|f\|_2)^2 \leq (\|f\|_{L^\infty})^2$来证明。

对任意$x \in [0, 1]$,我们有
\begin{align*}
  |f(x)| \leq \sup\limits_{x \in \mathbb{R}} |f(x)| = \sup\limits_{x \in [0, 1]} |f(x)| = \|f\|_{L^\infty}
\end{align*}

于是可得,
\begin{align*}
  (\|f\|_2)^2 = \int_{[0, 1]} |f(x)|^2dx  \leq \int_{[0, 1]} (\sup\limits_{x \in [0, 1]} |f(x)|)^2 dx
\end{align*}

又因为$\sup\limits_{x \in [0, 1]} |f(x)|$是常量,所以
\begin{align*}
  \int_{[0, 1]} (\sup\limits_{x \in [0, 1]} |f(x)|)^2 dx
   & = (1-0) \times (\sup\limits_{x \in [0, 1]} |f(x)|)^2 \\
   & = (\sup\limits_{x \in [0, 1]} |f(x)|)^2              \\
   & = (\|f\|_{L^\infty})^2
\end{align*}

综上可得,
\begin{align*}
  (\|f\|_2)^2 \leq (\|f\|_{L^\infty})^2 \\
  \implies                              \\
  \|f\|_2 \leq  \|f\|_{L^\infty}
\end{align*}

(2)

todo

\begin{align*}
  \|f\|_2^2 & = \int_{[0, 1]} |f(x)|^2dx   \\
            & = \int_{[0, 1]} c + dg(x) dx \\
            & = c + d \int_{[0, 1]} g(x)dx
\end{align*}
我们希望
\begin{equation*}
  \begin{cases*}
    \|f\|_2^2 = c + d \int_{[0, 1]} g(x)dx = A^2 \\
    \|f\|_\infty = \sup\limits_{x \in [0, 1]} |f(x)|
    = \sup\limits_{x \in [0, 1]} \sqrt{c + dg(x)}  = B
  \end{cases*}
\end{equation*}

\section*{16.2.4}

\begin{itemize}
  \item (a) 非退化性。

        由引理16.2.5(b)可得,
        \begin{align*}
          \|f\|_2 & = \sqrt{\langle f, f \rangle} \geq 0
        \end{align*}

  \item (b)柯西-施瓦茨不等式。

        如果$\|g\|_2 = 0$,由$\|g\|_2 = \sqrt{\langle g, g \rangle}$和引理16.2.5(b),
        $g = 0$,于是$|\langle f, g \rangle| = 0$,所以
        \begin{align*}
          |\langle f, g \rangle| = \|f\|_2\|g\|_2 = 0
        \end{align*}

        如果$\|g\|_2 > 0$,我们有
        \begin{align*}
           & \langle f\|g\|_2^2 - \langle f, g \rangle g, f\|g\|_2^2 - \langle f, g \rangle g \rangle                                      \\
           & = \langle  f\|g\|_2^2, f\|g\|_2^2 - \langle f, g \rangle g \rangle
          - \langle \langle f, g \rangle g, f\|g\|_2^2 - \langle f, g \rangle g \rangle                                                    \\
           & = \langle  f\|g\|_2^2, f\|g\|_2^2 \rangle - \langle  f\|g\|_2^2, \langle f, g \rangle g \rangle
          - (\langle \langle f, g \rangle g, f\|g\|_2^2 \rangle - \langle \langle f, g \rangle g, \langle f, g \rangle g \rangle)          \\
           & = \langle  f\|g\|_2^2, f\|g\|_2^2 \rangle - \langle  f\|g\|_2^2, \langle f, g \rangle g \rangle
          - \langle \langle f, g \rangle g, f\|g\|_2^2 \rangle + \langle \langle f, g \rangle g, \langle f, g \rangle g \rangle            \\
           & = \|g\|_2^4 \langle  f, f \rangle - \|g\|_2^2 \langle  f, \langle f, g \rangle g \rangle
          - \|g\|_2^2 \langle \langle f, g \rangle g, f \rangle + \langle \langle f, g \rangle g, \langle f, g \rangle g \rangle           \\
           & = \|g\|_2^4 \langle  f, f \rangle - \|g\|_2^2 \overline{\langle f, g \rangle}\langle  f, g \rangle
          - \|g\|_2^2 \langle f, g \rangle\langle g, f \rangle + \langle f, g \rangle \overline{\langle f, g \rangle} \langle g, g \rangle \\
           & = \|g\|_2^4 \langle  f, f \rangle - \|g\|_2^2 \overline{\langle f, g \rangle}\langle  f, g \rangle
          - \|g\|_2^2 \langle f, g \rangle\langle g, f \rangle + \langle f, g \rangle \langle g, f \rangle \|g\|_2^2                       \\
           & = \|g\|_2^4 \langle  f, f \rangle - \|g\|_2^2 \overline{\langle f, g \rangle}\langle  f, g \rangle                            \\
           & = \|g\|_2^4\|f\|_2^2 - \|g\|_2^2|\langle  f, g \rangle|^2                                                                     \\
        \end{align*}
        (注意,在拆分的时候,$\langle *, * \rangle, \|g\|_2, -1$都是复数,可以使用引理16.2.5(c)(d);
        最后一个等式利用了复数共轭相乘的性质,引理15.6.11中有说明)

        因为
        \begin{align*}
          \langle f\|g\|_2^2 - \langle f, g \rangle g, f\|g\|_2^2 - \langle f, g \rangle g \rangle \geq 0 \\
          \|g\|_2^4\|f\|_2^2 - \|g\|_2^2|\langle  f, g \rangle|^2 \geq 0                                  \\
          \|g\|_2^2\|f\|_2^2 - |\langle  f, g \rangle|^2 \geq 0                                           \\
          \|g\|_2^2\|f\|_2^2 \geq |\langle  f, g \rangle|^2                                               \\
          \|g\|_2\|f\|_2 \geq |\langle  f, g \rangle|
        \end{align*}


  \item (c) 三角不等式

        我们有,
        \begin{align*}
          \|f + g\|_2^2 & = \langle f + g, f + g \rangle                                                              \\
                        & = \langle f, f \rangle + \langle f, g \rangle + \langle g, f \rangle + \langle g, g \rangle \\
                        & \leq \|f\|_2^2 + \|f\|_2\|g\|_2 + \|g\|_2\|f\|_2 + \|g\|_2^2                                \\
                        & = (\|f\|_2 + \|g\|_2)^2
        \end{align*}
        于是
        \begin{align*}
          \|f + g\|_2 \leq \|f\|_2 + \|g\|_2
        \end{align*}

  \item (d) 毕达哥拉斯定理

        因为$\langle f, g \rangle = 0$,于是$\langle g, f \rangle = \overline{\langle f, g \rangle} = 0$。

        我们有,
        \begin{align*}
          \|f + g\|_2^2 & = \langle f + g, f + g \rangle                                                              \\
                        & = \langle f, f \rangle + \langle f, g \rangle + \langle g, f \rangle + \langle g, g \rangle \\
                        & = \|f\|_2^2 + 0 + 0 + \|g\|_2^2                                                             \\
                        & = \|f\|_2^2 + \|g\|_2^2
        \end{align*}

  \item (e) 齐次性

        \begin{align*}
          \|cf\|_2^2 & = \langle cf, cf \rangle  \\
                     & = c^2\langle f, f \rangle \\
                     & = c^2 \|f\|_2^2
        \end{align*}

        于是可得,
        \begin{align*}
          \|cf\|_2 = |c| \|f\|_2
        \end{align*}
\end{itemize}

\section*{16.2.5}

先定义不连续的$\mathbb{Z}$周期方波函数
\begin{equation*}
  f(x) =
  \begin{cases*}
    1 \ if \ x \in [n, n + \frac{1}{2}), \\
    0 \ if \ x \in [n + \frac{1}{2}, n + 1)
  \end{cases*}
\end{equation*}

构造连续的$\mathbb{Z}$周期函数序列$(f_n)_{n = 1}^\infty$。

\begin{equation*}
  f_n(x) =
  \begin{cases*}
    1 \ \ if \ x \in [0, \frac{1}{2} - \frac{1}{2n}),                                                              \\
    1 - n(x - \frac{1}{2} + \frac{1}{2n}) \ \ if \ x \in [\frac{1}{2} - \frac{1}{2n}, \frac{1}{2} + \frac{1}{2n}), \\
    0 \ \ if \ x \in [\frac{1}{2} + \frac{1}{2n}, 1)
  \end{cases*}
\end{equation*}
(注意,以上只定义了$[0, 1)$)于是要做一下补充:$x \in \mathbb{R}, f_n(x + 1) = f_n(x)$。

以上定义的$f_n(x)$关键在于$[\frac{1}{2} - \frac{1}{2n}, \frac{1}{2} + \frac{1}{2n})$上,函数线性从1到0。

\section*{16.2.6}

\begin{itemize}
  \item (a)

        对任意$\sqrt{\epsilon} > 0$,因为$f_n$一致收敛于$f$,
        于是存在$N \geq 1$,使得只要$n \geq N$和$x \in \mathbb{R}$,都有
        \begin{align*}
          |f_n(x) - f(x)| < \sqrt{\epsilon}
        \end{align*}

        于是,当$n \geq N$时
        \begin{align*}
          \int_{[0, 1]} |f_n(x) - f(x)|^2 dx \leq \int_{[0, 1]} \epsilon dx = \epsilon
        \end{align*}

        由$\epsilon$的任意性可知,
        \begin{align*}
          \lim\limits_{n \to \infty} \int_{[0, 1]} |f_n(x) - f(x)|^2 dx = 0
        \end{align*}

  \item (b)

        todo

  \item (c)

        todo

  \item (d)

        todo

\end{itemize}


\end{document}


