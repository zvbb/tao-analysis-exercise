\documentclass{article}
\usepackage{mathtools} 
\usepackage{fontspec}
\usepackage[UTF8]{ctex}
\usepackage{amsthm}
\usepackage{mdframed}
\usepackage{xcolor}
\usepackage{amssymb}
\usepackage{amsmath}


% 定义新的带灰色背景的说明环境 zremark
\newmdtheoremenv[
  backgroundcolor=gray!10,
  % 边框与背景一致,边框线会消失
  linecolor=gray!10
]{zremark}{说明}


\begin{document}
\title{16.4 习题}
\author{张志聪}
\maketitle

\section*{16.4.1}

反证法,假设$f$不是恒等于零的。

$f$是紧支撑的,不妨设其支撑在区间$[a, b]$上,
于是,对所有的$x \in [a, b]$都有$f(x) \neq 0$。

因为$f$不是恒等于零的,所以存在$x_0 \in [a, b]$,使得$f(x_0) \neq 0$。
又存在整数$N$,使得$N + x_0 > b$,因为$x_0 + N \notin [a, b]$,
所以$f(x_0 + N) = 0$。

又因为$f$是$\mathbb{Z}$周期函数,
所以$f(x_0 + N) = f(x_0) \neq 0$。

存在矛盾。

\section*{16.4.2}

先证明$f$是一致连续的,因为$f$在$\mathbb{R}$上是连续的,
所以$f$在$([0, 1], d)$这个紧致度量空间上是连续的,
于是由定理13.3.5可知,$f$是一致连续的。这可以周期性地推广到整个$\mathbb{R}$上。
(这里无法直接使用定理9.9.16,因为这里的值域是复数)。

同理可得,$g,h$是一致连续的。

\begin{itemize}
  \item (a) 封闭性

        \begin{itemize}
          \item (1)连续性

                因为$f$是有界的,所以存在一个$M > 0$,使得对于所有的$x \in \mathbb{R}$都有$|f(x)| \leq M$。

                设$\epsilon > 0$是任意的,因为$g$是一致连续的,所以存在一个$\delta > 0$,使得只要
                $|x - y| \leq \delta$,就有$|g(x) - g(y)| \leq \epsilon$。


                对任意$x_0 \in \mathbb{R}, |x - x_0| < \delta$,
                \begin{align*}
                   & |f \ast g(x) - f \ast g(x_0)|                                                    \\
                   & = \left|\int_{[0, 1]} f(y) g(x - y) dy - \int_{[0, 1]} f(y) g(x_0 - y) dy\right| \\
                   & = \left|\int_{[0, 1]} f(y) (g(x - y) - g(x_0 - y)) dy \right|                    \\
                   & \leq \int_{[0, 1]} |f(y) (g(x - y) - g(x_0 - y))| dy                             \\
                   & \leq \int_{[0, 1]} M|(g(x - y) - g(x_0 - y))| dy                                 \\
                   & \leq M \int_{[0, 1]} \epsilon dy                                                 \\
                   & = M \epsilon
                \end{align*}

                于是有$|f \ast g(x) - f \ast g(x_0)| \leq M \epsilon$。由于$M$是定制并且$\epsilon$是任意的,
                因此我我们可以得出$f \ast g$在$x_0$处连续的。

                由$x_0$的任意性,$f \ast g$连续。
          \item (2)$\mathbb{Z}$周期

                设$k$是整数,因为$f, g \in C(\mathbb{R}/\mathbb{Z}; \mathbb{C})$,
                我们有
                \begin{align*}
                  f \ast  g(x + k) & = \int_{[0, 1]} f(y) g(x+k - y) dy \\
                                   & = \int_{[0, 1]} f(y) g(x - y) dy   \\
                                   & = f \ast g(x)
                \end{align*}

                所以$f \ast g$是$\mathbb{Z}$周期的。
        \end{itemize}

  \item (b) 交换性

        \begin{align*}
          f \ast g (x) & = \int_{[0, 1]} f(y) g(x - y) dy
        \end{align*}

        令$u = x - y$,则$y = x - u$。当$y$从$0 \to 1$时,$u$从$x \to x - 1$。
        但由于$f,g$都是周期为$1$的函数,积分可以调整到任意长度为$1$的区间,
        因此我们将积分限改为$0 \to 1$:
        \begin{align*}
          g \ast f (x) & = \int_{[0, 1]} g(y) f(x - y) dy      \\
                       & = \int_{[x, x-1]} g(x - u) f(u) (-du) \\
                       & = \int_{[x - 1, x]} g(x - u) f(u) du  \\
                       & = \int_{[0, 1]} f(u) g(x - u)  du     \\
                       & = \int_{[0, 1]} f(y) g(x - y) dy
        \end{align*}

        所以,$f \ast g = g \ast f$。

  \item (c) 双线性性质
\end{itemize}

\end{document}


