\documentclass{article}
\usepackage{mathtools} 
\usepackage{fontspec}
\usepackage[UTF8]{ctex}
\usepackage{amsthm}
\usepackage{mdframed}
\usepackage{xcolor}
\usepackage{amssymb}
\usepackage{amsmath}


% 定义新的带灰色背景的说明环境 zremark
\newmdtheoremenv[
  backgroundcolor=gray!10,
  % 边框与背景一致,边框线会消失
  linecolor=gray!10
]{zremark}{说明}


\begin{document}
\title{16.4 习题}
\author{张志聪}
\maketitle

\section*{16.4.1}

反证法,假设$f$不是恒等于零的。

$f$是紧支撑的,不妨设其支撑在区间$[a, b]$上,
于是,对所有的$x \in [a, b]$都有$f(x) \neq 0$。

因为$f$不是恒等于零的,所以存在$x_0 \in [a, b]$,使得$f(x_0) \neq 0$。
又存在整数$N$,使得$N + x_0 > b$,因为$x_0 + N \notin [a, b]$,
所以$f(x_0 + N) = 0$。

又因为$f$是$\mathbb{Z}$周期函数,
所以$f(x_0 + N) = f(x_0) \neq 0$。

存在矛盾。

\section*{16.4.2}

先证明$f$是一致连续的,因为$f$在$\mathbb{R}$上是连续的,
所以$f$在$([0, 1], d)$这个紧致度量空间上是连续的,
于是由定理13.3.5可知,$f$是一致连续的。这可以周期性地推广到整个$\mathbb{R}$上。
(这里无法直接使用定理9.9.16,因为这里的值域是复数)。

同理可得,$g,h$是一致连续的。

\begin{itemize}
  \item (a) 封闭性

        \begin{itemize}
          \item (1)连续性

                因为$f$是有界的,所以存在一个$M > 0$,使得对于所有的$x \in \mathbb{R}$都有$|f(x)| \leq M$。

                设$\epsilon > 0$是任意的,因为$g$是一致连续的,所以存在一个$\delta > 0$,使得只要
                $|x - y| \leq \delta$,就有$|g(x) - g(y)| \leq \epsilon$。


                对任意$x_0 \in \mathbb{R}, |x - x_0| < \delta$,
                \begin{align*}
                   & |f \ast g(x) - f \ast g(x_0)|                                                    \\
                   & = \left|\int_{[0, 1]} f(y) g(x - y) dy - \int_{[0, 1]} f(y) g(x_0 - y) dy\right| \\
                   & = \left|\int_{[0, 1]} f(y) (g(x - y) - g(x_0 - y)) dy \right|                    \\
                   & \leq \int_{[0, 1]} |f(y) (g(x - y) - g(x_0 - y))| dy                             \\
                   & \leq \int_{[0, 1]} M|(g(x - y) - g(x_0 - y))| dy                                 \\
                   & \leq M \int_{[0, 1]} \epsilon dy                                                 \\
                   & = M \epsilon
                \end{align*}

                于是有$|f \ast g(x) - f \ast g(x_0)| \leq M \epsilon$。由于$M$是定制并且$\epsilon$是任意的,
                因此我我们可以得出$f \ast g$在$x_0$处连续的。

                由$x_0$的任意性,$f \ast g$连续。
          \item (2)$\mathbb{Z}$周期

                设$k$是整数,因为$f, g \in C(\mathbb{R}/\mathbb{Z}; \mathbb{C})$,
                我们有
                \begin{align*}
                  f \ast  g(x + k) & = \int_{[0, 1]} f(y) g(x+k - y) dy \\
                                   & = \int_{[0, 1]} f(y) g(x - y) dy   \\
                                   & = f \ast g(x)
                \end{align*}

                所以$f \ast g$是$\mathbb{Z}$周期的。
        \end{itemize}

  \item (b) 交换性

        \begin{align*}
          f \ast g (x) & = \int_{[0, 1]} f(y) g(x - y) dy
        \end{align*}

        令$u = x - y$,则$y = x - u$。当$y$从$0 \to 1$时,$u$从$x \to x - 1$。
        但由于$f,g$都是周期为$1$的函数,积分可以调整到任意长度为$1$的区间,
        因此我们将积分限改为$0 \to 1$:
        \begin{align*}
          g \ast f (x) & = \int_{[0, 1]} g(y) f(x - y) dy         \\
                       & = \int_{[x, x-1]} g(x - u) f(u) d(x - u) \\
                       & = \int_{[x - 1, x]} g(x - u) f(u) du     \\
                       & = \int_{[0, 1]} f(u) g(x - u)  du        \\
                       & = \int_{[0, 1]} f(y) g(x - y) dy
        \end{align*}

        所以,$f \ast g = g \ast f$。

  \item (c) 双线性性质
        \begin{align*}
          f \ast (g + h)
           & = \int_{[0, 1]} f(y) (g + h)(x - y) dy                         \\
           & = \int_{[0, 1]} f(y) (g(x - y) + h(x - y)) dy                  \\
           & = \int_{[0, 1]} f(y)g(x - y) + f(y)h(x - y) dy                 \\
           & = \int_{[0, 1]} f(y)g(x - y)dy + \int_{[0, 1]} f(y)h(x - y) dy \\
           & = f \ast g + f \ast h
        \end{align*}

        \begin{align*}
          (f + g) \ast h
           & = \int_{[0, 1]} (f+g)(y) h(x - y) dy           \\
           & = \int_{[0, 1]} (f(y)+g(y)) h(x - y) dy        \\
           & = \int_{[0, 1]} f(y)h(x - y) + g(y)h(x - y) dy \\
           & = f \ast h + g \ast h
        \end{align*}

        \begin{align*}
          c(f \ast g)
           & = c \int_{[0, 1]} f(y) g(x - y) dy \\
           & = \int_{[0, 1]} cf(y) g(x - y) dy  \\
           & = (cf) \ast g
        \end{align*}

        \begin{align*}
          c(f \ast g)
           & = c \int_{[0, 1]} f(y) g(x - y) dy \\
           & = \int_{[0, 1]} f(y) cg(x - y) dy  \\
           & = f \ast (cg)
        \end{align*}
\end{itemize}

\section*{16.4.3}

\begin{itemize}
  \item (1)

        \begin{align*}
          F_N & = \sum \limits_{n = -N}^{N} (1 - \frac{|n|}{N})e_n   \\
              & = \sum \limits_{n = -N}^{N} (\frac{N - |n|}{N})e_n   \\
              & = \frac{1}{N} \sum \limits_{n = -N}^{N} (N - |n|)e_n
        \end{align*}

        接下来,我们证明$\sum \limits_{n = -N}^{N} (N - |n|)e_n = |\sum \limits_{n = 0}^{N-1} e_n|^2$,即可完成证明。

        对$N$进行归纳。

        归纳基始,$N = 1$时,
        \begin{align*}
          \sum \limits_{n = -N}^{N} (N - |n|)e_n
           & = \sum \limits_{n = -1}^{1} (1 - |n|)e_n      \\
           & = (1 - |-1|)e_n + (1 - |0|)e_n + (1 - |n|)e_n \\
           & = e_n
        \end{align*}

        \begin{align*}
          |\sum \limits_{n = 0}^{N-1} e_n|^2
           & = \left|\sum\limits_{n = 0}^{0}e_n\right|^2 \\
           & = |e_0|^2                                   \\
           & = |e^{2\pi 0 i}|^2                          \\
           & = |1|^2                                     \\
           & = 1
        \end{align*}
        命题成立。

        归纳假设$N = K$时,命题成立。

        $N = K + 1$时,
        \begin{align*}
           & \sum \limits_{n = -(K + 1)}^{K + 1} (K + 1 - |n|)e_n                     \\
           & = \sum \limits_{n = -K}^{K} (K + 1 - |n|)e_n
          + \sum \limits_{n = -(K + 1)}^{-(K + 1)} (K + 1 - |n|)e_n
          + \sum \limits_{n = K + 1}^{K + 1} (K + 1 - |n|)e_n                         \\
           & = \sum \limits_{n = -K}^{K} e_n + \sum \limits_{n = -K}^{K} (K - |n|)e_n
          +  (K + 1 - |-(K + 1)|)e_{-(K + 1)}
          +  (K + 1 - |K + 1|)e_{K + 1}                                               \\
           & = \sum \limits_{n = -K}^{K} e_n + \sum \limits_{n = -K}^{K} (K - |n|)e_n
        \end{align*}

        \begin{align*}
           & \left|\sum\limits_{n = 0}^{K} e_n\right|^2                                                                        \\
           & = \left|\sum\limits_{n = 0}^{K-1} e_n + \sum\limits_{n = K}^{K} e_n\right|^2                                      \\
           & = \left|(\sum\limits_{n = 0}^{K-1} e_n) + e_K\right|^2                                                            \\
           & = \left((\sum\limits_{n = 0}^{K-1} e_n) + e_K\right)\overline{\left((\sum\limits_{n = 0}^{K-1} e_n) + e_K\right)} \\
           & = \left((\sum\limits_{n = 0}^{K-1} e_n) + e_K\right)\left((\sum\limits_{n = 0}^{K-1} e_{-n}) + e_{-K}\right)      \\
           & = \left((\sum\limits_{n = 0}^{K-1} e_n) + e_K\right)\left((\sum\limits_{n = 0}^{K-1} e_{-n}) + e_{-K}\right)      \\
           & = (\sum\limits_{n = 0}^{K-1} e_n)(\sum\limits_{n = 0}^{K-1} e_{-n})
          + (\sum\limits_{n = 0}^{K-1} e_n) e_{-K}
          + e_K(\sum\limits_{n = 0}^{K-1} e_{-n})
          + e_Ke_{-K}                                                                                                          \\
           & = \left|\sum\limits_{n = 0}^{K-1} e_n\right|^2
          + \sum\limits_{n = 0}^{K-1} e_{n - K}
          + (\sum\limits_{n = 0}^{K-1} e_{-n+K})
          + e_0                                                                                                                \\
           & = \sum \limits_{n = -K}^{K} (K - |n|)e_n
          + \sum\limits_{n = -K}^{-1} e_n
          + (\sum\limits_{n = 1}^{K} e_{n})
          + e_0                                                                                                                \\
           & = \sum \limits_{n = -K}^{K} (K - |n|)e_n + \sum \limits_{n = -K}^{K} e_n
        \end{align*}
        归纳完成,命题得证。

  \item (2)

        \begin{align*}
          \sum \limits_{n = 0}^{N - 1} e_n
           & = \sum \limits_{n = 0}^{\infty} e_n - \sum \limits_{n = N}^{\infty} e_n                  \\
           & = \sum \limits_{n = 0}^{\infty} e_n - e_N\sum \limits_{n = N}^{\infty} e_{n - N}         \\
           & = \sum \limits_{n = 0}^{\infty} e_n - e_N\sum \limits_{n = 0}^{\infty} e_n               \\
           & = \frac{1}{1 - e_1} - e_N\frac{1}{1 - e_1}                                               \\
           & = \frac{1 - e_N}{1 - e_1}                                                                \\
           & = \frac{e_N - 1}{e_1 - 1}                                                                \\
           & = \frac{e_N - e_0}{e_1 - e_0}                                                            \\
           & = \frac{e^{2\pi i N x} - e^0}{e^{2\pi i x} - e^0}                                        \\
           & = \frac{e^{2\pi i N x - \pi i x} - e^{-\pi i x}}{e^{\pi i x} - e^{-\pi i x}}             \\
           & = \frac{e^{\pi i N x + \pi i N x- \pi i x} - e^{-\pi i x}}{e^{\pi i x} - e^{-\pi i x}}   \\
           & = \frac{e^{\pi i (N - 1) x + \pi i N x} - e^{-\pi i x}}{e^{\pi i x} - e^{-\pi i x}}      \\
           & = \frac{e^{\pi i (N - 1) x}(e^{\pi i N x} - e^{-\pi i N x})}{e^{\pi i x} - e^{-\pi i x}} \\
           & = \frac{e^{\pi i (N - 1) x}(2isin(\pi Nx))}{2isin(\pi x)}                                \\
           & = \frac{e^{\pi i (N - 1) x}sin(\pi Nx)}{sin(\pi x)}
        \end{align*}

  \item (3)

        \begin{align*}
          \int_{[0, 1]} F_N(x) dx
           & = \int_{[0, 1]} \sum \limits_{n = -N}^N (1 - \frac{|n|}{N})e_n dx  \\
           & = \sum \limits_{n = -N}^N \int_{[0, 1]} (1 - \frac{|n|}{N})e_n dx  \\
           & = \sum \limits_{n = -N}^N (1 - \frac{|n|}{N}) \int_{[0, 1]} e_n dx \\
        \end{align*}
        因为$n = 0$时,$\int_{[0, 1]} e_n dx = 1$,$n > 0$时,$\int_{[0, 1]} e_n dx = 0$,
        所以
        \begin{align*}
           & \sum \limits_{n = -N}^N (1 - \frac{|n|}{N}) \int_{[0, 1]} e_n dx \\
           & = \sum \limits_{n = 0}^0(1 - \frac{|0|}{N}) 1                      \\
           & = (1 - \frac{|0|}{N}) 1                                            \\
           & = 1
        \end{align*}




\end{itemize}

\end{document}


