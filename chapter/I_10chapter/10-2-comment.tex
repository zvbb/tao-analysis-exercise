\documentclass{article}
\usepackage{mathtools} 
\usepackage{fontspec}
\usepackage[UTF8]{ctex}
\usepackage{amsthm}
\usepackage{mdframed}
\usepackage{xcolor}
\usepackage{amssymb}
\usepackage{amsmath}


% 定义新的带灰色背景的说明环境 zremark
\newmdtheoremenv[
  backgroundcolor=gray!10,
  % 边框与背景一致,边框线会消失
  linecolor=gray!10
]{zremark}{说明}


\begin{document}
\title{10.2 注释}
\author{张志聪}
\maketitle

\section*{1}

本科高等数学有一个重要的定理:“柯西中值定理”,这里做一个补充。

\begin{zremark}
  如果函数$f(x)$及$F(x)$满足
  \begin{itemize}
    \item (1)在闭区间$[a, b]$上连续;
    \item (2)在开区间$(a, b)$内可导;
    \item (3)对任一$x \in (a, b)$,$F^\prime(x) \not = 0$,
  \end{itemize}
  那么在$(a, b)$内至少有一个$\xi$,使等式
  \begin{align}
    \frac{f(b) - f(a)}{F(b) - F(a)} = \frac{f^\prime(\xi)}{F^\prime(\xi)}
  \end{align}
  成立。
\end{zremark}

证明:

注意这里无法直接使用推论10.2.9(中值定理)。

接下来的证明方法与同济版本一致。

设函数
\begin{align*}
  \phi(x) & = F(x)\frac{f(b) - f(a)}{F(b) - F(a)} - f(x)                    \\
          & = \frac{F(x)f(b) - F(x)f(a) - f(x)F(b) + f(x)F(a)}{F(b) - F(a)}
\end{align*}

因为
\begin{align*}
  \phi(a) = \frac{F(a)f(b) - F(a)f(a) - f(a)F(b) + f(a)F(a)}{F(b) - F(a)} = \frac{F(a)f(b)- f(a)F(b)}{F(b) - F(a)} \\
  \phi(b) = \frac{F(b)f(b) - F(b)f(a) - f(b)F(b) + f(b)F(a)}{F(b) - F(a)} = \frac{- F(b)f(a) + f(b)F(a)}{F(b) - F(a)}
\end{align*}

于是我们有
\begin{align*}
  \phi(a) = \phi(b)
\end{align*}

利用定理10.2.8(罗尔定理)可得,存在$\xi \in (a, b)$使得$\phi^\prime(\xi) = 0$,
即:
\begin{align*}
   & F^\prime(\xi)\frac{f(b) - f(a)}{F(b) - F(a)} - f^\prime(\xi) = 0      \\
   & \implies                                                              \\
   & \frac{f(b) - f(a)}{F(b) - F(a)} = \frac{f^\prime(\xi)}{F^\prime(\xi)}
\end{align*}

\end{document}
