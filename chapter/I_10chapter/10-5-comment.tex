\documentclass{article}
\usepackage{mathtools} 
\usepackage{fontspec}
\usepackage[UTF8]{ctex}
\usepackage{amsthm}
\usepackage{mdframed}
\usepackage{xcolor}
\usepackage{amssymb}
\usepackage{amsmath}


% 定义新的带灰色背景的说明环境 zremark
\newmdtheoremenv[
  backgroundcolor=gray!10,
  % 边框与背景一致,边框线会消失
  linecolor=gray!10
]{zremark}{}


\begin{document}
\title{10.5 注释}
\author{张志聪}
\maketitle

\begin{zremark}
  \textbf{个人判断,陶哲轩书中的洛必达法则的表述存在问题。}
\end{zremark}

主要有以下问题:
\begin{itemize}
  \item II中$\lim\limits_{x \to a; x \in (a, b]} \frac{f^\prime(x)}{g^\prime(x)}$是一定存在的。

        题设中$f,g$在$[a, b]$上是可微的,且任意$x \in [a,b], g^\prime(x) \neq 0$,
        即$f^\prime(a),g^\prime(a)$存在,且$g^\prime(a) \neq 0$,
        此时,按照定义10.1.1可得,
        \begin{align*}
          \lim\limits_{x \to a; x \in [a, b] \setminus \{a\}} \frac{f(x) - f(a)}{x - a}
          = f^\prime(a) \\
          \lim\limits_{x \to a; x \in [a, b] \setminus \{a\}} \frac{g(x) - g(a)}{x - a}
          = g^\prime(a) \\
        \end{align*}

        集合$[a, b] \setminus \{a\} = (a, b]$,按照命题9.3.14,极限就是,
        \begin{align*}
           & \lim\limits_{x \to a; x \in (a, b]} \frac{f^\prime(x)}{g^\prime(x)}                               \\
           & = \lim\limits_{x \to a; x \in (a, b]} \frac{\frac{f(x) - f(a)}{x - a}}{\frac{g(x) - g(a)}{x - a}} \\
           & = \frac{f^\prime(a)}{g^\prime(a)}
        \end{align*}
        此时为啥还要判断存在性呢???
\end{itemize}


\textcolor{red}{后来,在第1版的《陶哲轩实分析》勘误网页中发现,已经说明了其中问题。}

网页:https://terrytao.wordpress.com/books/analysis-i/

原文如下:
\begin{zremark}
  In Proposition 10.5.2,
  the hypothesis that f,g be differentiable on [a,b] may be weakened
  to being continuous on [a,b] and differentiable on (a,b],
  with g' only assumed to be non-zero on (a,b] rather than [a,b].

  翻译:
  在命题10.5.2中,
  关于f和g在$[a,b]$上可微的假设可以弱化为:在$[a,b]$上连续且在$(a,b]$上可微,
  且$g^\prime$仅需在$(a,b]$上非零(而非原条件要求的$[a,b]$上)。
\end{zremark}


\section*{2 \textbf{和我本科期间学的洛必达定理表达不一样。}}

\begin{itemize}
  \item 本科期间是$\lim\limits_{x \to a} f(x) = 0$,而不是$f(a) = 0$,后者的条件更强了。
\end{itemize}


\section*{2.1 定理1}

\begin{zremark}
  设
  \begin{itemize}
    \item (1) 当$x \to a$时,函数$f(x)$及$F(x)$都趋于零。
    \item (2)在点$a$的某个去心领域内,$f^\prime(x)$及$F^\prime(x)$都存在且$F^\prime(x) \not = 0$。
    \item (3)$\lim\limits_{x \to a} \frac{f^\prime(x)}{F^\prime(x)}$存在(或为无穷大),
          则
          \begin{align*}
            \lim\limits_{x \to a} \frac{f(x)}{F(x)} = \lim\limits_{x \to a} \frac{f^\prime(x)}{F^\prime(x)}
          \end{align*}
  \end{itemize}
\end{zremark}

\textbf{证明}:

需做如下转换:

设函数$h$如下:$x = a$时$h(a) = 0$(这是为了保证$h$在$a$处是连续的),$x \not = a$时,$h(x) = f(x)$。

同理设函数$H$如下,$x = a$时$H(a) = 0$,$x \not = a$时,$H(x) = h(x)$。

由前置条件(2),我们设$a$的去心领域为$(a - \epsilon, a + \epsilon), 0 < \delta < \epsilon$。

\begin{itemize}
  \item  方法一:直接利用命题10.5.2进行证明。

        于是
        \begin{align*}
          \lim\limits_{x \to a; x \in (a, \delta]} \frac{f^\prime(x)}{F^\prime(x)}
           & = \lim\limits_{x \to a; x \in (a, \delta]} \frac{h^\prime(x)}{H^\prime(x)}
        \end{align*}

        利用命题10.5.2,得
        \begin{align*}
          \lim\limits_{x \to a; x \in (a, \delta]} \frac{h^\prime(x)}{H^\prime(x)}
           & = \lim\limits_{x \to a; x \in (a, \delta]} \frac{h(x)}{H(x)} \\
           & = \lim\limits_{x \to a; x \in (a, \delta]} \frac{f(x)}{F(x)}
        \end{align*}

        左极限证明类似。

        综上,命题成立。

  \item 同济版证明(有改动)

        同济版的证明使用了“柯西中值定理”,定理的具体内容,
        在10-2-comment.tex中有说明。

        因为对任意$x \in (a, \delta), \delta > 0$,
        由“柯西中值定理”可知,存在$\xi(x) \in (a, x)$($\xi$是关于$x$的函数),使得
        \begin{align*}
           & \frac{h(x) - h(a)}{H(x) - H(a)} = \frac{h^\prime(\xi(x))}{H^\prime(\xi(x))} \\
           & \implies                                                                    \\
           & \frac{h(x)}{H(x)} = \frac{h^\prime(\xi(x))}{H^\prime(\xi(x))}
        \end{align*}

        当$x \to a$时,$\xi(x)$也收敛于$a$,
        由题设(3)和$h,H$的构造方式,我们有,
        \begin{align*}
          \lim\limits_{x \to a} \frac{h^\prime(\xi(x))}{H^\prime(\xi(x))}
          =
          \lim\limits_{x \to a} \frac{f^\prime(\xi(x))}{F^\prime(\xi(x))}
        \end{align*}

        于是由极限定律可知,
        \begin{align*}
          \lim\limits_{x \to a} \frac{h(x)}{H(x)} = \lim\limits_{x \to a} \frac{h^\prime(\xi(x))}{H^\prime(\xi(x))}
        \end{align*}

        又因为$x \in (0, \delta)$,我们有
        \begin{align*}
          \frac{f(x)}{F(x)}
          = \frac{h(x)}{H(x)}
          = \frac{h^\prime(\xi(x))}{H^\prime(\xi(x))}
          = \frac{f^\prime(\xi(x))}{F^\prime(\xi(x))}
        \end{align*}

        综上可得,
        \begin{align*}
          \lim\limits_{x \to a} \frac{f(x)}{F(x)}
          =\lim\limits_{x \to a} \frac{h(x)}{H(x)}
          = \lim\limits_{x \to a} \frac{h^\prime(\xi(x))}{H^\prime(\xi(x))}
          = \lim\limits_{x \to a} \frac{f^\prime(\xi(x))}{F^\prime(\xi(x))}
        \end{align*}

        左极限同理。
\end{itemize}


\section*{2.2 定理}

\begin{zremark}
  设
  \begin{itemize}
    \item (1)当$x \to a$时,函数$f(x)$及$F(x)$都趋于无穷大。
    \item (2)在点$a$的某个去心领域内,$f^\prime(x)$及$F^\prime(x)$都存在且$F^\prime(x) \not = 0$。
    \item (3)$\lim\limits_{x \to a} \frac{f^\prime(x)}{F^\prime(x)}$存在(或为无穷大),
          则
          \begin{align*}
            \lim\limits_{x \to a} \frac{f(x)}{F(x)} = \lim\limits_{x \to a} \frac{f^\prime(x)}{F^\prime(x)}
          \end{align*}
  \end{itemize}
\end{zremark}

\textbf{证明}:

证明使用了“柯西中值定理”,定理的具体内容,
在10-2-comment.tex中有说明。

因为$f(x)$及$F(x)$都趋于无穷大,所以无法保证在$a$处的连续性,为使用“柯西中值定理”需要特殊处理下。

任意$x \in (a, a + \delta)$,取$y = \frac{1}{2}(x + a)$,于是可得$a < y < x$,

由柯西中值定理,我们有
\begin{align*}
  \frac{f(x)}{F(x)} = \frac{f^\prime(\xi(x))}{F^\prime(\xi(x))}
\end{align*}
其中$\xi(x) \in (y, x)$。

因为$x \to a$,$\xi(x)$也收敛于$a$,我们有
\begin{align*}
  \lim\limits_{x \to a} \frac{f^\prime(\xi(x))}{F^\prime(\xi(x))} = \lim\limits_{x \to a} \frac{f^\prime(x)}{F^\prime(x)}
\end{align*}

不妨设$\lim\limits_{x \to a} \frac{f^\prime(\xi(x))}{F^\prime(\xi(x))} = L$。

对任意$\epsilon > 0$,存在$\delta_1 > 0$,使得只要$|x - a| < \delta_1$,就有
\begin{align*}
  |\frac{f^\prime(\xi(x))}{F^\prime(\xi(x))} - L| < \epsilon
\end{align*}

因为$\frac{f(x)}{F(x)} = \frac{f^\prime(\xi(x))}{F^\prime(\xi(x))}$,所以
\begin{align*}
  |\frac{f(x)}{F(x)} - L| < \epsilon
\end{align*}

由$\epsilon$的任意性可得,$\lim\limits_{x \to a} \frac{f(x)}{F(x)} = L$。

综上,
\begin{align*}
  \lim\limits_{x \to a} \frac{f(x)}{F(x)} = \lim\limits_{x \to a} \frac{f^\prime(x)}{F^\prime(x)}
\end{align*}

左极限证明类似。

\section*{2.3 定理}
\begin{zremark}
  设
  \begin{itemize}
    \item (1)当$x \to +\infty$时,函数$f(x)$及$F(x)$都趋于零。
    \item (2)存在$r > 0$,使得$x > r, f^\prime(x)$及$F^\prime(x)$都存在且$F^\prime(x) \not = 0$。
    \item (3)$\lim\limits_{x \to \infty} \frac{f^\prime(x)}{F^\prime(x)}$存在(或为无穷大),
          则
          \begin{align*}
            \lim\limits_{x \to +\infty} \frac{f(x)}{F(x)} = \lim\limits_{x \to +\infty} \frac{f^\prime(x)}{F^\prime(x)}
          \end{align*}
  \end{itemize}
\end{zremark}

\textbf{证明}:

定义$\phi(x) = \frac{1}{x}$,于是
\begin{align*}
  \lim\limits_{x \to 0^{+}} \phi(x) = +\infty
\end{align*}

于是
\begin{align*}
  \lim\limits_{x \to 0^{+}} \frac{f^\prime(\phi(x))}{F^\prime(\phi(x))}
  =
  \lim\limits_{x \to \infty} \frac{f^\prime(x)}{F^\prime(x)}
\end{align*}
(注意:使用了下文中的3定理)

利用2.1定理可得,
\begin{align*}
  \lim\limits_{x \to 0^{+}} \frac{f(\phi(x))}{F(\phi(x))}
  =
  \lim\limits_{x \to 0^{+}} \frac{f^\prime(\phi(x))}{F^\prime(\phi(x))}
\end{align*}

反方向使用刚才的操作,我们有
\begin{align*}
  \lim\limits_{x \to +\infty} \frac{f(x)}{F(x)}
  =
  \lim\limits_{x \to 0^{+}} \frac{f(\phi(x))}{F(\phi(x))}
\end{align*}

综上,
\begin{align*}
  \lim\limits_{x \to +\infty} \frac{f(x)}{F(x)}
  =
  \lim\limits_{x \to \infty} \frac{f^\prime(x)}{F^\prime(x)}
\end{align*}

\section*{2.4 定理}
\begin{zremark}
  设
  \begin{itemize}
    \item (1)当$x \to +\infty$时,函数$f(x)$及$F(x)$都趋于无穷大。
    \item (2)存在$r > 0$,使得$x > r, f^\prime(x)$及$F^\prime(x)$都存在且$F^\prime(x) \not = 0$。
    \item (3)$\lim\limits_{x \to \infty} \frac{f^\prime(x)}{F^\prime(x)}$存在(或为无穷大),
          则
          \begin{align*}
            \lim\limits_{x \to +\infty} \frac{f(x)}{F(x)} = \lim\limits_{x \to +\infty} \frac{f^\prime(x)}{F^\prime(x)}
          \end{align*}
  \end{itemize}
\end{zremark}

\textbf{证明:}

定义$\phi(x) = \frac{1}{x}$,于是
\begin{align*}
  \lim\limits_{x \to 0^{+}} \phi(x) = +\infty
\end{align*}

于是
\begin{align*}
  \lim\limits_{x \to 0^{+}} \frac{f^\prime(\phi(x))}{F^\prime(\phi(x))}
  =
  \lim\limits_{x \to \infty} \frac{f^\prime(x)}{F^\prime(x)}
\end{align*}
(注意:使用了下文中的3定理)

利用2.3可知,
\begin{align*}
  \lim\limits_{x \to 0^{+}} \frac{f(\phi(x))}{F(\phi(x))}
  =
  \lim\limits_{x \to 0^{+}} \frac{f^\prime(\phi(x))}{F^\prime(\phi(x))}
\end{align*}

反方向使用刚才的操作,我们有
\begin{align*}
  \lim\limits_{x \to +\infty} \frac{f(x)}{F(x)}
  =
  \lim\limits_{x \to 0^{+}} \frac{f(\phi(x))}{F(\phi(x))}
\end{align*}

综上,
\begin{align*}
  \lim\limits_{x \to +\infty} \frac{f(x)}{F(x)}
  =
  \lim\limits_{x \to \infty} \frac{f^\prime(x)}{F^\prime(x)}
\end{align*}

\section*{3 定理}

\begin{zremark}

  函数$f: \mathbb{R} \to \mathbb{R}$的函数,并且$\lim\limits_{x \to x_0} f(x) = L$,
  函数$\phi: \mathbb{R} \to \mathbb{R}$,且$\lim\limits_{x \to a} \phi(x) = x_0$,
  于是$\lim\limits_{x \to a} f\circ\phi(x) = L$成立。

\end{zremark}

\textbf{证明}:

任意$\epsilon > 0$。

由$\lim\limits_{x \to x_0} f(x) = L$可知,存在$\delta_0 > 0$,使得只要$|x - x_0| < \delta_0$,就有
\begin{align*}
  |f(x) - L| < \epsilon
\end{align*}

又因为$\lim\limits_{x \to a} \phi(x) = x_0$,存在$\delta > 0$,使得只要$|x - a| < \delta$,就有
\begin{align*}
  |\phi(x) - x_0| < \delta_0
\end{align*}

综上,只要$|x - a| < \delta_1$,就有
\begin{align*}
  |f(\phi(x)) - L| < \epsilon
\end{align*}

即:
\begin{align*}
  \lim\limits_{x \to a} f\circ\phi(x) = L
\end{align*}


\end{document}
