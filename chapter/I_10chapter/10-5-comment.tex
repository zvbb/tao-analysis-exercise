\documentclass{article}
\usepackage{mathtools} 
\usepackage{fontspec}
\usepackage[UTF8]{ctex}
\usepackage{amsthm}
\usepackage{mdframed}
\usepackage{xcolor}
\usepackage{amssymb}
\usepackage{amsmath}


% 定义新的带灰色背景的说明环境 zremark
\newmdtheoremenv[
  backgroundcolor=gray!10,
  % 边框与背景一致,边框线会消失
  linecolor=gray!10
]{zremark}{说明}


\begin{document}
\title{10.5 注释}
\author{张志聪}
\maketitle

\begin{zremark}
  \textbf{个人判断,陶哲轩书中的洛必达法则的表述存在问题。}
\end{zremark}

主要有以下问题:
\begin{itemize}
  \item II中$\lim\limits_{x \to a; x \in (a, b]} \frac{f^\prime(x)}{g^\prime(x)}$是一定存在的。

        题设中$f,g$在$[a, b]$上是可微的,且任意$x \in [a,b], g^\prime(x) \neq 0$,
        即$f^\prime(a),g^\prime(a)$存在,且$g^\prime(a) \neq 0$,
        此时,按照定义10.1.1可得,
        \begin{align*}
          \lim\limits_{x \to a; x \in [a, b] \setminus \{a\}} \frac{f(x) - f(a)}{x - a}
          = f^\prime(a) \\
          \lim\limits_{x \to a; x \in [a, b] \setminus \{a\}} \frac{g(x) - g(a)}{x - a}
          = g^\prime(a) \\
        \end{align*}

        集合$[a, b] \setminus \{a\} = (a, b]$,按照命题9.3.14,极限就是,
        \begin{align*}
           & \lim\limits_{x \to a; x \in (a, b]} \frac{f^\prime(x)}{g^\prime(x)}                               \\
           & = \lim\limits_{x \to a; x \in (a, b]} \frac{\frac{f(x) - f(a)}{x - a}}{\frac{g(x) - g(a)}{x - a}} \\
           & = \frac{f^\prime(a)}{g^\prime(a)}
        \end{align*}
        此时为啥还要判断存在性呢???
\end{itemize}


\textcolor{red}{后来,在第1版的《陶哲轩实分析》勘误网页中发现,已经说明了其中问题。}

网页:https://terrytao.wordpress.com/books/analysis-i/

原文如下:
\begin{zremark}
  In Proposition 10.5.2,
  the hypothesis that f,g be differentiable on [a,b] may be weakened
  to being continuous on [a,b] and differentiable on (a,b],
  with g' only assumed to be non-zero on (a,b] rather than [a,b].

  翻译:
  在命题10.5.2中,
  关于f和g在$[a,b]$上可微的假设可以弱化为:在$[a,b]$上连续且在$(a,b]$上可微,
  且$g^\prime$仅需在$(a,b]$上非零(而非原条件要求的$[a,b]$上)。
\end{zremark}


\section*{2 \textbf{和我本科期间学的洛必达定理表达不一样。}}

\begin{itemize}
  \item 本科期间是$\lim\limits_{x \to a} f(x) = 0$,而不是$f(a) = 0$,后者的条件更强了。
\end{itemize}


\section*{2.1 定理1}

\begin{zremark}
  设
  \begin{itemize}
    \item (1) 当$x \to a$时,函数$f(x)$及$F(x)$都趋于零。
    \item (2)在点$a$的某个去心领域内,$f^\prime(x)$及$F^\prime(x)$都存在且$F^\prime(x) \not = 0$。
    \item (3)$\lim\limits_{x \to a} \frac{f^\prime(x)}{F^\prime(x)}$存在(或为无穷大),
          则
          \begin{align*}
            \lim\limits_{x \to a} \frac{f(x)}{F(x)} = \lim\limits_{x \to a} \frac{f^\prime(x)}{F^\prime(x)}
          \end{align*}
  \end{itemize}
\end{zremark}

同济版本的证明,个人觉得是不对的。可以直接利用命题10.5.2进行证明。
只需做如下转换即可:

设函数$h$如下:$x = a$时$h(a) = 0$,$x \not = a$时,$h(x) = f(x)$。

同理设函数$H$如下,$x = a$时$H(a) = 0$,$x \not = a$时,$H(x) = h(x)$。

由前置条件(2),我们存在一个点$a$的去心领域$(a - \epsilon, a + \epsilon)$。
设$(a, \delta] \subseteq (a - \epsilon, a + \epsilon)$。

于是
\begin{align*}
  \lim\limits_{x \to a; x \in (a, \delta]} \frac{f^\prime(x)}{F^\prime(x)}
   & = \lim\limits_{x \to a; x \in (a, \delta]} \frac{h^\prime(x)}{H^\prime(x)}
\end{align*}

利用命题10.5.2,得
\begin{align*}
  \lim\limits_{x \to a; x \in (a, \delta]} \frac{h^\prime(x)}{H^\prime(x)}
   & = \lim\limits_{x \to a; x \in (a, \delta]} \frac{h(x)}{H(x)} \\
   & = \lim\limits_{x \to a; x \in (a, \delta]} \frac{f(x)}{F(x)}
\end{align*}

左极限证明类似。

综上,命题成立。

\end{document}
