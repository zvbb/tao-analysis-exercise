\documentclass{article}
\usepackage{mathtools} 
\usepackage{fontspec}
\usepackage[UTF8]{ctex}
\usepackage{amsthm}
\usepackage{mdframed}
\usepackage{xcolor}
\usepackage{amssymb}
\usepackage{amsmath}


% 定义新的带灰色背景的说明环境 zremark
\newmdtheoremenv[
  backgroundcolor=gray!10,
  % 边框与背景一致,边框线会消失
  linecolor=gray!10
]{zremark}{说明}


\begin{document}
\title{10.5 注释}
\author{张志聪}
\maketitle

\begin{zremark}
  \textbf{综合1,2,个人判断,陶哲轩书中的洛必达法则的表述存在问题。}
\end{zremark}

\section*{1 \textbf{书中洛必达法则I和II有区别,感觉两者表达的不一致?}}
主要有以下问题:
\begin{itemize}
  \item I中直接就是$\frac{f^\prime(x_0)}{g^\prime(x_0)}$,但II中却要
        $\lim\limits_{x \to a; x \in (a, b]} \frac{f^\prime(x)}{g^\prime(x)}$存在。
  \item II中
        $\lim\limits_{x \to a; x \in (a, b]} \frac{f^\prime(x)}{g^\prime(x)}$难道不是一定存在么?

        题设中$f,g$在$[a, b]$上是可微的,且任意$x \in [a,b], g^\prime(x) \neq 0$,
        即$f^\prime(a),g^\prime(a)$存在,且$g^\prime(a) \neq 0$,
        此时,按照定义10.1.1可得,
        \begin{align*}
          \lim\limits_{x \to a; x \in [a, b] \setminus \{a\}} \frac{f(x) - f(a)}{x - a}
          = f^\prime(a) \\
          \lim\limits_{x \to a; x \in [a, b] \setminus \{a\}} \frac{g(x) - g(a)}{x - a}
          = g^\prime(a) \\
        \end{align*}

        集合$[a, b] \setminus \{a\} = (a, b]$,按照命题9.3.14,极限就是,
        \begin{align*}
           & \lim\limits_{x \to a; x \in (a, b]} \frac{f^\prime(x)}{g^\prime(x)}                               \\
           & = \lim\limits_{x \to a; x \in (a, b]} \frac{\frac{f(x) - f(a)}{x - a}}{\frac{g(x) - g(a)}{x - a}} \\
           & = \frac{f^\prime(a)}{g^\prime(a)}
        \end{align*}
        此时为啥还要判断存在性呢???

\end{itemize}


\section*{2 \textbf{和我本科期间学的洛必达定理表达不一样?}}

\begin{itemize}
  \item 本科期间是$\lim\limits_{x \to a} f(x) = 0$,而不是$f(a) = 0$,后者的条件更强了。
  \item 维基百科与百度百科表达的方式与本科期间一致。
\end{itemize}




\end{document}
