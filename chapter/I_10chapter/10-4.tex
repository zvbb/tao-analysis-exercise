\documentclass{article}
\usepackage{mathtools} 
\usepackage{fontspec}
\usepackage[UTF8]{ctex}
\usepackage{amsthm}
\usepackage{mdframed}
\usepackage{xcolor}
\usepackage{amssymb}
\usepackage{amsmath}


% 定义新的带灰色背景的说明环境 zremark
\newmdtheoremenv[
  backgroundcolor=gray!10,
  % 边框与背景一致,边框线会消失
  linecolor=gray!10
]{zremark}{说明}


\begin{document}
\title{10.4 习题}
\author{张志聪}
\maketitle

\section*{10.4.1}

\begin{itemize}
  \item (a)

        $g$的反函数为$g^{-1}: (0, +\infty) \to (0, +\infty),g^{-1}(y) = y^n$,
        因为$g^{-1}(x) = x^n$是既连续又严格单调递增(这里没有做详细证明),由命题9.8.3可知,
        它的反函数$g$也是既连续又严格单调递增的。

  \item (b)

        由题设和(a),并利用定理10.4.3(反函数定理)可知,$g$在$(0, +\infty)$上可微的,并且,
        对任意$x_0 \in (0, +\infty), y_0 = g(x_0) = x_0^{\frac{1}{n}}$
        \begin{align*}
          g^\prime(x_0) & = \frac{1}{g^{-1}(y_0)}                          \\
                        & = \frac{1}{ny_0^{n-1}}                           \\
                        & = \frac{1}{n}\frac{1}{y_0^{n-1}}                 \\
                        & = \frac{1}{n}\frac{1}{(x_0^{\frac{1}{n}})^{n-1}} \\
                        & = \frac{1}{n}\frac{1}{x_0^{1 - \frac{1}{n}}}     \\
                        & = \frac{1}{n}x_0^{\frac{1}{n} - 1}               \\
        \end{align*}
\end{itemize}

\section*{10.4.2}

$q$是有理数,所以可以表示成$q = \frac{a}{b}$其中$a, b$都是整数,且$b$是不等于零的正整数。

于是$f(x) = x^q$可以表示成$f(x) = x^{\frac{a}{b}}$。

\begin{itemize}
  \item (a)

        \begin{align*}
          f(x) & = x^{\frac{a}{b}}     \\
               & = (x^{\frac{1}{b}})^a
        \end{align*}

        \begin{itemize}
          \item[$\circ$]  $p = 0, a = 0$

                此时,$f(x) = x^0 = 1$,$f^\prime(x) = 0$,命题成立。
          \item[$\circ$]  $p > 0, a > 0$

                令$g : (0, +\infty) \to (0, +\infty), g(x) = x^{\frac{1}{b}}$、
                $h : (0, +\infty) \to (0, +\infty), h(x) = x^a$。

                于是可得$f(x) = (h \circ g)(x)$,因为$g, h$在定义域上可微($g$的可微习题10.4.1(b)保证),
                由定理10.1.15(链式法则)可得,$f$在定义域上可微。

                由定理10.1.15(链式法则)和习题10.4.1(b)可得,
                对任意$x_0 \in (0, +\infty), y_0 = g(x_0) = x_0^{\frac{1}{b}}$
                \begin{align*}
                  f^\prime(x_0) & = (h \circ g)^\prime(x_0)                                      \\
                                & = h^\prime(y_0)g^\prime(x_0)                                   \\
                                & = a(y_0)^{a - 1}\frac{1}{b}x_0^{\frac{1}{b} - 1}               \\
                                & = a(x_0^{\frac{1}{b}})^{a - 1}\frac{1}{b}x_0^{\frac{1}{b} - 1} \\
                                & = \frac{a}{b}x_0^{\frac{a-1}{b}} x_0^{\frac{1}{b} - 1}         \\
                                & = \frac{a}{b}x_0^{\frac{a}{b} - 1}                             \\
                                & = q x_0^{q - 1}
                \end{align*}

          \item[$\circ$] $q < 0, a < 0$

                令$p = -q$,于是$p > 0, f(x) = x^p = x^{-p} = \frac{1}{x^p}$。

                由前一个证明和定理10.1.13(g)可知,
                \begin{align*}
                  f^\prime(x_0) & = - \frac{p x_0^{p - 1}}{(x_0^p)^2} \\
                                & = - \frac{p x_0^{p - 1}}{x_0^{2p}}  \\
                                & = - p x_0^{-p - 1}                  \\
                                & = q x_0^{q - 1}
                \end{align*}

        \end{itemize}

  \item (b)

        \begin{zremark}
          习题有点问题$x - 1$是不能等于零的,所以$x \in (0, +\infty) \ \{1\}$
        \end{zremark}

        \begin{align*}
          \lim\limits_{x \to 1; x \in (0, +\infty) \setminus \{1\}} \frac{x^q - 1}{x - 1}
          = \lim\limits_{x \to 1; x \in (0, +\infty) \setminus \{1\}} \frac{x^q - 1^q}{x - 1} \\
        \end{align*}
        可见,这里是函数在$x=1$处的导数,于是$f^\prime(1) = q1^0 = q$
\end{itemize}

\section*{10.4.3}


\begin{itemize}
  \item (a)

        证明框架:通过证明左右极限存在且相等,证明极限的存在。

        证明右极限,我们必须证明
        \begin{align*}
          \lim\limits_{x \to 1; x \in (1, +\infty) } \frac{f(x) - f(1)}{x - 1}
        \end{align*}
        根据命题9.3.9可知,只需证明:对任意一个由$(1, +\infty)$中的元素构成的且收敛于
        $1$的序列$(a_n)_{n=1}^\infty$,对应的以下极限存在
        \begin{align*}
          \lim\limits_{n \to \infty} \frac{f(a_n) - f(1)}{a_n - 1}
        \end{align*}
        即可。

        对任意$\epsilon > 0$,由命题5.4.14可知,存在有理数$q_1, q_2$使得
        \begin{align*}
          \alpha - \frac{1}{2}\epsilon < q_0 < \alpha \\
          \alpha < q_1 < \alpha + \frac{1}{2}\epsilon \\
        \end{align*}
        由命题4.3.3(g)可知$d(q_1, q_2) < \epsilon$。

        由引理5.6.9(e)可知,$a_n > 1$时,$a_n^{q_0} < a_n^\alpha < a_n^{q_1}$,
        于是,
        \begin{align*}
          \frac{a_n^{q_1} - f(1)}{a_n - 1} < \frac{f(a_n) - f(1)}{a_n - 1} < \frac{a_n^{q_1} - f(1)}{a_n - 1}
        \end{align*}
        由命题10.4.2(b)可知,
        \begin{align*}
          \lim\limits_{n \to \infty} \frac{a_n^{q_0} - f(1)}{x - 1} = q_0 \\
          \lim\limits_{n \to \infty} \frac{a_n^{q_1} - f(1)}{x - 1} = q_1 \\
        \end{align*}

        由夹逼定理可得
        \begin{align*}
          q_0 \leq \sup(\frac{f(a_n) - f(1)}{a_n - 1})_{n=1}^\infty  \leq q_1 \\
          q_0 \leq \inf(\frac{f(a_n) - f(1)}{a_n - 1})_{n=1}^\infty  \leq q_1 \\
        \end{align*}
        由$d(q_0, q_1) < \epsilon$且$\epsilon$是任意的,可得,
        \begin{align*}
          \sup(\frac{f(a_n) - f(1)}{a_n - 1})_{n=1}^\infty = \inf(\frac{f(a_n) - f(1)}{a_n - 1})_{n=1}^\infty = \alpha
        \end{align*}
        如果不相等,则会出现以下矛盾,(注意:上确界与下确界是确定,所以两者的差值也是确定的)
        \begin{align*}
          d(q_0, q_1) > \sup(\frac{f(a_n) - f(1)}{a_n - 1})_{n=1}^\infty - \inf(\frac{f(a_n) - f(1)}{a_n - 1})_{n=1}^\infty
        \end{align*}
        由上可得,右极限是存在的,不妨设为$L$。

        如果$L \neq \alpha$,设$L > \alpha, \delta = L - \alpha$,取$\epsilon < \delta $,
        此时$L > \alpha + \epsilon > p_1$,与$L \leq q_1$存在矛盾。

        由此可得右极限存在且等于$\alpha$。

        类似地,左极限也存在且等于$\alpha$。(注意在左极限中$x - 1 < 0, x^{q_0} > x^{p_1}$)

  \item (b)

        由定义10.1.1可知,要证明对任意$x_0 \in (0, +\infty)$,以下极限的存在性,
        \begin{align*}
           & \lim \limits_{x \to x_0; x \in (0, +\infty) \setminus \{x_0\}} \frac{f(x) - f(x_0)}{x - x_0}                                                    \\
           & = \lim \limits_{x \to x_0; x \in (0, +\infty) \setminus \{x_0\}} \frac{x^{\alpha} - x_0^{\alpha}}{x - x_0}                                      \\
           & = \lim \limits_{x \to x_0; x \in (0, +\infty) \setminus \{x_0\}} \frac{(\frac{x}{x_0})^{\alpha} - 1}{\frac{x}{x_0} - 1}\frac{x_0^{\alpha}}{x_0} \\
           & = \lim \limits_{x \to x_0; x \in (0, +\infty) \setminus \{x_0\}} \frac{(\frac{x}{x_0})^{\alpha} - 1}{\frac{x}{x_0} - 1}x_0^{\alpha - 1}         \\
           & = \alpha x_0^{\alpha - 1}
        \end{align*}
        最后一个等式使用了命题9.3.14(函数的极限定律)

\end{itemize}

\begin{zremark}
  \begin{align*}
    \lim \limits_{x \to x_0; x \in (0, +\infty) \setminus \{x_0\}} \frac{(\frac{x}{x_0})^{\alpha} - 1}{\frac{x}{x_0} - 1} = \alpha
  \end{align*}
  这一步可以通过定义9.3.6证明。
\end{zremark}


\end{document}
