\documentclass{article}
\usepackage{mathtools} 
\usepackage{fontspec}
\usepackage[UTF8]{ctex}
\usepackage{amsthm}
\usepackage{mdframed}
\usepackage{xcolor}
\usepackage{amssymb}
\usepackage{amsmath}


% 定义新的带灰色背景的说明环境 zremark
\newmdtheoremenv[
  backgroundcolor=gray!10,
  % 边框与背景一致,边框线会消失
  linecolor=gray!10
]{zremark}{说明}


\begin{document}
\title{10.2 习题}
\author{张志聪}
\maketitle

\section*{10.2.1}

$f$在$x_0$处可微,设其导数是$L$,即极限
\begin{align*}
  \lim\limits_{x \to x_0; x \in X \setminus \{x_0\}} \frac{f(x) - f(x_0)}{x - x_0} = L
\end{align*}

接下来证明$L > 0$和$L < 0$都会导致矛盾,来确定$L$只能等于$0$。

定义序列$(a_n)_{n=1}^\infty$如下,
\begin{equation*}
  a_n
  \begin{cases*}
    = x_0; \text{if } (x_0 + (-1)^n\frac{1}{n}) \notin (a, b)                  \\
    = x_0 + (-1)^n\frac{1}{n}; \text{if } (x_0 + (-1)^n\frac{1}{n}) \in (a, b) \\
  \end{cases*}
\end{equation*}

因为$\lim\limits_{x \to x_0; x \in X \setminus \{x_0\}} \frac{f(x) - f(x_0)}{x - x_0} = L$,
由定义9.3.9(b)可知,对于完全由$X$中元素构成并且收敛于$x_0$的序列$(a_n)_{n=1}^\infty$,序列
$(f(a_n))_{n=1}^\infty$都收敛于$L$。

\begin{itemize}
  \item $L > 0$

        那么,存在正整数$N$使得
        \begin{align*}
           & \left| \frac{f(a_n) - f(x_0)}{a_n - x_0} - L\right| \leq \frac{1}{2}L  \\
           & \implies                                                               \\
           & \frac{1}{2} L \leq \frac{f(a_n) - f(x_0)}{a_n - x_0} \leq \frac{3}{2}L \\
           & \implies                                                               \\
           & \frac{f(a_n) - f(x_0)}{a_n - x_0} > 0
        \end{align*}
        对$n \geq N$均成立。

        当$a_n > x_0$时(由序列的构造方式可知,这样的$a_n$是存在的),$f(x) > f(x_0)$;
        当$a_n < x_0$时(由序列的构造方式可知,这样的$a_n$是存在的),$f(x) < f(x_0)$;
        此时$f(x_0)$既不是局部最大值,也不是局部最小值。

  \item $L < 0$

        同理可得,$L < 0$时,此时$f(x_0)$既不是局部最大值,也不是局部最小值。

\end{itemize}

综上可得$L = 0$,即$f^\prime(x_0) = 0$

\section*{10.2.2}

函数$f: (-1, 1) \to \mathbb{R}$定义为$f(x) = -|x|$。

与命题10.2.6不矛盾的原因是,不满足命题的前置条件:$f$在$0$处可微。

\section*{10.2.3}

\begin{align*}
  f(x)
  \begin{cases*}
    = x; \text{if } x \in (-1, -\frac{1}{2}] \cup [\frac{1}{2}, 1) \\
    = 0; \text{if } x \in (-\frac{1}{2}, \frac{1}{2})              \\
  \end{cases*}
\end{align*}

与命题10.2.6不矛盾的原因是,不满足命题的前置条件:$f$在$0$处达到局部最大值或局部最小值。

\section*{10.2.4}

因为$f$是$[a, b]$上的连续函数,由命题9.6.7(最大值原理)可知,
存在$x_{min}, x_{max} \in [a, b]$分别取到最小值和最大值。

\begin{itemize}
  \item $g(a)$同时是最大值和最小值

        此时任意$x \in [a, b]$都有$g(x) = g(a)$,
        由定理10.1.13(a)可知,任意$x_0 \in [a, b]$都有$f^\prime(x_0) = 0$。

        命题成立。

  \item $g(a)$处是最大值或最小值

        $g(a)$是最小值,因为$g(a) = g(b)$所以$g(b)$也是最小值,
        于是$x_{max} \in (a, b)$,又函数$f$在$(a, b)$上可微,所以$f$在$x_{max}$处是可微的,
        并且$f$在$x_{max}$处是全局最大值,那么也是局部最大值,由命题10.2.6可知,
        \begin{align*}
          f^\prime(x_{max}) = 0
        \end{align*}

        类似的,$g(a)$是最大值,那么,
        \begin{align*}
          f^\prime(x_{min}) = 0
        \end{align*}

  \item $x_{min}, x_{max} \in (a, b)$

        $x_{max} \in (a, b)$,又函数$f$在$(a, b)$上可微,所以$f$在$x_{max}$处是可微的,
        并且$f$在$x_{max}$处是全局最大值,那么也是局部最大值,由命题10.2.6可知,
        \begin{align*}
          f^\prime(x_{max}) = 0
        \end{align*}

        类似的,
        \begin{align*}
          f^\prime(x_{min}) = 0
        \end{align*}

\end{itemize}


\end{document}
