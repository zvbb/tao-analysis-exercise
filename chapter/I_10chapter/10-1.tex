\documentclass{article}
\usepackage{mathtools} 
\usepackage{fontspec}
\usepackage[UTF8]{ctex}
\usepackage{amsthm}
\usepackage{mdframed}
\usepackage{xcolor}
\usepackage{amssymb}
\usepackage{amsmath}


% 定义新的带灰色背景的说明环境 zremark
\newmdtheoremenv[
  backgroundcolor=gray!10,
  % 边框与背景一致,边框线会消失
  linecolor=gray!10
]{zremark}{说明}


\begin{document}
\title{10.1 习题}
\author{张志聪}
\maketitle

\section*{10.1.1}

(1)
$f$在$x_0$处可微分,由定义10.1.1可知,极限
\begin{align*}
  \lim\limits_{x \to x_0; x \in X \setminus \{x_0\}} \frac{f(x) - f(x_0)}{x - x_0}
\end{align*}
是存在的,不妨设极限是$L$。由定义9.3.6可知,
对任意$\epsilon > 0$,存在$\delta > 0$,使得
\begin{align*}
  |\frac{f(x) - f(x_0)}{x - x_0} - L | < \epsilon
\end{align*}
对任意$x \in \big((x_0 - \delta, x_0 + \delta) \cap X \setminus \{x_0\}\big)$
均成立。

任意$y \in \big((x_0 - \delta, x_0 + \delta) \cap Y \setminus \{x_0\}\big)$,
因为$Y \subset X$,所以$y \in \big((x_0 - \delta, x_0 + \delta) \cap X \setminus \{x_0\}\big)$,
所以
\begin{align*}
  |\frac{f(y) - f(x_0)}{y - x_0} - L | < \epsilon
\end{align*}
由定义9.3.6可知,
\begin{align*}
  \lim\limits_{y \to x_0; y \in Y \setminus \{x_0\}} \frac{f|_Y(y) - f|_Y(x_0)}{y - x_0}
\end{align*}
的极限存在,所以$f|_Y$在$x_0$处可微。

(2)
与10.1.2不矛盾的原因:

点$3$不是$[1,2] \cup \{3\}$的极限点,不满足习题10.1.1习题的前置条件。


\end{document}
