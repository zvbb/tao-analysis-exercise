\documentclass{article}
\usepackage{mathtools} 
\usepackage{fontspec}
\usepackage[UTF8]{ctex}
\usepackage{amsthm}
\usepackage{mdframed}
\usepackage{xcolor}
\usepackage{amssymb}
\usepackage{amsmath}


% 定义新的带灰色背景的说明环境 zremark
\newmdtheoremenv[
  backgroundcolor=gray!10,
  % 边框与背景一致,边框线会消失
  linecolor=gray!10
]{zremark}{说明}


\begin{document}
\title{10.1 习题}
\author{张志聪}
\maketitle

\section*{10.1.1}

(1)
$f$在$x_0$处可微分,由定义10.1.1可知,极限
\begin{align*}
  \lim\limits_{x \to x_0; x \in X \setminus \{x_0\}} \frac{f(x) - f(x_0)}{x - x_0}
\end{align*}
是存在的,不妨设极限是$L$。由定义9.3.6可知,
对任意$\epsilon > 0$,存在$\delta > 0$,使得
\begin{align*}
  |\frac{f(x) - f(x_0)}{x - x_0} - L | \leq \epsilon
\end{align*}
对任意$x \in \big((x_0 - \delta, x_0 + \delta) \cap X \setminus \{x_0\}\big)$
均成立。

任意$y \in \big((x_0 - \delta, x_0 + \delta) \cap Y \setminus \{x_0\}\big)$,
因为$Y \subset X$,所以$y \in \big((x_0 - \delta, x_0 + \delta) \cap X \setminus \{x_0\}\big)$,
所以
\begin{align*}
  |\frac{f(y) - f(x_0)}{y - x_0} - L | \leq \epsilon
\end{align*}
由定义9.3.6可知,
\begin{align*}
  \lim\limits_{y \to x_0; y \in Y \setminus \{x_0\}} \frac{f|_Y(y) - f|_Y(x_0)}{y - x_0}
\end{align*}
的极限存在,所以$f|_Y$在$x_0$处可微。

(2)
与10.1.2不矛盾的原因:

点$3$不是$[1,2] \cup \{3\}$的极限点,不满足习题10.1.1习题的前置条件。

\section*{10.1.2}

\begin{itemize}
  \item $(a) \implies (b)$

        $f$在$X$中的$x_0$处是可微的,且导数为$L$,由定义10.1.1可知,极限
        \begin{align*}
          \lim\limits_{x \to x_0; x \in X \setminus \{x_0\}} \frac{f(x) - f(x_0)}{x - x_0} = L
        \end{align*}

        于是,由定义9.3.6可知,
        对任意$\epsilon > 0$,存在$\delta > 0$,使得
        \begin{align*}
          |\frac{f(x) - f(x_0)}{x - x_0} - L | \leq \epsilon
        \end{align*}
        对$|x - x_0| \leq \delta, x \in X \setminus \{x_0\}$均成立。

        对上式进行算术运算,
        \begin{align*}
          |\frac{f(x) - f(x_0)}{x - x_0} - L |       & \leq \epsilon           \\
          |\frac{f(x) - f(x_0) - L(x-x_0)}{x - x_0}| & \leq \epsilon           \\
          |f(x) - f(x_0) - L(x-x_0)|                 & \leq \epsilon |x - x_0| \\
          |f(x) - \big( f(x_0) + L(x-x_0) \big)|     & \leq \epsilon |x - x_0|
        \end{align*}
        对$|x - x_0| \leq \delta, x \in X \setminus \{x_0\}$均成立。
        当$x = x_0$时,公式也成立。

  \item $(b) \implies (a)$

        直接进行算术运算,略
\end{itemize}

\section*{10.1.3}

\begin{itemize}
  \item 方法1(利用极限定律,命题9.3.14)

        不妨设$x_0$处的导数为$L$,于是
        \begin{align*}
          \lim\limits_{x \to x_0; x \in X \setminus \{x_0\}} \frac{f(x) - f(x_0)}{x - x_0} = L
        \end{align*}
        即函数$f$在$x_0$处沿着$X \setminus \{x_0\}$收敛于$L$。

        我们易证
        \begin{align*}
          \lim\limits_{x \to x_0; x \in X \setminus \{x_0\}} x - x_0 = 0
        \end{align*}

        即函数$g$在$x_0$处沿着$X \setminus \{x_0\}$收敛于$0$。

        于是
        \begin{align*}
          fg & = \frac{f(x) - f(x_0)}{x - x_0}  (x - x_0) \\
             & = f(x) - f(x_0)
        \end{align*}

        按照极限定律(命题9.3.14)可得
        \begin{align*}
           & \lim\limits_{x \to x_0; x \in X \setminus \{x_0\}} fg              \\
           & = \lim\limits_{x \to x_0; x \in X \setminus \{x_0\}} f(x) - f(x_0) \\
           & = L \times 0                                                       \\
           & = 0
        \end{align*}

        于是对任意$\epsilon > 0$,都存在$\delta > 0$使得$|f(x) - f(x_0)| < \epsilon$对所有满足
        $|x - x_0| < \delta$的$x \in X \setminus \{x_0\}$均成立。特别地$x = x_0$时$f(x) - f(x_0) = 0 < \delta$也成立。

        所以由命题9.4.7(d)可知,$f$在$x_0$处连续。



  \item 方法2(利用命题10.1.7)

        $f$在$x_0$处可微,由命题10.1.7(b)可知,对任意的$\epsilon > 0$都存在$\delta > 0$,
        当$x \in X$且$|x - x_0| \leq \delta$时,那么就有
        \begin{align*}
          |f(x) - \big( f(x_0) + L(x-x_0) \big)| & \leq \epsilon |x - x_0|
        \end{align*}
        进过算术运算,
        \begin{align*}
          |f(x) - \big( f(x_0) + L(x-x_0) \big)| & \leq \epsilon |x - x_0|                \\
          |f(x) - f(x_0)|                        & \leq \epsilon |x - x_0| + |L||x - x_0| \\
          |f(x) - f(x_0)| \leq (\epsilon + |L|)|x - x_0|
        \end{align*}
        取$\delta = \frac{\epsilon}{\epsilon + |L|}$,此时$|x - x_0| \leq \frac{\epsilon}{\epsilon + |L|} = \delta$,
        使得
        \begin{align*}
          |f(x) - f(x_0)| \leq \epsilon
        \end{align*}
        由命题9.4.7(d)可知,$f$在$x_0$处连续。
\end{itemize}

\section*{10.1.4}

\begin{itemize}
  \item (a)

        对任意$\epsilon > 0$,对任意$\delta > 0$,都存在
        \begin{align*}
           & \frac{f(x) - f(x_0)}{x - x_0} \\
           & = \frac{c - c}{x - x_0}       \\
          %  & = \frac{0}{x - x_0}           \\
           & = 0
        \end{align*}
        对所有满足$|x - x_0| < \delta$的$x \in X \setminus \{x_0\}$均成立。
        于是由命题9.4.7(c)可得
        \begin{align*}
           & \lim\limits_{x \to x_0; x \in X \setminus \{x_0\}}  \frac{f(x) - f(x_0)}{x - x_0} = 0
        \end{align*}
        即$f^\prime(x_0) = 0$

  \item (b)

        对任意$\epsilon > 0$,对任意$\delta > 0$,都存在
        \begin{align*}
           & \frac{f(x) - f(x_0)}{x - x_0} \\
           & = \frac{x - x_0}{x - x_0}     \\
           & = 1
        \end{align*}
        对所有满足$|x - x_0| < \delta$的$x \in X \setminus \{x_0\}$均成立。
        于是由命题9.4.7(c)可得
        \begin{align*}
           & \lim\limits_{x \to x_0; x \in X \setminus \{x_0\}}  \frac{f(x) - f(x_0)}{x - x_0} = 1
        \end{align*}
        即$f^\prime(x_0) = 1$

  \item (c)

        设$f^\prime(x_0) = L, g^\prime(x_0) = M$

        $f$在$x_0$处可微,由定义10.1.1可知,极限
        \begin{align*}
          \lim\limits_{x \to x_0; x \in X \setminus \{x_0\}} \frac{f(x) - f(x_0)}{x - x_0} = L
        \end{align*}
        由定义9.3.6可知,
        对任意$\epsilon > 0, \frac{1}{2}\epsilon > 0$,存在$\delta_f > 0$,使得
        \begin{align*}
          |\frac{f(x) - f(x_0)}{x - x_0} - L | \leq \frac{1}{2}\epsilon
        \end{align*}
        对所有满足$|x - x_0| < \delta_f$的$x \in X \setminus \{x_0\}$均成立。

        同理可得,
        存在$\delta_g > 0$,使得
        \begin{align*}
          |\frac{g(x) - g(x_0)}{x - x_0} - M | \leq \frac{1}{2}\epsilon
        \end{align*}
        对所有满足$|x - x_0| < \delta_g$的$x \in X \setminus \{x_0\}$均成立。

        令$\delta = min(\delta_f, \delta_g)$,于是,
        \begin{align*}
           & \big|\frac{(f(x) + g(x)) - (f(x_0) - g(x_0))}{x - x_0} - (L + M)  \big|                          \\
           & = \big|\frac{f(x) - f(x_0)}{x - x_0} - L + \frac{g(x) - g(x_0)}{x - x_0} - M \big|               \\
           & \leq \big|\frac{f(x) - f(x_0)}{x - x_0} - L \big| + \big|\frac{g(x) - g(x_0)}{x - x_0} - M \big| \\
           & = \frac{1}{2}\epsilon + \frac{1}{2}\epsilon                                                      \\
           & = \epsilon
        \end{align*}
        对所有满足$|x - x_0| < \delta$的$x \in X \setminus \{x_0\}$均成立。
        所以$(f + g)^\prime(x_0) = L + M$。

        于是$(f + g)^\prime(x_0) = f^\prime(x_0) + g^\prime(x_0) = L + M$。

  \item (d)

        设$f^\prime(x_0) = L, g^\prime(x_0) = M$

        $f$在$x_0$处可微,由定义10.1.1可知,极限
        \begin{align*}
          \lim\limits_{x \to x_0; x \in X \setminus \{x_0\}} \frac{f(x) - f(x_0)}{x - x_0} = L
        \end{align*}

        同理可得,
        \begin{align*}
          \lim\limits_{x \to x_0; x \in X \setminus \{x_0\}} \frac{g(x) - g(x_0)}{x - x_0} = M
        \end{align*}

        \begin{align*}
           & \lim\limits_{x \to x_0; x \in X \setminus \{x_0\}} \frac{f(x)g(x) - f(x_0)g(x_0)}{x - x_0} \\
           & = \lim\limits_{x \to x_0; x \in X \setminus \{x_0\}} \frac{f(x)\big(g(x) - g(x_0)\big)
          + \big(f(x) - f(x_0)\big)g(x_0)}{x - x_0}                                                     \\
           & = \lim\limits_{x \to x_0; x \in X \setminus \{x_0\}} f(x) \frac{g(x) - g(x_0)}{x - x_0}
          + g(x_0) \frac{f(x) - f(x_0)}{x - x_0}                                                        \\
        \end{align*}
        又$f$在$x_0$处可微,由命题10.1.0可知$f$在$x_0$处连续
        \begin{align*}
           & \lim\limits_{x \to x_0; x \in X} f(x) = f(x_0) \\
        \end{align*}
        于是通过命题9.4.7(c)可知($x = x_0$是特例),
        \begin{align*}
           & \lim\limits_{x \to x_0; x \in X \setminus \{x_0\}} f(x) = f(x_0)
        \end{align*}
        于是利用极限定律(命题9.3.14)可得
        \begin{align*}
           & \lim\limits_{x \to x_0; x \in X \setminus \{x_0\}} f(x) \frac{g(x) - g(x_0)}{x - x_0}
          + g(x_0) \frac{f(x) - f(x_0)}{x - x_0}                                                   \\
           & = f(x_0)M + g(x_0)L
        \end{align*}
        所以$(fg)^\prime(x_0) = f^\prime(x_0)g(x_0) + f(x_0)g^\prime(x_0)$。

  \item (e)

        设函数$h : X \to \mathbb{R}$为$h(x) = c$,于是$(cf)(x) = h(x)f(x)$,函数相等一定有相同的导数(注10.1.4),
        于是利用(d)可得
        \begin{align*}
           & (cf)^\prime(x_0)                            \\
           & = (hf)^\prime(x_0)                          \\
           & = h^\prime(x_0)f(x_0) + h(x_0)f^\prime(x_0) \\
           & = 0 \times f(x_0) + c \times f^\prime(x_0)  \\
           & = cf^\prime(x_0)
        \end{align*}

  \item (f)

        设函数$h : X \to \mathbb{R}$为$h(x) = -g(x)$,于是$(f-h)(x) = (f+g)(x)$,函数相等一定有相同的导数(注10.1.4),
        于是利用(d)可得

        \begin{align*}
           & (f-g)^\prime(x_0)               \\
           & = (f+h)^\prime(x_0)             \\
           & = f^\prime(x_0) + h^\prime(x_0)
        \end{align*}

        由(e)可知,$h^\prime(x_0) = (-g)^\prime(x_0) = -g^\prime(x_0)$,把(e)中的$c$看做$-1$,

        综上可得
        \begin{align*}
           & (f-g)^\prime(x_0)               \\
           & = f^\prime(x_0) - g^\prime(x_0)
        \end{align*}


  \item (g)

        \begin{align*}
           & \lim\limits_{x \to x_0; x \in X \setminus \{x_0\}} \frac{\frac{1}{g(x)} - \frac{1}{g(x_0)}}{x - x_0}               \\
           & = \lim\limits_{x \to x_0; x \in X \setminus \{x_0\}} \frac{\frac{g(x_0) - g(x)}{g(x)g(x_0)}}{x - x_0}              \\
           & = \lim\limits_{x \to x_0; x \in X \setminus \{x_0\}} \frac{1}{g(x)g(x_0)} \frac{g(x_0) - g(x)}{x - x_0}            \\
           & = \lim\limits_{x \to x_0; x \in X \setminus \{x_0\}} \frac{1}{g(x)} \frac{1}{g(x_0)} \frac{g(x_0) - g(x)}{x - x_0} \\
        \end{align*}

        由于$g$在$x_0$处可微,所以
        \begin{align*}
           & \lim\limits_{x \to x_0; x \in X \setminus \{x_0\}} \frac{g(x_0) - g(x)}{x - x_0} \\
           & = f^\prime(x_0)
        \end{align*}

        由于$g$在$x_0$处可微,由命题10.1.0可知$g$在$x_0$处连续,且$g(x)$在$X$上不为零,
        由命题9.3.14(函数的极限定理)可得
        \begin{align*}
          \lim\limits_{x \to x_0; x \in X \setminus \{x_0\}} \frac{1}{g(x)} = \frac{1}{g(x_0)}
        \end{align*}


        再次利用命题9.3.14(函数的极限定理)可得
        \begin{align*}
           & \lim\limits_{x \to x_0; x \in X \setminus \{x_0\}} \frac{1}{g(x)} \frac{1}{g(x_0)} \frac{g(x_0) - g(x)}{x - x_0} \\
           & = \frac{1}{g(x_0)} \frac{1}{g(x_0)} \big( - g^\prime(x_0)\big)                                                   \\
           & = -\frac{g^\prime(x_0)}{g(x_0)^{2}}
        \end{align*}

  \item (h)

        因为$\big(\frac{f}{g}\big)(x) = f(x)\frac{1}{g(x)}$,于是利用(d)(g)可知,
        \begin{align*}
          \big(\frac{f}{g}\big)^\prime(x_0) & = \big(f\frac{1}{g}\big)^\prime(x_0)                                           \\
                                            & = f^\prime(x_0)\frac{1}{g(x_0)} + f(x_0)\big(\frac{1}{g}\big)^\prime(x_0)            \\
                                            & = f^\prime(x_0)\frac{1}{g(x_0)} + f(x_0) \big(-\frac{g^\prime(x_0)}{g(x_0)^{2}}\big) \\
                                            & = \frac{f^\prime(x_0)}{g(x_0)} - \frac{f(x_0) g^\prime(x_0)}{g(x_0)^{2}}             \\
                                            & = \frac{f^\prime(x_0) g(x_0) - f(x_0) g^\prime(x_0)}{g(x_0)^{2}}
        \end{align*}


\end{itemize}



\end{document}
