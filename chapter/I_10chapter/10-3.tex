\documentclass{article}
\usepackage{mathtools} 
\usepackage{fontspec}
\usepackage[UTF8]{ctex}
\usepackage{amsthm}
\usepackage{mdframed}
\usepackage{xcolor}
\usepackage{amssymb}
\usepackage{amsmath}


% 定义新的带灰色背景的说明环境 zremark
\newmdtheoremenv[
  backgroundcolor=gray!10,
  % 边框与背景一致,边框线会消失
  linecolor=gray!10
]{zremark}{说明}


\begin{document}
\title{10.3 习题}
\author{张志聪}
\maketitle

\section*{10.3.1}


反证法,假设$f^\prime(x_0) < 0$。

$f$在$x_0$处可微,那么,以下极限存在
\begin{align*}
  \lim\limits_{x \to x_0; x \in X \setminus \{x_0\}} \frac{f(x) - f(x_0)}{x - x_0} = f^\prime(x_0)
\end{align*}
由定义9.3.6可知,对$\epsilon = -\frac{1}{2}f^\prime(x_0) > 0$,存在$\delta > 0$使得
\begin{align*}
   & \left|\frac{f(x) - f(x_0)}{x - x_0} - f^\prime(x_0) \right| \leq -\frac{1}{2}f^\prime(x_0) \\
   & \implies                                                                                   \\
   & \frac{3}{2}f^\prime(x_0) \leq \frac{f(x) - f(x_0)}{x - x_0} \leq \frac{1}{2}f^\prime(x_0)  \\
   & \implies                                                                                   \\
   & \frac{f(x) - f(x_0)}{x - x_0} < 0
\end{align*}
对所有满足$|x - x_0| < \delta$的$x \in X$均成立。

这显然是矛盾的,因为,
当$x > x_0$时,由$\frac{f(x) - f(x_0)}{x - x_0} < 0$可得,$f(x) < f(x_0)$;
当$x < x_0$是,由$\frac{f(x) - f(x_0)}{x - x_0} < 0$可得,$f(x) > f(x_0)$;
这与$f$是单调递增的矛盾。

\section*{10.3.2}

\begin{equation*}
  f(x)=
  \begin{cases*}
    x; x \in (-1, 0) \\
    x^2; x \in [0, 1)
  \end{cases*}
\end{equation*}

与命题10.3.1不矛盾的原因是,不满足命题的前置条件:$f$在$0$处是可微的。

\section*{10.3.3}

\begin{equation*}
  f(x) = x^3; x \in (-1, 1)
\end{equation*}
$f$在$0$处可微的,且$f^\prime(0) = 0$,而$f$是严格单调递增的。

\section*{10.3.4}

只证明$f$严格单调递增的情况,严格单调递减证明类似。

反证法,假设$f$不是严格单调递增的,那么,存在$x, y \in [a, b], x < y$且$f(x) \geq f(y)$,
由推论10.2.9(中值定理)可知,存在$c \in (x, y)$使得
\begin{align*}
  f^\prime(c) = \frac{f(y) - f(x)}{y - x} \leq 0
\end{align*}
与题设任意$x \in [a, b]$均有$f^\prime(x) > 0$矛盾。

\section*{10.3.5}

\begin{equation*}
  f(x)=
  \begin{cases*}
    x + 1; x \in (-1, -\frac{1}{2}) \\
    x - 1; x \in (\frac{1}{2}, 1)
  \end{cases*}
\end{equation*}

区别在于:定义域不是连续的

\end{document}
