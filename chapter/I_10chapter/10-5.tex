\documentclass{article}
\usepackage{mathtools} 
\usepackage{fontspec}
\usepackage[UTF8]{ctex}
\usepackage{amsthm}
\usepackage{mdframed}
\usepackage{xcolor}
\usepackage{amssymb}
\usepackage{amsmath}


% 定义新的带灰色背景的说明环境 zremark
\newmdtheoremenv[
  backgroundcolor=gray!10,
  % 边框与背景一致,边框线会消失
  linecolor=gray!10
]{zremark}{说明}


\begin{document}
\title{10.5 习题}
\author{张志聪}
\maketitle

\section*{10.5.1}

首先证明存在$\delta > 0$使得对所有的$x \in \left( X \cap (x_0 - \delta, x_0 + \delta) \right) \setminus \{x_0\}$
都有$g(x) \neq 0$。
由命题10.1.7(牛顿逼近法)(b)可得,特别地,取$\epsilon = \frac{1}{2}|g^\prime(x_0)| > 0$,存在$\delta > 0$使得,
\begin{align*}
  |g(x) - (g(x_0) + g^\prime(x_0)(x - x_0))| \leq \frac{1}{2}|g^\prime(x_0)||x - x_0| \\
  |g(x) - g^\prime(x_0)(x - x_0)| \leq \frac{1}{2}|g^\prime(x_0)||x - x_0|
\end{align*}
此时,如果存在$x \in (X \cap (x_0 - \delta, x_0 + \delta))$,使得$g(x) = 0$,那么,
\begin{align*}
  |g(x) - g^\prime(x_0)(x - x_0)| \leq \frac{1}{2}|g^\prime(x_0)||x - x_0| \\
  |0 - g^\prime(x_0)(x - x_0)| \leq \frac{1}{2}|g^\prime(x_0)||x - x_0|    \\
  |g^\prime(x_0)(x - x_0)| \leq \frac{1}{2}|g^\prime(x_0)||x - x_0|
\end{align*}
显然以上等式是不成立的,所以$g(x) \neq 0$。

由于$f, g$都在$x_0$处可微,所以
\begin{align*}
  \lim\limits_{x \to x_0; x \in (X \cap (x_0 - \delta, x_0 + \delta))} \frac{f(x) - f(x_0)}{x - x_0} = f^\prime(x_0) \\
  \lim\limits_{x \to x_0; x \in (X \cap (x_0 - \delta, x_0 + \delta))} \frac{g(x) - g(x_0)}{x - x_0} = g^\prime(x_0)
\end{align*}

因为
\begin{align*}
   & \lim\limits_{x \to x_0; x \in (X \cap (x_0 - \delta, x_0 + \delta))} \frac{f(x)}{g(x)}                                                       \\
   & =   \lim\limits_{x \to x_0; x \in (X \cap (x_0 - \delta, x_0 + \delta))} \frac{\frac{f(x) - f(x_0)}{x - x_0}}{\frac{g(x) - g(x_0)}{x - x_0}} \\
\end{align*}
于是利用命题9.3.14(函数的极限定律)可知,
\begin{align*}
   & \lim\limits_{x \to x_0; x \in (X \cap (x_0 - \delta, x_0 + \delta))} \frac{f(x)}{g(x)} \\
   & = \frac{f^\prime(x_0)}{g^\prime(x_0)}
\end{align*}

\section*{10.5.2}

(1)不满足命题的前置条件:$g(0) = 0$

(2)不满足命题的前置条件:$\lim\limits_{x \to a; x \in (a, b]} \frac{f^\prime(x)}{g^\prime(x)}$存在。

\end{document}
