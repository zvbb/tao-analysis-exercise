\documentclass{article}
\usepackage{mathtools} 
\usepackage{fontspec}
\usepackage[UTF8]{ctex}
\usepackage{amsthm}
\usepackage{mdframed}
\usepackage{xcolor}
\usepackage{amssymb}
\usepackage{amsmath}


% 定义新的带灰色背景的说明环境 zremark
\newmdtheoremenv[
  backgroundcolor=gray!10,
  % 边框与背景一致,边框线会消失
  linecolor=gray!10
]{zremark}{说明}


\begin{document}
\title{12.1 习题}
\author{张志聪}
\maketitle

\section*{12.1.1}

令$a_n = d(x_n, x) = |x_n - x|$,于是$(a_n)_{n=m}^\infty$是一个实数序列。\\
$(x_n)_{n = m}^\infty$收敛与$x$ \\
$\Leftrightarrow$ \\
对任意$\epsilon > 0$,存在正整数$N$,当$n \geq N$时,$|x_n - x| \leq \epsilon$ \\
$\Leftrightarrow$ \\
对任意$\epsilon > 0$,存在正整数$N$,当$n \geq N$时,$a_n \leq \epsilon$ \\
$\Leftrightarrow$ \\
$\lim\limits_{n \to \infty} a_n = \lim\limits_{n \to \infty} d(x_n, x) = 0$

\section*{12.1.2}

需要证明这里的度量$d(x, y) := |x - y|$满足定义12.1.2。

命题4.3.3(a),满足了定义12.1.2中的公理(a),(b)(因为任意$x, y \in \mathbb{R}, x - y$都是实数);

命题4.3.3(f),满足了定义12.1.2中的公理(c);

命题4.3.3(g),满足定义12.1.2中的公理(d)。

\section*{12.1.3}

\begin{itemize}
      \item (a)

            把离散度量$d_{disc}$做一点修改:当$x = y$时,$d_{disc}(x, y) := 2$。

      \item (b)

            把离散度量$d_{disc}$做一点修改:当$x \neq y$时,$d_{disc}(x, y) := 0$。

      \item (c)

            定义$X := \mathbb{R} - \{0\}$ 并且度量$d$如下:
            \begin{align*}
                  d(x, x) := 0                     \\
                  d(x, y) := |x| \; if \; x \neq y \\
            \end{align*}

      \item (d)

            定义$X := \{a, b, c\}$ 并且度量$d$如下:
            \begin{align*}
                  d(x, x) := 0 \; if \; x \in X               \\
                  d(a, b) = d(b, a) = 1, d(b,c) = d(c, b) = 1 \\
                  d(a, c) = d(c, a) = 3                       \\
            \end{align*}

\end{itemize}

\section*{12.1.4}

任意$x, y \in Y; x, y \in X$,所以$d|_{Y \times Y}(x, y) = d(x, y)$,
因为$d$满足所有的公理,那么$d|_{Y \times Y}$也满足所有的公理。

\section*{12.1.5}

略,敲起来太费劲了。
% (1)恒等式的证明

% 对$n$进行归纳。

% $n = 1$时,
% \begin{align*}
%    & \left(\sum\limits_{i = 1}^n a_ib_i\right)^2 + \frac{1}{2}\sum\limits_{i=1}^n\sum\limits_{j=1}^n(a_ib_j - a_jb_i)^2 \\
%    & = (a_1b_1)^2 + \frac{1}{2} (a_1b_1 - a_1b_1)                                                                       \\
%    & = (a_1b_1)^2                                                                                                       \\
%    & = a_1^2b_1^2
% \end{align*}
% 又
% \begin{align*}
%    & \left(\sum\limits_{i = 1}^n a_i^2\right) \left(\sum\limits_{j = 1}^n b_j^2\right) \\
%    & = a_1^2b_1^2
% \end{align*}
% 此时,等式成立。

% 归纳假设$n = k$时,等式成立。

% $n = k+1$时,
% \begin{align*}
%    & \left(\sum\limits_{i = 1}^{k+1} a_ib_i\right)^2 + \frac{1}{2}\sum\limits_{i=1}^{k+1}\sum\limits_{j=1}^{k+1}(a_ib_j - a_jb_i)^2 \\
%    & = \left(\sum\limits_{i = 1}^{k} a_ib_i + a_{k+1}b_{k+1}\right)^2 
%    + \frac{1}{2}\left(\sum\limits_{i=1}^{k}\sum\limits_{j=1}^{k}(a_ib_j - a_jb_i)^2 + \sum\limits_{i=1}^{k}\sum\limits_{j=1}^{k}(a_ib_j - a_jb_i)^2\right) \\
% \end{align*}

\section*{12.1.6}

\begin{itemize}
      \item 公理(a)

            当$x = y$时,对任意第$i$个坐标分量都有$(x_i - y_i)^2 = 0$, 所以
            \begin{align*}
                  d_{l^2}(x, y) & = \left(\sum\limits_{i=1}^n (x_i - y_i)^2\right)^{1/2} \\
                                & = \left(\sum\limits_{i=1}^n 0 \right)^{1/2}            \\
                                & = 0
            \end{align*}
      \item 公理(b)

            当$x \neq y$时,存在第$i$个坐标分量使得$(x_i - y_i)^2 > 0$,所以$d_{l^2}(x, y) > 0$。

      \item 公理(c)

            任意的$x, y \in X$,它们的任意第$i$个坐标分量都有$(x_i - y_i)^2 = (y_i - x_i)^2$,所以$d_{l^2}(x, y) = d_{l^2}(y, x)$。

      \item 公理(d)

            定义$x_i - y_i = a_i, y_i - z_i = b_i$于是
            \begin{align*}
                  d_{l^2}(x, z) & = \left(\sum\limits_{i=1}^n (x_i - z_i)^2\right)^{1/2}                                                        \\
                                & = \left(\sum\limits_{i=1}^n (a_i + b_i)^2\right)^{1/2}                                                        \\
                                & \leq \left(\sum\limits_{i=1}^n a_i^2\right)^{1/2} + \left(\sum\limits_{i=1}^n b_i^2\right)^{1/2}              \\
                                & = \left(\sum\limits_{i=1}^n (x_i - y_i)^2\right)^{1/2} + \left(\sum\limits_{i=1}^n (y_i - z_i)^2\right)^{1/2} \\
                                & = d_{l^2}(x, y) + d_{l^2}(y, z)
            \end{align*}
            综上可得
            \begin{align*}
                  d_{l^2}(x, z) \leq d_{l^2}(x, y) + d_{l^2}(y, z)
            \end{align*}

\end{itemize}

\section*{12.1.7}

\begin{itemize}
      \item 公理(a)

            当$x = y$时,对任意第$i$个坐标分量都有$|x_i - y_i| = 0$,所以
            \begin{align*}
                  \sum \limits_{i = 1}^n |x_i - y_i|^2 = 0
            \end{align*}

      \item 公理(b)

            当$x \neq y$时,存在第$i$个坐标分量使得$|x_i - y_i| > 0$,
            其他坐标分量$j$都有$|x_j - y_j| \geq 0$,
            所以$d_{l^1}(x, y) > 0$。

      \item 公理(c)

            对任意第$i$个坐标分量都有$|x_i - y_i| = |y_i - x_i|$,所以
            \begin{align*}
                  d_{l^1}(x, y) = d_{l^1}(y, x)
            \end{align*}

      \item 公理(d)

            对任意第$i$个坐标分量,由命题4.3.3(g)可知,
            \begin{align*}
                  |x_i - z_i| \leq |x_i - y_i| + |y_i - z_i|
            \end{align*}
            所以,
            \begin{align*}
                  d_{l^1}(x, z) \leq d_{l^1}(x, y) + d_{l^1}(y, z)
            \end{align*}

\end{itemize}

\section*{12.1.8}
(1) 第一个不等式
\begin{align*}
      d_{l^2}(x, y) & = \left(\sum \limits_{i = 1}^n (x_i - y_i)^2\right)^{1/2} \\
      d_{l^1}(x, y) & = |x_1 - y_1| + ... + |x_n - y_n|
\end{align*}
两边同时取平方
\begin{align*}
      (d_{l^2}(x, y))^2 & = \sum \limits_{i = 1}^n (x_i - y_i)^2                                                      \\
      (d_{l^1}(x, y))^2 & = \left(|x_1 - y_1| + ... + |x_n - y_n|\right) \left(|x_1 - y_1| + ... + |x_n - y_n|\right) \\
                        & = \sum \limits_{i = 1}^n (x_i - y_i)^2
      + \sum \limits_{i=1}^n \sum \limits_{1 \leq j \leq n, i \neq j} |x_i - y_i||x_j - y_j|
\end{align*}
由上式可得
\begin{align*}
       & (d_{l^2}(x, y))^2 \leq (d_{l^1}(x, y))^2 \\
       & \implies                                 \\
       & d_{l^2}(x, y) \leq d_{l^1}(x, y)
\end{align*}

(2)第二个不等式

\begin{align*}
      d_{l^1}(x, y) = \sum \limits_{i = 1}^n |x_i - y_i|
\end{align*}
利用柯西-施瓦茨不等式可得
\begin{align*}
      d_{l^1}(x, y) = \sum \limits_{i = 1}^n |x_i - y_i| & = \left|\sum \limits_{i = 1}^n |x_i - y_i|\right|           \\
                                                         & \leq \left(\sum\limits_{i = 1}^n (x_i - y_i)^2\right)^{1/2}
      \left(\sum \limits_{i = 1}^n 1\right)^{1/2}                                                                      \\
                                                         & = \sqrt{n}d_{l^2}(x, y)
\end{align*}

\section*{12.1.9}

\begin{itemize}
      \item 公理(a)

            当$x = y$时,对任意第$i$个坐标分量都有$|x_i - y_i| = 0$,所以
            \begin{align*}
                  \sup\{|x_i - y_i|: 1 \leq i \leq n\} = 0
            \end{align*}

      \item 公理(b)

            当$x \neq y$时,存在第$i$个坐标分量使得$|x_i - y_i| > 0$,所以
            \begin{align*}
                  \sup\{|x_i - y_i|: 1 \leq i \leq n\} > 0
            \end{align*}

      \item 公理(c)

            对任意第$i$个坐标分量都有$|x_i - y_i| = |y_i - x_i|$,所以
            \begin{align*}
                  d_{l^\infty}(x, y) = d_{l^\infty}(y, x)
            \end{align*}

      \item 公理(d)

            对任意第$i$个坐标分量,由命题4.3.3(g)可知,
            \begin{align*}
                  |x_i - z_i| \leq |x_i - y_i| + |y_i - z_i|
            \end{align*}
            所以,
            \begin{align*}
                  d_{l^\infty}(x, z) \leq d_{l^\infty}(x, y) + d_{l^\infty}(y, z)
            \end{align*}
\end{itemize}

\section*{12.1.10}

设$\sup\{|x_i - y_i|: 1 \leq i \leq n\} = r$。

(1)第一个不等式
\begin{align*}
      \frac{1}{\sqrt{n}}d_{l^2}(x, y) & = \frac{1}{\sqrt{n}} \left(\sum \limits_{i = 1}^n (x_i - y_i)^2\right)^{1/2} \\
                                      & \leq \frac{1}{\sqrt{n}} \left(\sum \limits_{i = 1}^n r^2\right)^{1/2}        \\
                                      & = r                                                                          \\
                                      & = d_{l^\infty}(x, y)
\end{align*}

(2)第二个不等式

存在一个$i$使得$|x_i - y_i| = r$。反证法,对任意$i$都有
\begin{align*}
      |x_i - y_i| < r
\end{align*}
有定义可知,度量都是有限维的,即$n$不会是$+\infty$,
取$m = max|x_i - y_i|, 1 \leq i \leq n$(引理5.1.14),此时$m < r$且$m$是上界,
这与$r$是最小上界矛盾(定义5.5.5)。

所以
\begin{align*}
      d_{l^2}(x, y) = \left(\sum \limits_{i = 1}^n (x_i - y_i)^2\right)^{1/2} \geq (r^2)^{1/2} = r = d_{l^\infty}(x, y)
\end{align*}



\section*{12.1.11}
略

\section*{12.1.12}

为了利用式(12.1)、(12.2),我们的证明步骤如下(需要保证是环状的)
\begin{align*}
      (d) \implies (c) \implies (b) \implies (a) \\
      (a) \implies (d)
\end{align*}

\begin{itemize}
      \item $(b) \implies (a)$

            (b)成立,即$\lim\limits_{k \to \infty}d_{l^1}(x^{(k)}, x) = 0$,
            即对任意$\epsilon > 0$,存在一个$N \geq m$使得$d_{l^1}(x^{(k)}, x) \leq \epsilon$对所有的$k \geq N$均成立。

            由式12.1可知
            \begin{align*}
                  d_{l^2}(x^{(k)}, x) \leq d_{l^1}(x^{(k)}, x) \leq \epsilon
            \end{align*}
            对所有的$k \geq N$均成立。于是$(a)$成立。

      \item $(c) \implies (b)$

            任意$\epsilon > 0$于是$\epsilon^\prime = \frac{1}{\sqrt{n}}\epsilon > 0$(注意这里的$n$是固定值),因为(c)成立,
            所以存在一个$N \geq m$使得$d_{l^\infty}(x^{(k)}, x) \leq \epsilon^\prime$对所有的$k \geq N$均成立。

            由式12.2可知
            \begin{align*}
                   & \frac{1}{\sqrt{n}}d_{l^2}(x^{(k)}, x) \leq d_{l^\infty}(x^{(k)}, x) \leq \epsilon^\prime = \frac{1}{\sqrt{n}}\epsilon \\
                   & \implies                                                                                                              \\
                   & d_{l^2}(x^{(k)}, x) \leq \epsilon
            \end{align*}

      \item $(d) \implies (c)$

            (d)成立,那么对任意$1 \leq i \leq n, \epsilon > 0$存在一个$N \geq m$使得
            $|x_i^{(k)} - x_i| \leq \epsilon$对所有的$k \geq N$均成立。

            由$l^{\infty}$度量的定义可知
            $d_{l^\infty}(x^{(k)}, x) = \sup\{|x_i - y_i|: 1 \leq i \leq n\} \leq \epsilon$对所有的$k \geq N$均成立。
            于是(c)成立。

      \item $(a) \implies (d)$

            (a)成立,那么对任意$\epsilon > 0 > 0$存在一个$N \geq m$使得
            $d_{l^2}(x^{(k)}, x) \leq \epsilon$对所有的$k \geq N$均成立。
            即(注意:以下的$x_i^{(k)}$要看做实数,不能看做“点”)
            \begin{align*}
                  \left( \sum \limits_{i = 1}^n (x_i^{(k)} - x_i)^2 \right)^{1/2} \leq \epsilon \\
                  \sum \limits_{i = 1}^n (x_i^{(k)} - x_i)^2 \leq \epsilon^2                    \\
            \end{align*}
            那么,对每一个$1 \leq i \leq n$都有
            \begin{align*}
                  (x_i^{(k)} - x_i)^2 & \leq \epsilon^2 \\
                  \implies                              \\
                  |x_i^{(k)} - x_i|   & \leq \epsilon
            \end{align*}
            对所有的$k \geq N$均成立。

            所以(d)成立。

\end{itemize}

\section*{12.1.13}

反证法,假设$d_{disc}$收敛于$x$,但$N$不存在。那么,任意$N \geq m$都至少存在一个$n \geq N$使得
\begin{align*}
      x^{(n)} \neq x
\end{align*}
由离散度量$d_{disc}$的定义可知,此时$d_{disc}(x, x^{(n)}) = 1$。由于$N$的任意性可知,
定义12.1.14(度量空间中序列的收敛)中的定义无法满足,与假设$d_{disc}$收敛于$x$矛盾。

\section*{12.1.14}

为了推出矛盾,我们假设$(x^{(n)})_{n = m}^\infty$依度量$d$同时收敛于$x, x^\prime$,
设$\epsilon = d(x, x^\prime)/3$。注意,因为$x \neq x^\prime$,所以$\epsilon$是正的。
由$(x^{(n)})_{n = m}^\infty$依度量$d$收敛于$x$可知,存在一个$N \geq m$使得$d(x^{(n)}, x) \leq \epsilon$
对所有的$n \geq N$均成立。类似地,存在一个$M \geq m$使得$d(x^{(n)}, x^\prime) \leq \epsilon$对所有的$n \geq M$均成立。
特别地,如果令$n := max(N, M)$,那么有$d(x^{(n)}, x) \leq \epsilon$和$d(x^{(n)}, x^\prime) \leq \epsilon$,
于是根据三角不等式(定义12.1.2公理(d))可得,
$d(x, x^\prime) \leq 2\epsilon = 2d(x, x^\prime)/3$。所以$d(x, x^\prime) \leq 2d(x, x^\prime)/3$,这和$d(x, x^\prime) > 0$矛盾。
所以度量空间中的序列不能同时收敛于两个不同的点。

\section*{12.1.15}

(1)$d_{l^1}$是$X$上的度量。
\begin{itemize}
      \item 公理(a)

            如果序列$(a_n)_{n=0}^\infty$与$(b_n)_{n=0}^\infty$是相等的,那么对任意$n \in \mathbb{N}$都有
            $a_n = b_n$,所有
            \begin{align*}
                  d_{l^1}((a_n)_{n = 0}^\infty, (b_n)_{n = 0}^\infty) = \sum \limits_{n = 0}^\infty |a_n - b_n| = 0
            \end{align*}

      \item 公理(b)

            序列$(a_n)_{n=0}^\infty$与$(b_n)_{n=0}^\infty$不相等,对任意的$n \in \mathbb{N}$都有
            $|a_n - b_n| \geq 0$,且两个序列不相等,所以至少存在一个$n \in \mathbb{N}$使得
            $|a_n - b_n| > 0$,所以
            \begin{align*}
                  d_{l^1}((a_n)_{n = 0}^\infty, (b_n)_{n = 0}^\infty) = \sum \limits_{n = 0}^\infty |a_n - b_n| > 0
            \end{align*}

      \item 公理(c)

            序列$(a_n)_{n=0}^\infty$与$(b_n)_{n=0}^\infty$,对任意的$n \in \mathbb{N}$都有
            $|a_n - b_n| = |b_n - a_n|$,所以
            \begin{align*}
                  d_{l^1}((a_n)_{n = 0}^\infty, (b_n)_{n = 0}^\infty) = d_{l^1}((b_n)_{n = 0}^\infty, (a_n)_{n = 0}^\infty)
            \end{align*}

      \item 公理(d)

            序列$(a_n)_{n=0}^\infty,(b_n)_{n=0}^\infty$与$(c_n)_{n=0}^\infty$,对任意的$n \in \mathbb{N}$都有
            \begin{align*}
                  |a_n - c_n| \leq |a_n - b_n| + |b_n - c_n|
            \end{align*}
            于是
            \begin{align*}
                  d_{l^1}((a_n)_{n = 0}^\infty, (c_n)_{n = 0}^\infty) & = \sum \limits_{n = 0}^\infty |a_n - c_n|                                                                   \\
                                                                      & \leq \sum \limits_{n = 0}^\infty |a_n - b_n| + |b_n - c_n|                                                  \\
                                                                      & = \sum \limits_{n = 0}^\infty |a_n - b_n| + \sum \limits_{n = 0}^\infty |b_n - c_n|                         \\
                                                                      & = d_{l^1}((a_n)_{n = 0}^\infty, (b_n)_{n = 0}^\infty) + d_{l^1}((b_n)_{n = 0}^\infty, (c_n)_{n = 0}^\infty)
            \end{align*}

\end{itemize}

(2)$d_{l^\infty}$是$X$上的度量。

\begin{itemize}
      \item 公理(a)

            如果序列$(a_n)_{n=0}^\infty$与$(b_n)_{n=0}^\infty$是相等的,那么对任意$n \in \mathbb{N}$都有
            $a_n = b_n$,于是$|a_n - b_n| = 0$,所有
            \begin{align*}
                  d_{l^\infty}((a_n)_{n = 0}^\infty, (b_n)_{n = 0}^\infty) = \sup \limits_{n \in \mathbb{N}} |a_n - b_n| = 0
            \end{align*}

      \item 公理(b)

            序列$(a_n)_{n=0}^\infty$与$(b_n)_{n=0}^\infty$不相等,对任意的$n \in \mathbb{N}$都有
            $|a_n - b_n| \geq 0$,且两个序列不相等,所以至少存在一个$n \in \mathbb{N}$使得
            $|a_n - b_n| > 0$,所以
            \begin{align*}
                  d_{l^\infty}((a_n)_{n = 0}^\infty, (b_n)_{n = 0}^\infty) = \sup \limits_{n \in \mathbb{N}} |a_n - b_n| > 0
            \end{align*}

      \item 公理(c)

            序列$(a_n)_{n=0}^\infty$与$(b_n)_{n=0}^\infty$,对任意的$n \in \mathbb{N}$都有
            $|a_n - b_n| = |b_n - a_n|$,所以
            \begin{align*}
                  d_{l^\infty}((a_n)_{n = 0}^\infty, (b_n)_{n = 0}^\infty) = d_{l^\infty}((b_n)_{n = 0}^\infty, (a_n)_{n = 0}^\infty)
            \end{align*}

      \item 公理(d)

            序列$(a_n)_{n=0}^\infty,(b_n)_{n=0}^\infty$与$(c_n)_{n=0}^\infty$,对任意的$n \in \mathbb{N}$都有
            \begin{align*}
                  |a_n - c_n| \leq |a_n - b_n| + |b_n - c_n|
            \end{align*}
            于是(有用到引理6.4.13(比较原理))
            \begin{align*}
                  d_{l^\infty}((a_n)_{n = 0}^\infty, (c_n)_{n = 0}^\infty) & = \sup \limits_{n \in \mathbb{N}} |a_n - c_n|                  \\
                                                                           & \leq \sup \limits_{n \in \mathbb{N}} |a_n - b_n| + |b_n - c_n|
            \end{align*}
            接下来,我们需要证明
            \begin{align*}
                  \sup \limits_{n \in \mathbb{N}} |a_n - b_n| + |b_n - c_n| = \sup \limits_{n \in \mathbb{N}} |a_n - b_n| + \sup \limits_{n \in \mathbb{N}} |b_n - c_n|
            \end{align*}
            不妨设$M := \sup \limits_{n \in \mathbb{N}} |a_n - b_n| + |b_n - c_n|, p := \sup \limits_{n \in \mathbb{N}} |a_n - b_n|, q := \sup \limits_{n \in \mathbb{N}} |b_n - c_n|$\\
            反证法,假设$M \neq p + q$,那么$M > p + q$或者$M < p + q$,我们以$M > p + q$为例,
            于是存在$m, M > m > p + q$,由最小上界的性质可知(命题6.3.6),存在$n \in \mathbb{N}$使得
            \begin{align*}
                  |a_n - b_n| + |b_n - c_n| > m > p + q
            \end{align*}
            于是$|a_n - b_n| > p$或$|b_n - c_n| > q$,这与$p,q$是最小上确界矛盾。

            所以
            \begin{align*}
                  d_{l^\infty}((a_n)_{n = 0}^\infty, (c_n)_{n = 0}^\infty) & \leq \sup \limits_{n \in \mathbb{N}} |a_n - b_n| + |b_n - c_n|                                                        \\
                                                                           & = \sup \limits_{n \in \mathbb{N}} |a_n - b_n| + \sup \limits_{n \in \mathbb{N}} |b_n - c_n|                           \\
                                                                           & = d_{l^\infty}((a_n)_{n = 0}^\infty, (b_n)_{n = 0}^\infty) + d_{l^\infty}((b_n)_{n = 0}^\infty, (c_n)_{n = 0}^\infty)
            \end{align*}
\end{itemize}

(3)存在一个由$X$中元素构成的序列$x^{(1)}, x^{(2)}, \cdots$(即序列的序列),它依度量$d_{l^\infty}$收敛但不依度量$d_{l^1}$收敛。

这里可以手动构造一个
\begin{align*}
      (a_n)_{n=0}^\infty, a_n = 1/k \\
      x^{(k)} = (a_n)_{n=0}^\infty
\end{align*}
于是$x^{(k)}$收敛于$1/k$,序列$x^{(1)}, x^{(2)}, \cdots$依度量$d_{l^\infty}$收敛于$(a_n)_{n=0}^\infty, a_n = 0$,
但不依度量$d_{l^1}$收敛,这里简单说明下原因,因为两个点$x^{(p)}, x^{(q)}$,每个坐标分量$i$都有
\begin{align*}
      |x^{(p)}_i - x^{(q)}_i| = |1/p - 1/q| = \frac{1}{pq}
\end{align*}
而每个点的坐标分量个数是$\infty$,所以$d_{l^1}(x^{(p)}, x^{(q)}) = \infty$

(4)任何一个依度量$d_{l^1}$收敛的序列都依度量$d_{l^\infty}$收敛。

不妨设序列依度量$d_{l^1}$收敛于$(a_n)_{n=0}^\infty$,那么对于任意$\epsilon > 0$存在一个$N$,使得
\begin{align*}
      \sum \limits_{i = 0}^\infty |a_i - x_i| \leq \epsilon
\end{align*}
对$n \geq N$均成立。

于是可得
\begin{align*}
      \sup \limits_{i \in \mathbb{N}} |a_i - x_i| \leq \epsilon
\end{align*}
所以,序列依度量$d_{l^\infty}$收敛于$(a_n)_{n=0}^\infty$。

\section*{12.1.16}

由$(x_n)_{n=1}^\infty$收敛于点$x \in X$,对于任意$\epsilon > 0, \frac{1}{2} \epsilon > 0$,存在一个$N \geq 1$使得
\begin{align*}
      d(x_n, x) \leq \frac{1}{2} \epsilon
\end{align*}
对所有的$n \geq N$均成立。

类似地,存在$M \geq 1$使得
\begin{align*}
      d(y_n, y) \leq \frac{1}{2} \epsilon
\end{align*}
对所有的$n \geq M$均成立。

取$m = max(N, M)$使得
\begin{align*}
      d(x_n, y_n)           & \leq d(x_n, x) + d(x, y_n)                \\
                            & \leq d(x_n, x) + d(x, y) + d(y, y_n)      \\
                            & \implies                                  \\
      d(x_n, y_n) - d(x, y) & \leq  d(x_n, x) + d(y, y_n) \leq \epsilon
\end{align*}
对所有的$n \geq m$均成立。

由$\epsilon$的任意性和$d(x, y)$是常量可知
\begin{align*}
      \lim\limits_{n \to \infty} d(x_n, y_n) = d(x, y)
\end{align*}



\end{document}