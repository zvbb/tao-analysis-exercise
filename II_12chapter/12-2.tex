\documentclass{article}
\usepackage{mathtools} 
\usepackage{fontspec}
\usepackage[UTF8]{ctex}
\usepackage{amsthm}
\usepackage{mdframed}
\usepackage{xcolor}
\usepackage{amssymb}
\usepackage{amsmath}


% 定义新的带灰色背景的说明环境 zremark
\newmdtheoremenv[
  backgroundcolor=gray!10,
  % 边框与背景一致,边框线会消失
  linecolor=gray!10
]{zremark}{说明}


\begin{document}
\title{12.2 习题}
\author{张志聪}
\maketitle

\section*{12.2.1}

任意$x_0 \in X$,要么$x_0 \in E$要么$x_0 \notin E$。

\begin{itemize}
      \item $x_0 \in E$

            任意$0 < r < 1$,由$d_{disc}$度量的定义可知,
            $B(x_0, r) = \{x_0\}$,所以$B(x_0, r) \subseteq E$,
            所以$x_0$是$E$的内点。

      \item $x_0 \notin E$

            任意$0 < r < 1$,由$d_{disc}$度量的定义可知,
            $B(x_0, r) = \{x_0\}$,所以$B(x_0, r) \cap E = \varnothing$,
            所以$x_0$是$E$的外点。
\end{itemize}

\section*{12.2.2}

证明路径:$(a) \implies (b) \implies (c) \implies (a)$

\begin{itemize}
      \item $(a) \implies (b)$

            由闭包的定义(定义12.2.9)可知,如果(a)成立,那么对任意的半径$r > 0$,球$B(x_0, r)$与$E$的交集总是非空的。
            所以$x_0$不可能是$E$的外点。

            球$B(x_0, r)$与$E$的交集总是非空的,于是有两种情况。    \\
            情况1:$B(x_0, r) \subseteq E$,此时$x_0$是$E$的内点。 \\
            情况2:存在$x \in B(x_0, r), x \notin E$,此时$x_0$是$E$的边界点。

            综上,(b)成立。


      \item $(b) \implies (c)$

            (b)成立。

            \begin{itemize}
                  \item $x_0$是$E$的内点

                        那么可以把序列$(x_n)_{n=1}^\infty$设置为常量序列$(x_0)_{n=1}^\infty$。

                  \item $x_0$是$E$的边界点

                        那么任意$r > 0$,
                        都有$B(x_0, r) \cap E \neq \varnothing$(因为如果$B(x_0, r) \cap E = \varnothing$,那么$x_0$是$E$的外点)。

                        仿照引理8.4.5的证明,构造序列。

                        对于任意的正整数$n$,设$X_n$表示集合
                        \begin{align*}
                              X_n := \{x \in E: x \in B(x_0, \frac{1}{n})\}
                        \end{align*}
                        由之前的分析可得,对每一个$n$都有$X_n$是非空的。利用选择公理(或者可数选择公理),能够找到一个
                        序列$(x_n)_{n = 1}^\infty$使得$x_n \in X_n$对所有的$n \geq 1$均成立。
                        特别地,对所有的$n$均有$x_n \in E \cap B(x_0, \frac{1}{n})$,于是
                        \begin{align*}
                              0 \leq d(x_0, x_n) \leq \frac{1}{n}
                        \end{align*}
                        根据夹逼定理(推论6.4.14)有$\lim\limits_{n \to \infty} d(x_0, x_n) = 0$,
                        所以序列$(x_n)_{n = 1}^\infty$依度量$d$收敛于点$x_0$。
            \end{itemize}

      \item $(c) \implies (a)$

            (c)成立,由收敛定义(定义12.1.14)可知,对任意$\epsilon > 0$,存在一个$N \geq 1$使得
            \begin{align*}
                  d(x_n, x_0) < \epsilon
            \end{align*}
            对所有$n \geq N$均成立(注意:这里的把定义中的$\leq$改成了$<$,并不影响正确性)。

            做一下变形,把$\epsilon$看做半径,球$B(x_0, \epsilon)$与$E$的交集是非空的,
            这是因为对$n \geq N$的$x_n$我们有$d(x_n, x_0) < \epsilon$,所以$x_n \in B(x_0, \epsilon)$
            且$x_n \in E$。

            由$\epsilon$的任意性可知,$x_0$是$E$的附着点。
\end{itemize}

\section*{12.2.3}

\begin{zremark}
      先证明以下命题:

      \textbf{设$(X, d)$是一个度量空间,$E$是$X$的子集,并设$x_0$是$X$中的一个点。那么$x_0$要么是$E$的内点,
            要么是$E$的外点,要么是$E$的边界点(存在三歧性)。}


      证明:
      以下的情况是互斥的:
      \begin{itemize}
            \item $x_0 \in E$

                  首先$x_0$不可能是$E$的外点,如果$x_0$是$E$的外点,那么存在$r > 0$使得$B(x_0, r) \cap E = \varnothing$,
                  因为$d(x_0, x_0) = 0$所以$x_0 \in B(x_0, r)$,于是$x_0 \notin E$,存在矛盾。

                  以下的情况是互斥的:
                  \begin{itemize}
                        \item $x_0$是$E$的边界点

                              由定义12.2.5可知,$x_0$不可能同时是$E$的内点。

                        \item $x_0$不是$E$的边界点

                              之前已经说明$x_0$不是$E$的外点,假设$x_0$也不是$E$的内点,那么
                              $x_0$就是$E$的边界点,存在矛盾,所以$x_0$是$E$的内点。

                  \end{itemize}
            \item $x_0 \notin E$

                  首先$x_0$不可能是$E$的内点,如果$x_0$是$E$的内点,那么存在$r > 0$使得$B(x_0, r) \subseteq E$,
                  因为$x_0 \in B(x_0, r)$,所以$x_0 \in E$,这与$x_0 \notin E$矛盾。

                  以下的情况是互斥的:
                  \begin{itemize}
                        \item $x_0$是$E$的外点

                        \item $x_0$不是$E$的外点

                              由定义12.2.5可知,$x_0$既不是$E$的内点也不是$E$的外点,所以$x_0$是$E$的边界点。
                  \end{itemize}
      \end{itemize}

      综上,$x_0 \in E$,$x_0$要么是$E$的内点,要么是$E$的边界点;
      $x_0 \notin E$,$x_0$要么是$E$的外点,要么是$E$的边界点。命题得证。


\end{zremark}


\begin{itemize}
      \item (a)
            \begin{itemize}
                  \item $\Rightarrow$

                        由注12.2.6可知$int(E) \subseteq E$。

                        任意$x_0 \in E$,因为$E$是开的,那么$E$不包含自身的任意边界点,
                        所以$x_0 \notin \partial E $;由$d(x_0, x_0)=0$可知$x_0 \notin ext(E)$。
                        于是由说明1可知$x_0 \in int(E)$,所以$E \subseteq int(E)$。

                        所以$E = int(E)$
                  \item $\Leftarrow$

                        $E = int(E)$,那么任意$x_0 \in E$,都有$x_0 \in int(E)$,即$E$中不包含边界点,
                        由定义12.2.12可知,$E$是开的。
            \end{itemize}

      \item (b)
            \begin{itemize}
                  \item $\Rightarrow$

                        反证法,假设存在$x_0$是附着点且$x_0 \notin E$。
                        $x_0$是$E$的附着点,那么由定义12.2.9可知,对任意的半径$r > 0$,
                        球$B(x_0, r) \cap E \neq \varnothing$,所以可得$x_0$不可能是$E$的外点。
                        又由说明1可得$x_0$要么是$E$的边界点,要么是$E$的内点。如果
                        $x_0$是$E$的边界点,由于$E$是闭的,所以$x_0 \in E$,与假设矛盾;如果$x_0$是$E$的内点,
                        于是$x_0 \in E$,与假设矛盾。

                        综上,假设不成立。

                  \item $\Leftarrow$

                        反证法,假设$E$不是闭的,即存在边界点$x_0$且$x_0 \notin E$。
                        由推论12.2.11可知$x_0$是$E$的附着点,由题设可知$x_0 \in E$,与假设矛盾。
            \end{itemize}

      \item (c)
            \begin{itemize}
                  \item 球$B(x_0, r)$是开集

                        对任意的$x \in B(x_0, r)$,都有$d(x_0, x) < r$,令$r^\prime = r - d(x_0, x)$,
                        于是$B(x, r^\prime) \subseteq B(x_0, r)$,因为任意$y \in B(x, r^\prime)$,都有
                        \begin{align*}
                              d(x_0, y) \leq d(x_0, x) + d(x, y) < d(x_0, x) + r^\prime = r
                        \end{align*}
                        由(a)可知,球$B(x_0, r)$是开集。

                  \item 闭球是闭集

                        $B := \{x \in X : d(x, x_0) \leq r \}$,让$(x_n)_{n = m}^\infty$是$B$中任意一个收敛序列,
                        假设$\lim\limits_{n \to \infty} x_n = b \notin E$,于是$d(x_0, b) > r$,
                        令$\epsilon = d(x_0, b) - r > 0$,于是存在$N \geq m$使得
                        \begin{align*}
                              d(x_n, b) & < \epsilon              \\
                              d(x_n, b) & < d(x_0, b) - r         \\
                              r         & < d(x_0, b) + d(x_n, b) \\
                              r         & < d(x_0, x_n)
                        \end{align*}
                        对所有$n \geq N$均成立,这与$x_n \in B$矛盾。

                        于是$b \in B$,由(b)可知,$B$是闭集。

            \end{itemize}


      \item (d)

            令$E := \{x_0\}$,$E$中的任意一个收敛序列$(x_n)_{n = m}^\infty$都是与$(x_0)_{n=m}^\infty$相等,
            所以$\lim\limits_{n \to \infty} x_n = x_0 \in E$。由(b)可知,$E$是闭集。

      \item (e)
            由于$int(E) = ext(X \setminus E), ext(E) = int(X \setminus E)$,于是可得$\partial E = \partial (X \setminus E)$。
            \begin{itemize}
                  \item $\Rightarrow$

                        $E$是开的,则$\partial E \cap E = \varnothing$,于是可得$\partial E \subseteq (X \setminus E)$,
                        即$\partial E = \partial (X \setminus E) \subseteq (X \setminus E)$,所以$X \setminus E$是闭的。

                  \item $\Leftarrow$

                        $X \setminus E$是闭的,则$\partial (X \setminus E) \subseteq (X \setminus E)$,由$\partial E = \partial (X \setminus E)$
                        可得$\partial E \cap E = \varnothing$,所以$E$是开的。
            \end{itemize}

      \item (f)

            (f.1)

            使用(a)可知,对任意$x \in E_1 \cap E_2 \cap ... \cap E_n$,
            对任意$E_i (1 \leq i \leq n)$存在一个$r_i > 0$使得$B(x, r_i) \subseteq E_i$。

            由于$n$是有限的,所以可取$r = \min\{r_1, r_2, ..., r_n\}$,此时$B(x, r) \subseteq E_i$,
            于是再次利用(a)可得,$E_1 \cap E_2 \cap ... \cap E_n$是开的。

            (f.2)

            $F_1,...,F_2$是闭的,由(e)可知,$X \setminus F_1,...,X \setminus F_n$是开的,
            $F_1 \cup F_2 \cup ... \cup F_n = X \setminus \left( (X \setminus F_1) \cap (X \setminus F_2)\cap ... \cap (X \setminus F_n) \right)$,
            再次利用(e)可知,$F_1 \cup F_2 \cup ... \cup F_n$是闭的。

      \item (g)

            (g.1)

            任意$x \in \bigcup_{\alpha \in I} E_{\alpha}$,那么,存在某个$\alpha \in I$使得$x \in E_{\alpha}$,
            又因为$E_{\alpha}$是开的,所以存在$r > 0$使得$B(x, r) \subseteq E_{\alpha} \in \bigcup_{\alpha \in I} E_{\alpha}$,
            所以$\bigcup_{\alpha \in I} E_{\alpha}$是开的。

            (g.2)

            % 因为$\bigcap_{\alpha \in I} F_{\alpha} = X \setminus \bigcup_{\alpha \in I} (X \setminus F_{\alpha})$,
            % 由(e)可知,$X \setminus F_{\alpha}$是开的,所以利用(g.1)可得$\bigcup_{\alpha \in I} (X \setminus F_{\alpha})$是开的,
            % 再次利用(e)可知$X \setminus \bigcup_{\alpha \in I} (X \setminus F_{\alpha})$是闭的。
            % todo


      \item (h)

            (h.1)

            反证法,假设$int(E)$不是包含在$E$中的最大开集,即存在$V \subseteq E, V \not \subseteq int(E)$。

            由假设可知,存在$x \in V, x \notin int(E)$,由于$V \subseteq E$,所以$x \in E$,于是$x \in int(E)$或$x \in \partial E$,
            因为$x \notin int(E)$,所以$x \in \partial E$。

            由于$V$是开集,所以存在$r > 0$,使得$B(x, r) \subseteq V \subseteq E$,
            于是$x$是$E$的内点,即$x \in int(E)$,存在矛盾。

            (h.2)

            反证法,假设$\overline{E}$不是包含$E$的最小闭集,即存在$K \supset E, K \not \supset \overline{E}$。

            由假设可知,存在$x \in \overline{E}, x \notin K$。
            因为$x$是$E$的附着点,于是由命题12.2.10(c)可知在$E$中(也在$K$中)构造一个收敛于$x$的序列$(x_n)_{n = m}^\infty$,
            但$x \notin K$,这与(b)矛盾。
\end{itemize}

\section*{12.2.4}

\begin{itemize}
      \item (a)

            反证法,假设存在$x \in \overline{B}, x \notin C$。

            $x \notin C$,可知$x \in X \setminus C$,而$X \setminus C = \{x \in X: d(x, x_0) > r\}$,
            那么$d(x, x_0) > r$。

            因为$x \in \overline{B}$,所以对任意半径$r^\prime > 0$都有$B(x, r^\prime) \cap B \neq \varnothing$,
            于是令$r^\prime = d(x_0, x) - r > 0, y \in B(x, r^\prime) \cap B$。

            按照定义12.1.2我们有
            \begin{align*}
                  d(x_0, x)             & \leq d(x_0, y) + d(x, y) \\
                  d(x_0, x) - d(x_0, y) & \leq d(x, y)
            \end{align*}
            因为$y \in B(x, r^\prime)$于是$d(x, y) < r^\prime$,所以
            \begin{align*}
                  d(x_0, x) - d(x_0, y) \leq d(x, y) & < r^\prime = d(x_0, x) - r \\
                  d(x_0, x) - d(x_0, y)              & < d(x_0, x) - r            \\
                  r                                  & < d(x_0, y)                
            \end{align*}
            这与$y \in B$矛盾。


      \item (b)

            在离散度量$d_{disc}$中,$B := B(x_0, 1)$,是单点集,由命题12.2.15(d)可知,$B$是闭集,
            由命题12.2.15(b)可知,$B = \overline{B}$。

            而$C := \{x \in X: d_{disc}(x_0, x) \leq 1\}$就是$X$本身,此时$B \subset C$。
\end{itemize}

\end{document}