\documentclass{article}
\usepackage{mathtools} 
\usepackage{fontspec}
\usepackage[UTF8]{ctex}
\usepackage{amsthm}
\usepackage{mdframed}
\usepackage{xcolor}
\usepackage{amssymb}
\usepackage{amsmath}


% 定义新的带灰色背景的说明环境 zremark
\newmdtheoremenv[
  backgroundcolor=gray!10,
  % 边框与背景一致,边框线会消失
  linecolor=gray!10
]{zremark}{说明}


\begin{document}
\title{12.3 习题}
\author{张志聪}
\maketitle

\section*{12.3.1}

\begin{itemize}
  \item $\Rightarrow$

        $E$是关于$Y$相对闭的,那么$F = Y \setminus E$是关于$Y$相对开的,
        由命题12.3.4(a)可知,存在$X$中的开集$V \subseteq X$使得$F = V \cap Y$是关于$Y$相对开的。

        令$K = X \setminus V$,于是可得$K$是关于$X$的闭集。现在我们来证明
        $E = K \cap Y$。

        任意$x \in E$,于是
        \begin{align*}
           & x \in E             \\
           & \implies            \\
           & x \in Y, x \notin F \\
           & \implies            \\
           & x \notin V          \\
           & \implies            \\
           & x \in K
        \end{align*}

        任意$y \in K \cap Y$,于是
        \begin{align*}
           & y \in K \cap Y                                                             \\
           & \implies                                                                   \\
           & y \in K, y \in Y                                                           \\
           & \implies                                                                   \\
           & y \notin V                                                                 \\
           & \implies                                                                   \\
           & y \notin F                                                                 \\
           & y \in E       \;\;   \text{因为$y \notin E, F = Y \setminus E, y \in F$存在矛盾} \\
        \end{align*}

  \item $\Leftarrow$

        假设存在某个闭集$K$使得$E = K \cap Y$,那么$V = X \setminus K$是关于$X$相对开的,于是$V \cap Y$是关于$Y$相对开的,
        于是
        \begin{align*}
          Y \setminus (V \cap Y) = (Y \setminus V) \cup (Y \setminus Y) = Y \setminus V 
        \end{align*}
        是关于$Y$相对闭的。

        接下来我们证明$E = Y \setminus V$。

        任意$x \in E$,我们有$x \in K, x \in Y$,所以$x \notin V = X \setminus K$,
        于是$x \in Y \setminus V$,所以$E \subseteq (Y \setminus V)$。

        任意$x \in Y \setminus V$,我们有$x \in Y, x \notin V$,于是$x \in X, x \notin V$,
        所以$x \in K$,因为$V = X \setminus K$,于是$x \in Y \cap K = E$,所以$(Y \setminus V) \subseteq E$。


\end{itemize}

\end{document}